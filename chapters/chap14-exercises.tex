% Exercises (Exercises) for Chapter 14

\section*{Exercises}
\addcontentsline{toc}{section}{Exercises}

\subsection*{Easy}

\begin{exercisebox}[easy]
\begin{problem}[Undercomplete AE]
Explain why undercomplete AEs can avoid trivial identity mapping.
\end{problem}
\begin{hintbox}
Bottleneck limits capacity.
\end{hintbox}
\end{exercisebox}


\begin{exercisebox}[easy]
\begin{problem}[Denoising Noise]
How does noise type affect learned features?
\end{problem}
\begin{hintbox}
Gaussian vs. masking vs. salt-and-pepper.
\end{hintbox}
\end{exercisebox}


\begin{exercisebox}[easy]
\begin{problem}[Sparse Codes]
List benefits of sparsity in latent codes.
\end{problem}
\begin{hintbox}
Interpretability, robustness, compression.
\end{hintbox}
\end{exercisebox}


\begin{exercisebox}[easy]
\begin{problem}[Contractive Penalty]
What does a Jacobian penalty encourage?
\end{problem}
\begin{hintbox}
Local invariance.
\end{hintbox}
\end{exercisebox}


\subsection*{Medium}

\begin{exercisebox}[medium]
\begin{problem}[Loss Choices]
Compare MSE vs. cross-entropy for images.
\end{problem}
\begin{hintbox}
Data scale/likelihood assumptions.
\end{hintbox}
\end{exercisebox}


\begin{exercisebox}[medium]
\begin{problem}[Regularisation Trade-offs]
Contrast denoising vs. contractive penalties.
\end{problem}
\begin{hintbox}
Noise robustness vs. local smoothing.
\end{hintbox}
\end{exercisebox}


\subsection*{Hard}

\begin{exercisebox}[hard]
\begin{problem}[Jacobian Penalty]
Derive gradient of contractive loss w.r.t. encoder parameters.
\end{problem}
\begin{hintbox}
Chain rule through Jacobian norm.
\end{hintbox}
\end{exercisebox}


\begin{exercisebox}[hard]
\begin{problem}[Generative Link]
Explain links between AEs and VAEs/flows.
\end{problem}
\begin{hintbox}
Likelihood vs. reconstruction objectives.
\end{hintbox}
\end{exercisebox}



\begin{exercisebox}[hard]
\begin{problem}[Advanced Topic 1]
Explain a key concept from this chapter and its practical applications.
\end{problem}
\begin{hintbox}
Consider the theoretical foundations and real-world implications.
\end{hintbox}
\end{exercisebox}


\begin{exercisebox}[hard]
\begin{problem}[Advanced Topic 2]
Analyse the relationship between different techniques covered in this chapter.
\end{problem}
\begin{hintbox}
Look for connections and trade-offs between methods.
\end{hintbox}
\end{exercisebox}


\begin{exercisebox}[hard]
\begin{problem}[Advanced Topic 3]
Design an experiment to test a hypothesis related to this chapter's content.
\end{problem}
\begin{hintbox}
Consider experimental design, metrics, and potential confounding factors.
\end{hintbox}
\end{exercisebox}


\begin{exercisebox}[hard]
\begin{problem}[Advanced Topic 4]
Compare different approaches to solving a problem from this chapter.
\end{problem}
\begin{hintbox}
Consider computational complexity, accuracy, and practical considerations.
\end{hintbox}
\end{exercisebox}


\begin{exercisebox}[hard]
\begin{problem}[Advanced Topic 5]
Derive a mathematical relationship or prove a theorem from this chapter.
\end{problem}
\begin{hintbox}
Start with the definitions and work through the logical steps.
\end{hintbox}
\end{exercisebox}


\begin{exercisebox}[hard]
\begin{problem}[Advanced Topic 6]
Implement a practical solution to a problem discussed in this chapter.
\end{problem}
\begin{hintbox}
Consider the implementation details and potential challenges.
\end{hintbox}
\end{exercisebox}


\begin{exercisebox}[hard]
\begin{problem}[Advanced Topic 7]
Evaluate the limitations and potential improvements of techniques from this chapter.
\end{problem}
\begin{hintbox}
Consider both theoretical limitations and practical constraints.
\end{hintbox}
\end{exercisebox}

