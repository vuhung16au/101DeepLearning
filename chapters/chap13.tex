% Chapter 13: Linear Factor Models

\chapter{Linear Factor Models}
\label{chap:linear-factor-models}

This chapter introduces probabilistic models with linear structure, which form the foundation for many unsupervised learning methods.


\section*{Learning Objectives}
\addcontentsline{toc}{section}{Learning Objectives}

After studying this chapter, you will be able to:

\begin{enumerate}
    \item Explain probabilistic PCA and factor analysis and their relationships to PCA.
    \item Derive EM updates for latent variable models with linear-Gaussian structure.
    \item Compare identifiability and rotation issues in factor models.
    \item Connect linear latent models to modern representation learning.
\end{enumerate}



\section*{Intuition}
\addcontentsline{toc}{section}{Intuition}

Linear factor models posit a small set of hidden sources generating observed data through linear mixing plus noise. The learning problem is to recover those hidden coordinates that best explain variance while respecting uncertainty.


% Chapter 13, Section 1

\section{Probabilistic PCA \difficultyInline{intermediate}}
\label{sec:prob-pca}

Probabilistic PCA extends classical PCA by providing a probabilistic framework that handles uncertainty and enables principled treatment of noise and missing data in dimensionality reduction.

\subsection{Principal Component Analysis Review}

PCA finds orthogonal directions of maximum variance through the transformation $\vect{z} = \mat{W}^\top (\vect{x} - \boldsymbol{\mu})$ where $\mat{W}$ contains principal components (eigenvectors of covariance matrix). In deep learning, this formula is fundamental for dimensionality reduction in preprocessing steps, where for example, when training a neural network on high-dimensional image data, PCA can reduce the input dimensionality from thousands of pixels to a smaller set of principal components, making training more efficient while preserving the most important variance in the data.

\subsection{Probabilistic Formulation}

The probabilistic formulation models observations through a generative process:
\begin{align}
\vect{z} &\sim \mathcal{N}(\boldsymbol{0}, \mat{I}) \\
\vect{x} \,|\, \vect{z} &\sim \mathcal{N}(\mat{W}\vect{z} + \boldsymbol{\mu}, \sigma^2 \mat{I})
\end{align}
\noindent\textbf{Explanations.}
\begin{itemize}[leftmargin=1.5em]
  \item $\vect{z} \sim \mathcal{N}(\boldsymbol{0}, \mat{I})$ means the latent variable has a \emph{standard normal} prior: zero mean and identity covariance (independent unit-variance components).
  \item $\vect{x} | \vect{z} \sim \mathcal{N}(\mat{W}\vect{z}+\boldsymbol{\mu}, \sigma^2\mat{I})$ is a \emph{Gaussian likelihood}: data is a linear mapping of $\vect{z}$ via loadings $\mat{W}$ plus mean $\boldsymbol{\mu}$, with isotropic noise variance $\sigma^2$.
\end{itemize}
Marginalizing the latent variable yields
\begin{equation}
\vect{x} \sim \mathcal{N}\big(\boldsymbol{\mu},\; \underbrace{\mat{W}\mat{W}^\top}_{\text{signal}} + \underbrace{\sigma^2\mat{I}}_{\text{noise}}\big).
\end{equation}
Here $\mat{W}\mat{W}^\top + \sigma^2\mat{I}$ is the \emph{covariance matrix} of the observed data: $\mat{W}\mat{W}^\top$ captures variance explained by latent factors and $\sigma^2\mat{I}$ is residual (isotropic) noise.

\subsection{Learning}

Learning in probabilistic PCA maximizes the likelihood using the EM algorithm, where the E-step computes the posterior distribution $p(\vect{z}|\vect{x})$ to estimate the latent variables given the observed data, and the M-step updates the parameters $\mat{W}$, $\boldsymbol{\mu}$, and $\sigma^2$ to maximize the expected log-likelihood. This iterative process allows the model to learn the optimal linear transformation and noise parameters that best explain the observed data, where the EM algorithm is particularly useful because it handles the latent variables naturally and provides a principled way to estimate parameters in the presence of missing data. As the noise variance $\sigma^2 \to 0$, the probabilistic PCA recovers standard PCA, showing that classical PCA is a special case of the probabilistic formulation when we assume no noise in the data generation process.

% \subsection{Visual aids}
% \addcontentsline{toc}{subsubsection}{Visual aids (PPCA)}

% \begin{figure}[h]
%   \centering
%   \begin{tikzpicture}
%     \begin{axis}[
%       width=0.48\textwidth,height=0.36\textwidth,
%       xlabel={$x_1$}, ylabel={$x_2$}, grid=both]
%       % data cloud
%       \addplot+[only marks,mark=*,mark size=0.8pt,bookpurple!60] coordinates{(-1.2,-0.9) (-0.8,-0.7) (-0.2,-0.3) (0.2,0.1) (0.6,0.4) (1.0,0.8)};
%       % principal axis
%       \addplot[bookred,very thick,domain=-1.5:1.5]{0.8*x};
%     \end{axis}
%   \end{tikzpicture}
%   \caption{PCA principal axis capturing maximum variance.}
%   \label{fig:ppca-axis}
% \end{figure}

\subsection{Historical context and references}

The probabilistic formulation of PCA was developed to connect classical PCA with latent variable models and enable principled handling of noise and missing data, representing a significant advancement in dimensionality reduction techniques. This probabilistic approach has been widely adopted in machine learning and deep learning applications, where it provides a foundation for understanding how linear transformations can capture the essential structure in high-dimensional data. Previous work has been establishing the theoretical foundations and practical applications of probabilistic PCA, while Goodfellow and colleagues have shown how these concepts extend to modern deep learning architectures. Real-world applications include image compression, noise reduction in signal processing, and preprocessing for neural network training, where the probabilistic framework allows for better handling of uncertainty and missing data compared to classical PCA.

% Chapter 13, Section 2

\section{Factor Analysis \difficultyInline{intermediate}}
\label{sec:factor-analysis}

Factor analysis extends probabilistic PCA by allowing different noise variances for each observed dimension, providing more flexibility in modeling measurement error and enabling better handling of heterogeneous data where different variables may have different levels of reliability. The model assumes that observed data $\vect{x} | \vect{z} \sim \mathcal{N}(\mat{W}\vect{z} + \boldsymbol{\mu}, \boldsymbol{\Psi})$ is generated from latent factors through linear transformation plus diagonal noise covariance $\boldsymbol{\Psi}$, where each observed dimension has its own noise variance, allowing for more realistic modeling of real-world data where different measurements may have different levels of precision. In deep learning, factor analysis is particularly useful for understanding the underlying structure of high-dimensional data and for preprocessing steps where we need to account for different levels of noise in different features.

\textbf{Applications:} Psychology, social sciences, finance - Factor analysis has been extensively used in psychology to identify underlying personality traits and cognitive abilities from questionnaire responses, in social sciences to understand latent social factors from survey data, and in finance to identify common risk factors in portfolio management and asset pricing models.

\subsection{Learning via EM}

EM alternates between inferring posteriors over factors and updating loadings $\mat{W}$ and noise $\boldsymbol{\Psi}$. Diagonal noise permits modeling idiosyncratic measurement error per dimension \textcite{Bishop2006}.

\begin{figure}[h]
  \centering
  \begin{tikzpicture}
    \begin{axis}[
      width=0.48\textwidth,height=0.36\textwidth,
      xlabel={Latent dim}, ylabel={Explained variance}, ymode=linear, grid=both]
      \addplot[bookpurple,very thick] coordinates{(1,0.55) (2,0.7) (3,0.78) (4,0.82) (5,0.84)};
    \end{axis}
  \end{tikzpicture}
  \caption{Explained variance as a function of latent dimensionality (illustrative).}
  \label{fig:fa-variance}
\end{figure}


% Chapter 13, Section 3

\section{Independent Component Analysis \difficultyInline{intermediate}}
\label{sec:ica}

\subsection{Objective}

Find independent sources from linear mixtures:
\begin{equation}
\vect{x} = \mat{A}\vect{s}
\end{equation}

where $\vect{s}$ contains independent sources.

\subsection{Non-Gaussianity}

ICA exploits that independent signals are typically non-Gaussian.

\textbf{Applications:}
\begin{itemize}
    \item Blind source separation (cocktail party problem)
    \item Signal processing
    \item Feature extraction
\end{itemize}

% \subsection{Visual aids}
% \addcontentsline{toc}{subsubsection}{Visual aids (ICA)}

% \begin{figure}[h]
%   \centering
%   \begin{tikzpicture}
%     \begin{axis}[
%       width=0.48\textwidth,height=0.36\textwidth,
%       xlabel={$s_1$}, ylabel={$s_2$}, grid=both]
%       \addplot+[only marks,mark=*,mark size=0.9pt,bookpurple!70] coordinates{(-1,0) (-0.8,0) (0.8,0) (1,0)};
%       \addplot+[only marks,mark=*,mark size=0.9pt,bookred!70] coordinates{(0,-1) (0,-0.8) (0,0.8) (0,1)};
%     \end{axis}
%   \end{tikzpicture}
%   \caption{Independent sparse sources aligned with axes (illustrative).}
%   \label{fig:ica-sources}
% \end{figure}

% \subsection{Historical context and references}

% ICA popularized as a blind source separation tool and applied widely in signal processing and neuroscience \textcite{Bishop2006,GoodfellowEtAl2016}.


% Chapter 13, Section 4

\section{Sparse Coding \difficultyInline{intermediate}}
\label{sec:sparse-coding}

Learn overcomplete dictionary where data has sparse representation:
\begin{equation}
\min_{\mat{D}, \vect{z}} \|\vect{x} - \mat{D}\vect{z}\|^2 + \lambda \|\vect{z}\|_1
\end{equation}

\textbf{Applications:}
\begin{itemize}
    \item Image denoising
    \item Feature learning
    \item Compression
\end{itemize}

\subsection{Optimization and interpretation}

The $\ell_1$ penalty promotes sparsity, yielding parts-based representations and robust denoising. Alternating minimization over dictionary $\mat{D}$ and codes $\vect{z}$ is common; convolutional variants are used in images \textcite{GoodfellowEtAl2016}.

\begin{figure}[h]
  \centering
  \begin{tikzpicture}
    \begin{axis}[
      width=0.48\textwidth,height=0.36\textwidth,
      xlabel={$z_1$}, ylabel={$z_2$}, grid=both]
      \addplot[bookpurple,very thick,domain=-2:2]{1.0 - abs(x)}; % diamond-like L1 contour (schematic)
    \end{axis}
  \end{tikzpicture}
  \caption{Schematic illustrating $\ell_1$-induced sparsity geometry.}
  \label{fig:l1-geometry}
\end{figure}


% Chapter 13: Real World Applications

\section{Real World Applications}
\label{sec:linear-factor-real-world}


Linear factor models, including PCA, ICA, and sparse coding, provide interpretable representations of complex data. These techniques underpin many practical systems for data compression, signal processing, and feature extraction.

\subsection{Facial Recognition Systems}

Efficient and robust face identification:

\begin{itemize}
    \item \textbf{Eigenfaces for face recognition:} One of the earliest successful face recognition systems used PCA to represent faces efficiently. Each face is expressed as a weighted combination of "eigenfaces" (principal components). This reduces storage requirements dramatically—instead of storing full images, systems store just a few dozen numbers per person while maintaining recognition accuracy.
    
    \item \textbf{Robust to lighting and expression:} Factor models capture the most important variations in face appearance (identity) while being less sensitive to less important variations (lighting, expression). This makes recognition work under different conditions without requiring massive datasets of each person.
    
    \item \textbf{Privacy-preserving representations:} The compressed representations from factor models can enable face recognition without storing actual face images, providing better privacy protection. The low-dimensional codes contain enough information for matching but can't easily be reversed to reconstruct recognizable faces.
\end{itemize}

\subsection{Audio Signal Processing}

Extracting meaning from sound:

\begin{itemize}
    \item \textbf{Music analysis and recommendation:} Spotify and similar services use factor models to decompose songs into latent features (mood, genre, tempo, instrumentation). These compact representations enable efficient similarity search across millions of songs. When you like a song, the system finds others with similar factor patterns.
    
    \item \textbf{Noise reduction in hearing aids:} Modern hearing aids use sparse coding to separate speech from background noise. The factor model learns to represent speech efficiently with few active components while requiring many more components for noise. This distinction enables selective amplification of speech while suppressing noise.
    
    \item \textbf{Source separation:} Isolating individual instruments in music recordings or separating overlapping speakers in recordings uses independent component analysis (ICA). This enables remixing old recordings, improving audio quality, and creating karaoke tracks from normal songs.
\end{itemize}

\subsection{Anomaly Detection in Systems}

Finding unusual patterns in complex data:

\begin{itemize}
    \item \textbf{Network intrusion detection:} Cybersecurity systems use factor models to represent normal network traffic patterns compactly. Unusual activities (potential attacks) don't fit well into this low-dimensional representation, triggering alerts. This approach detects novel attacks without explicitly programming rules for every possible threat.
    
    \item \textbf{Manufacturing quality control:} Production lines use factor models to analyze sensor data from equipment. Normal operations cluster in low-dimensional space; deviations indicate problems like tool wear, calibration drift, or defects. Early detection prevents defective products and costly equipment damage.
    
    \item \textbf{Healthcare monitoring:} Wearable devices compress continuous health metrics (heart rate, activity, sleep patterns) into factor representations. Doctors can spot concerning trends without examining raw data streams, and anomaly detection alerts patients to unusual patterns warranting attention.
\end{itemize}

\subsection{Practical Advantages}

Why factor models remain valuable:
\begin{itemize}
    \item \textbf{Interpretability:} Components often correspond to meaningful concepts
    \item \textbf{Efficiency:} Dramatically reduce data storage and transmission costs
    \item \textbf{Generalization:} Capture essential patterns while ignoring noise
    \item \textbf{Foundation:} Serve as building blocks for more complex deep learning systems
\end{itemize}

These applications demonstrate that relatively simple factor models continue to provide practical value, either standalone or as components within larger deep learning systems.

% Index entries
\index{applications!facial recognition}
\index{applications!audio processing}
\index{applications!anomaly detection}
\index{linear factor models!applications}


% Chapter summary and problems
% Key Takeaways for Chapter 13

\section*{Key Takeaways}
\addcontentsline{toc}{section}{Key Takeaways}

\begin{keytakeaways}
\begin{itemize}[leftmargin=2em]
    \item \textbf{Linear-Gaussian latent models} provide probabilistic PCA and factor analysis with uncertainty estimates.
    \item \textbf{EM algorithm} alternates inference of latents and parameter updates, exploiting conjugacy.
    \item \textbf{Identifiability} requires fixing rotations/scales; solutions are not unique without conventions.
    \item \textbf{Bridge to deep models}: Linear factors motivate nonlinear representation learning and VAEs.
\end{itemize}
\end{keytakeaways}



% Exercises (Exercises) for Chapter 13

\section*{Exercises}
\addcontentsline{toc}{section}{Exercises}

\subsection*{Easy}

\begin{exercisebox}[easy]
\begin{problem}[PPCA vs PCA]
Contrast PCA and probabilistic PCA in assumptions and outputs.
\end{problem}
\begin{hintbox}
Deterministic vs. probabilistic view; noise model; likelihood.
\end{hintbox}
\end{exercisebox}


\begin{exercisebox}[easy]
\begin{problem}[Dimensionality Choice]
List two heuristics to select latent dimensionality.
\end{problem}
\begin{hintbox}
Explained variance, information criteria.
\end{hintbox}
\end{exercisebox}


\begin{exercisebox}[easy]
\begin{problem}[Gaussian Conditionals]
Recall the conditional of a joint Gaussian and its role in E-steps.
\end{problem}
\begin{hintbox}
Use block-partitioned mean and covariance formulas.
\end{hintbox}
\end{exercisebox}


\begin{exercisebox}[easy]
\begin{problem}[Rotation Ambiguity]
Explain why factor loadings are identifiable only up to rotation.
\end{problem}
\begin{hintbox}
Orthogonal transforms preserve latent covariance.
\end{hintbox}
\end{exercisebox}


\subsection*{Medium}

\begin{exercisebox}[medium]
\begin{problem}[PPCA Likelihood]
Derive the log-likelihood of PPCA and the M-step for $\sigma^2$.
\end{problem}
\begin{hintbox}
Marginalise latents; differentiate w.r.t. variance.
\end{hintbox}
\end{exercisebox}


\begin{exercisebox}[medium]
\begin{problem}[EM for FA]
Write the E and M steps for Factor Analysis with diagonal noise.
\end{problem}
\begin{hintbox}
Use expected sufficient statistics of latents.
\end{hintbox}
\end{exercisebox}


\subsection*{Hard}

\begin{exercisebox}[hard]
\begin{problem}[Equivalence of PPCA Solution]
Show that the MLE loading matrix spans the top-$k$ eigenvectors of the sample covariance.
\end{problem}
\begin{hintbox}
Use spectral decomposition of covariance.
\end{hintbox}
\end{exercisebox}


\begin{exercisebox}[hard]
\begin{problem}[Nonlinear Extension]
Sketch how to generalise linear factor models to VAEs.
\end{problem}
\begin{hintbox}
Replace linear-Gaussian with neural encoder/decoder.
\end{hintbox}
\end{exercisebox}




\begin{exercisebox}[hard]
\begin{problem}[Advanced Topic 1]
Explain a key concept from this chapter and its practical applications.
\end{problem}
\begin{hintbox}
Consider the theoretical foundations and real-world implications.
\end{hintbox}
\end{exercisebox}


\begin{exercisebox}[hard]
\begin{problem}[Advanced Topic 2]
Analyse the relationship between different techniques covered in this chapter.
\end{problem}
\begin{hintbox}
Look for connections and trade-offs between methods.
\end{hintbox}
\end{exercisebox}


\begin{exercisebox}[hard]
\begin{problem}[Advanced Topic 3]
Design an experiment to test a hypothesis related to this chapter's content.
\end{problem}
\begin{hintbox}
Consider experimental design, metrics, and potential confounding factors.
\end{hintbox}
\end{exercisebox}


\begin{exercisebox}[hard]
\begin{problem}[Advanced Topic 4]
Compare different approaches to solving a problem from this chapter.
\end{problem}
\begin{hintbox}
Consider computational complexity, accuracy, and practical considerations.
\end{hintbox}
\end{exercisebox}


\begin{exercisebox}[hard]
\begin{problem}[Advanced Topic 5]
Derive a mathematical relationship or prove a theorem from this chapter.
\end{problem}
\begin{hintbox}
Start with the definitions and work through the logical steps.
\end{hintbox}
\end{exercisebox}


\begin{exercisebox}[hard]
\begin{problem}[Advanced Topic 6]
Implement a practical solution to a problem discussed in this chapter.
\end{problem}
\begin{hintbox}
Consider the implementation details and potential challenges.
\end{hintbox}
\end{exercisebox}


\begin{exercisebox}[hard]
\begin{problem}[Advanced Topic 7]
Evaluate the limitations and potential improvements of techniques from this chapter.
\end{problem}
\begin{hintbox}
Consider both theoretical limitations and practical constraints.
\end{hintbox}
\end{exercisebox}


