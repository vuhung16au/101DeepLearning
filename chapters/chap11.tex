% Chapter 11: Practical Methodology

\chapter{Practical Methodology}
\label{chap:practical-methodology}

This chapter provides practical guidelines for successfully applying deep learning to real-world problems.


\section*{Learning Objectives}
\addcontentsline{toc}{section}{Learning Objectives}

After studying this chapter, you will be able to:

\begin{enumerate}
    \item \textbf{Scope a deep learning project}: define objectives, constraints, and success metrics.
    \item \textbf{Design data pipelines}: split datasets, manage leakage, and ensure reproducibility.
    \item \textbf{Select architectures and baselines}: choose strong baselines and iterate systematically.
    \item \textbf{Tune hyperparameters}: apply principled search and learning-rate schedules.
    \item \textbf{Diagnose failures}: use loss curves, ablations, and sanity checks to debug.
    \item \textbf{Deploy responsibly}: monitor drift, handle distribution shift, and document models.
\end{enumerate}



\section*{Intuition}
\addcontentsline{toc}{section}{Intuition}

Practical deep learning succeeds when we reduce uncertainty early and iterate quickly. Start with \emph{simple, auditable baselines} to validate data and objectives, then progressively add complexity only when it measurably helps. Prefer experiments that answer the biggest unknowns first (e.g., data quality vs. model capacity). Treat metrics, validation splits, and ablations as your instrumentation layer; they convert intuition into evidence. See also \textcite{GoodfellowEtAl2016} for methodology patterns.


% Chapter 11, Section 1

\section{Performance Metrics \difficultyInline{intermediate}}
\label{sec:performance-metrics}

Performance metrics provide quantitative measures to evaluate model effectiveness, guide model selection, and ensure models meet business requirements across different machine learning tasks.

\subsection{Classification Metrics}

Robust model evaluation depends on selecting metrics aligned with task requirements and operational costs \index{metrics}. Accuracy alone can be misleading under class imbalance \index{class imbalance}; prefer precision/recall, AUC, PR-AUC, calibration, and cost-sensitive metrics when appropriate \textcite{GoodfellowEtAl2016,Prince2023}.

\begin{definition}[Confusion Matrix]
For binary classification with positive/negative classes, define true positives (TP), false positives (FP), true negatives (TN), and false negatives (FN). The confusion matrix \index{confusion matrix} summarizes counts:

\begin{center}
\begin{tabular}{@{}lcc@{}}\toprule
 & \textbf{Predicted +} & \textbf{Predicted --} \\
\midrule
\textbf{Actual +} & TP & FN \\
\textbf{Actual --} & FP & TN \\
\bottomrule
\end{tabular}
\end{center}
\end{definition}

\begin{figure}[h]
  \centering
  \begin{tikzpicture}[x=1.8cm,y=1.8cm]
    % Axes labels
    \node at (1,2.3) {Predicted};
    \node[rotate=90] at (-0.3,1) {Actual};
    % Grid and cells (values: TP=100, FP=15, FN=10, TN=875)
    % Normalize colors roughly by mapping value/ max
    \definecolor{cellA}{RGB}{60,16,83}    % TP - bookpurple
    \definecolor{cellB}{RGB}{242,18,12}   % FP - bookred
    \definecolor{cellC}{RGB}{242,18,12}   % FN - bookred
    \definecolor{cellD}{RGB}{90,25,120}   % TN - lighter purple
    % Draw borders
    \draw[bookblack] (0,0) rectangle (2,2);
    \draw[bookblack] (1,0) -- (1,2);
    \draw[bookblack] (0,1) -- (2,1);
    % Fill cells
    \fill[cellA] (0,1) rectangle (1,2);
    \fill[cellB] (1,1) rectangle (2,2);
    \fill[cellC] (0,0) rectangle (1,1);
    \fill[cellD] (1,0) rectangle (2,1);
    % Ticks and labels
    \node at (0.5,-0.25) {$-$};
    \node at (1.5,-0.25) {$+$};
    \node at (-0.25,0.5) {$-$};
    \node at (-0.25,1.5) {$+$};
    % Counts
    \node[bookwhite] at (0.5,1.5) {100};
    \node[bookwhite] at (1.5,1.5) {15};
    \node[bookwhite] at (0.5,0.5) {10};
    \node[bookwhite] at (1.5,0.5) {875};
  \end{tikzpicture}
  \caption{Confusion matrix heatmap (example counts). High diagonal values indicate good performance.}
  \label{fig:confusion-heatmap}
\end{figure}

\begin{definition}[Accuracy]
\gls{accuracy} \index{accuracy} measures overall correctness but can obscure minority-class performance:
\begin{equation}
\text{Accuracy} = \frac{\text{TP} + \text{TN}}{\text{TP} + \text{FP} + \text{TN} + \text{FN}}.
\end{equation}
\end{definition}

\begin{definition}[Precision and Recall]
\gls{precision} and \gls{recall} quantify quality on the positive class:
\begin{align}
\text{Precision} &= \frac{\text{TP}}{\text{TP} + \text{FP}}, \\
\text{Recall} &= \frac{\text{TP}}{\text{TP} + \text{FN}}.
\end{align}
\end{definition}

\begin{definition}[F1 Score]
The harmonic mean balances precision and recall:
\begin{equation}
F_1 = 2 \cdot \frac{\text{Precision} \cdot \text{Recall}}{\text{Precision} + \text{Recall}}.
\end{equation}
\end{definition}

\begin{definition}[ROC and AUC]
The ROC curve \index{ROC curve} plots TPR vs. FPR as the decision threshold varies; AUC summarizes ranking quality and is threshold-independent.
\begin{align}
\text{TPR} &= \frac{\text{TP}}{\text{TP} + \text{FN}}, & \text{FPR} &= \frac{\text{FP}}{\text{FP} + \text{TN}}.
\end{align}
\end{definition}

\begin{definition}[Precision-Recall (PR) Curve]
Under heavy class imbalance, the PR curve \index{precision--recall curve} and average precision (AP) are often more informative than ROC \textcite{Prince2023}.
\end{definition}

\begin{definition}[Calibration]
A calibrated classifier's predicted probabilities match observed frequencies. Use reliability diagrams and expected calibration error (ECE) \index{calibration}. Calibration matters in risk-sensitive applications \textcite{GoodfellowEtAl2016}.
\end{definition}

% \subsubsection*{Visual aids}

\addcontentsline{toc}{subsubsection}{Visual aids (classification)}

\paragraph{ROC and PR curves.} The following figures illustrate key classification metrics:

\begin{figure}[ht]
  \centering
  \begin{tikzpicture}
    \begin{axis}[
      width=0.45\textwidth,
      height=0.32\textwidth,
      xlabel={False Positive Rate}, 
      ylabel={True Positive Rate},
      xmin=0,xmax=1,ymin=0,ymax=1,
      legend pos=south east, 
      grid=both,
      axis lines=left]
      \addplot[very thick,bookpurple] coordinates {(0,0) (0.1,0.6) (0.2,0.78) (0.4,0.9) (1,1)};\addlegendentry{Model A}
      \addplot[very thick,bookred,dashed] coordinates {(0,0) (0.2,0.55) (0.5,0.8) (0.8,0.9) (1,1)};\addlegendentry{Model B}
      \addplot[bookpurple!40] coordinates {(0,0) (1,1)};\addlegendentry{Random}
    \end{axis}
  \end{tikzpicture}
  \caption{ROC curves for two models. Higher AUC indicates better ranking quality.}
  \label{fig:roc-curves}
\end{figure}

\begin{figure}[ht]
  \centering
  \begin{tikzpicture}
    \begin{axis}[
      width=0.45\textwidth,
      height=0.32\textwidth,
      xlabel={Recall}, 
      ylabel={Precision},
      xmin=0,xmax=1,ymin=0,ymax=1,
      legend pos=south west, 
      grid=both,
      axis lines=left]
      \addplot[very thick,bookpurple] coordinates {(0.0,1.0) (0.2,0.92) (0.4,0.88) (0.6,0.80) (0.8,0.65) (1.0,0.45)};\addlegendentry{Model A}
      \addplot[very thick,bookred,dashed] coordinates {(0.0,1.0) (0.2,0.85) (0.4,0.78) (0.6,0.70) (0.8,0.55) (1.0,0.40)};\addlegendentry{Model B}
    \end{axis}
  \end{tikzpicture}
  \caption{Precision--recall curves emphasize performance on the positive class under imbalance.}
  \label{fig:pr-curves}
\end{figure}

\paragraph{Calibration diagram.} Shows how well predicted probabilities match observed frequencies:

\begin{figure}[ht]
  \centering
  \begin{tikzpicture}
    \begin{axis}[
      width=0.45\textwidth,
      height=0.32\textwidth,
      xlabel={Predicted probability}, 
      ylabel={Observed frequency},
      xmin=0,xmax=1,ymin=0,ymax=1, 
      grid=both,
      axis lines=left]
      \addplot[bookpurple,very thick] coordinates{(0.0,0.0) (0.1,0.05) (0.2,0.10) (0.3,0.18) (0.4,0.28) (0.5,0.40) (0.6,0.55) (0.7,0.68) (0.8,0.80) (0.9,0.90) (1.0,1.0)};
      \addplot[bookpurple!50] coordinates{(0,0) (1,1)};
    \end{axis}
  \end{tikzpicture}
  \caption{Reliability diagram illustrating calibration. The diagonal is perfect calibration.}
  \label{fig:calibration}
\end{figure}

\subsection{Regression Metrics}

Choose metrics that reflect business loss and robustness to outliers \index{regression metrics}. Mean squared error (MSE) penalizes large errors more heavily than mean absolute error (MAE). Root mean squared error (RMSE) is in the original units. Coefficient of determination $R^2$ measures variance explained.

\begin{align}
\text{MSE} &= \frac{1}{n} \sum_{i=1}^{n} (y_i - \hat{y}_i)^2, &
\text{MAE} &= \frac{1}{n} \sum_{i=1}^{n} \lvert y_i - \hat{y}_i \rvert, &
\text{RMSE} &= \sqrt{\text{MSE}}.
\end{align}

\begin{remark}[Regression Metrics for Robustness]
For heavy-tailed noise, consider Huber loss and quantile losses for pinball objectives. Huber loss is a robust loss function that combines the benefits of squared loss (for small errors) and absolute loss (for large errors), making it less sensitive to outliers than mean squared error while still being differentiable everywhere.
\end{remark}

\begin{definition}[Huber Loss]
Huber loss is defined as:
\begin{equation}
L_\delta(y, \hat{y}) = \begin{cases}
\frac{1}{2}(y - \hat{y})^2 & \text{if } |y - \hat{y}| \leq \delta \\
\delta|y - \hat{y}| - \frac{1}{2}\delta^2 & \text{otherwise}
\end{cases}
\end{equation}
where $\delta$ is a threshold parameter that determines the transition point between squared and absolute loss.
\end{definition}

\begin{figure}[h]
  \centering
  \begin{tikzpicture}
    \begin{axis}[
      width=0.48\textwidth,
      height=0.36\textwidth,
      xlabel={$\hat y - y$}, ylabel={Loss}, grid=both, legend pos=north west]
      \addplot[bookpurple,very thick,domain=-3:3,samples=200]{x^2};\addlegendentry{Squared}
      \addplot[bookred,very thick,domain=-3:3,samples=200]{abs(x)};\addlegendentry{Absolute}
      \addplot[black,very thick,domain=-3:3,samples=200]{(abs(x)<=1)*0.5*x^2 + (abs(x)>1)*(abs(x)-0.5)};\addlegendentry{Huber($\delta{=}1$)}
    \end{axis}
  \end{tikzpicture}
  \caption{Comparison of squared, absolute, and Huber losses.}
  \label{fig:huber}
\end{figure}

\subsection{NLP and Sequence Metrics}

Sequence generation quality is commonly measured by \gls{bleu} \index{BLEU} and \gls{rouge} \index{ROUGE} (n-gram overlap), while language models use \emph{perplexity} (negative log-likelihood in exponential form) \textcite{GoodfellowEtAl2016,D2LChapterRNN}:
\begin{equation}
\text{PPL} = \exp\left(-\frac{1}{N} \sum_{i=1}^{N} \log P(x_i)\right).
\end{equation}

\begin{remark}[Mean Average Precision (mAP)]
Mean Average Precision (mAP) measures how well a system ranks relevant items by computing the average precision across different recall levels, making it crucial for information retrieval where ranking quality matters more than binary classification.
\end{remark}

\begin{remark}[Normalized Discounted Cumulative Gain (nDCG)]
Normalized Discounted Cumulative Gain (nDCG) evaluates ranking quality by considering both relevance and position, giving higher weight to items ranked earlier in the list, which reflects real-world user behavior where top results are most important.
\end{remark}

\begin{remark}[Recall@k]
Recall@k measures the proportion of relevant items found in the top-k results, providing a practical metric for applications where users only examine the first few results.
\end{remark}

For retrieval and ranking, report mean average precision (mAP), normalized discounted cumulative gain (nDCG), and recall@k. These metrics are essential because they capture the nuanced performance of ranking systems where the order of results significantly impacts user experience, unlike traditional classification metrics that treat all predictions equally regardless of their position in a ranked list.

\index{mAP}\index{nDCG}\index{recall@k}

\subsection{Worked examples}

\begin{example}[Imbalanced Disease Detection]
In a 1\% prevalence setting, a classifier with 99\% accuracy can be worthless. Reporting PR-AUC and calibration surfaces early detection quality and absolute risk estimates valued by clinicians \textcite{Ronneberger2015}.
\end{example}

\begin{example}[Threshold Selection]
Optimize thresholds against a cost matrix or utility function (e.g., false negative cost \(\gg\) false positive). Plot utility vs. threshold to choose operating points.
\end{example}

\begin{example}[Macro vs. Micro Averaging]
For multi-class, macro-averaged F1 treats classes equally; micro-averaged F1 weights by support. Choose based on fairness vs. prevalence alignment \textcite{Prince2023}.
\end{example}


% Chapter 11, Section 2

\section{Baseline Models and Debugging \difficultyInline{intermediate}}
\label{sec:baselines-debugging}

Before diving into complex architectures, establish simple reference models to validate data quality, metrics, and the fundamental feasibility of your approach.

\subsection{Establishing Baselines}

A \textbf{baseline model} is a simple, well-understood reference system that provides a performance floor for comparison against more complex approaches. Strong baselines de-risk projects by validating data quality, metrics, and feasibility \textcite{GoodfellowEtAl2016,Prince2023}. They serve as sanity checks to ensure that sophisticated models actually improve upon simple solutions rather than introducing unnecessary complexity. Baselines help identify whether poor performance stems from model limitations or fundamental issues with data quality, preprocessing, or evaluation methodology. By establishing multiple baselines across different complexity levels, practitioners can quantify the marginal value of each architectural choice and avoid over-engineering solutions. These reference points become immutable benchmarks that prevent performance regression and provide confidence that improvements are genuine rather than artifacts of experimental variance. \index{baseline}

Starting with the simplest possible approach, a random baseline generates predictions through pure chance, establishing the absolute minimum performance threshold that any reasonable model must exceed. Moving up in sophistication, simple heuristics employ rule-based systems that encode domain knowledge without learning from data, such as predicting the most frequent class in classification tasks or using moving averages for time series. These rule-based approaches often perform surprisingly well and provide valuable insights into the problem structure. Classical machine learning methods like logistic regression and random forests represent the next level of complexity, offering interpretable models with proven track records across diverse domains. Finally, simple neural networks with just a few layers serve as a bridge between classical approaches and deep architectures, helping isolate whether performance gains truly require depth or stem from other factors like better optimization or data augmentation.

When comparing deep learning improvements against these baselines, it's crucial to maintain experimental rigour through careful attention to \emph{data leakage} \index{data leakage} prevention. This means using identical preprocessing pipelines across all baselines and ensuring that validation splits respect the problem's temporal or hierarchical structure—for example, using time-based splits for forecasting tasks or patient-level splits in medical datasets to prevent information leakage across training and validation sets.

\subsection{Debugging Strategy}

Deep learning models often fail silently or produce unexpected results due to the complexity of neural architectures and the non-convex optimization landscape. Systematic debugging is essential because model failures can stem from multiple sources: implementation bugs, data quality issues, hyperparameter choices, or fundamental limitations of the approach. Without proper debugging methodology, practitioners may waste significant time pursuing ineffective solutions or miss critical insights about their data and model behavior.

\index{silent failures}

\textbf{Step 1: Overfit a small dataset}
\begin{itemize}
    \item Take 10-100 examples
    \item Turn off regularization
    \item If can't overfit, suspect implementation, data, or optimization bugs
\end{itemize}

\textbf{Step 2: Check intermediate outputs}
\begin{itemize}
    \item Visualize activations
    \item Check gradient magnitudes
    \item Verify loss decreases on training set
    \item Plot learning-rate vs. loss; test different seeds
\end{itemize}

\textbf{Step 3: Diagnose underfitting vs. overfitting}
\begin{itemize}
    \item \textbf{Underfitting:} Poor train performance $\to$ increase capacity
    \item \textbf{Overfitting:} Good train, poor validation $\to$ add regularization
\end{itemize}

\subsection{Common Issues}

\textbf{Vanishing/exploding gradients} represent one of the most fundamental challenges in training deep networks, occurring when gradients either decay to near-zero or grow exponentially as they propagate backwards through layers. Batch normalization addresses this by stabilizing training through normalizing inputs to each layer, preventing gradients from becoming too small or large during backpropagation—for example, in a 50-layer network without batch norm, gradients might vanish to near-zero values by layer 20, but with batch norm they remain stable throughout the entire network. Gradient clipping provides another line of defence by capping gradient magnitudes at a threshold (e.g., 1.0 or 5.0), which is particularly important in RNNs where gradients can grow exponentially over long sequences, causing training instability and preventing convergence. Proper weight initialization using Xavier or He initialization ensures gradients start at reasonable magnitudes rather than vanishing or exploding from the first forward pass—for instance, He initialization for ReLU networks sets weights to $\sqrt{2/n}$ where $n$ is the input size, preventing the "dead neuron" problem where all activations become zero. Finally, residual connections provide direct paths for gradient flow, allowing information to bypass layers where gradients might vanish; in ResNet architectures, the skip connection $y = F(x) + x$ ensures that even if $F(x)$ becomes zero, the gradient can still flow through the identity connection.

\index{batch normalization}\index{gradient clipping}

\textbf{Dead ReLUs} occur when neurons never activate because their weights become too negative, often due to aggressive learning rates. Reducing the learning rate from 0.01 to 0.001 can prevent neurons from being "killed" during early training, allowing them to recover and contribute to learning. Alternatively, using Leaky ReLU or ELU instead of standard ReLU addresses the root cause directly: unlike standard ReLU which outputs zero for negative inputs, Leaky ReLU allows small negative values (e.g., 0.01x) and ELU provides smooth negative outputs, preventing the "dying ReLU" problem where neurons become permanently inactive—a phenomenon that can affect 30-50\% of neurons in poorly configured networks.

\textbf{Loss not decreasing} signals fundamental problems with training dynamics that require systematic investigation. First, check the learning rate, as rates that are too high cause the optimizer to overshoot the minimum and oscillate around it (e.g., a learning rate of 0.1 might cause loss to bounce between 0.5 and 0.7), while rates that are too low make training painfully slow (e.g., 0.0001 might show no improvement for 100 epochs). Second, verify gradient computation, as bugs in backpropagation implementations, wrong loss function derivatives, or gradient accumulation errors can cause the optimizer to move in wrong directions or not move at all. Third, check data preprocessing, as incorrect preprocessing can make learning impossible—for example, normalizing images to [0,1] when the model expects [-1,1], or accidentally including future information in time series data can prevent the model from learning meaningful patterns. Finally, confirm label alignment and class indexing, as misaligned labels or incorrect indexing (such as using 1-based indexing when the model expects 0-based) can cause the model to learn completely wrong mappings, resulting in consistently incorrect predictions.

\subsection{Ablation and sanity checks}

Perform \emph{ablation studies} \index{ablation study} to quantify the contribution of each component (augmentation, architecture blocks, regularizers). Use \emph{label shuffling} to verify the pipeline cannot learn when labels are randomized. Train with \emph{frozen features} to isolate head capacity.

% \subsection{Visual aids}
% \addcontentsline{toc}{subsubsection}{Visual aids (debugging)}

\begin{figure}[h]
  \centering
  \begin{tikzpicture}
    \begin{axis}[
      width=0.48\textwidth,
      height=0.36\textwidth,
      xlabel={Epoch}, ylabel={Loss}, grid=both, legend pos=north east]
      \addplot[bookpurple,very thick] coordinates{(0,1.0) (1,0.8) (2,0.65) (3,0.55) (4,0.50) (5,0.48)};\addlegendentry{Train}
      \addplot[bookred,very thick,dashed] coordinates{(0,1.1) (1,0.95) (2,0.90) (3,0.92) (4,1.00) (5,1.10)};\addlegendentry{Val}
    \end{axis}
  \end{tikzpicture}
  \caption{Typical overfitting: training loss decreases while validation loss bottoms out and rises.}
  \label{fig:overfit-curve}
\end{figure}

\begin{figure}[h]
  \centering
  \begin{tikzpicture}
    \begin{axis}[
      width=0.48\textwidth,
      height=0.36\textwidth,
      ymode=log,
      xlabel={Layer depth}, ylabel={$\lVert g \rVert_2$}, grid=both]
      \addplot[bookpurple,very thick] coordinates{(1,1e-1) (2,8e-2) (3,5e-2) (4,2e-2) (5,8e-3) (6,3e-3)};
    \end{axis}
  \end{tikzpicture}
  \caption{Gradient norms vanishing with depth; motivates normalization and residual connections.}
  \label{fig:vanishing-grad}
\end{figure}

\begin{figure}[h]
  \centering
  \begin{tikzpicture}
    \begin{axis}[
      width=0.48\textwidth,
      height=0.36\textwidth,
      xlabel={Learning rate}, ylabel={Final loss}, grid=both]
      \addplot[bookpurple,very thick] coordinates{(1e-5,1.2) (5e-5,0.9) (1e-4,0.7) (5e-4,0.55) (1e-3,0.54) (5e-3,1.3) (1e-2,3.0)};
    \end{axis}
  \end{tikzpicture}
  \caption{Learning-rate sweep to identify a stable training regime.}
  \label{fig:lr-sweep}
\end{figure}

\subsection{Historical notes and references}

Debugging by overfitting a tiny subset and systematic ablations has roots in classical ML practice and was emphasized in early deep learning methodology \textcite{GoodfellowEtAl2016}. Modern best practices are also surveyed in open textbooks \textcite{Prince2023,D2LChapterOptimization}.


% Chapter 11, Section 3

\section{Hyperparameter Tuning \difficultyInline{intermediate}}
\label{sec:hyperparameter-tuning}

\subsection{Key Hyperparameters (Priority Order)}

Effective tuning prioritizes learning rate, regularization, and capacity before fine details \index{hyperparameter tuning}. Treat the validation set as your instrumentation layer and control randomness via fixed seeds \textcite{GoodfellowEtAl2016,Prince2023,D2LChapterOptimization}.

\begin{enumerate}
    \item \textbf{Learning rate:} Most critical; consider warmup and cosine decay
    \item \textbf{Network architecture:} Depth/width, normalization, residuals
    \item \textbf{Batch size:} Affects noise scale and generalization
    \item \textbf{Regularization:} Weight decay, dropout, label smoothing
    \item \textbf{Optimizer parameters:} Momentum, $\beta$ values in Adam
\end{enumerate}

\subsection{Search Strategies}

\textbf{Manual Search:}
\begin{itemize}
    \item Start with educated guesses
    \item Adjust based on results
    \item Time-consuming but insightful
\end{itemize}

\textbf{Grid Search:}
\begin{itemize}
    \item Try all combinations from predefined values
    \item Exhaustive but expensive
    \item Better for 2-3 hyperparameters
\end{itemize}

\textbf{Random Search:}
\begin{itemize}
    \item Sample hyperparameters randomly
    \item More efficient than grid search
    \item Better for high-dimensional spaces
\end{itemize}

\textbf{Bayesian Optimization:}
\begin{itemize}
    \item Model hyperparameter performance
    \item Choose next trials intelligently
    \item More sample-efficient
\end{itemize}

\subsection{Best Practices}

\begin{itemize}
    \item Use logarithmic scale for learning rate; sweep $[10^{-5},10^{-1}]$
    \item Vary batch size and adjust learning rate proportionally
    \item Track results with a consistent random seed and multiple repeats
    \item Early-stop poor runs; allocate budget adaptively
    \item Use a fixed validation protocol to avoid leakage
    \item Retrain with best setting on train+val and report on held-out test
\end{itemize}

% \subsection{Visual aids}
% \addcontentsline{toc}{subsubsection}{Visual aids (tuning)}

% \begin{figure}[h]
%   \centering
%   \begin{tikzpicture}
%     \begin{axis}[
%       width=0.48\textwidth,
%       height=0.36\textwidth,
%       xmode=log, log basis x=10,
%       xlabel={Learning rate}, ylabel={Val. loss}, grid=both]
%       \addplot[bookpurple,very thick] coordinates{(1e-5,0.90) (3e-5,0.80) (1e-4,0.70) (3e-4,0.62) (1e-3,0.60) (3e-3,0.95) (1e-2,2.0)};
%     \end{axis}
%   \end{tikzpicture}
%   \caption{Learning-rate sweep identifies a stable operating region.}
%   \label{fig:lr-range-test}
% \end{figure}

\subsection{Historical notes}

Random search and Bayesian optimization rose to prominence as models and search spaces grew large; they offer better coverage than grid for high-dimensional spaces. Practical DL texts emphasize learning-rate schedules (step, cosine) and warmup for stability \textcite{GoodfellowEtAl2016,Prince2023,D2LChapterOptimization}.


% Chapter 11, Section 4

\section{Data Preparation and Preprocessing \difficultyInline{intermediate}}
\label{sec:data-preparation}

\subsection{Data Splitting}

\textbf{Train/Validation/Test split:} This three-way split ensures unbiased model evaluation by keeping the test set completely isolated until final assessment, while the validation set guides hyperparameter tuning without contaminating the final performance estimate. The validation set acts as a proxy for the test set during development, allowing you to make informed decisions about model architecture and hyperparameters without peeking at the true test performance.

\begin{figure}[h]
  \centering
  \begin{tikzpicture}
    \begin{axis}[
      width=0.6\textwidth,
      height=0.25\textwidth,
      ybar, bar width=20pt,
      xlabel={Dataset Split}, ylabel={Percentage},
      xtick=data,
      xticklabels={Training, Validation, Test},
      ymin=0, ymax=100,
      grid=both]
      \addplot[bookpurple,fill=bookpurple!60] coordinates{(1,70)};
      \addplot[bookred,fill=bookred!60] coordinates{(2,15)};
      \addplot[blue,fill=blue!60] coordinates{(3,15)};
    \end{axis}
  \end{tikzpicture}
  \caption{Train/Validation/Test split with typical proportions: 70\% training, 15\% validation, 15\% test.}
  \label{fig:data-split}
\end{figure}

\textbf{Cross-validation:} For small datasets where a single train/validation split might not provide reliable estimates, k-fold cross-validation uses all available data for both training and validation by rotating which subset serves as the validation set. This approach maximizes the use of limited data while providing more robust performance estimates, especially crucial when you have fewer than 1000 examples and need to make the most of every data point.

\begin{figure}[h]
  \centering
  \begin{tikzpicture}
    \begin{axis}[
      width=0.8\textwidth,
      height=0.4\textwidth,
      xlabel={Data Points (100 total)}, ylabel={Fold},
      ymin=0.5, ymax=10.5,
      xmin=0, xmax=100,
      grid=both,
      ytick={1,2,3,4,5,6,7,8,9,10},
      yticklabels={Fold 1, Fold 2, Fold 3, Fold 4, Fold 5, Fold 6, Fold 7, Fold 8, Fold 9, Fold 10},
      xtick={0,10,20,30,40,50,60,70,80,90,100}]
      
      % Fold 1: Validation = 0-9
      \addplot[bookred,very thick] coordinates{(0,1) (9,1)};
      \addplot[bookpurple,very thick] coordinates{(10,1) (99,1)};
      
      % Fold 2: Validation = 10-19
      \addplot[bookred,very thick] coordinates{(10,2) (19,2)};
      \addplot[bookpurple,very thick] coordinates{(0,2) (9,2)};
      \addplot[bookpurple,very thick] coordinates{(20,2) (99,2)};
      
      % Fold 3: Validation = 20-29
      \addplot[bookred,very thick] coordinates{(20,3) (29,3)};
      \addplot[bookpurple,very thick] coordinates{(0,3) (19,3)};
      \addplot[bookpurple,very thick] coordinates{(30,3) (99,3)};
      
      % Fold 4: Validation = 30-39
      \addplot[bookred,very thick] coordinates{(30,4) (39,4)};
      \addplot[bookpurple,very thick] coordinates{(0,4) (29,4)};
      \addplot[bookpurple,very thick] coordinates{(40,4) (99,4)};
      
      % Fold 5: Validation = 40-49
      \addplot[bookred,very thick] coordinates{(40,5) (49,5)};
      \addplot[bookpurple,very thick] coordinates{(0,5) (39,5)};
      \addplot[bookpurple,very thick] coordinates{(50,5) (99,5)};
      
      % Fold 6: Validation = 50-59
      \addplot[bookred,very thick] coordinates{(50,6) (59,6)};
      \addplot[bookpurple,very thick] coordinates{(0,6) (49,6)};
      \addplot[bookpurple,very thick] coordinates{(60,6) (99,6)};
      
      % Fold 7: Validation = 60-69
      \addplot[bookred,very thick] coordinates{(60,7) (69,7)};
      \addplot[bookpurple,very thick] coordinates{(0,7) (59,7)};
      \addplot[bookpurple,very thick] coordinates{(70,7) (99,7)};
      
      % Fold 8: Validation = 70-79
      \addplot[bookred,very thick] coordinates{(70,8) (79,8)};
      \addplot[bookpurple,very thick] coordinates{(0,8) (69,8)};
      \addplot[bookpurple,very thick] coordinates{(80,8) (99,8)};
      
      % Fold 9: Validation = 80-89
      \addplot[bookred,very thick] coordinates{(80,9) (89,9)};
      \addplot[bookpurple,very thick] coordinates{(0,9) (79,9)};
      \addplot[bookpurple,very thick] coordinates{(90,9) (99,9)};
      
      % Fold 10: Validation = 90-99
      \addplot[bookred,very thick] coordinates{(90,10) (99,10)};
      \addplot[bookpurple,very thick] coordinates{(0,10) (89,10)};
      
    \end{axis}
  \end{tikzpicture}
  \caption{10-fold cross-validation with 100 data points. Red bars show validation sets (10 points each), purple bars show training sets (90 points each).}
  \label{fig:kfold-cv}
\end{figure}

\begin{itemize}
    \item \textbf{k-fold cross-validation:} Divides data into k equal folds, using each fold as validation set once
    \item \textbf{Stratified splits for imbalanced data:} Ensures each fold maintains the same class distribution as the original dataset
\end{itemize}

\index{cross-validation}

\subsection{Normalization}

Normalization is essential because neural networks are sensitive to the scale of input features, and features with vastly different ranges can cause training instability and poor convergence. When one feature ranges from 0 to 1 while another spans 0 to 1000, the larger-scale feature dominates the learning process, causing the network to ignore the smaller-scale feature entirely. This scale imbalance leads to slow convergence, as the optimizer struggles to find appropriate learning rates that work for both features simultaneously.

\textbf{Min-Max Scaling:} This method rescales features to a fixed range, typically [0,1], by subtracting the minimum value and dividing by the range. Min-max scaling preserves the original distribution shape and is particularly useful when you know the expected range of your data or when you need features to have the same scale for distance-based algorithms.
\begin{equation}
x' = \frac{x - x_{\min}}{x_{\max} - x_{\min}}
\end{equation}

\textbf{Standardization (Z-score):} This approach transforms features to have zero mean and unit variance, making them follow a standard normal distribution. Standardization is more robust to outliers than min-max scaling and is the preferred method for most deep learning applications, as it centers the data around zero and gives equal importance to all features regardless of their original scale.
\begin{equation}
x' = \frac{x - \mu}{\sigma}
\end{equation}

Always compute statistics on training set only! \index{data leakage} Using validation or test set statistics would create data leakage, as the model would have access to information from future data during training, leading to overly optimistic performance estimates that don't generalize to truly unseen data.

\index{min-max scaling}\index{normalization}\index{standardization}

\subsection{Handling Imbalanced Data}

\textbf{Imbalanced data} occurs when one or more classes have significantly fewer examples than others, creating a skewed class distribution that can severely bias model training toward the majority class. This imbalance is problematic because standard machine learning algorithms assume balanced class distributions and will naturally favor the majority class, leading to poor performance on minority classes that are often the most important to identify correctly.

\begin{itemize}
    \item \textbf{Oversampling:} Duplicate minority class examples to balance the dataset artificially. This approach increases the representation of minority classes by creating exact copies of existing examples, which helps the model see more minority class instances during training. However, simple duplication can lead to overfitting since the model sees identical examples multiple times, potentially memorizing specific patterns rather than learning generalizable features.
    
    \item \textbf{Undersampling:} Remove majority class examples to create a more balanced dataset by randomly discarding instances from the overrepresented class. This method reduces computational cost and training time while forcing the model to pay more attention to minority classes. The main drawback is the loss of potentially valuable information from the majority class, which can hurt overall model performance if the discarded examples contain important patterns.
    
    \item \textbf{SMOTE:} Synthetic minority oversampling creates new synthetic examples for minority classes by interpolating between existing minority class instances in feature space. SMOTE generates realistic synthetic data points by finding k-nearest neighbors of minority examples and creating new instances along the line segments connecting them. This approach provides more diverse training examples than simple duplication while maintaining the statistical properties of the original minority class distribution.
    
    \item \textbf{Class weights:} Penalize errors on minority class more heavily during training by assigning higher loss weights to minority class misclassifications. This technique adjusts the loss function to make the model more sensitive to minority class errors, effectively forcing it to prioritize learning the underrepresented classes. The weights are typically set inversely proportional to class frequency, so a class with 10\% representation gets 10x higher weight than a class with 100\% representation.
    
    \item \textbf{Focal loss:} Focus on hard examples by down-weighting easy examples and up-weighting difficult-to-classify instances, particularly useful for extreme class imbalance. This loss function automatically adapts to the difficulty of each example, reducing the contribution of well-classified majority class examples while emphasizing misclassified minority class instances. Focal loss is especially effective for object detection and segmentation tasks where background pixels vastly outnumber foreground objects.
\end{itemize}

\index{oversampling}\index{undersampling}\index{SMOTE}\index{class weights}\index{focal loss}

\subsection{Data Augmentation}

\textbf{Data augmentation} is a crucial strategy in deep learning to artificially increase the size and diversity of a training dataset, which is vital for achieving generalization and mitigating overfitting, especially when working with limited real-world samples. By generating additional examples through domain-specific transformations, such as flips, crops, or color jitter for images, the model learns to recognize the core object or pattern regardless of minor variations. In the NLP space, techniques like back-translation (translating text to another language and back) introduce crucial syntactic and vocabulary variance that stabilizes large language models. The primary challenge lies in calibrating the strength of these augmentations, as excessive or unrealistic noise, such as extreme time stretching for audio or radical color shifts for images, can distort the underlying signal and cause a debilitating distribution shift that undermines model performance. Ultimately, intelligent data augmentation expands the effective manifold of the training data without the cost of collecting new samples.

For images: flips, crops, color jitter; for text: back-translation; for audio: time stretch, noise. Calibrate augmentation strength to avoid distribution shift \textcite{Prince2023}.

\subsection{Visual aids}
\addcontentsline{toc}{subsubsection}{Visual aids (data)}

\begin{figure}[ht]
  \centering
  \begin{tikzpicture}
    \begin{axis}[
      width=0.45\textwidth,
      height=0.32\textwidth,
      ybar, bar width=15pt, grid=both,
      xlabel={Class}, ylabel={Count}, xtick=data,
      xticklabels={A,B,C,D,E},
      ymin=0, ymax=1200,
      axis lines=left]
      \addplot[bookpurple,fill=bookpurple!60] coordinates{(1,1000) (2,800) (3,120) (4,80) (5,40)};
    \end{axis}
  \end{tikzpicture}
  \caption{Imbalanced dataset example motivating class weights or resampling.}
  \label{fig:imbalance-bar}
\end{figure}

\begin{figure}[ht]
  \centering
  \begin{tikzpicture}
    \begin{axis}[
      width=0.45\textwidth,
      height=0.32\textwidth,
      xlabel={Original value $x$}, ylabel={Normalized value $x'$}, 
      grid=both,
      axis lines=left,
      xmin=0, xmax=1, ymin=0, ymax=1]
      \addplot[bookpurple,very thick,domain=0:1]{x};
      \addplot[bookred,very thick,domain=-3:3]({(x+3)/6},{(x-0)/2});
      \node at (0.5,0.8) {Min-Max};
      \node at (0.5,0.2) {Standardization};
    \end{axis}
  \end{tikzpicture}
  \caption{Min-max scaling (purple) vs. standardization (red) schematic.}
  \label{fig:scaling}
\end{figure}

\subsection{Historical notes}

Careful dataset design (train/val/test segregation, leakage prevention) has long underpinned reliable evaluation in ML and remains essential at scale in deep learning \textcite{Bishop2006,GoodfellowEtAl2016}. The evolution from classical ML to deep learning fundamentally transformed data preprocessing requirements, as traditional methods like linear regression and decision trees were relatively robust to feature scaling, while neural networks require careful normalization to prevent gradient instability. The introduction of batch normalization by Ioffe and Szegedy in 2015 marked a pivotal moment, as it automated the normalization process during training, eliminating the need for manual feature scaling in many cases. Unlike classical methods where data splitting was primarily about preventing overfitting, deep learning's data augmentation techniques (pioneered in computer vision by Krizhevsky et al. in 2012) became essential for generalization, as neural networks' high capacity made them prone to memorizing training data. The rise of transfer learning and pre-trained models further complicated data preparation, as practitioners now needed to understand how to adapt datasets for models trained on different distributions, a challenge that classical ML rarely faced. Modern frameworks like TensorFlow and PyTorch have democratized these sophisticated preprocessing techniques, making advanced data preparation accessible to practitioners who previously relied on simpler methods like one-hot encoding for categorical variables or basic standardization for continuous features.


% Chapter 11, Section 5

\section{Production Considerations \difficultyInline{intermediate}}
\label{sec:production}

\subsection{Model Deployment}

\begin{itemize}
    \item Model compression (pruning, quantization) to meet latency/size budgets
    \item Model serving infrastructure (A/B testing, canary deploys)
    \item Latency budgets and tail performance (p95/p99)
    \item Batch vs. online inference and feature freshness
\end{itemize}

\subsection{Monitoring}

Track in production:
\begin{itemize}
    \item Prediction distribution shifts (covariate/label shift) \index{distribution shift}
    \item Model performance metrics and calibration drift
    \item System latency and throughput; autoscaling behavior
    \item Error analysis with human-in-the-loop review
\end{itemize}

\subsection{Iterative Improvement}

\begin{enumerate}
    \item Deploy initial model
    \item Monitor performance
    \item Collect more data
    \item Retrain and improve
    \item A/B test new models
\end{enumerate}

\subsection{Visual aids}
\addcontentsline{toc}{subsubsection}{Visual aids (production)}

\begin{figure}[h]
  \centering
  \begin{tikzpicture}
    \begin{axis}[
      width=0.48\textwidth,height=0.36\textwidth,
      xlabel={Day}, ylabel={KS statistic}, grid=both]
      \addplot[bookpurple,very thick] coordinates{(1,0.05) (2,0.06) (3,0.07) (4,0.09) (5,0.14) (6,0.22) (7,0.28)};
      \addplot[bookred,dashed,very thick] coordinates{(1,0.2) (7,0.2)};
    \end{axis}
  \end{tikzpicture}
  \caption{Simple drift monitor: KS statistic over time with an alert threshold.}
  \label{fig:drift-monitor}
\end{figure}

\subsection{Applications and context}

Compression and low-latency serving are crucial in mobile vision, speech on device, and recommender systems; calibration and post-deployment monitoring are critical in healthcare and finance \textcite{Ronneberger2015,Prince2023}.


% Chapter 11: Real World Applications

\section{Real World Applications}
\label{sec:methodology-real-world}


Practical methodology—the systematic approach to designing, training, and deploying deep learning systems—is what separates successful real-world projects from academic experiments.

\subsection{Healthcare Diagnostic System Deployment}

Bringing AI from lab to clinic:

\begin{itemize}
    \item \textbf{FDA-approved medical imaging systems:} Companies developing AI diagnostic tools must follow rigorous methodologies: careful dataset collection from diverse hospitals, systematic validation on held-out test sets, extensive clinical trials, and continuous monitoring post-deployment. A stroke detection system, for example, must work reliably across different scanners, patient populations, and hospital settings before doctors trust it with patient care.
    
    \item \textbf{Handling data quality issues:} Real medical data is messy—images have artifacts, labels contain errors, and rare diseases are underrepresented. Practical methodology includes data cleaning procedures, handling class imbalance, and establishing confidence thresholds for when the system should defer to human experts.
    
    \item \textbf{Continuous learning and monitoring:} Once deployed, medical AI systems need ongoing validation. Methodology includes establishing monitoring dashboards, detecting distribution shift (when patient populations change), and protocols for updating models without disrupting clinical workflows.
\end{itemize}

\subsection{Recommendation System Development}

Building and maintaining large-scale personalization:

\begin{itemize}
    \item \textbf{A/B testing and evaluation:} When Netflix develops new recommendation algorithms, they don't just optimize offline metrics. Practical methodology involves carefully designed A/B tests with real users, balancing multiple objectives (user engagement, diversity, content discovery, business goals), and understanding long-term effects beyond immediate clicks.
    
    \item \textbf{Cold start problem:} New users have no history, and new items have no ratings. Practical methodology addresses this through strategic initialization, hybrid approaches combining content features with collaborative filtering, and active learning to quickly gather useful information.
    
    \item \textbf{Production infrastructure:} Serving recommendations to millions of users simultaneously requires careful system design. Methodology includes choosing appropriate model architectures that balance accuracy with inference speed, caching strategies, and gradual rollouts to detect problems early.
\end{itemize}

\subsection{Autonomous Vehicle Development}

The most safety-critical deep learning application:

\begin{itemize}
    \item \textbf{Simulation and testing methodology:} Self-driving cars must handle rare but critical scenarios (a child running into the street). Companies use systematic methodologies combining real-world data collection, photorealistic simulation of dangerous scenarios, and extensive closed-track testing before public road trials.
    
    \item \textbf{Failure analysis and iteration:} When test vehicles make mistakes, teams follow rigorous procedures to understand root causes, reproduce issues in simulation, develop fixes, and validate improvements. This includes systematic logging of all sensor data and decisions for later analysis.
    
    \item \textbf{Multi-stage validation:} Models progress through increasingly realistic testing: simulation, closed tracks, controlled public roads, then broader deployment. Each stage has specific success criteria and methodologies for objective evaluation.
\end{itemize}

\subsection{Key Methodological Principles}

What makes real-world projects succeed:
\begin{itemize}
    \item \textbf{Start simple:} Baseline models first, then increase complexity as needed
    \item \textbf{Measure what matters:} Align metrics with actual business or user goals
    \item \textbf{Understand your data:} Invest time in data exploration and cleaning
    \item \textbf{Iterate systematically:} Change one thing at a time to understand impact
    \item \textbf{Plan for production:} Consider deployment constraints from the beginning
    \item \textbf{Monitor continuously:} Real-world conditions change; models must adapt
\end{itemize}

These examples show that methodology—the "how" of deep learning—is just as important as the "what" when building systems that work reliably in practice.

% Index entries
\index{applications!healthcare systems}
\index{applications!recommendation systems}
\index{applications!autonomous vehicles}
\index{practical methodology!applications}


% Chapter summary and problems
% Key Takeaways for Chapter 11

\section*{Key Takeaways}
\addcontentsline{toc}{section}{Key Takeaways}

\begin{keytakeaways}
\begin{itemize}[leftmargin=2em]
    \item \textbf{Start simple}: Establish a reliable baseline to validate data, metrics, and training code.
    \item \textbf{Measure relentlessly}: Use clear validation splits, confidence intervals, and learning curves.
    \item \textbf{Ablate to learn}: Prefer small, controlled changes to isolate causal effects.
    \item \textbf{Prioritise data}: Label quality, coverage, and augmentation often beat model complexity.
    \item \textbf{Tune methodically}: Track hyperparameters, seeds, and environments for reproducibility.
    \item \textbf{Deploy with monitoring}: Watch for drift, performance decay, and fairness regressions; plan periodic re-training.
\end{itemize}
\end{keytakeaways}



% Exercises (Exercises) for Chapter 11

\section*{Exercises}
\addcontentsline{toc}{section}{Exercises}

\subsection*{Easy}

\begin{exercisebox}[easy]
\begin{problem}[Define Success Metrics]
You are building a classifier for defect detection. Propose suitable metrics beyond accuracy and justify validation splits.
\end{problem}
\begin{hintbox}
Consider precision/recall, AUROC vs. AUPRC under class imbalance, and stratified splits.
\end{hintbox}
\end{exercisebox}


\begin{exercisebox}[easy]
\begin{problem}[Sanity Checks]
List three quick sanity checks to run before large-scale training and explain expected outcomes.
\end{problem}
\begin{hintbox}
Overfit a tiny subset; randomise labels; train with shuffled pixels.
\end{hintbox}
\end{exercisebox}


\begin{exercisebox}[easy]
\begin{problem}[Baseline First]
Explain why a strong non-deep baseline can accelerate iteration on a deep model.
\end{problem}
\begin{hintbox}
Separates data/metric issues from model capacity; provides performance floor.
\end{hintbox}
\end{exercisebox}


\begin{exercisebox}[easy]
\begin{problem}[Data Leakage]
Define data leakage and give two concrete examples.
\end{problem}
\begin{hintbox}
Temporal leakage; using normalised stats computed on the full dataset.
\end{hintbox}
\end{exercisebox}


\subsection*{Medium}

\begin{exercisebox}[medium]
\begin{problem}[Hyperparameter Search Budget]
Given budget for 30 runs, propose an allocation between exploration (random search) and exploitation (local search). Defend your choice.
\end{problem}
\begin{hintbox}
Start broad (e.g., 20 random), then refine top configurations (e.g., 10 local).
\end{hintbox}
\end{exercisebox}


\begin{exercisebox}[medium]
\begin{problem}[Early Stopping vs. Schedules]
Compare early stopping with cosine decay schedules under limited training budget.
\end{problem}
\begin{hintbox}
Consider variance, bias, and checkpoint selection.
\end{hintbox}
\end{exercisebox}


\subsection*{Hard}

\begin{exercisebox}[hard]
\begin{problem}[Confidence Intervals]
Derive a 95\% Wilson interval for a classifier with $n$ samples and accuracy $\hat{p}$.
\end{problem}
\begin{hintbox}
Use $\frac{\hat{p}+z^2/(2n) \pm z\sqrt{\frac{\hat{p}(1-\hat{p})}{n}+\frac{z^2}{4n^2}}}{1+z^2/n}$ with $z\approx1.96$.
\end{hintbox}
\end{exercisebox}


\begin{exercisebox}[hard]
\begin{problem}[Causal Confounding]
Your model uses a spurious feature. Propose an experimental protocol to detect and mitigate it.
\end{problem}
\begin{hintbox}
Counterfactual augmentation, environment splitting, invariant risk minimisation.
\end{hintbox}
\end{exercisebox}




\begin{exercisebox}[hard]
\begin{problem}[Advanced Topic 1]
Explain a key concept from this chapter and its practical applications.
\end{problem}
\begin{hintbox}
Consider the theoretical foundations and real-world implications.
\end{hintbox}
\end{exercisebox}


\begin{exercisebox}[hard]
\begin{problem}[Advanced Topic 2]
Analyse the relationship between different techniques covered in this chapter.
\end{problem}
\begin{hintbox}
Look for connections and trade-offs between methods.
\end{hintbox}
\end{exercisebox}


\begin{exercisebox}[hard]
\begin{problem}[Advanced Topic 3]
Design an experiment to test a hypothesis related to this chapter's content.
\end{problem}
\begin{hintbox}
Consider experimental design, metrics, and potential confounding factors.
\end{hintbox}
\end{exercisebox}


\begin{exercisebox}[hard]
\begin{problem}[Advanced Topic 4]
Compare different approaches to solving a problem from this chapter.
\end{problem}
\begin{hintbox}
Consider computational complexity, accuracy, and practical considerations.
\end{hintbox}
\end{exercisebox}


\begin{exercisebox}[hard]
\begin{problem}[Advanced Topic 5]
Derive a mathematical relationship or prove a theorem from this chapter.
\end{problem}
\begin{hintbox}
Start with the definitions and work through the logical steps.
\end{hintbox}
\end{exercisebox}


\begin{exercisebox}[hard]
\begin{problem}[Advanced Topic 6]
Implement a practical solution to a problem discussed in this chapter.
\end{problem}
\begin{hintbox}
Consider the implementation details and potential challenges.
\end{hintbox}
\end{exercisebox}


\begin{exercisebox}[hard]
\begin{problem}[Advanced Topic 7]
Evaluate the limitations and potential improvements of techniques from this chapter.
\end{problem}
\begin{hintbox}
Consider both theoretical limitations and practical constraints.
\end{hintbox}
\end{exercisebox}


