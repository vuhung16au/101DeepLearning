% Exercises (Hands-On Exercises) for Chapter 3: Probability and Information Theory

\section*{Exercises}
\addcontentsline{toc}{section}{Exercises}

\subsection*{Easy}

\begin{exercisebox}[easy]
\begin{problem}[Bayes' Theorem Application]
Given that P(Disease) = 0.01, P(Positive Test | Disease) = 0.95, and P(Positive Test | No Disease) = 0.05, calculate P(Disease | Positive Test).
\end{problem}
\begin{hintbox}
Use Bayes' theorem: $P(A|B) = \frac{P(B|A)P(A)}{P(B)}$. Remember to compute P(Positive Test) first.
\end{hintbox}
\end{exercisebox}


\begin{exercisebox}[easy]
\begin{problem}[Expectation and Variance]
A discrete random variable $X$ takes values 1, 2, 3, 4 with probabilities 0.1, 0.2, 0.4, 0.3 respectively. Calculate $\mathbb{E}[X]$ and $\text{Var}(X)$.
\end{problem}
\begin{hintbox}
$\mathbb{E}[X] = \sum_i x_i P(X=x_i)$ and $\text{Var}(X) = \mathbb{E}[X^2] - (\mathbb{E}[X])^2$.
\end{hintbox}
\end{exercisebox}


\begin{exercisebox}[easy]
\begin{problem}[Entropy Calculation]
Calculate the entropy of a fair coin flip and compare it to the entropy of a biased coin with P(Heads) = 0.9.
\end{problem}
\begin{hintbox}
Entropy $H(X) = -\sum_i p_i \log_2 p_i$. Higher entropy means more uncertainty.
\end{hintbox}
\end{exercisebox}


\begin{exercisebox}[easy]
\begin{problem}[Independence Test]
Given P(A) = 0.3, P(B) = 0.4, and P(A $\cap$ B) = 0.12, determine if events A and B are independent.
\end{problem}
\begin{hintbox}
Events are independent if P(A $\cap$ B) = P(A)P(B).
\end{hintbox}
\end{exercisebox}


\begin{exercisebox}[easy]
\begin{problem}[Conditional Probability]
In a deck of 52 cards, what is the probability of drawing a heart given that the card drawn is red?
\end{problem}
\begin{hintbox}
Use the definition of conditional probability: P(A|B) = P(A $\cap$ B)/P(B).
\end{hintbox}
\end{exercisebox}


\begin{exercisebox}[easy]
\begin{problem}[Joint Probability]
Given P(A) = 0.6, P(B) = 0.4, and P(A|B) = 0.8, find P(A $\cap$ B) and P(B|A).
\end{problem}
\begin{hintbox}
Use the multiplication rule: P(A $\cap$ B) = P(A|B)P(B).
\end{hintbox}
\end{exercisebox}


\begin{exercisebox}[easy]
\begin{problem}[Probability Distributions]
A random variable X follows a uniform distribution on [0, 2]. Find P(X > 1.5) and the expected value E[X].
\end{problem}
\begin{hintbox}
For uniform distribution on [a, b], the density is f(x) = 1/(b-a) for x in [a, b].
\end{hintbox}
\end{exercisebox}


\begin{exercisebox}[easy]
\begin{problem}[Binomial Distribution]
A fair coin is flipped 10 times. What is the probability of getting exactly 7 heads?
\end{problem}
\begin{hintbox}
Use the binomial probability formula: $P(X = k) = C(n,k) p^k (1-p)^{(n-k)}$.
\end{hintbox}
\end{exercisebox}


\begin{exercisebox}[easy]
\begin{problem}[Normal Distribution]
If X ~ N(50, 25), find P(45 < X < 55) using the standard normal distribution.
\end{problem}
\begin{hintbox}
Standardise using Z = (X - μ)/σ, then use standard normal tables.
\end{hintbox}
\end{exercisebox}


\begin{exercisebox}[easy]
\begin{problem}[Information Content]
Calculate the information content of an event with probability 0.1, and compare it to an event with probability 0.5.
\end{problem}
\begin{hintbox}
Information content is I(x) = -log₂ P(x).
\end{hintbox}
\end{exercisebox}


\begin{exercisebox}[easy]
\begin{problem}[Mutual Information]
Given P(X=0) = 0.6, P(X=1) = 0.4, P(Y=0|X=0) = 0.8, P(Y=1|X=0) = 0.2, P(Y=0|X=1) = 0.3, P(Y=1|X=1) = 0.7, calculate I(X;Y).
\end{problem}
\begin{hintbox}
Use I(X;Y) = H(X) - H(X|Y) = H(Y) - H(Y|X).
\end{hintbox}
\end{exercisebox}


\subsection*{Medium}

\begin{exercisebox}[medium]
\begin{problem}[KL Divergence for Model Comparison]
Explain why Kullback-Leibler (KL) divergence is not symmetric and discuss its implications when comparing probability distributions in machine learning.
\end{problem}
\begin{hintbox}
Consider $D_{KL}(P||Q)$ versus $D_{KL}(Q||P)$ and their behaviour when $P$ or $Q$ is close to zero.
\end{hintbox}
\end{exercisebox}


\begin{exercisebox}[medium]
\begin{problem}[Cross-Entropy Loss]
Show that minimising cross-entropy loss is equivalent to maximising the log-likelihood for classification tasks. Derive the relationship mathematically.
\end{problem}
\begin{hintbox}
Start with the cross-entropy $H(p,q) = -\sum_i p_i \log q_i$ where $p$ is the true distribution and $q$ is the predicted distribution.
\end{hintbox}
\end{exercisebox}


\begin{exercisebox}[medium]
\begin{problem}[Maximum Likelihood Estimation]
Given a sample of n independent observations from a normal distribution N(μ, σ²), derive the maximum likelihood estimators for μ and σ².
\end{problem}
\begin{hintbox}
Write the likelihood function L(μ, σ²) and take partial derivatives with respect to μ and σ².
\end{hintbox}
\end{exercisebox}


\begin{exercisebox}[medium]
\begin{problem}[Jensen's Inequality Application]
Use Jensen's inequality to show that the entropy of a mixture of distributions is at least the weighted average of the individual entropies.
\end{problem}
\begin{hintbox}
Consider H(∑ᵢ αᵢpᵢ) ≥ ∑ᵢ αᵢH(pᵢ) where ∑ᵢ αᵢ = 1 and αᵢ ≥ 0.
\end{hintbox}
\end{exercisebox}


\begin{exercisebox}[medium]
\begin{problem}[Central Limit Theorem]
Explain how the Central Limit Theorem applies to the convergence of sample means and discuss its implications for machine learning.
\end{problem}
\begin{hintbox}
Consider the distribution of sample means and how it approaches normality regardless of the original distribution.
\end{hintbox}
\end{exercisebox}


\begin{exercisebox}[medium]
\begin{problem}[Concentration Inequalities]
Use Markov's inequality to bound P(X ≥ 2E[X]) for a non-negative random variable X, and compare with Chebyshev's inequality.
\end{problem}
\begin{hintbox}
Markov's inequality: P(X ≥ a) ≤ E[X]/a for a > 0.
\end{hintbox}
\end{exercisebox}


\begin{exercisebox}[medium]
\begin{problem}[Bayesian Inference]
Given a prior Beta(2, 2) distribution and observing 7 successes in 10 trials, find the posterior distribution and the Bayesian estimate.
\end{problem}
\begin{hintbox}
Use the Beta-Binomial conjugate relationship: Beta(α, β) + Binomial(n, p) → Beta(α + k, β + n - k).
\end{hintbox}
\end{exercisebox}


\subsection*{Hard}

\begin{exercisebox}[hard]
\begin{problem}[Information Theory in Neural Networks]
Analyse how mutual information between layers in a neural network can be used to understand information flow during training. Discuss the information bottleneck principle.
\end{problem}
\begin{hintbox}
Consider $I(X;Y) = H(X) - H(X|Y)$ and how it relates to representation learning.
\end{hintbox}
\end{exercisebox}


\begin{exercisebox}[hard]
\begin{problem}[Variational Inference]
Derive the Evidence Lower Bound (ELBO) for variational inference and explain its relationship to the Kullback-Leibler divergence.
\end{problem}
\begin{hintbox}
Start with log p(x) = log ∫ p(x,z) dz and use Jensen's inequality with a variational distribution q(z).
\end{hintbox}
\end{exercisebox}


\begin{exercisebox}[hard]
\begin{problem}[Information Bottleneck Theory]
Prove that the information bottleneck principle leads to a trade-off between compression and prediction accuracy in representation learning.
\end{problem}
\begin{hintbox}
Consider the Lagrangian L = I(X;T) - βI(T;Y) where T is the representation and β controls the trade-off.
\end{hintbox}
\end{exercisebox}


\begin{exercisebox}[hard]
\begin{problem}[PAC-Bayes Bounds]
Derive a PAC-Bayes bound for generalisation error in terms of the KL divergence between prior and posterior distributions.
\end{problem}
\begin{hintbox}
Use the change of measure inequality and the union bound over the hypothesis space.
\end{hintbox}
\end{exercisebox}


\begin{exercisebox}[hard]
\begin{problem}[Maximum Entropy Principle]
Show that the maximum entropy distribution under moment constraints is exponential family, and derive the dual optimisation problem.
\end{problem}
\begin{hintbox}
Use Lagrange multipliers to maximise H(p) subject to E[φᵢ(X)] = μᵢ for moment constraints.
\end{hintbox}
\end{exercisebox}


\begin{exercisebox}[hard]
\begin{problem}[Causal Inference and Information Theory]
Analyse the relationship between causal discovery and information-theoretic measures, particularly in the context of conditional independence testing.
\end{problem}
\begin{hintbox}
Consider how mutual information relates to conditional independence: X ⊥ Y | Z if and only if I(X;Y|Z) = 0.
\end{hintbox}
\end{exercisebox}


