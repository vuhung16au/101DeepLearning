% Chapter 16, Section 1

\section{Graphical Models \difficultyInline{advanced}}
\label{sec:graphical-models}

\subsection{Motivation}

Graphical models represent complex probability distributions using graphs:
\begin{itemize}
    \item Nodes: Random variables
    \item Edges: Probabilistic dependencies
\end{itemize}

\subsection{Bayesian Networks}

\textbf{Directed acyclic graphs} (DAGs) represent conditional dependencies:
\begin{equation}
p(\vect{x}) = \prod_{i=1}^{n} p(x_i | \text{Pa}(x_i))
\end{equation}

where $\text{Pa}(x_i)$ are parents of $x_i$.

\textbf{Example:} Naive Bayes classifier
\begin{equation}
p(y, \vect{x}) = p(y) \prod_{i=1}^{d} p(x_i|y)
\end{equation}

\subsection{Markov Random Fields}

\textbf{Undirected graphs} with potential functions:
\begin{equation}
p(\vect{x}) = \frac{1}{Z} \prod_{c \in \mathcal{C}} \psi_c(\vect{x}_c)
\end{equation}

where $\mathcal{C}$ are cliques and $Z$ is the partition function.

\textbf{Example:} Ising model, Conditional Random Fields (CRFs)


% \subsection{Visual aids}
% \addcontentsline{toc}{subsubsection}{Visual aids (graphical models)}

% \begin{figure}[h]
%   \centering
%   \begin{tikzpicture}[>=stealth]
%     % Simple DAG: X1 -> X3 <- X2
%     \node[circle,draw] (x1) at (0,0) {$x_1$};
%     \node[circle,draw] (x2) at (2,0) {$x_2$};
%     \node[circle,draw] (x3) at (1,-1.2) {$x_3$};
%     \draw[->] (x1) -- (x3);
%     \draw[->] (x2) -- (x3);
%   \end{tikzpicture}
%   \caption{A simple Bayesian network (DAG) encoding conditional dependencies.}
%   \label{fig:bn-dag}
% \end{figure}

% \begin{figure}[h]
%   \centering
%   \begin{tikzpicture}
%     % 3x3 MRF grid (undirected)
%     \foreach \i in {0,1,2} {
%       \foreach \j in {0,1,2} {
%         \node[circle,draw,inner sep=1.5pt] (n\i\j) at (\i*0.9,-\j*0.9) {};
%       }
%     }
%     \foreach \i in {0,1,2} {
%       \foreach \j in {0,1} {
%         \draw (n\i\j) -- (n\i\the\numexpr\j+1\relax);
%       }
%     }
%     \foreach \i in {0,1} {
%       \foreach \j in {0,1,2} {
%         \draw (n\i\j) -- (n\the\numexpr\i+1\relax\j);
%       }
%     }
%   \end{tikzpicture}
%   \caption{Undirected MRF grid with pairwise potentials between neighbors.}
%   \label{fig:mrf-grid}
% \end{figure}

% \subsection{Notes and references}

% Foundations and factorization properties are covered extensively in \textcite{Bishop2006,GoodfellowEtAl2016,Prince2023}.
