% Chapter 18, Section 1

\section{The Partition Function Problem \difficultyInline{advanced}}
\label{sec:partition-problem}

Many models have form:
\begin{equation}
p(\vect{x}) = \frac{1}{Z} \tilde{p}(\vect{x})
\end{equation}

where partition function $Z = \sum_{\vect{x}} \tilde{p}(\vect{x})$ or $Z = \int \tilde{p}(\vect{x}) d\vect{x}$ is intractable.

\subsection{Why It's Hard}

Computing $Z$ requires:
\begin{itemize}
    \item Summing/integrating over all configurations
    \item Exponential in dimensionality
    \item Prohibitive for high-dimensional models
\end{itemize}

\subsection{Impact}

Cannot directly:
\begin{itemize}
    \item Evaluate likelihood $p(\vect{x})$
    \item Compute gradients for learning
    \item Compare models
\end{itemize}

% \subsection{Visual aids}
% \addcontentsline{toc}{subsubsection}{Visual aids (partition function)}

% \begin{figure}[h]
%   \centering
%   \begin{tikzpicture}
%     \begin{axis}[
%       width=0.48\textwidth,height=0.36\textwidth,
%       xlabel={Dimension $d$}, ylabel={Configurations $\sim 2^d$}, ymode=log, grid=both]
%       \addplot[bookpurple,very thick] coordinates{(4,16) (6,64) (8,256) (10,1024) (12,4096)};
%     \end{axis}
%   \end{tikzpicture}
%   \caption{Configuration count grows exponentially with dimension, making $Z$ intractable (illustrative).}
%   \label{fig:pf-exp}
% \end{figure}

% \subsection{Notes and references}

% See \textcite{GoodfellowEtAl2016,Prince2023} for treatments of partition functions in energy-based models and practical workarounds.

