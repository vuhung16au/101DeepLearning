% Exercises (Hands-On Exercises) for Chapter 8: Optimization for Training Deep Models

\section*{Exercises}
\addcontentsline{toc}{section}{Exercises}

\subsection*{Easy}

\begin{exercisebox}[easy]
\begin{problem}[Batch Size Selection]
Explain the trade-offs between using a large batch size versus a small batch size for training. Consider computation time, memory usage, and convergence properties.
\end{problem}
\begin{hintbox}
Think about GPU utilization, gradient noise, and generalization gap.
\end{hintbox}
\end{exercisebox}


\begin{exercisebox}[easy]
\begin{problem}[Momentum Intuition]
Describe how momentum helps accelerate optimization. Use the analogy of a ball rolling down a hill to explain the concept.
\end{problem}
\begin{hintbox}
Consider how previous gradients influence the current update and help overcome small local variations.
\end{hintbox}
\end{exercisebox}


\begin{exercisebox}[easy]
\begin{problem}[Learning Rate Scheduling]
List three common learning rate scheduling strategies and explain when each is most appropriate.
\end{problem}
\begin{hintbox}
Consider step decay, exponential decay, cosine annealing, and cyclical learning rates.
\end{hintbox}
\end{exercisebox}


\begin{exercisebox}[easy]
\begin{problem}[Adam Hyperparameters]
Adam optimizer has hyperparameters $\beta_1$ (typically 0.9) and $\beta_2$ (typically 0.999). Explain the role of each parameter.
\end{problem}
\begin{hintbox}
$\beta_1$ controls momentum (first moment), $\beta_2$ controls adaptive learning rates (second moment).
\end{hintbox}
\end{exercisebox}


\subsection*{Medium}

\begin{exercisebox}[medium]
\begin{problem}[Optimizer Comparison]
Compare SGD with momentum, RMSProp, and Adam on a simple optimization problem. Discuss their convergence behavior and when to prefer one over another.
\end{problem}
\begin{hintbox}
Consider sparse gradients, non-stationary objectives, and computational cost.
\end{hintbox}
\end{exercisebox}


\begin{exercisebox}[medium]
\begin{problem}[Gradient Clipping]
Explain why gradient clipping is important for training recurrent neural networks. Derive the gradient clipping formula and discuss the choice of threshold.
\end{problem}
\begin{hintbox}
Consider exploding gradients and the norm $\|\nabla L\|$. Clip by value or by norm.
\end{hintbox}
\end{exercisebox}


\subsection*{Hard}

\begin{exercisebox}[hard]
\begin{problem}[Natural Gradient Descent]
Derive the natural gradient update rule and explain why it is invariant to reparameterisations. Discuss the computational challenges of using natural gradient in deep learning.
\end{problem}
\begin{hintbox}
Consider the Fisher information matrix $\mat{F}$ and the update $\boldsymbol{\theta} \leftarrow \boldsymbol{\theta} - \eta \mat{F}^{-1} \nabla L$.
\end{hintbox}
\end{exercisebox}


\begin{exercisebox}[hard]
\begin{problem}[Learning Rate Warmup]
Analyse why learning rate warmup is beneficial when training with large batch sizes. Provide theoretical justification based on the optimization landscape.
\end{problem}
\begin{hintbox}
Consider the stability of gradients in early training and the sharpness of the loss landscape.
\end{hintbox}
\end{exercisebox}


\begin{exercisebox}[hard]
\begin{problem}[Optimizer Comparison]
Compare the convergence properties of SGD, Adam, and RMSprop on a non-convex optimization problem.
\end{problem}
\begin{hintbox}
Consider the adaptive learning rates and momentum effects of each optimizer.
\end{hintbox}
\end{exercisebox}


\begin{exercisebox}[hard]
\begin{problem}[Gradient Clipping]
Explain when and why gradient clipping is necessary. How does it affect the optimization dynamics?
\end{problem}
\begin{hintbox}
Consider the exploding gradient problem and the relationship between gradient norms and learning stability.
\end{hintbox}
\end{exercisebox}


\begin{exercisebox}[hard]
\begin{problem}[Learning Rate Scheduling]
Design a learning rate schedule for training a deep network. Compare step decay, exponential decay, and cosine annealing.
\end{problem}
\begin{hintbox}
Consider the trade-off between exploration and exploitation in different phases of training.
\end{hintbox}
\end{exercisebox}


\begin{exercisebox}[hard]
\begin{problem}[Second-Order Methods]
Explain why second-order optimization methods are rarely used in deep learning despite their theoretical advantages.
\end{problem}
\begin{hintbox}
Consider computational complexity, memory requirements, and the stochastic nature of deep learning.
\end{hintbox}
\end{exercisebox}


\begin{exercisebox}[hard]
\begin{problem}[Optimization Landscape]
Analyse the relationship between the optimization landscape and the choice of optimizer for deep networks.
\end{problem}
\begin{hintbox}
Consider saddle points, local minima, and the role of noise in optimization.
\end{hintbox}
\end{exercisebox}


\begin{exercisebox}[hard]
\begin{problem}[Batch Size Effects]
Investigate how batch size affects optimization dynamics and generalization in deep learning.
\end{problem}
\begin{hintbox}
Consider the relationship between batch size, gradient noise, and the implicit regularization effect.
\end{hintbox}
\end{exercisebox}


