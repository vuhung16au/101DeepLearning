% Exercises (Exercises) for Chapter 13

\section*{Exercises}
\addcontentsline{toc}{section}{Exercises}

\subsection*{Easy}

\begin{exercisebox}[easy]
\begin{problem}[PPCA vs PCA]
Contrast PCA and probabilistic PCA in assumptions and outputs.
\end{problem}
\begin{hintbox}
Deterministic vs. probabilistic view; noise model; likelihood.
\end{hintbox}
\end{exercisebox}


\begin{exercisebox}[easy]
\begin{problem}[Dimensionality Choice]
List two heuristics to select latent dimensionality.
\end{problem}
\begin{hintbox}
Explained variance, information criteria.
\end{hintbox}
\end{exercisebox}


\begin{exercisebox}[easy]
\begin{problem}[Gaussian Conditionals]
Recall the conditional of a joint Gaussian and its role in E-steps.
\end{problem}
\begin{hintbox}
Use block-partitioned mean and covariance formulas.
\end{hintbox}
\end{exercisebox}


\begin{exercisebox}[easy]
\begin{problem}[Rotation Ambiguity]
Explain why factor loadings are identifiable only up to rotation.
\end{problem}
\begin{hintbox}
Orthogonal transforms preserve latent covariance.
\end{hintbox}
\end{exercisebox}


\subsection*{Medium}

\begin{exercisebox}[medium]
\begin{problem}[PPCA Likelihood]
Derive the log-likelihood of PPCA and the M-step for $\sigma^2$.
\end{problem}
\begin{hintbox}
Marginalise latents; differentiate w.r.t. variance.
\end{hintbox}
\end{exercisebox}


\begin{exercisebox}[medium]
\begin{problem}[EM for FA]
Write the E and M steps for Factor Analysis with diagonal noise.
\end{problem}
\begin{hintbox}
Use expected sufficient statistics of latents.
\end{hintbox}
\end{exercisebox}


\subsection*{Hard}

\begin{exercisebox}[hard]
\begin{problem}[Equivalence of PPCA Solution]
Show that the MLE loading matrix spans the top-$k$ eigenvectors of the sample covariance.
\end{problem}
\begin{hintbox}
Use spectral decomposition of covariance.
\end{hintbox}
\end{exercisebox}


\begin{exercisebox}[hard]
\begin{problem}[Nonlinear Extension]
Sketch how to generalise linear factor models to VAEs.
\end{problem}
\begin{hintbox}
Replace linear-Gaussian with neural encoder/decoder.
\end{hintbox}
\end{exercisebox}




\begin{exercisebox}[hard]
\begin{problem}[Advanced Topic 1]
Explain a key concept from this chapter and its practical applications.
\end{problem}
\begin{hintbox}
Consider the theoretical foundations and real-world implications.
\end{hintbox}
\end{exercisebox}


\begin{exercisebox}[hard]
\begin{problem}[Advanced Topic 2]
Analyse the relationship between different techniques covered in this chapter.
\end{problem}
\begin{hintbox}
Look for connections and trade-offs between methods.
\end{hintbox}
\end{exercisebox}


\begin{exercisebox}[hard]
\begin{problem}[Advanced Topic 3]
Design an experiment to test a hypothesis related to this chapter's content.
\end{problem}
\begin{hintbox}
Consider experimental design, metrics, and potential confounding factors.
\end{hintbox}
\end{exercisebox}


\begin{exercisebox}[hard]
\begin{problem}[Advanced Topic 4]
Compare different approaches to solving a problem from this chapter.
\end{problem}
\begin{hintbox}
Consider computational complexity, accuracy, and practical considerations.
\end{hintbox}
\end{exercisebox}


\begin{exercisebox}[hard]
\begin{problem}[Advanced Topic 5]
Derive a mathematical relationship or prove a theorem from this chapter.
\end{problem}
\begin{hintbox}
Start with the definitions and work through the logical steps.
\end{hintbox}
\end{exercisebox}


\begin{exercisebox}[hard]
\begin{problem}[Advanced Topic 6]
Implement a practical solution to a problem discussed in this chapter.
\end{problem}
\begin{hintbox}
Consider the implementation details and potential challenges.
\end{hintbox}
\end{exercisebox}


\begin{exercisebox}[hard]
\begin{problem}[Advanced Topic 7]
Evaluate the limitations and potential improvements of techniques from this chapter.
\end{problem}
\begin{hintbox}
Consider both theoretical limitations and practical constraints.
\end{hintbox}
\end{exercisebox}

