% Chapter 6, Section 6

\section{Problems \difficultyInline{intermediate}}
\label{sec:problems}

This section provides exercises to reinforce your understanding of feedforward networks. Problems are categorized by difficulty level, and hints are provided for each problem.

\subsection{Easy Problems (6 problems)}

\begin{problem}[Forward Pass Calculation]
\label{prob:forward-pass}

Consider a simple neural network with:
\begin{itemize}
    \item Input layer: 2 neurons with values $x_1 = 1$, $x_2 = -0.5$
    \item Hidden layer: 2 neurons with weights $\mat{W} = \begin{bmatrix} 0.5 & -0.3 \\ 0.8 & 0.2 \end{bmatrix}$ and bias $\vect{b} = [0.1, -0.2]$
    \item Activation function: ReLU
\end{itemize}

Calculate the output of the hidden layer.

\textbf{Hint:} Apply the formula $\vect{h} = \text{ReLU}(\mat{W}\vect{x} + \vect{b})$ step by step.
\end{problem}

\begin{problem}[Activation Function Properties]
\label{prob:activation-properties}

For each activation function, state whether it is:
\begin{enumerate}
    \item Bounded or unbounded
    \item Monotonic or non-monotonic
    \item Zero-centered or not zero-centered
\end{enumerate}

Functions: (a) Sigmoid, (b) Tanh, (c) ReLU, (d) Leaky ReLU

\textbf{Hint:} Consider the mathematical properties and plot each function to understand their behavior.
\end{problem}

\begin{problem}[Loss Function Selection]
\label{prob:loss-selection}

Match each task with the appropriate loss function:
\begin{enumerate}
    \item Predicting house prices
    \item Classifying emails as spam/not spam
    \item Recognizing digits 0-9
    \item Predicting stock market direction (up/down)
\end{enumerate}

Loss functions: MSE, Binary Cross-entropy, Categorical Cross-entropy

\textbf{Hint:} Consider the output format needed for each task and the mathematical properties of each loss function.
\end{problem}

\begin{problem}[Network Architecture]
\label{prob:architecture}

Design a neural network architecture for each task:
\begin{enumerate}
    \item Binary classification with 10 input features
    \item Regression with 5 input features
    \item 3-class classification with 8 input features
\end{enumerate}

Specify the number of layers, neurons per layer, and activation functions.

\textbf{Hint:} Consider the complexity of each task and the output requirements.
\end{problem}

\begin{problem}[Gradient Calculation]
\label{prob:gradient-calc}

For the function $f(x, y) = x^2 + 2xy + y^2$, calculate:
\begin{enumerate}
    \item $\frac{\partial f}{\partial x}$
    \item $\frac{\partial f}{\partial y}$
    \item The gradient vector $\nabla f$
\end{enumerate}

\textbf{Hint:} Use the power rule and product rule for differentiation.
\end{problem}

\begin{problem}[Chain Rule Application]
\label{prob:chain-rule}

Given $f(x) = (x^2 + 1)^3$, find $\frac{df}{dx}$ using the chain rule.

\textbf{Hint:} Let $u = x^2 + 1$, then $f = u^3$. Apply the chain rule: $\frac{df}{dx} = \frac{df}{du} \cdot \frac{du}{dx}$.
\end{problem}

\subsection{Medium Problems (5 problems)}

\begin{problem}[Backpropagation Derivation]
\label{prob:backprop-derivation}

Derive the backpropagation equations for a 2-layer network with:
\begin{itemize}
    \item Input: $\vect{x} \in \mathbb{R}^d$
    \item Hidden layer: $\vect{h} = \sigma(\mat{W}_1 \vect{x} + \vect{b}_1)$
    \item Output: $\hat{y} = \mat{W}_2 \vect{h} + \vect{b}_2$
    \item Loss: $L = \frac{1}{2}(\hat{y} - y)^2$
\end{itemize}

Find $\frac{\partial L}{\partial \mat{W}_1}$, $\frac{\partial L}{\partial \mat{W}_2}$, $\frac{\partial L}{\partial \vect{b}_1}$, and $\frac{\partial L}{\partial \vect{b}_2}$.

\textbf{Hint:} Use the chain rule systematically, starting from the loss and working backwards through each layer.
\end{problem}

\begin{problem}[Vanishing Gradient Analysis]
\label{prob:vanishing-gradient}

Consider a deep network with sigmoid activation functions. Show that the gradient can vanish exponentially with depth.

\textbf{Hint:} The derivative of sigmoid is $\sigma'(z) = \sigma(z)(1-\sigma(z))$. Since $\sigma(z) \in (0,1)$, the derivative is bounded by $\frac{1}{4}$. What happens when you multiply many such terms?
\end{problem}

\begin{problem}[Weight Initialization]
\label{prob:weight-init}

Derive the Xavier initialization for a layer with $n_{\text{in}}$ inputs and $n_{\text{out}}$ outputs, assuming the weights are drawn from a normal distribution.

\textbf{Hint:} Consider the variance of the output when inputs have unit variance. For the output to also have unit variance, what should be the variance of the weights?
\end{problem}

\begin{problem}[Universal Approximation]
\label{prob:universal-approx}

Explain why a single hidden layer with sigmoid activation can approximate any continuous function, but in practice, deeper networks are often preferred.

\textbf{Hint:} Consider the difference between theoretical capacity and practical learnability. What are the computational and optimization challenges?
\end{problem}

\begin{problem}[Activation Function Comparison]
\label{prob:activation-comparison}

Compare ReLU and sigmoid activation functions in terms of:
\begin{enumerate}
    \item Computational efficiency
    \item Gradient flow
    \item Saturation behavior
    \item Zero-centering
\end{enumerate}

When would you choose each one?

\textbf{Hint:} Consider the mathematical properties, training dynamics, and practical considerations for each activation function.
\end{problem}

\subsection{Hard Problems (5 problems)}

\begin{problem}[Deep Network Optimization]
\label{prob:deep-optimization}

Analyze why very deep networks are difficult to train and discuss modern solutions (residual connections, batch normalization, etc.).

\textbf{Hint:} Consider the relationship between network depth, gradient flow, and optimization landscape. How do modern techniques address these challenges?
\end{problem}

\begin{problem}[Loss Function Design]
\label{prob:loss-design}

Design a custom loss function for a multi-task learning problem where you need to:
\begin{enumerate}
    \item Classify images into 10 categories
    \item Predict bounding boxes for objects
    \item Estimate object confidence scores
\end{enumerate}

\textbf{Hint:} Consider how to combine different loss functions, handle different scales, and balance the importance of each task.
\end{problem}

\begin{problem}[Network Capacity Analysis]
\label{prob:capacity-analysis}

Analyze the relationship between network capacity (number of parameters) and generalization. When does more capacity help, and when does it hurt?

\textbf{Hint:} Consider the bias-variance tradeoff, the role of regularization, and the relationship between model complexity and data size.
\end{problem}

\begin{problem}[Gradient Flow Analysis]
\label{prob:gradient-flow}

Derive the gradient flow equations for a network with skip connections (residual blocks) and analyze how they help with training.

\textbf{Hint:} Consider how skip connections affect the gradient computation and what this means for the optimization dynamics.
\end{problem}

\begin{problem}[Architecture Search]
\label{prob:architecture-search}

Design a systematic approach to find the optimal network architecture for a given dataset and task. Consider:
\begin{enumerate}
    \item Search space definition
    \item Evaluation strategy
    \item Computational constraints
    \item Generalization considerations
\end{enumerate}

\textbf{Hint:} Consider automated machine learning (AutoML) techniques, neural architecture search (NAS), and the tradeoffs between search efficiency and architecture quality.
\end{problem}

% Index entries for problems
\index{problems!feedforward networks}
\index{exercises!neural networks}
\index{backpropagation!exercises}
\index{activation functions!exercises}
\index{gradient descent!exercises}
