% Exercises (Exercises) for Chapter 17

\section*{Exercises}
\addcontentsline{toc}{section}{Exercises}

\subsection*{Easy}

\begin{problem}[Self-Attention Intuition]
Explain why self-attention captures long-range dependencies.

\textbf{Hint:} Direct pairwise interactions.
\end{problem}

\begin{problem}[Positional Encoding]
Why do Transformers need positional encodings?

\textbf{Hint:} Permutation invariance of self-attention.
\end{problem}

\begin{problem}[Multi-Head Attention]
State the benefit of multiple attention heads.

\textbf{Hint:} Different representation subspaces.
\end{problem}

\begin{problem}[Masked Attention]
Explain the role of masking in causal attention.

\textbf{Hint:} Prevent future information leakage.
\end{problem}

\subsection*{Medium}

\begin{problem}[Computational Complexity]
Derive the computational complexity of self-attention.

\textbf{Hint:} $O(n^2 d)$ for sequence length $n$, dimension $d$.
\end{problem}

\begin{problem}[LayerNorm vs. BatchNorm]
Compare LayerNorm and BatchNorm in Transformers.

\textbf{Hint:} Independence from batch; sequence-level statistics.
\end{problem}

\subsection*{Hard}

\begin{problem}[Sparse Attention]
Design a sparse attention pattern and analyse complexity savings.

\textbf{Hint:} Local windows; strided patterns; $O(n \log n)$ or $O(n\sqrt{n})$.
\end{problem}

\begin{problem}[Attention Visualisation]
Propose methods to interpret attention weights and discuss limitations.

\textbf{Hint:} Attention rollout; gradient-based; correlation vs. causation.
\end{problem}


\begin{problem}[Advanced Topic 1]
Explain a key concept from this chapter and its practical applications.

\textbf{Hint:} Consider the theoretical foundations and real-world implications.
\end{problem}

\begin{problem}[Advanced Topic 2]
Analyse the relationship between different techniques covered in this chapter.

\textbf{Hint:} Look for connections and trade-offs between methods.
\end{problem}

\begin{problem}[Advanced Topic 3]
Design an experiment to test a hypothesis related to this chapter's content.

\textbf{Hint:} Consider experimental design, metrics, and potential confounding factors.
\end{problem}

\begin{problem}[Advanced Topic 4]
Compare different approaches to solving a problem from this chapter.

\textbf{Hint:} Consider computational complexity, accuracy, and practical considerations.
\end{problem}

\begin{problem}[Advanced Topic 5]
Derive a mathematical relationship or prove a theorem from this chapter.

\textbf{Hint:} Start with the definitions and work through the logical steps.
\end{problem}

\begin{problem}[Advanced Topic 6]
Implement a practical solution to a problem discussed in this chapter.

\textbf{Hint:} Consider the implementation details and potential challenges.
\end{problem}

\begin{problem}[Advanced Topic 7]
Evaluate the limitations and potential improvements of techniques from this chapter.

\textbf{Hint:} Consider both theoretical limitations and practical constraints.
\end{problem}
