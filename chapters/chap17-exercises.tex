% Exercises (Exercises) for Chapter 17

\section*{Exercises}
\addcontentsline{toc}{section}{Exercises}

\subsection*{Easy}

\begin{exercisebox}[easy]
\begin{problem}[Self-Attention Intuition]
Explain why self-attention captures long-range dependencies.
\end{problem}
\begin{hintbox}
Direct pairwise interactions.
\end{hintbox}
\end{exercisebox}


\begin{exercisebox}[easy]
\begin{problem}[Positional Encoding]
Why do Transformers need positional encodings?
\end{problem}
\begin{hintbox}
Permutation invariance of self-attention.
\end{hintbox}
\end{exercisebox}


\begin{exercisebox}[easy]
\begin{problem}[Multi-Head Attention]
State the benefit of multiple attention heads.
\end{problem}
\begin{hintbox}
Different representation subspaces.
\end{hintbox}
\end{exercisebox}


\begin{exercisebox}[easy]
\begin{problem}[Masked Attention]
Explain the role of masking in causal attention.
\end{problem}
\begin{hintbox}
Prevent future information leakage.
\end{hintbox}
\end{exercisebox}


\subsection*{Medium}

\begin{exercisebox}[medium]
\begin{problem}[Computational Complexity]
Derive the computational complexity of self-attention.
\end{problem}
\begin{hintbox}
$O(n^2 d)$ for sequence length $n$, dimension $d$.
\end{hintbox}
\end{exercisebox}


\begin{exercisebox}[medium]
\begin{problem}[LayerNorm vs. BatchNorm]
Compare LayerNorm and BatchNorm in Transformers.
\end{problem}
\begin{hintbox}
Independence from batch; sequence-level statistics.
\end{hintbox}
\end{exercisebox}


\subsection*{Hard}

\begin{exercisebox}[hard]
\begin{problem}[Sparse Attention]
Design a sparse attention pattern and analyse complexity savings.
\end{problem}
\begin{hintbox}
Local windows; strided patterns; $O(n \log n)$ or $O(n\sqrt{n})$.
\end{hintbox}
\end{exercisebox}


\begin{exercisebox}[hard]
\begin{problem}[Attention Visualisation]
Propose methods to interpret attention weights and discuss limitations.
\end{problem}
\begin{hintbox}
Attention rollout; gradient-based; correlation vs. causation.
\end{hintbox}
\end{exercisebox}



\begin{exercisebox}[hard]
\begin{problem}[Advanced Topic 1]
Explain a key concept from this chapter and its practical applications.
\end{problem}
\begin{hintbox}
Consider the theoretical foundations and real-world implications.
\end{hintbox}
\end{exercisebox}


\begin{exercisebox}[hard]
\begin{problem}[Advanced Topic 2]
Analyse the relationship between different techniques covered in this chapter.
\end{problem}
\begin{hintbox}
Look for connections and trade-offs between methods.
\end{hintbox}
\end{exercisebox}


\begin{exercisebox}[hard]
\begin{problem}[Advanced Topic 3]
Design an experiment to test a hypothesis related to this chapter's content.
\end{problem}
\begin{hintbox}
Consider experimental design, metrics, and potential confounding factors.
\end{hintbox}
\end{exercisebox}


\begin{exercisebox}[hard]
\begin{problem}[Advanced Topic 4]
Compare different approaches to solving a problem from this chapter.
\end{problem}
\begin{hintbox}
Consider computational complexity, accuracy, and practical considerations.
\end{hintbox}
\end{exercisebox}


\begin{exercisebox}[hard]
\begin{problem}[Advanced Topic 5]
Derive a mathematical relationship or prove a theorem from this chapter.
\end{problem}
\begin{hintbox}
Start with the definitions and work through the logical steps.
\end{hintbox}
\end{exercisebox}


\begin{exercisebox}[hard]
\begin{problem}[Advanced Topic 6]
Implement a practical solution to a problem discussed in this chapter.
\end{problem}
\begin{hintbox}
Consider the implementation details and potential challenges.
\end{hintbox}
\end{exercisebox}


\begin{exercisebox}[hard]
\begin{problem}[Advanced Topic 7]
Evaluate the limitations and potential improvements of techniques from this chapter.
\end{problem}
\begin{hintbox}
Consider both theoretical limitations and practical constraints.
\end{hintbox}
\end{exercisebox}

