% Chapter 7, Section 2

\section{Dataset Augmentation \difficultyInline{intermediate}}
\label{sec:data-augmentation}

\textbf{Data augmentation} artificially increases training set size by applying transformations that preserve labels.

\subsection{Intuition: Seeing the Same Thing in Many Ways}

Humans recognize an object despite different viewpoints, lighting, or small occlusions. Augmentation teaches models the same robustness by showing multiple, label-preserving variations of each example. This reduces overfitting by making spurious correlations less useful and forcing the model to focus on invariant structure.

\subsection{Image Augmentation}

Common transformations:
\begin{itemize}
    \item \textbf{Geometric:} rotation, translation, scaling, flipping, cropping
    \item \textbf{Color:} brightness, contrast, saturation adjustments
    \item \textbf{Noise:} Gaussian noise, blur
    \item \textbf{Cutout/Erasing:} randomly mask regions
    \item \textbf{Mixup:} blend pairs of images and labels
\end{itemize}

Example: horizontal flip
\begin{equation}
\vect{x}_{\text{aug}} = \text{flip}(\vect{x}), \quad y_{\text{aug}} = y
\end{equation}

\subsection{Text Augmentation}

For NLP:
\begin{itemize}
    \item Synonym replacement
    \item Random insertion/deletion
    \item Back-translation
    \item Paraphrasing
\end{itemize}

\subsection{Audio Augmentation}

For speech/audio:
\begin{itemize}
    \item Time stretching
    \item Pitch shifting
    \item Adding background noise
    \item SpecAugment (masking frequency/time regions)
\end{itemize}

\begin{figure}[htbp]
\centering
\begin{tikzpicture}
    % Original image placeholder
    \draw[fill=bookpurple!10] (0,0) rectangle (3,2);
    \node at (1.5,1) {Original};
    % Rotated
    \begin{scope}[xshift=120]
        \draw[fill=bookpurple!10,rotate=10] (0,0) rectangle (3,2);
        \node at (1.5,1) {Rotate};
    \end{scope}
    % Flipped
    \begin{scope}[xshift=240]
        \draw[fill=bookpurple!10] (0,0) rectangle (3,2);
        \node at (1.5,1) {Flip};
    \end{scope}
    % Cropped
    \begin{scope}[yshift=-80]
        \draw[fill=bookpurple!10] (0,0) rectangle (3,2);
        \draw[bookred,thick] (0.5,0.5) rectangle (2.5,1.5);
        \node at (1.5,1) {Crop};
    \end{scope}
    % Color jitter
    \begin{scope}[xshift=120,yshift=-80]
        \shade[left color=bookpurple!10,right color=bookred!20] (0,0) rectangle (3,2);
        \node at (1.5,1) {Color};
    \end{scope}
    % Cutout
    \begin{scope}[xshift=240,yshift=-80]
        \draw[fill=bookpurple!10] (0,0) rectangle (3,2);
        \draw[fill=bookblack!60] (1,0.7) rectangle (2,1.3);
        \node at (1.5,1) {Cutout};
    \end{scope}
\end{tikzpicture}
\caption{Illustration of common image augmentations. Variants preserve labels while encouraging invariances.}
\label{fig:augmentation-examples}
\end{figure}

% Index entries
\index{data augmentation}
\index{augmentation!vision}
\index{augmentation!text}
\index{augmentation!audio}

