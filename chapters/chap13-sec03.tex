% Chapter 13, Section 3

\section{Independent Component Analysis \difficultyInline{intermediate}}
\label{sec:ica}

\subsection{Objective}

Find independent sources from linear mixtures:
\begin{equation}
\vect{x} = \mat{A}\vect{s}
\end{equation}

where $\vect{s}$ contains independent sources.

\subsection{Non-Gaussianity}

ICA exploits that independent signals are typically non-Gaussian.

\textbf{Applications:}
\begin{itemize}
    \item Blind source separation (cocktail party problem)
    \item Signal processing
    \item Feature extraction
\end{itemize}

% \subsection{Visual aids}
% \addcontentsline{toc}{subsubsection}{Visual aids (ICA)}

% \begin{figure}[h]
%   \centering
%   \begin{tikzpicture}
%     \begin{axis}[
%       width=0.48\textwidth,height=0.36\textwidth,
%       xlabel={$s_1$}, ylabel={$s_2$}, grid=both]
%       \addplot+[only marks,mark=*,mark size=0.9pt,bookpurple!70] coordinates{(-1,0) (-0.8,0) (0.8,0) (1,0)};
%       \addplot+[only marks,mark=*,mark size=0.9pt,bookred!70] coordinates{(0,-1) (0,-0.8) (0,0.8) (0,1)};
%     \end{axis}
%   \end{tikzpicture}
%   \caption{Independent sparse sources aligned with axes (illustrative).}
%   \label{fig:ica-sources}
% \end{figure}

% \subsection{Historical context and references}

% ICA popularized as a blind source separation tool and applied widely in signal processing and neuroscience \textcite{Bishop2006,GoodfellowEtAl2016}.

