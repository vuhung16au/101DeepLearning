% Chapter 8, Section 3

\section{Adaptive Learning Rate Methods \difficultyInline{intermediate}}
\label{sec:adaptive-methods}

Adaptive learning rate methods automatically adjust step sizes per parameter based on historical gradient information, allowing faster progress on rarely-updated dimensions while stabilizing highly-volatile parameters.

\subsection{Intuition: Per-Parameter Step Sizes}

Different parameters learn at different speeds: some directions are steep, others are flat. \textbf{Adaptive methods} adjust the step size per parameter based on recent gradient information, allowing faster progress on rarely-updated or low-variance dimensions while stabilizing steps on highly-volatile ones.\index{adaptive optimization}\index{AdaGrad}\index{RMSProp}\index{Adam}

Historical note: AdaGrad emerged for sparse exercises; RMSProp stabilized AdaGrad's decay; Adam blended momentum with RMSProp-style adaptation and became a widely used default in deep learning \cite{Duchi2011,Tieleman2012,Kingma2014,GoodfellowEtAl2016}.

\subsection{AdaGrad}

Adapts learning rate per parameter based on historical gradients:
\begin{align}
\vect{g}_t &= \nabla_{\vect{\theta}} L(\vect{\theta}_t) \\
\vect{r}_t &= \vect{r}_{t-1} + \vect{g}_t \odot \vect{g}_t \\
\vect{\theta}_{t+1} &= \vect{\theta}_t - \frac{\alpha}{\sqrt{\vect{r}_t + \epsilon}} \odot \vect{g}_t
\end{align}

where $\epsilon$ (e.g., $10^{-8}$) prevents division by zero. AdaGrad is well-suited to sparse features: infrequent parameters receive larger effective steps, accelerating learning in NLP and recommender settings \cite{Duchi2011,WebOptimizationDLBook,D2LChapterOptimization}. A drawback is the ever-growing accumulator \(\vect{r}_t\), which can shrink steps too aggressively over long runs.

\subsection{RMSProp}

RMSProp (Root Mean Square Propagation) addresses AdaGrad's aggressive decay using exponential moving average:
\begin{align}
\vect{r}_t &= \rho \vect{r}_{t-1} + (1-\rho) \vect{g}_t \odot \vect{g}_t \\
\vect{\theta}_{t+1} &= \vect{\theta}_t - \frac{\alpha}{\sqrt{\vect{r}_t + \epsilon}} \odot \vect{g}_t
\end{align}

Add a small \(\epsilon\) for numerical stability and tune decay \(\rho\in[0.9,0.99]\). RMSProp prevents the learning rate from decaying to zero as in AdaGrad, making it effective for non-stationary objectives typical in deep networks \cite{Tieleman2012,WebOptimizationDLBook,D2LChapterOptimization}.

\subsection{Adam (Adaptive Moment Estimation)}

Combines momentum and adaptive learning rates:
\begin{align}
\vect{m}_t &= \beta_1 \vect{m}_{t-1} + (1-\beta_1) \vect{g}_t \\
\vect{v}_t &= \beta_2 \vect{v}_{t-1} + (1-\beta_2) \vect{g}_t \odot \vect{g}_t \\
\hat{\vect{m}}_t &= \frac{\vect{m}_t}{1 - \beta_1^t} \\
\hat{\vect{v}}_t &= \frac{\vect{v}_t}{1 - \beta_2^t} \\
\vect{\theta}_{t+1} &= \vect{\theta}_t - \frac{\alpha \hat{\vect{m}}_t}{\sqrt{\hat{\vect{v}}_t} + \epsilon}
\end{align}

Default hyperparameters: $\beta_1 = 0.9$, $\beta_2 = 0.999$, $\epsilon = 10^{-8}$, $\alpha = 0.001$ \cite{Kingma2014}. Adam often converges quickly and is robust to poorly scaled gradients. For best generalization in some vision tasks, SGD with momentum can still outperform Adam; consider switching optimizers during fine-tuning \cite{GoodfellowEtAl2016,D2LChapterOptimization,He2016}.

\subsection{Learning Rate Schedules}\index{learning rate schedule}

Learning rate schedules systematically adjust the learning rate during training to balance exploration and exploitation, often improving convergence speed and final performance. Different schedules provide various strategies for managing the learning rate over time, each suited to different optimization scenarios and model architectures.

Step decay reduces the learning rate by a factor at predetermined intervals, providing a simple yet effective approach for many applications:
\begin{equation}
\alpha_t = \alpha_0 \cdot \gamma^{\lfloor t / s \rfloor}
\end{equation}

Exponential decay provides smooth reduction of the learning rate over time, often useful for fine-tuning and convergence:
\begin{equation}
\alpha_t = \alpha_0 e^{-\lambda t}
\end{equation}

Cosine annealing creates a smooth, periodic learning rate schedule that can help escape local minima:
\begin{equation}
\alpha_t = \alpha_{\min} + \frac{1}{2}(\alpha_{\max} - \alpha_{\min})\left(1 + \cos\left(\frac{t}{T}\pi\right)\right)
\end{equation}

Additional scheduling strategies include warmup techniques that start from a small learning rate and increase linearly over the first \(T_w\) steps to reduce early instabilities in large-batch training. The one-cycle policy increases then anneals the learning rate, often paired with momentum decay, to speed convergence and improve generalization. Practical implementation typically combines cosine decay with warmup for transformer-like models, step decay for CNNs trained with SGD, and exponential decay for simple baselines.\index{learning-rate-schedule}\cite{WebOptimizationDLBook,D2LChapterOptimization}

