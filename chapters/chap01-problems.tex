% Problems (Hands-On Exercises) for Chapter 1: Introduction

\section*{Problems}
\addcontentsline{toc}{section}{Problems}

\subsection*{Easy}

\begin{problem}[Historical Milestones]
List three key breakthroughs that enabled the rise of deep learning in the 21st century and explain their significance.

\textbf{Hint:} Consider computational advances, data availability, and algorithmic innovations.
\end{problem}

\begin{problem}[Deep Learning vs Traditional ML]
Explain the main difference between deep learning and traditional machine learning approaches in terms of feature engineering.

\textbf{Hint:} Think about automatic feature learning versus manual feature extraction.
\end{problem}

\begin{problem}[Application Domains]
Name three real-world domains where deep learning has achieved significant success and briefly describe one application in each domain.

\textbf{Hint:} Consider computer vision, natural language processing, and speech recognition.
\end{problem}

\subsection*{Medium}

\begin{problem}[Enabling Factors]
Analyse how the availability of large datasets and computational resources (GPUs) together enabled the practical success of deep learning. Why wasn't one factor alone sufficient?

\textbf{Hint:} Consider the computational requirements of training deep networks and the need for diverse training examples.
\end{problem}

\begin{problem}[Challenges and Limitations]
Identify two major challenges or limitations of current deep learning approaches and propose potential research directions to address them.

\textbf{Hint:} Think about interpretability, data efficiency, generalisation, or robustness to adversarial examples.
\end{problem}

