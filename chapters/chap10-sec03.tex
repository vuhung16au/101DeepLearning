% Chapter 10, Section 3

\section{Long Short-Term Memory (LSTM) \difficultyInline{intermediate}}
\label{sec:lstm}

\subsection*{Intuition}

The LSTM adds a highway for information (the cell state) that can pass signals forward with minimal modification. Gates act like valves to forget unhelpful information, write new content, and reveal outputs, which preserves gradients over long spans \cite{Hochreiter1997,GoodfellowEtAl2016}.

\subsection*{Historical Context}

Introduced in the 1990s to address vanishing gradients \cite{Hochreiter1997}, LSTMs unlocked practical sequence learning across speech, language, and time-series tasks before attention-based Transformers became dominant \cite{Vaswani2017}.

% Index and glossary
\index{long short-term memory}
\glsadd{long-short-term-memory}

\subsection{Architecture}

LSTM uses \textbf{gating mechanisms} to control information flow and maintain a persistent cell state that supports long-range credit assignment \cite{Hochreiter1997,GoodfellowEtAl2016}. Let $\vect{h}_{t-1} \in \mathbb{R}^h$ denote the hidden state from the previous time step, $\vect{x}_t \in \mathbb{R}^d$ the current input, and $\vect{c}_{t-1} \in \mathbb{R}^h$ the previous cell state. The concatenation $[\vect{h}_{t-1}, \vect{x}_t] \in \mathbb{R}^{h+d}$ combines both sources of information. The LSTM computes the following:

\begin{align}
\vect{f}_t &= \sigma(\mat{W}_f [\vect{h}_{t-1}, \vect{x}_t] + \vect{b}_f) \quad \text{(forget gate)} \\
\vect{i}_t &= \sigma(\mat{W}_i [\vect{h}_{t-1}, \vect{x}_t] + \vect{b}_i) \quad \text{(input gate)} \\
\tilde{\vect{c}}_t &= \tanh(\mat{W}_c [\vect{h}_{t-1}, \vect{x}_t] + \vect{b}_c) \quad \text{(candidate)} \\
\vect{c}_t &= \vect{f}_t \odot \vect{c}_{t-1} + \vect{i}_t \odot \tilde{\vect{c}}_t \quad \text{(cell state)} \\
\vect{o}_t &= \sigma(\mat{W}_o [\vect{h}_{t-1}, \vect{x}_t] + \vect{b}_o) \quad \text{(output gate)} \\
\vect{h}_t &= \vect{o}_t \odot \tanh(\vect{c}_t) \quad \text{(hidden state)}
\end{align}

where $\sigma(\cdot)$ denotes the sigmoid function $\sigma(z) = 1/(1+e^{-z})$ applied element-wise, $\tanh(\cdot)$ is the hyperbolic tangent activation, and $\odot$ represents the Hadamard (element-wise) product. The weight matrices have dimensions $\mat{W}_f, \mat{W}_i, \mat{W}_c, \mat{W}_o \in \mathbb{R}^{h \times (h+d)}$ and bias vectors $\vect{b}_f, \vect{b}_i, \vect{b}_c, \vect{b}_o \in \mathbb{R}^h$.

The cell state update equation $\vect{c}_t = \vect{f}_t \odot \vect{c}_{t-1} + \vect{i}_t \odot \tilde{\vect{c}}_t$ can be understood as a gated linear interpolation. When $\vect{f}_t \approx \vect{1}$ (forget gate fully open) and $\vect{i}_t \approx \vect{0}$ (input gate closed), the cell state is preserved: $\vect{c}_t \approx \vect{c}_{t-1}$. When $\vect{f}_t \approx \vect{0}$ and $\vect{i}_t \approx \vect{1}$, the cell state is replaced: $\vect{c}_t \approx \tilde{\vect{c}}_t$. In practice, both gates take intermediate values, allowing partial forgetting and partial updating.

\subsection{Key Ideas}

The LSTM's key innovations address fundamental limitations of traditional RNNs and feedforward networks. Unlike standard RNNs where information must pass through repeated nonlinear transformations that cause gradient decay, the LSTM introduces a dedicated \textbf{cell state} $\vect{c}_t$ that acts as a highway for information flow with minimal transformation, enabling gradients to flow directly across long time spans. The \textbf{gating mechanisms} (forget, input, and output gates) provide selective control over information flow, allowing the network to learn when to remember, forget, and output information—a capability that was impossible with fixed-weight feedforward networks or vanilla RNNs. This selective memory management solves the vanishing gradient problem by creating direct paths for gradient flow while maintaining the ability to learn complex temporal dependencies that exceed the capacity of traditional sequence models.

\textbf{Cell state} $\vect{c}_t$: Long-term memory
\begin{itemize}
    \item Information flows with minimal transformation
    \item Gates control what to remember/forget
\end{itemize}

\textbf{Forget gate} $\vect{f}_t$: Decides what to discard from cell state

\textbf{Input gate} $\vect{i}_t$: Decides what new information to store

\textbf{Output gate} $\vect{o}_t$: Decides what to output

\subsection{Advantages}

\textbf{Addresses vanishing gradient problem:} LSTMs mitigate vanishing gradients through their cell state and gating mechanisms. Consider the gradient of the loss $L$ with respect to the cell state at an earlier time $k$:
\begin{equation}
\frac{\partial L}{\partial \vect{c}_k} = \frac{\partial L}{\partial \vect{c}_t} \prod_{i=k+1}^{t} \frac{\partial \vect{c}_i}{\partial \vect{c}_{i-1}}
\end{equation}
where $\frac{\partial \vect{c}_i}{\partial \vect{c}_{i-1}} = \text{diag}(\vect{f}_i)$ from the cell state update rule. When forget gates are close to 1, this product involves multiplications close to identity, preventing gradient decay. Unlike vanilla RNNs where $\frac{\partial \vect{h}_i}{\partial \vect{h}_{i-1}} = \mat{W}^\top \text{diag}(\sigma'(\vect{z}_i))$ involves repeated matrix multiplications that compound eigenvalue effects, the LSTM's cell state path provides a more direct gradient highway.

\textbf{Can learn long-term dependencies:} The forget and input gates explicitly control information retention through learnable parameters. The effective memory span can be approximated by considering when $\vect{f}_i \approx \vect{1}$ for all $i \in [k, t]$, the cell state propagates with minimal modification: $\vect{c}_t \approx \vect{c}_k + \sum_{i=k+1}^{t} \vect{i}_i \odot \tilde{\vect{c}}_i$. This additive structure, rather than multiplicative, allows information to persist over hundreds of time steps. The selective gating learns to retain task-relevant information whilst discarding noise.

\textbf{Gradients flow more easily through cell state:} The gradient flow through the cell state can be written as:
\begin{equation}
\frac{\partial L}{\partial \vect{c}_{t-1}} = \frac{\partial L}{\partial \vect{c}_t} \odot \vect{f}_t + \frac{\partial L}{\partial \vect{h}_t} \odot \vect{o}_t \odot (1 - \tanh^2(\vect{c}_t)) \odot \vect{f}_t
\end{equation}
The first term shows that when $\vect{f}_t \approx \vect{1}$, gradients pass through almost unchanged: $\frac{\partial L}{\partial \vect{c}_{t-1}} \approx \frac{\partial L}{\partial \vect{c}_t}$. This linear-like flow, especially when the forget gate is close to 1, prevents the repeated multiplication by small weights that causes gradients to vanish in vanilla RNNs where $\frac{\partial L}{\partial \vect{h}_{t-1}} \propto \mat{W}^\top \sigma'(\cdot)$. The gating mechanism creates a more stable gradient propagation environment by adaptively controlling gradient flow.

\textbf{Widely used for sequential tasks:} Due to their ability to handle long-term dependencies and mitigate gradient issues, LSTMs have become a de facto standard for various sequential data processing tasks. Their robust performance across diverse applications, from natural language processing to speech recognition, underscores their versatility and effectiveness. The architecture's success has made it a go-to choice for practitioners working with temporal data.

\begin{figure}[h]
    \centering
    \begin{tikzpicture}[>=stealth, node distance=1.8cm]
        \tikzstyle{gate}=[draw, rounded corners, minimum width=1.5cm, minimum height=0.9cm]
        \node[gate] (ft) {$\vect{f}_t$};
        \node[gate, below of=ft] (it) {$\vect{i}_t$};
        \node[gate, below of=it] (ot) {$\vect{o}_t$};
        \node[right of=it, xshift=2.2cm] (ct1) {$\vect{c}_{t-1}$};
        \node[right of=ct1, xshift=1.2cm] (sum) {$+$};
        \node[right of=sum, xshift=1.2cm] (ct) {$\vect{c}_t$};
        \draw[->] (ct1) -- node[midway, above] {$\odot\,\vect{f}_t$} (sum);
        \draw[->] (it) -- node[midway, above] {$\odot\,\tilde{\vect{c}}_t$} (sum);
        \draw[->] (sum) -- (ct);
        \draw[->] (ot) |- ++(1.2,1.8) node[pos=0.55, right] {$\odot\,\tanh(\vect{c}_t)$} -| ++(0,0);
    \end{tikzpicture}
    \caption{Visual aid: A compact LSTM cell diagram.}
    \label{fig:lstm_cell_diagram}
\end{figure}

