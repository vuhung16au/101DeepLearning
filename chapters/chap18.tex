% Chapter 18: Confronting the Partition Function

\chapter{Confronting the Partition Function}
\label{chap:partition-function}

This chapter addresses computational challenges in probabilistic models arising from intractable partition functions.


\begin{learningobjectives}
\objective{Why partition functions are hard and where they arise}
\objective{Strategies: importance sampling, AIS, and contrastive methods}
\objective{Bias/variance trade-offs in partition function estimation}
\objective{Practical estimators under compute constraints}
\end{learningobjectives}



\section*{Intuition}
\addcontentsline{toc}{section}{Intuition}

Partition functions are normalizing constants that ensure probability distributions sum to one by dividing unnormalized probabilities by their sum over all possible states. They normalize probabilities by summing over exponentially many states, making exact computation intractable for high-dimensional models.

For example, in a binary image with 100 pixels, computing the partition function requires summing over $2^{100}$ possible configurations, which is computationally impossible. Like trying to count every possible arrangement of a massive jigsaw puzzle where each piece can be in one of two orientations, partition functions represent the total "weight" of all possible configurations that must be computed to get proper probabilities.

Think of partition functions as the "denominator" in a fraction where the numerator is the probability of a specific configuration and the denominator is the sum of probabilities of all possible configurations. Just as you can't know what fraction of a pizza you're eating without knowing the total size of the pizza, you can't compute exact probabilities without knowing the partition function.


% Chapter 18, Section 1

\section{The Partition Function Problem \difficultyInline{advanced}}
\label{sec:partition-problem}

Many probabilistic models have the form:
\begin{equation}
p(\vect{x}) = \frac{1}{Z} \tilde{p}(\vect{x})
\end{equation}

This equation (18.1) shows that the normalized probability $p(\vect{x})$ is obtained by dividing the unnormalized probability $\tilde{p}(\vect{x})$ by the partition function $Z$. The partition function $Z = \sum_{\vect{x}} \tilde{p}(\vect{x})$ (for discrete variables) or $Z = \int \tilde{p}(\vect{x}) d\vect{x}$ (for continuous variables) is intractable because it requires summing or integrating over all possible configurations.

In deep learning, this problem arises in energy-based models like Restricted Boltzmann Machines and modern generative models, where we need to compute likelihoods for training but cannot evaluate the partition function exactly. This forces us to use approximate methods like contrastive divergence, noise contrastive estimation, or score matching to train these models effectively.

\subsection{Why It's Hard}

Computing the partition function $Z$ presents fundamental computational challenges that grow exponentially with the dimensionality of the problem. The core difficulty lies in the requirement to sum or integrate over all possible configurations, which becomes computationally prohibitive as the number of variables increases.

The exponential growth in dimensionality means that for a model with $d$ binary variables, we must consider $2^d$ possible configurations, making exact computation impossible for realistic model sizes. This exponential explosion affects not just the computational cost but also the memory requirements, as we need to store and process information about exponentially many states.

The solution to this intractability involves developing approximate methods that avoid computing the partition function directly. These approaches include contrastive divergence, which uses short Markov chains to approximate gradients; noise contrastive estimation, which transforms the problem into binary classification; and score matching, which works with gradients rather than probabilities. Each method trades off between computational efficiency and approximation accuracy, enabling practical training of complex probabilistic models.

\subsection{Impact}

The intractability of partition functions has profound implications for probabilistic modeling and machine learning, fundamentally limiting our ability to work with complex models in their most natural form. When we cannot compute the partition function, we lose the ability to evaluate exact likelihoods, which are essential for model comparison, parameter estimation, and uncertainty quantification.

This limitation prevents us from directly computing gradients needed for learning, forcing us to develop alternative training procedures that approximate the true gradients. The inability to evaluate exact likelihoods also makes it difficult to compare different models or assess their relative performance, as we cannot compute the standard likelihood-based metrics that would naturally arise from the probabilistic framework.

These challenges have driven the development of specialized techniques that work around the partition function problem, including approximate inference methods, contrastive learning approaches, and score-based training procedures. While these methods provide practical solutions, they often introduce bias or require careful tuning, highlighting the fundamental tension between theoretical elegance and computational feasibility in probabilistic modeling.

% \subsection{Visual aids}
% \addcontentsline{toc}{subsubsection}{Visual aids (partition function)}

% \begin{figure}[h]
%   \centering
%   \begin{tikzpicture}
%     \begin{axis}[
%       width=0.48\textwidth,height=0.36\textwidth,
%       xlabel={Dimension $d$}, ylabel={Configurations $\sim 2^d$}, ymode=log, grid=both]
%       \addplot[bookpurple,very thick] coordinates{(4,16) (6,64) (8,256) (10,1024) (12,4096)};
%     \end{axis}
%   \end{tikzpicture}
%   \caption{Configuration count grows exponentially with dimension, making $Z$ intractable (illustrative).}
%   \label{fig:pf-exp}
% \end{figure}

% \subsection{Notes and references}

% See \textcite{GoodfellowEtAl2016,Prince2023} for treatments of partition functions in energy-based models and practical workarounds.


% Chapter 18, Section 2

\section{Contrastive Divergence \difficultyInline{advanced}}
\label{sec:contrastive-divergence}

Contrastive divergence is a practical training algorithm that approximates the intractable gradients in energy-based models by using short Markov chains instead of requiring exact sampling from the model distribution.

\subsection{Motivation}

For Restricted Boltzmann Machines (RBMs), the joint probability distribution is given by:
\begin{equation}
p(\vect{v}, \vect{h}) = \frac{1}{Z} \exp(-E(\vect{v}, \vect{h}))
\end{equation}

This equation (18.2) shows that the probability of a configuration is proportional to the exponential of the negative energy, normalized by the partition function $Z$. The exact gradient for learning requires computing expectations under the model distribution, which involves the intractable partition function.

The gradient computation involves two terms:
\begin{equation}
\frac{\partial \log p(\vect{v})}{\partial \theta} = -\mathbb{E}_{p(\vect{h}|\vect{v})}\left[\frac{\partial E}{\partial \theta}\right] + \mathbb{E}_{p(\vect{v}, \vect{h})}\left[\frac{\partial E}{\partial \theta}\right]
\end{equation}

This equation (18.3) shows that the gradient consists of a positive term (expectation under the data distribution) and a negative term (expectation under the model distribution). The second term requires sampling from the full model distribution, which is intractable due to the partition function, motivating the need for approximate methods like contrastive divergence.

\subsection{CD-k Algorithm}

The CD-k algorithm approximates the intractable second term in the gradient by using a short Markov Chain Monte Carlo (MCMC) chain. An MCMC chain is a sequence of samples generated by iteratively applying a transition operator that preserves the target distribution, allowing us to sample from complex distributions without computing the partition function.

The algorithm works by starting from the data and running a short chain of k Gibbs sampling steps, which provides a biased but computationally efficient approximation to the true gradient. The key insight is that we don't need to run the chain until convergence; even a few steps provide a useful approximation that enables effective learning. This approach trades off between computational efficiency and approximation accuracy, making it practical for training energy-based models on large datasets.

Despite being biased, the CD-k algorithm works surprisingly well in practice because the bias tends to be in a direction that still provides useful gradient information for learning. The algorithm has been successfully applied to training Restricted Boltzmann Machines and other energy-based models, demonstrating that approximate methods can be highly effective even when they don't provide exact gradients.

% \subsection{Visual aids}
% \addcontentsline{toc}{subsubsection}{Visual aids (contrastive divergence)}

% \begin{figure}[h]
%   \centering
%   \begin{tikzpicture}
%     \begin{axis}[
%       width=0.48\textwidth,height=0.36\textwidth,
%       xlabel={$k$ steps}, ylabel={Reconstruction error}, grid=both]
%       \addplot[bookpurple,very thick] coordinates{(1,0.20) (2,0.15) (5,0.12) (10,0.11)};
%     \end{axis}
%   \end{tikzpicture}
%   \caption{CD-$k$: reconstruction error vs. number of Gibbs steps (illustrative).}
%   \label{fig:cdk}
% \end{figure}

% \subsection{Notes and references}

% Contrastive divergence for RBMs is covered in \textcite{GoodfellowEtAl2016,Prince2023}.


% Chapter 18, Section 3

\section{Noise-Contrastive Estimation \\difficultyInline{advanced}}
\label{sec:nce}


\subsection{Key Idea}

Turn density estimation into binary classification:
\begin{itemize}
    \item Distinguish data samples from noise samples
    \item Avoids computing partition function
\end{itemize}

\subsection{NCE Objective}

\begin{equation}
\mathcal{L} = \mathbb{E}_{p_{\text{data}}}[\log h(\vect{x})] + k \cdot \mathbb{E}_{p_{\text{noise}}}[\log(1-h(\vect{x}))]
\end{equation}

where:
\begin{equation}
h(\vect{x}) = \frac{p_{\text{model}}(\vect{x})}{p_{\text{model}}(\vect{x}) + k \cdot p_{\text{noise}}(\vect{x})}
\end{equation}

\subsection{Applications}

\begin{itemize}
    \item Word embeddings (word2vec)
    \item Language models
    \item Energy-based models
\end{itemize}

% \subsection{Visual aids}
% \addcontentsline{toc}{subsubsection}{Visual aids (NCE)}

% \begin{figure}[h]
%   \centering
%   \begin{tikzpicture}
%     \begin{axis}[
%       width=0.48\textwidth,height=0.36\textwidth,
%       xlabel={Score $h(x)$}, ylabel={Density}, grid=both]
%       \addplot[bookpurple,very thick,domain=0:1,samples=100]{4*x*(1-x)};
%       \addplot[bookred,very thick,dashed,domain=0:1,samples=100]{2*(1-x)};
%     \end{axis}
%   \end{tikzpicture}
%   \caption{Schematic of classifier scores for data (solid) vs. noise (dashed).}
%   \label{fig:nce-scores}
% \end{figure}

\subsection{Notes and references}

NCE as a technique to bypass partition functions is discussed in \textcite{GoodfellowEtAl2016,Prince2023}.


% Chapter 18, Section 4

\section{Score Matching \difficultyInline{advanced}}
\label{sec:score-matching}

Score matching is a powerful technique that learns probability distributions by matching the gradients of the log-density (score function) rather than the densities themselves. The score function is defined as:
\begin{equation}
\psi(\vect{x}) = \nabla_{\vect{x}} \log p(\vect{x})
\end{equation}

This equation shows that the score function represents the gradient of the log-probability with respect to the input variables. The key insight is that the score function can be computed without knowing the partition function, as the normalization constant cancels out when taking the gradient.

The score matching objective minimizes the squared difference between the model's score function and the data's score function:
\begin{equation}
\mathcal{L} = \frac{1}{2} \mathbb{E}_{p_{\text{data}}}[\|\psi_{\theta}(\vect{x}) - \nabla_{\vect{x}} \log p_{\text{data}}(\vect{x})\|^2]
\end{equation}

This approach avoids the partition function problem entirely because the normalization constant cancels out in the gradient computation, making the objective tractable to optimize. Score matching has been particularly successful in training energy-based models and has connections to modern generative modeling techniques like diffusion models.

% \subsection{Visual aids}
% \addcontentsline{toc}{subsubsection}{Visual aids (score matching)}

% \begin{figure}[h]
%   \centering
%   \begin{tikzpicture}
%     \begin{axis}[
%       width=0.48\textwidth,height=0.36\textwidth,
%       xlabel={$x$}, ylabel={Score $\nabla_x \log p(x)$}, grid=both]
%       \addplot[bookpurple,very thick,domain=-2:2,samples=100]{-x};
%     \end{axis}
%   \end{tikzpicture}
%   \caption{Score function for a standard normal $p(x) \propto e^{-x^2/2}$ is $-x$ (illustrative).}
%   \label{fig:score-normal}
% \end{figure}

% \subsection{Notes and references}

% See \textcite{GoodfellowEtAl2016,Prince2023} for score matching and related denoising score matching.


% Chapter 18: Real World Applications

\section{Real World Applications}
\label{sec:partition-real-world}


Confronting the partition function—computing normalizing constants in probabilistic models—is a fundamental challenge. Practical applications require approximations and specialized techniques to make inference tractable in complex models.

\subsection{Recommender Systems at Scale}

Efficient scoring of millions of items:

\begin{itemize}
    \item \textbf{YouTube video recommendations:} YouTube must score millions of videos for each user. Computing exact probabilities requires evaluating partition functions over all possible videos—computationally infeasible. Instead, systems use approximate methods like negative sampling and importance sampling, providing good recommendations efficiently. These approximations enable real-time personalization for billions of users.
    
    \item \textbf{E-commerce product ranking:} Online retailers face similar challenges ranking products. Models learn to score product-user compatibility, but exact probability computations are intractable. Contrastive learning methods approximate partition functions by sampling negative examples, enabling practical deployment at scale while maintaining recommendation quality.
    
    \item \textbf{Music playlist generation:} Streaming services create personalized playlists by modeling sequential song compatibility. Full probabilistic models would require intractable partition functions over all possible song sequences. Practical systems use locally normalized models and sampling-based approximations, generating engaging playlists efficiently.
\end{itemize}

\subsection{Natural Language Processing}

Handling large vocabularies efficiently:

\begin{itemize}
    \item \textbf{Language model training:} Modern language models predict next words from vocabularies of 50,000+ tokens. Computing partition functions over all possible next words for every training example is expensive. Techniques like noise contrastive estimation and self-normalization make training practical, enabling language models that power translation, autocomplete, and conversational AI.
    
    \item \textbf{Neural machine translation:} Translation models generate target sentences word by word, considering vast numbers of possible continuations. Exact probability computation would require intractable partition functions. Beam search with approximate scoring enables practical translation systems, producing high-quality translations in real-time.
    
    \item \textbf{Named entity recognition:} Identifying people, places, and organizations in text involves structured prediction over exponentially many possible tag sequences. Conditional random fields require computing partition functions efficiently. The forward-backward algorithm provides exact computation for chain structures, enabling accurate entity extraction in applications from news analysis to medical record processing.
\end{itemize}

\subsection{Computer Vision}

Structured prediction in image understanding:

\begin{itemize}
    \item \textbf{Semantic segmentation:} Labeling every pixel in images requires modeling dependencies between neighboring pixels. Fully modeling these dependencies involves intractable partition functions over pixel labelings. Practical systems use approximate inference (mean field approximation, pseudo-likelihood) or structured models with tractable partition functions (chain or tree structures).
    
    \item \textbf{Pose estimation:} Estimating human body poses involves predicting joint locations with anatomical constraints (arms connect to shoulders, legs have limited range). Models encoding these constraints have complex partition functions. Approximate inference techniques enable real-time pose estimation for applications from gaming to physical therapy.
    
    \item \textbf{Object detection:} Detecting objects requires scoring countless possible bounding boxes. Models learning to rank boxes face partition function challenges similar to recommendation systems. Techniques like contrastive learning and hard negative mining make training practical, enabling accurate detection in applications from autonomous driving to retail analytics.
\end{itemize}

\subsection{Practical Solutions}

Key strategies for handling partition functions:
\begin{itemize}
    \item \textbf{Approximation methods:} Monte Carlo sampling, variational inference
    \item \textbf{Negative sampling:} Approximate partition functions using sampled negatives
    \item \textbf{Structured models:} Design models with tractable partition functions
    \item \textbf{Unnormalized models:} Use score-based approaches avoiding normalization
\end{itemize}

These applications show how confronting the partition function is not just a theoretical concern—it's a practical challenge requiring clever approximations to deploy probabilistic models at scale.

% Index entries
\index{applications!recommender systems}
\index{applications!language models}
\index{applications!computer vision}
\index{partition function!applications}


% Chapter summary and problems
% Key Takeaways for Chapter 18

\section*{Key Takeaways}
\addcontentsline{toc}{section}{Key Takeaways}

\begin{keytakeaways}
\begin{itemize}[leftmargin=2em]
    \item \textbf{Partition functions} create intractable normalisers in many models.
    \item \textbf{Estimators} (IS, AIS, contrastive) trade bias and variance differently.
    \item \textbf{Practicality} depends on proposal quality and compute budget.
\end{itemize}
\end{keytakeaways}



% Exercises (Exercises) for Chapter 18

\section*{Exercises}
\addcontentsline{toc}{section}{Exercises}

\subsection*{Easy}

\begin{problem}[MDP Definition]
Define the components of a Markov Decision Process.

\textbf{Hint:} States, actions, rewards, transition dynamics, discount factor.
\end{problem}

\begin{problem}[Value Function]
Explain the difference between $V(s)$ and $Q(s,a)$.

\textbf{Hint:} State value vs. action-value.
\end{problem}

\begin{problem}[Policy Types]
Contrast deterministic and stochastic policies.

\textbf{Hint:} Mapping vs. distribution over actions.
\end{problem}

\begin{problem}[Exploration vs. Exploitation]
Give two exploration strategies in RL.

\textbf{Hint:} $\epsilon$-greedy; UCB; entropy regularisation.
\end{problem}

\subsection*{Medium}

\begin{problem}[Bellman Equation]
Derive the Bellman equation for $Q(s,a)$.

\textbf{Hint:} Recursive relationship with successor states.
\end{problem}

\begin{problem}[Policy Gradient]
Explain why policy gradient methods are useful for continuous action spaces.

\textbf{Hint:} Direct parameterisation; differentiability.
\end{problem}

\subsection*{Hard}

\begin{problem}[Actor-Critic Derivation]
Derive the advantage actor-critic update rule.

\textbf{Hint:} Baseline subtraction; variance reduction.
\end{problem}

\begin{problem}[Off-Policy Correction]
Analyse importance sampling for off-policy learning and its variance.

\textbf{Hint:} Likelihood ratio; distribution mismatch.
\end{problem}


\begin{problem}[Advanced Topic 1]
Explain a key concept from this chapter and its practical applications.

\textbf{Hint:} Consider the theoretical foundations and real-world implications.
\end{problem}

\begin{problem}[Advanced Topic 2]
Analyse the relationship between different techniques covered in this chapter.

\textbf{Hint:} Look for connections and trade-offs between methods.
\end{problem}

\begin{problem}[Advanced Topic 3]
Design an experiment to test a hypothesis related to this chapter's content.

\textbf{Hint:} Consider experimental design, metrics, and potential confounding factors.
\end{problem}

\begin{problem}[Advanced Topic 4]
Compare different approaches to solving a problem from this chapter.

\textbf{Hint:} Consider computational complexity, accuracy, and practical considerations.
\end{problem}

\begin{problem}[Advanced Topic 5]
Derive a mathematical relationship or prove a theorem from this chapter.

\textbf{Hint:} Start with the definitions and work through the logical steps.
\end{problem}

\begin{problem}[Advanced Topic 6]
Implement a practical solution to a problem discussed in this chapter.

\textbf{Hint:} Consider the implementation details and potential challenges.
\end{problem}

\begin{problem}[Advanced Topic 7]
Evaluate the limitations and potential improvements of techniques from this chapter.

\textbf{Hint:} Consider both theoretical limitations and practical constraints.
\end{problem}

