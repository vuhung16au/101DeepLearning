% Chapter 18: Confronting the Partition Function

\chapter{Confronting the Partition Function}
\label{chap:partition-function}

This chapter addresses computational challenges in probabilistic models arising from intractable partition functions.


\begin{learningobjectives}
\objective{Why partition functions are hard and where they arise}
\objective{Strategies: importance sampling, AIS, and contrastive methods}
\objective{Bias/variance trade-offs in partition function estimation}
\objective{Practical estimators under compute constraints}
\end{learningobjectives}



\section*{Intuition}
\addcontentsline{toc}{section}{Intuition}

Partition functions are normalizing constants that ensure probability distributions sum to one by dividing unnormalized probabilities by their sum over all possible states. They normalize probabilities by summing over exponentially many states, making exact computation intractable for high-dimensional models.

For example, in a binary image with 100 pixels, computing the partition function requires summing over $2^{100}$ possible configurations, which is computationally impossible. Like trying to count every possible arrangement of a massive jigsaw puzzle where each piece can be in one of two orientations, partition functions represent the total "weight" of all possible configurations that must be computed to get proper probabilities.

Think of partition functions as the "denominator" in a fraction where the numerator is the probability of a specific configuration and the denominator is the sum of probabilities of all possible configurations. Just as you can't know what fraction of a pizza you're eating without knowing the total size of the pizza, you can't compute exact probabilities without knowing the partition function.


% Chapter 18, Section 1

\section{The Partition Function Problem \difficultyInline{advanced}}
\label{sec:partition-problem}

Many probabilistic models have the form:
\begin{equation}
p(\vect{x}) = \frac{1}{Z} \tilde{p}(\vect{x})
\end{equation}

This equation (18.1) shows that the normalized probability $p(\vect{x})$ is obtained by dividing the unnormalized probability $\tilde{p}(\vect{x})$ by the partition function $Z$. The partition function $Z = \sum_{\vect{x}} \tilde{p}(\vect{x})$ (for discrete variables) or $Z = \int \tilde{p}(\vect{x}) d\vect{x}$ (for continuous variables) is intractable because it requires summing or integrating over all possible configurations.

In deep learning, this problem arises in energy-based models like Restricted Boltzmann Machines and modern generative models, where we need to compute likelihoods for training but cannot evaluate the partition function exactly. This forces us to use approximate methods like contrastive divergence, noise contrastive estimation, or score matching to train these models effectively.

\subsection{Why It's Hard}

Computing the partition function $Z$ presents fundamental computational challenges that grow exponentially with the dimensionality of the problem. The core difficulty lies in the requirement to sum or integrate over all possible configurations, which becomes computationally prohibitive as the number of variables increases.

The exponential growth in dimensionality means that for a model with $d$ binary variables, we must consider $2^d$ possible configurations, making exact computation impossible for realistic model sizes. This exponential explosion affects not just the computational cost but also the memory requirements, as we need to store and process information about exponentially many states.

The solution to this intractability involves developing approximate methods that avoid computing the partition function directly. These approaches include contrastive divergence, which uses short Markov chains to approximate gradients; noise contrastive estimation, which transforms the problem into binary classification; and score matching, which works with gradients rather than probabilities. Each method trades off between computational efficiency and approximation accuracy, enabling practical training of complex probabilistic models.

\subsection{Impact}

The intractability of partition functions has profound implications for probabilistic modeling and machine learning, fundamentally limiting our ability to work with complex models in their most natural form. When we cannot compute the partition function, we lose the ability to evaluate exact likelihoods, which are essential for model comparison, parameter estimation, and uncertainty quantification.

This limitation prevents us from directly computing gradients needed for learning, forcing us to develop alternative training procedures that approximate the true gradients. The inability to evaluate exact likelihoods also makes it difficult to compare different models or assess their relative performance, as we cannot compute the standard likelihood-based metrics that would naturally arise from the probabilistic framework.

These challenges have driven the development of specialized techniques that work around the partition function problem, including approximate inference methods, contrastive learning approaches, and score-based training procedures. While these methods provide practical solutions, they often introduce bias or require careful tuning, highlighting the fundamental tension between theoretical elegance and computational feasibility in probabilistic modeling.

% \subsection{Visual aids}
% \addcontentsline{toc}{subsubsection}{Visual aids (partition function)}

% \begin{figure}[h]
%   \centering
%   \begin{tikzpicture}
%     \begin{axis}[
%       width=0.48\textwidth,height=0.36\textwidth,
%       xlabel={Dimension $d$}, ylabel={Configurations $\sim 2^d$}, ymode=log, grid=both]
%       \addplot[bookpurple,very thick] coordinates{(4,16) (6,64) (8,256) (10,1024) (12,4096)};
%     \end{axis}
%   \end{tikzpicture}
%   \caption{Configuration count grows exponentially with dimension, making $Z$ intractable (illustrative).}
%   \label{fig:pf-exp}
% \end{figure}

% \subsection{Notes and references}

% See \textcite{GoodfellowEtAl2016,Prince2023} for treatments of partition functions in energy-based models and practical workarounds.


% Chapter 18, Section 2

\section{Contrastive Divergence \difficultyInline{advanced}}
\label{sec:contrastive-divergence}

\subsection{Motivation}

For Restricted Boltzmann Machines (RBMs):
\begin{equation}
p(\vect{v}, \vect{h}) = \frac{1}{Z} \exp(-E(\vect{v}, \vect{h}))
\end{equation}

Exact gradient requires expectations under model:
\begin{equation}
\frac{\partial \log p(\vect{v})}{\partial \theta} = -\mathbb{E}_{p(\vect{h}|\vect{v})}\left[\frac{\partial E}{\partial \theta}\right] + \mathbb{E}_{p(\vect{v}, \vect{h})}\left[\frac{\partial E}{\partial \theta}\right]
\end{equation}

\subsection{CD-k Algorithm}

Approximate second term with short MCMC chain (k steps):
\begin{enumerate}
    \item Start from data: $\vect{v}_0 = \vect{v}$
    \item Run k Gibbs steps
    \item Use $\vect{v}_k$ for negative phase
\end{enumerate}

Works surprisingly well despite being biased.

% \subsection{Visual aids}
% \addcontentsline{toc}{subsubsection}{Visual aids (contrastive divergence)}

% \begin{figure}[h]
%   \centering
%   \begin{tikzpicture}
%     \begin{axis}[
%       width=0.48\textwidth,height=0.36\textwidth,
%       xlabel={$k$ steps}, ylabel={Reconstruction error}, grid=both]
%       \addplot[bookpurple,very thick] coordinates{(1,0.20) (2,0.15) (5,0.12) (10,0.11)};
%     \end{axis}
%   \end{tikzpicture}
%   \caption{CD-$k$: reconstruction error vs. number of Gibbs steps (illustrative).}
%   \label{fig:cdk}
% \end{figure}

% \subsection{Notes and references}

% Contrastive divergence for RBMs is covered in \textcite{GoodfellowEtAl2016,Prince2023}.


% Chapter 18, Section 3

\section{Noise-Contrastive Estimation \\difficultyInline{advanced}}
\label{sec:nce}


\subsection{Key Idea}

Turn density estimation into binary classification:
\begin{itemize}
    \item Distinguish data samples from noise samples
    \item Avoids computing partition function
\end{itemize}

\subsection{NCE Objective}

\begin{equation}
\mathcal{L} = \mathbb{E}_{p_{\text{data}}}[\log h(\vect{x})] + k \cdot \mathbb{E}_{p_{\text{noise}}}[\log(1-h(\vect{x}))]
\end{equation}

where:
\begin{equation}
h(\vect{x}) = \frac{p_{\text{model}}(\vect{x})}{p_{\text{model}}(\vect{x}) + k \cdot p_{\text{noise}}(\vect{x})}
\end{equation}

\subsection{Applications}

\begin{itemize}
    \item Word embeddings (word2vec)
    \item Language models
    \item Energy-based models
\end{itemize}

% \subsection{Visual aids}
% \addcontentsline{toc}{subsubsection}{Visual aids (NCE)}

% \begin{figure}[h]
%   \centering
%   \begin{tikzpicture}
%     \begin{axis}[
%       width=0.48\textwidth,height=0.36\textwidth,
%       xlabel={Score $h(x)$}, ylabel={Density}, grid=both]
%       \addplot[bookpurple,very thick,domain=0:1,samples=100]{4*x*(1-x)};
%       \addplot[bookred,very thick,dashed,domain=0:1,samples=100]{2*(1-x)};
%     \end{axis}
%   \end{tikzpicture}
%   \caption{Schematic of classifier scores for data (solid) vs. noise (dashed).}
%   \label{fig:nce-scores}
% \end{figure}

\subsection{Notes and references}

NCE as a technique to bypass partition functions is discussed in \textcite{GoodfellowEtAl2016,Prince2023}.


% Chapter 18, Section 4

\section{Score Matching \difficultyInline{advanced}}
\label{sec:score-matching}

Match gradients of log-density (score function):
\begin{equation}
\psi(\vect{x}) = \nabla_{\vect{x}} \log p(\vect{x})
\end{equation}

Objective:
\begin{equation}
\mathcal{L} = \frac{1}{2} \mathbb{E}_{p_{\text{data}}}[\|\psi_{\theta}(\vect{x}) - \nabla_{\vect{x}} \log p_{\text{data}}(\vect{x})\|^2]
\end{equation}

Avoids partition function since it cancels in gradient.

% \subsection{Visual aids}
% \addcontentsline{toc}{subsubsection}{Visual aids (score matching)}

% \begin{figure}[h]
%   \centering
%   \begin{tikzpicture}
%     \begin{axis}[
%       width=0.48\textwidth,height=0.36\textwidth,
%       xlabel={$x$}, ylabel={Score $\nabla_x \log p(x)$}, grid=both]
%       \addplot[bookpurple,very thick,domain=-2:2,samples=100]{-x};
%     \end{axis}
%   \end{tikzpicture}
%   \caption{Score function for a standard normal $p(x) \propto e^{-x^2/2}$ is $-x$ (illustrative).}
%   \label{fig:score-normal}
% \end{figure}

% \subsection{Notes and references}

% See \textcite{GoodfellowEtAl2016,Prince2023} for score matching and related denoising score matching.


% Chapter 18: Real World Applications

\section{Real World Applications}
\label{sec:partition-real-world}


Confronting the partition function—computing normalizing constants in probabilistic models—is a fundamental challenge. Practical applications require approximations and specialized techniques to make inference tractable in complex models.

\subsection{Recommender Systems at Scale}

The challenge of providing personalized recommendations at scale represents one of the most compelling applications of partition function approximation techniques, where the need to score millions of items for billions of users makes exact probability computation completely infeasible.

YouTube's video recommendation system exemplifies this challenge, where the platform must score millions of videos for each user in real-time. Computing exact probabilities would require evaluating partition functions over all possible videos, which is computationally impossible. Instead, the system uses approximate methods like negative sampling and importance sampling to provide good recommendations efficiently, enabling real-time personalization for billions of users while maintaining acceptable computational costs.

E-commerce platforms face similar challenges when ranking products for users, where models must learn to score product-user compatibility without being able to compute exact probabilities. These systems employ contrastive learning methods that approximate partition functions by sampling negative examples, enabling practical deployment at scale while maintaining recommendation quality. The success of these approaches demonstrates how approximate methods can provide effective solutions even when exact computation is impossible.

Music streaming services create personalized playlists by modeling sequential song compatibility, where full probabilistic models would require intractable partition functions over all possible song sequences. Practical systems use locally normalized models and sampling-based approximations to generate engaging playlists efficiently, showing how partition function approximation techniques enable complex sequential modeling at scale.

\subsection{Natural Language Processing}

Natural language processing presents unique challenges for partition function computation, where the exponential growth in possible sequences makes exact probability computation intractable for realistic applications. The field has developed sophisticated approximation techniques that enable practical deployment of probabilistic models at scale.

Modern language model training exemplifies these challenges, where models must predict next words from vocabularies of 50,000+ tokens. Computing partition functions over all possible next words for every training example would be prohibitively expensive, making traditional maximum likelihood estimation impossible. Techniques like noise contrastive estimation and self-normalization have made training practical, enabling language models that power translation, autocomplete, and conversational AI systems that would otherwise be impossible to train.

Neural machine translation systems face similar challenges, where translation models generate target sentences word by word while considering vast numbers of possible continuations. Exact probability computation would require intractable partition functions over all possible translation sequences, making traditional approaches infeasible. Beam search with approximate scoring enables practical translation systems that produce high-quality translations in real-time, demonstrating how approximation techniques can maintain performance while avoiding computational bottlenecks.

Named entity recognition represents another critical application, where identifying people, places, and organizations in text involves structured prediction over exponentially many possible tag sequences. While conditional random fields require computing partition functions efficiently, the forward-backward algorithm provides exact computation for chain structures, enabling accurate entity extraction in applications ranging from news analysis to medical record processing. This demonstrates how careful model design can sometimes avoid partition function intractability entirely.

\subsection{Computer Vision}

Computer vision applications present unique challenges for partition function computation, where the need to model complex spatial and structural relationships often leads to intractable probability distributions. The field has developed specialized approximation techniques that enable practical deployment of probabilistic models for image understanding tasks.

Semantic segmentation exemplifies these challenges, where labeling every pixel in images requires modeling dependencies between neighboring pixels to ensure coherent segmentation results. Fully modeling these dependencies would involve intractable partition functions over all possible pixel labelings, making exact inference impossible. Practical systems use approximate inference methods like mean field approximation and pseudo-likelihood, or employ structured models with tractable partition functions using chain or tree structures that maintain computational feasibility while preserving important spatial relationships.

Pose estimation represents another critical application, where estimating human body poses involves predicting joint locations subject to anatomical constraints such as arms connecting to shoulders and legs having limited range of motion. Models that encode these constraints naturally have complex partition functions, but approximate inference techniques enable real-time pose estimation for applications ranging from gaming to physical therapy. These systems demonstrate how careful approximation can maintain performance while avoiding computational bottlenecks.

Object detection systems face partition function challenges similar to recommendation systems, where detecting objects requires scoring countless possible bounding boxes. Models that learn to rank boxes face the same fundamental challenge of intractable partition functions, but techniques like contrastive learning and hard negative mining make training practical. These approaches enable accurate detection in applications from autonomous driving to retail analytics, showing how partition function approximation techniques can be adapted across different domains.

\subsection{Practical Solutions}

The practical solutions to partition function intractability have evolved into a sophisticated toolkit of approximation techniques, each with its own strengths and trade-offs. These methods represent the collective wisdom of the machine learning community in addressing one of the most fundamental challenges in probabilistic modeling.

Approximation methods like Monte Carlo sampling and variational inference provide general-purpose approaches to handling intractable partition functions, offering different trade-offs between computational efficiency and approximation accuracy. Monte Carlo methods use random sampling to approximate expectations, while variational inference provides deterministic approximations that can be more computationally efficient but may introduce bias. The choice between these approaches depends on the specific requirements of the application and the available computational resources.

Negative sampling techniques have proven particularly effective in large-scale applications, where they approximate partition functions using sampled negatives rather than computing exact expectations. This approach has been successfully applied in recommendation systems, language modeling, and computer vision, demonstrating its versatility across different domains. The key insight is that we often don't need exact probabilities but rather relative rankings or scores that can be computed efficiently.

Structured models represent another important strategy, where careful design can sometimes avoid partition function intractability entirely. By constraining the model structure to use tractable components like chains or trees, we can maintain computational feasibility while preserving important dependencies. This approach has been particularly successful in sequence modeling and structured prediction tasks, where the structure of the problem naturally suggests tractable approximations.

These applications demonstrate that confronting the partition function is not just a theoretical concern but a practical challenge requiring clever approximations to deploy probabilistic models at scale. The success of these methods across diverse applications shows that partition function intractability, while fundamental, is not insurmountable when approached with the right combination of theoretical understanding and practical engineering.

% Index entries
\index{applications!recommender systems}
\index{applications!language models}
\index{applications!computer vision}
\index{partition function!applications}


% Chapter summary and problems
% Key Takeaways for Chapter 18

\section*{Key Takeaways}
\addcontentsline{toc}{section}{Key Takeaways}

\begin{keytakeaways}
\begin{itemize}[leftmargin=2em]
    \item \textbf{Partition functions} create intractable normalisers in many models.
    \item \textbf{Estimators} (IS, AIS, contrastive) trade bias and variance differently.
    \item \textbf{Practicality} depends on proposal quality and compute budget.
\end{itemize}
\end{keytakeaways}



% Exercises (Exercises) for Chapter 18

\section*{Exercises}
\addcontentsline{toc}{section}{Exercises}

\subsection*{Easy}

\begin{exercisebox}[easy]
\begin{problem}[MDP Definition]
Define the components of a Markov Decision Process.
\end{problem}
\begin{hintbox}
States, actions, rewards, transition dynamics, discount factor.
\end{hintbox}
\end{exercisebox}


\begin{exercisebox}[easy]
\begin{problem}[Value Function]
Explain the difference between $V(s)$ and $Q(s,a)$.
\end{problem}
\begin{hintbox}
State value vs. action-value.
\end{hintbox}
\end{exercisebox}


\begin{exercisebox}[easy]
\begin{problem}[Policy Types]
Contrast deterministic and stochastic policies.
\end{problem}
\begin{hintbox}
Mapping vs. distribution over actions.
\end{hintbox}
\end{exercisebox}


\begin{exercisebox}[easy]
\begin{problem}[Exploration vs. Exploitation]
Give two exploration strategies in RL.
\end{problem}
\begin{hintbox}
$\epsilon$-greedy; UCB; entropy regularisation.
\end{hintbox}
\end{exercisebox}


\subsection*{Medium}

\begin{exercisebox}[medium]
\begin{problem}[Bellman Equation]
Derive the Bellman equation for $Q(s,a)$.
\end{problem}
\begin{hintbox}
Recursive relationship with successor states.
\end{hintbox}
\end{exercisebox}


\begin{exercisebox}[medium]
\begin{problem}[Policy Gradient]
Explain why policy gradient methods are useful for continuous action spaces.
\end{problem}
\begin{hintbox}
Direct parameterisation; differentiability.
\end{hintbox}
\end{exercisebox}


\subsection*{Hard}

\begin{exercisebox}[hard]
\begin{problem}[Actor-Critic Derivation]
Derive the advantage actor-critic update rule.
\end{problem}
\begin{hintbox}
Baseline subtraction; variance reduction.
\end{hintbox}
\end{exercisebox}


\begin{exercisebox}[hard]
\begin{problem}[Off-Policy Correction]
Analyse importance sampling for off-policy learning and its variance.
\end{problem}
\begin{hintbox}
Likelihood ratio; distribution mismatch.
\end{hintbox}
\end{exercisebox}



\begin{exercisebox}[hard]
\begin{problem}[Advanced Topic 1]
Explain a key concept from this chapter and its practical applications.
\end{problem}
\begin{hintbox}
Consider the theoretical foundations and real-world implications.
\end{hintbox}
\end{exercisebox}


\begin{exercisebox}[hard]
\begin{problem}[Advanced Topic 2]
Analyse the relationship between different techniques covered in this chapter.
\end{problem}
\begin{hintbox}
Look for connections and trade-offs between methods.
\end{hintbox}
\end{exercisebox}


\begin{exercisebox}[hard]
\begin{problem}[Advanced Topic 3]
Design an experiment to test a hypothesis related to this chapter's content.
\end{problem}
\begin{hintbox}
Consider experimental design, metrics, and potential confounding factors.
\end{hintbox}
\end{exercisebox}


\begin{exercisebox}[hard]
\begin{problem}[Advanced Topic 4]
Compare different approaches to solving a problem from this chapter.
\end{problem}
\begin{hintbox}
Consider computational complexity, accuracy, and practical considerations.
\end{hintbox}
\end{exercisebox}


\begin{exercisebox}[hard]
\begin{problem}[Advanced Topic 5]
Derive a mathematical relationship or prove a theorem from this chapter.
\end{problem}
\begin{hintbox}
Start with the definitions and work through the logical steps.
\end{hintbox}
\end{exercisebox}


\begin{exercisebox}[hard]
\begin{problem}[Advanced Topic 6]
Implement a practical solution to a problem discussed in this chapter.
\end{problem}
\begin{hintbox}
Consider the implementation details and potential challenges.
\end{hintbox}
\end{exercisebox}


\begin{exercisebox}[hard]
\begin{problem}[Advanced Topic 7]
Evaluate the limitations and potential improvements of techniques from this chapter.
\end{problem}
\begin{hintbox}
Consider both theoretical limitations and practical constraints.
\end{hintbox}
\end{exercisebox}


