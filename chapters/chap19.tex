% Chapter 19: Approximate Inference

\chapter{Approximate Inference}
\label{chap:approximate-inference}

This chapter explores methods for tractable inference in complex probabilistic models.


\section*{Learning Objectives}
\addcontentsline{toc}{section}{Learning Objectives}

After studying this chapter, you will be able to:

\begin{enumerate}
    \item Differentiate variational inference and sampling-based approaches.
    \item Derive ELBO objectives and coordinate ascent updates for simple models.
    \item Understand amortized inference and its benefits/limitations.
    \item Evaluate approximation quality using diagnostics and bounds.
\end{enumerate}



\section*{Intuition}
\addcontentsline{toc}{section}{Intuition}

Exact posteriors are rare. We instead optimize over a family of tractable distributions or draw dependent samples, trading bias and variance to approximate expectations we care about.

For example, when trying to understand a patient's disease risk from their genetic data, we need to integrate information across thousands of genes and environmental factors, which creates an intractable posterior distribution. Like a detective trying to piece together a complex case from scattered evidence, approximate inference methods help us make sense of overwhelming amounts of uncertain information by focusing on the most important patterns and relationships.

Think of approximate inference as using a simplified map instead of a detailed satellite image when navigating a city. While the simplified map might miss some details, it captures the essential information needed to reach your destination efficiently. Similarly, approximate inference methods trade some accuracy for computational efficiency, enabling us to make decisions in complex probabilistic models that would otherwise be impossible to solve exactly.


% Chapter 19, Section 1

\section{Variational Inference \difficultyInline{advanced}}
\label{sec:variational-inference}

Variational inference transforms intractable posterior inference into an optimization problem by approximating the true posterior with a simpler, tractable distribution that can be efficiently optimized.

\subsection{Evidence Lower Bound (ELBO)}

For latent variable model with intractable posterior $p(\vect{z}|\vect{x})$, we approximate with $q(\vect{z})$. The mathematical derivation shows how we can bound the log-evidence:

\begin{align}
\log p(\vect{x}) &= \mathbb{E}_{q(\vect{z})}[\log p(\vect{x})] \\
&= \mathbb{E}_{q(\vect{z})}\left[\log \frac{p(\vect{x}, \vect{z})}{p(\vect{z}|\vect{x})}\right] \\
&= \mathbb{E}_{q(\vect{z})}\left[\log \frac{p(\vect{x}, \vect{z})}{q(\vect{z})}\right] + D_{KL}(q(\vect{z}) \| p(\vect{z}|\vect{x})) \\
&\geq \mathbb{E}_{q(\vect{z})}\left[\log \frac{p(\vect{x}, \vect{z})}{q(\vect{z})}\right] = \mathcal{L}(q)
\end{align}

The key insight is that the KL divergence $D_{KL}(q(\vect{z}) \| p(\vect{z}|\vect{x}))$ is always non-negative, so the ELBO $\mathcal{L}(q)$ provides a lower bound on the log-evidence. The motivation behind this mathematical framework is that maximizing the ELBO simultaneously maximizes the log-evidence and minimizes the KL divergence between our approximation $q(\vect{z})$ and the true posterior $p(\vect{z}|\vect{x})$. This transforms the intractable inference problem into a tractable optimization problem where we can use standard optimization techniques to find the best approximation.

\subsection{Variational Family}

The choice of variational family determines the expressiveness and computational tractability of our approximation. We must balance between capturing the complexity of the true posterior and maintaining computational efficiency.

\textbf{Mean field:} Fully factorized approximation assumes all variables are independent:
\begin{equation}
q(\vect{z}) = \prod_{i=1}^{n} q_i(z_i)
\end{equation}

This equation shows that the joint distribution factors into a product of individual marginals, making computation tractable but potentially missing important dependencies between variables.

\textbf{Structured:} Allow some dependencies by grouping variables into cliques:
\begin{equation}
q(\vect{z}) = \prod_{c} q_c(\vect{z}_c)
\end{equation}

This equation permits dependencies within each clique $c$ while maintaining independence between cliques, providing a middle ground between mean field and full posterior approximation. The trade-off between expressiveness and tractability is fundamental to variational inference, as more expressive families can better approximate the true posterior but require more computational resources.

\subsection{Coordinate Ascent VI}

Coordinate ascent variational inference optimizes each factor of the variational distribution iteratively while keeping all other factors fixed. The update equation for each factor is:

\begin{equation}
q_j^*(z_j) \propto \exp\left(\mathbb{E}_{q_{-j}}[\log p(\vect{z}, \vect{x})]\right)
\end{equation}

This equation shows that the optimal factor $q_j^*(z_j)$ is proportional to the exponential of the expected log-joint probability, where the expectation is taken over all other factors $q_{-j}$. The key insight is that each factor can be optimized independently given the others, making the optimization problem tractable. This approach is guaranteed to converge to a local optimum of the ELBO, providing a principled way to find good approximations to the true posterior.

\subsection{Stochastic Variational Inference}

Stochastic variational inference addresses the scalability limitations of traditional variational inference by using stochastic gradients and mini-batch processing to handle large datasets efficiently. The method enables variational inference to scale to massive datasets by processing only small subsets of data at each iteration, making it practical for modern machine learning applications.

The approach combines mini-batch data processing with Monte Carlo estimation of expectations, allowing the algorithm to work with large datasets without requiring the full dataset to be loaded into memory. The reparameterization trick provides a crucial variance reduction technique that enables stable optimization by expressing the stochastic gradients in a form that has lower variance than naive Monte Carlo estimation. This combination of techniques makes stochastic variational inference the method of choice for large-scale probabilistic modeling, enabling the deployment of sophisticated probabilistic models in production systems.

% \subsection{Visual aids}
% \addcontentsline{toc}{subsubsection}{Visual aids (variational inference)}

% \begin{figure}[h]
%   \centering
%   \begin{tikzpicture}
%     \begin{axis}[
%       width=0.48\textwidth,height=0.36\textwidth,
%       xlabel={Iteration}, ylabel={ELBO}, grid=both]
%       \addplot[bookpurple,very thick] coordinates{(0,-300) (10,-220) (20,-180) (40,-150) (80,-140)};
%     \end{axis}
%   \end{tikzpicture}
%   \caption{ELBO increasing during variational optimization (illustrative).}
%   \label{fig:elbo-trace}
% \end{figure}

% \subsection{Notes and references}

% See \textcite{Bishop2006,GoodfellowEtAl2016,Prince2023} for derivations of the ELBO, mean-field updates, and stochastic VI.


% Chapter 19, Section 2

\section{Mean Field Approximation \difficultyInline{advanced}}
\label{sec:mean-field}

Mean field approximation is the simplest form of variational inference that assumes all latent variables are independent, providing a computationally efficient but potentially limited approximation to complex posterior distributions.

\subsection{Fully Factorized Approximation}

The fully factorized approximation assumes that all latent variables are independent, leading to the factorization:
\begin{equation}
q(\vect{z}) = \prod_{i=1}^{n} q_i(z_i)
\end{equation}

This equation shows that the joint variational distribution factors into a product of individual marginals, where each $q_i(z_i)$ represents the approximate posterior for variable $z_i$. This factorization makes the optimization problem tractable by allowing each factor to be optimized independently, but it may miss important dependencies between variables that exist in the true posterior.

\subsection{Update Equations}

The mean field update equations provide the optimal form for each factor when all other factors are held fixed. For each variable $z_j$, the optimal factor is:

\begin{equation}
\log q_j^*(z_j) = \mathbb{E}_{i \neq j}[\log p(\vect{z}, \vect{x})] + \text{const}
\end{equation}

This equation shows that the log of the optimal factor $q_j^*(z_j)$ is proportional to the expected log-joint probability, where the expectation is taken over all other variables $i \neq j$. The constant term ensures proper normalization. The algorithm iterates these updates until convergence, with each update guaranteed to increase the ELBO, leading to a local optimum of the variational objective.

\subsection{Properties}

Mean field approximation exhibits several important properties that make it both useful and limited in practice. The method tends to underestimate variance, leading to overconfident predictions that may not capture the full uncertainty in the posterior distribution. This bias arises from the independence assumption, which prevents the approximation from capturing correlations between variables that could lead to higher uncertainty estimates.

Despite these limitations, mean field approximation remains computationally efficient and often provides surprisingly good approximations in practice. The efficiency comes from the simple factorization that allows each factor to be optimized independently, making the algorithm scalable to high-dimensional problems. The practical success of mean field methods in many applications demonstrates that the independence assumption, while theoretically limiting, often captures enough of the important structure to be useful for decision-making and prediction tasks.

% \subsection{Visual aids}
% \addcontentsline{toc}{subsubsection}{Visual aids (mean field)}

% \begin{figure}[h]
%   \centering
%   \begin{tikzpicture}
%     \begin{axis}[
%       width=0.48\textwidth,height=0.36\textwidth,
%       xlabel={$z$}, ylabel={Density}, grid=both]
%       \addplot[bookpurple,very thick,domain=-3:3,samples=100]{exp(-0.5*(x^2))};
%       \addplot[bookred,very thick,dashed,domain=-3:3,samples=100]{exp(-0.5*((x/0.7)^2))};
%     \end{axis}
%   \end{tikzpicture}
%   \caption{Mean field (dashed) often underestimates posterior variance (illustrative).}
%   \label{fig:mf-variance}
% \end{figure}

% \subsection{Notes and references}

% For mean-field derivations and limitations, see \textcite{Bishop2006,GoodfellowEtAl2016,Prince2023}.


% Chapter 19, Section 3

\section{Loopy Belief Propagation \difficultyInline{advanced}}
\label{sec:loopy-bp}

\subsection{Message Passing}

On graphical models, pass messages between nodes:
\begin{equation}
m_{i \to j}(x_j) = \sum_{x_i} \psi(x_i, x_j) \psi(x_i) \prod_{k \in N(i) \setminus j} m_{k \to i}(x_i)
\end{equation}

\subsection{Beliefs}

Compute marginals from messages:
\begin{equation}
b_i(x_i) \propto \psi(x_i) \prod_{j \in N(i)} m_{j \to i}(x_i)
\end{equation}

\subsection{Exact on Trees}

For tree-structured graphs, converges to exact marginals.

\subsection{Loopy Graphs}

On graphs with cycles:
\begin{itemize}
    \item May not converge
    \item Often gives good approximations
    \item Used in error-correcting codes, computer vision
\end{itemize}



% \subsection{Visual aids}
% \addcontentsline{toc}{subsubsection}{Visual aids (loopy BP)}

% \begin{figure}[h]
%   \centering
%   \begin{tikzpicture}
%     \begin{axis}[
%       width=0.48\textwidth,height=0.36\textwidth,
%       xlabel={Iteration}, ylabel={Residual}, ymode=log, grid=both]
%       \addplot[bookpurple,very thick] coordinates{(1,1.0) (2,0.5) (3,0.3) (4,0.2) (5,0.18)};
%       \addplot[bookred,very thick,dashed] coordinates{(1,1.0) (2,1.2) (3,1.4) (4,1.7) (5,2.0)};
%     \end{axis}
%   \end{tikzpicture}
%   \caption{Loopy BP: a convergent case (solid) vs. a divergent case (dashed), illustrative.}
%   \label{fig:lbp-conv}
% \end{figure}

% \subsection{Notes and references}

% For background and practical considerations, see \textcite{Bishop2006,GoodfellowEtAl2016,Prince2023}.


% Chapter 19, Section 4

\section{Expectation Propagation \difficultyInline{advanced}}
\label{sec:ep}

Expectation propagation approximates complex probability distributions by replacing each factor with a simpler distribution from an exponential family, enabling tractable inference while preserving important statistical properties. The mathematical foundation is:

\begin{equation}
p(\vect{x}) = \frac{1}{Z} \prod_i f_i(\vect{x}) \approx \frac{1}{Z} \prod_i \tilde{f}_i(\vect{x})
\end{equation}

This equation shows that the true distribution $p(\vect{x})$ is approximated by replacing each complex factor $f_i(\vect{x})$ with a simpler factor $\tilde{f}_i(\vect{x})$ from a tractable family. The algorithm iteratively refines these approximations to match the moments of the true distribution, providing a principled way to approximate complex posteriors. Expectation propagation is particularly effective for multi-modal posteriors where mean field approximation would fail, as it can capture multiple modes by using more expressive approximating families.


% \subsection{Visual aids}
% \addcontentsline{toc}{subsubsection}{Visual aids (EP)}

% \begin{figure}[h]
%   \centering
%   \begin{tikzpicture}
%     \begin{axis}[
%       width=0.48\textwidth,height=0.36\textwidth,
%       xlabel={$x$}, ylabel={Density}, grid=both]
%       \addplot[bookpurple,very thick,domain=-3:3,samples=100]{0.5*exp(-0.5*((x-1)^2)) + 0.5*exp(-0.5*((x+1)^2))};
%       \addplot[bookred,very thick,dashed,domain=-3:3,samples=100]{exp(-0.5*((x)^2))};
%     \end{axis}
%   \end{tikzpicture}
%   \caption{EP (solid) can better match multi-modal targets than simple mean-field Gaussian (dashed), illustrative.}
%   \label{fig:ep-modes}
% \end{figure}

% \subsection{Notes and references}

% See \textcite{Bishop2006,GoodfellowEtAl2016,Prince2023} for EP algorithms and comparisons to mean field.


% Chapter 19: Real World Applications

\section{Real World Applications}
\label{sec:approx-inference-real-world}


Approximate inference makes complex probabilistic reasoning practical. When exact inference is intractable, approximate methods enable deploying sophisticated probabilistic models in real-world systems requiring fast, scalable inference.

\subsection{Autonomous Systems}

Autonomous systems represent one of the most demanding applications of approximate inference, where real-time decision making under uncertainty is essential for safety and performance. These systems must process vast amounts of sensor data, maintain probabilistic beliefs about their environment, and make decisions that could have serious consequences if incorrect.

Robot navigation in uncertain environments exemplifies these challenges, where robots operating in homes or warehouses face sensor noise and unpredictable obstacles that make exact inference impossible. Approximate inference methods like particle filters and variational methods enable real-time localization and mapping despite these uncertainties, allowing robots to continuously update their beliefs about position and surroundings. The robot makes navigation decisions based on approximate posterior distributions computed in milliseconds, demonstrating how approximate inference enables practical autonomous systems that would be impossible with exact methods.

Drone flight control presents similar challenges, where autonomous drones must track their position, velocity, and orientation while compensating for wind and sensor errors. Extended Kalman filters, which are a form of approximate inference, provide real-time state estimation that enables stable flight. This makes applications from package delivery to aerial photography practical, showing how approximate inference techniques can be adapted to different types of autonomous systems with varying requirements and constraints.

\subsection{Personalized Medicine}

Personalized medicine represents a transformative application of approximate inference, where the goal is to tailor treatments to individual patients based on their unique genetic, clinical, and lifestyle factors. This approach requires integrating vast amounts of heterogeneous data to make predictions about disease risk, treatment response, and optimal interventions.

Genomic data analysis exemplifies these challenges, where understanding disease risk from genetic variants requires integrating evidence across thousands of genes with complex interactions. Approximate inference in Bayesian models combines genetic data with clinical information, computing posterior probabilities for disease risk and treatment response that would be impossible to calculate exactly. This enables precision medicine decisions about preventive care and drug selection, demonstrating how approximate inference can handle the complexity of modern genomic medicine while providing actionable insights for patient care.

Real-time patient monitoring in intensive care units presents another critical application, where monitoring systems must track dozens of vital signs and detect deterioration early to prevent adverse events. Approximate inference in hierarchical models captures the normal variation versus concerning trends in patient data, enabling systems to trigger alerts while avoiding false alarms that cause alarm fatigue among medical staff. This balance between sensitivity and specificity is crucial for patient safety and demonstrates how approximate inference can be tailored to specific clinical requirements.

\subsection{Content Recommendation}

Content recommendation systems represent one of the most visible applications of approximate inference, where the challenge is to provide personalized content to billions of users in real-time while handling the massive scale and complexity of modern digital platforms. These systems must balance personalization with exploration, handle new users with minimal history, and provide recommendations that are both relevant and diverse.

Real-time feed ranking exemplifies these challenges, where social media platforms must rank posts for billions of users continuously, requiring inference methods that can scale to massive user bases while capturing uncertainty in preference estimates. Approximate inference in probabilistic models estimates user preferences from sparse interactions, computing rankings in milliseconds using variational methods that enable scaling while maintaining the probabilistic reasoning benefits that make recommendations robust and interpretable.

The explore-exploit tradeoff represents another critical challenge, where recommendation systems must balance showing proven content that users are likely to enjoy versus trying new items that might lead to discovery. Approximate Bayesian inference maintains uncertainty estimates about item quality, enabling the implementation of principled exploration strategies like Thompson sampling that prevent recommendation systems from getting stuck showing only popular content. This balance is essential for maintaining user engagement and discovering new content that users might enjoy.

\subsection{Natural Language Systems}

Natural language systems represent one of the most complex applications of approximate inference, where the goal is to understand and generate human language at scale while handling the inherent ambiguity and complexity of linguistic communication. These systems must process vast amounts of text, maintain probabilistic beliefs about meaning and intent, and provide responses that are both accurate and contextually appropriate.

Document understanding exemplifies these challenges, where extracting structured information from documents like contracts, medical records, and scientific papers involves uncertain entity recognition and relation extraction that requires sophisticated probabilistic reasoning. Approximate inference in structured models provides confidence estimates that enable systems to flag uncertain extractions for human verification while automating clear cases, demonstrating how approximate inference can be integrated into human-AI collaborative workflows.

Conversational AI systems face similar challenges, where chatbots must maintain beliefs about conversation state and user intent through approximate inference that can handle the ambiguity inherent in natural language. This enables systems to handle uncertainty gracefully by asking clarifying questions when uncertain about user goals rather than guessing wrongly, showing how approximate inference can be used to create more robust and user-friendly conversational interfaces.

\subsection{Why Approximation Is Essential}

Approximate inference is not merely a computational convenience but a fundamental necessity for making probabilistic modeling practical in real-world applications. The benefits of approximate inference extend far beyond simple computational efficiency, encompassing scalability, speed, flexibility, and the preservation of uncertainty quantification that makes probabilistic reasoning valuable.

Scalability represents perhaps the most critical benefit, as approximate inference enables the deployment of complex probabilistic models on real-world data sizes that would be impossible to handle with exact methods. This scalability is essential for modern applications that must process massive datasets while maintaining the sophisticated modeling capabilities that make probabilistic approaches valuable. The ability to handle large-scale problems while preserving model complexity is what makes approximate inference indispensable for practical machine learning systems.

Speed is another crucial advantage, as approximate inference provides results fast enough for interactive applications where real-time decision making is essential. This speed enables the deployment of probabilistic models in applications ranging from autonomous systems to recommendation engines, where delays can have serious consequences for user experience or system performance. The combination of speed and accuracy makes approximate inference the method of choice for many practical applications where exact inference would be too slow or impossible.

These applications demonstrate that approximate inference is not a compromise but rather what makes probabilistic modeling practical at scale. The success of approximate inference methods across diverse domains shows that they provide the right balance between theoretical rigor and computational feasibility, enabling the deployment of sophisticated probabilistic models in real-world systems that would otherwise be impossible to build.

% Index entries
\index{applications!autonomous systems}
\index{applications!personalized medicine}
\index{applications!recommendation systems}
\index{approximate inference!applications}


% Chapter summary and problems
% Key Takeaways for Chapter 19

\section*{Key Takeaways}
\addcontentsline{toc}{section}{Key Takeaways}

\begin{keytakeaways}
\begin{itemize}[leftmargin=2em]
    \item \textbf{VI vs. MCMC}: bias-variance trade-offs define suitability.
    \item \textbf{ELBO optimisation} turns inference into tractable learning.
    \item \textbf{Amortisation} speeds inference but can underfit the posterior.
\end{itemize}
\end{keytakeaways}



% Exercises (Exercises) for Chapter 19

\section*{Exercises}
\addcontentsline{toc}{section}{Exercises}

\subsection*{Easy}

\begin{exercisebox}[easy]
\begin{problem}[Why Approximate?]
Explain why exact inference is intractable in many models.
\end{problem}
\begin{hintbox}
Partition function; high-dimensional integration.
\end{hintbox}
\end{exercisebox}


\begin{exercisebox}[easy]
\begin{problem}[ELBO Connection]
Relate ELBO to KL divergence between $q$ and $p$.
\end{problem}
\begin{hintbox}
$\log p(x) = \text{ELBO} + D_{KL}(q||p)$.
\end{hintbox}
\end{exercisebox}


\begin{exercisebox}[easy]
\begin{problem}[Mean-Field Assumption]
State the mean-field independence assumption.
\end{problem}
\begin{hintbox}
Factored variational distribution.
\end{hintbox}
\end{exercisebox}


\begin{exercisebox}[easy]
\begin{problem}[MCMC vs. VI]
Compare MCMC and variational inference trade-offs.
\end{problem}
\begin{hintbox}
Asymptotic exactness vs. computational speed.
\end{hintbox}
\end{exercisebox}


\subsection*{Medium}

\begin{exercisebox}[medium]
\begin{problem}[Coordinate Ascent VI]
Derive the coordinate ascent update for a simple model.
\end{problem}
\begin{hintbox}
Fix all but one factor; optimise w.r.t. remaining factor.
\end{hintbox}
\end{exercisebox}


\begin{exercisebox}[medium]
\begin{problem}[Importance Sampling]
Explain how importance sampling estimates expectations.
\end{problem}
\begin{hintbox}
Reweight samples from proposal distribution.
\end{hintbox}
\end{exercisebox}


\subsection*{Hard}

\begin{exercisebox}[hard]
\begin{problem}[Amortised Inference]
Analyse the trade-offs of amortised inference in VAEs.
\end{problem}
\begin{hintbox}
Amortisation gap; scalability.
\end{hintbox}
\end{exercisebox}


\begin{exercisebox}[hard]
\begin{problem}[Reparameterisation Gradients]
Derive the reparameterisation gradient for a Gaussian variational distribution.
\end{problem}
\begin{hintbox}
$z = \mu + \sigma \epsilon$ where $\epsilon \sim \mathcal{N}(0,1)$.
\end{hintbox}
\end{exercisebox}



\begin{exercisebox}[hard]
\begin{problem}[Advanced Topic 1]
Explain a key concept from this chapter and its practical applications.
\end{problem}
\begin{hintbox}
Consider the theoretical foundations and real-world implications.
\end{hintbox}
\end{exercisebox}


\begin{exercisebox}[hard]
\begin{problem}[Advanced Topic 2]
Analyse the relationship between different techniques covered in this chapter.
\end{problem}
\begin{hintbox}
Look for connections and trade-offs between methods.
\end{hintbox}
\end{exercisebox}


\begin{exercisebox}[hard]
\begin{problem}[Advanced Topic 3]
Design an experiment to test a hypothesis related to this chapter's content.
\end{problem}
\begin{hintbox}
Consider experimental design, metrics, and potential confounding factors.
\end{hintbox}
\end{exercisebox}


\begin{exercisebox}[hard]
\begin{problem}[Advanced Topic 4]
Compare different approaches to solving a problem from this chapter.
\end{problem}
\begin{hintbox}
Consider computational complexity, accuracy, and practical considerations.
\end{hintbox}
\end{exercisebox}


\begin{exercisebox}[hard]
\begin{problem}[Advanced Topic 5]
Derive a mathematical relationship or prove a theorem from this chapter.
\end{problem}
\begin{hintbox}
Start with the definitions and work through the logical steps.
\end{hintbox}
\end{exercisebox}


\begin{exercisebox}[hard]
\begin{problem}[Advanced Topic 6]
Implement a practical solution to a problem discussed in this chapter.
\end{problem}
\begin{hintbox}
Consider the implementation details and potential challenges.
\end{hintbox}
\end{exercisebox}


\begin{exercisebox}[hard]
\begin{problem}[Advanced Topic 7]
Evaluate the limitations and potential improvements of techniques from this chapter.
\end{problem}
\begin{hintbox}
Consider both theoretical limitations and practical constraints.
\end{hintbox}
\end{exercisebox}


