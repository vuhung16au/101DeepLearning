% Chapter 19: Approximate Inference

\chapter{Approximate Inference}
\label{chap:approximate-inference}

This chapter explores methods for tractable inference in complex probabilistic models.


\section*{Learning Objectives}
\addcontentsline{toc}{section}{Learning Objectives}

After studying this chapter, you will be able to:

\begin{enumerate}
    \item Differentiate variational inference and sampling-based approaches.
    \item Derive ELBO objectives and coordinate ascent updates for simple models.
    \item Understand amortized inference and its benefits/limitations.
    \item Evaluate approximation quality using diagnostics and bounds.
\end{enumerate}



\section*{Intuition}
\addcontentsline{toc}{section}{Intuition}

Exact posteriors are rare. We instead optimize over a family of tractable distributions or draw dependent samples, trading bias and variance to approximate expectations we care about.

For example, when trying to understand a patient's disease risk from their genetic data, we need to integrate information across thousands of genes and environmental factors, which creates an intractable posterior distribution. Like a detective trying to piece together a complex case from scattered evidence, approximate inference methods help us make sense of overwhelming amounts of uncertain information by focusing on the most important patterns and relationships.

Think of approximate inference as using a simplified map instead of a detailed satellite image when navigating a city. While the simplified map might miss some details, it captures the essential information needed to reach your destination efficiently. Similarly, approximate inference methods trade some accuracy for computational efficiency, enabling us to make decisions in complex probabilistic models that would otherwise be impossible to solve exactly.


% Chapter 19, Section 1

\section{Variational Inference \difficultyInline{advanced}}
\label{sec:variational-inference}

Variational inference transforms intractable posterior inference into an optimization problem by approximating the true posterior with a simpler, tractable distribution that can be efficiently optimized.

\subsection{Evidence Lower Bound (ELBO)}

For latent variable model with intractable posterior $p(\vect{z}|\vect{x})$, we approximate with $q(\vect{z})$. The mathematical derivation shows how we can bound the log-evidence:

\begin{align}
\log p(\vect{x}) &= \mathbb{E}_{q(\vect{z})}[\log p(\vect{x})] \\
&= \mathbb{E}_{q(\vect{z})}\left[\log \frac{p(\vect{x}, \vect{z})}{p(\vect{z}|\vect{x})}\right] \\
&= \mathbb{E}_{q(\vect{z})}\left[\log \frac{p(\vect{x}, \vect{z})}{q(\vect{z})}\right] + D_{KL}(q(\vect{z}) \| p(\vect{z}|\vect{x})) \\
&\geq \mathbb{E}_{q(\vect{z})}\left[\log \frac{p(\vect{x}, \vect{z})}{q(\vect{z})}\right] = \mathcal{L}(q)
\end{align}

The key insight is that the KL divergence $D_{KL}(q(\vect{z}) \| p(\vect{z}|\vect{x}))$ is always non-negative, so the ELBO $\mathcal{L}(q)$ provides a lower bound on the log-evidence. The motivation behind this mathematical framework is that maximizing the ELBO simultaneously maximizes the log-evidence and minimizes the KL divergence between our approximation $q(\vect{z})$ and the true posterior $p(\vect{z}|\vect{x})$. This transforms the intractable inference problem into a tractable optimization problem where we can use standard optimization techniques to find the best approximation.

\subsection{Variational Family}

The choice of variational family determines the expressiveness and computational tractability of our approximation. We must balance between capturing the complexity of the true posterior and maintaining computational efficiency.

\textbf{Mean field:} Fully factorized approximation assumes all variables are independent:
\begin{equation}
q(\vect{z}) = \prod_{i=1}^{n} q_i(z_i)
\end{equation}

This equation shows that the joint distribution factors into a product of individual marginals, making computation tractable but potentially missing important dependencies between variables.

\textbf{Structured:} Allow some dependencies by grouping variables into cliques:
\begin{equation}
q(\vect{z}) = \prod_{c} q_c(\vect{z}_c)
\end{equation}

This equation permits dependencies within each clique $c$ while maintaining independence between cliques, providing a middle ground between mean field and full posterior approximation. The trade-off between expressiveness and tractability is fundamental to variational inference, as more expressive families can better approximate the true posterior but require more computational resources.

\subsection{Coordinate Ascent VI}

Coordinate ascent variational inference optimizes each factor of the variational distribution iteratively while keeping all other factors fixed. The update equation for each factor is:

\begin{equation}
q_j^*(z_j) \propto \exp\left(\mathbb{E}_{q_{-j}}[\log p(\vect{z}, \vect{x})]\right)
\end{equation}

This equation shows that the optimal factor $q_j^*(z_j)$ is proportional to the exponential of the expected log-joint probability, where the expectation is taken over all other factors $q_{-j}$. The key insight is that each factor can be optimized independently given the others, making the optimization problem tractable. This approach is guaranteed to converge to a local optimum of the ELBO, providing a principled way to find good approximations to the true posterior.

\subsection{Stochastic Variational Inference}

Stochastic variational inference addresses the scalability limitations of traditional variational inference by using stochastic gradients and mini-batch processing to handle large datasets efficiently. The method enables variational inference to scale to massive datasets by processing only small subsets of data at each iteration, making it practical for modern machine learning applications.

The approach combines mini-batch data processing with Monte Carlo estimation of expectations, allowing the algorithm to work with large datasets without requiring the full dataset to be loaded into memory. The reparameterization trick provides a crucial variance reduction technique that enables stable optimization by expressing the stochastic gradients in a form that has lower variance than naive Monte Carlo estimation. This combination of techniques makes stochastic variational inference the method of choice for large-scale probabilistic modeling, enabling the deployment of sophisticated probabilistic models in production systems.

% \subsection{Visual aids}
% \addcontentsline{toc}{subsubsection}{Visual aids (variational inference)}

% \begin{figure}[h]
%   \centering
%   \begin{tikzpicture}
%     \begin{axis}[
%       width=0.48\textwidth,height=0.36\textwidth,
%       xlabel={Iteration}, ylabel={ELBO}, grid=both]
%       \addplot[bookpurple,very thick] coordinates{(0,-300) (10,-220) (20,-180) (40,-150) (80,-140)};
%     \end{axis}
%   \end{tikzpicture}
%   \caption{ELBO increasing during variational optimization (illustrative).}
%   \label{fig:elbo-trace}
% \end{figure}

% \subsection{Notes and references}

% See \textcite{Bishop2006,GoodfellowEtAl2016,Prince2023} for derivations of the ELBO, mean-field updates, and stochastic VI.


% Chapter 19, Section 2

\section{Mean Field Approximation \difficultyInline{advanced}}
\label{sec:mean-field}

\subsection{Fully Factorized Approximation}

Assume all variables independent:
\begin{equation}
q(\vect{z}) = \prod_{i=1}^{n} q_i(z_i)
\end{equation}

\subsection{Update Equations}

For each variable:
\begin{equation}
\log q_j^*(z_j) = \mathbb{E}_{i \neq j}[\log p(\vect{z}, \vect{x})] + \text{const}
\end{equation}

Iterate until convergence.

\subsection{Properties}

\begin{itemize}
    \item Underestimates variance (overconfident)
    \item Computationally efficient
    \item Often good approximation in practice
\end{itemize}

% \subsection{Visual aids}
% \addcontentsline{toc}{subsubsection}{Visual aids (mean field)}

% \begin{figure}[h]
%   \centering
%   \begin{tikzpicture}
%     \begin{axis}[
%       width=0.48\textwidth,height=0.36\textwidth,
%       xlabel={$z$}, ylabel={Density}, grid=both]
%       \addplot[bookpurple,very thick,domain=-3:3,samples=100]{exp(-0.5*(x^2))};
%       \addplot[bookred,very thick,dashed,domain=-3:3,samples=100]{exp(-0.5*((x/0.7)^2))};
%     \end{axis}
%   \end{tikzpicture}
%   \caption{Mean field (dashed) often underestimates posterior variance (illustrative).}
%   \label{fig:mf-variance}
% \end{figure}

% \subsection{Notes and references}

% For mean-field derivations and limitations, see \textcite{Bishop2006,GoodfellowEtAl2016,Prince2023}.


% Chapter 19, Section 3

\section{Loopy Belief Propagation \difficultyInline{advanced}}
\label{sec:loopy-bp}

Loopy belief propagation extends the exact message passing algorithm from trees to general graphs with cycles, providing an approximate inference method that often works well despite lacking convergence guarantees.

\subsection{Message Passing}

Message passing algorithms compute marginal probabilities by iteratively passing messages between nodes in a graphical model. The message from node $i$ to node $j$ is computed as:

\begin{equation}
m_{i \to j}(x_j) = \sum_{x_i} \psi(x_i, x_j) \psi(x_i) \prod_{k \in N(i) \setminus j} m_{k \to i}(x_i)
\end{equation}

This equation shows that the message $m_{i \to j}(x_j)$ combines the local potential $\psi(x_i, x_j)$ between nodes $i$ and $j$, the node potential $\psi(x_i)$ at node $i$, and all incoming messages from other neighbors $k \in N(i) \setminus j$. The algorithm iteratively updates these messages until convergence, providing an efficient way to compute approximate marginals in complex graphical models.

\subsection{Beliefs}

The beliefs represent the approximate marginal probabilities for each variable, computed from the incoming messages:

\begin{equation}
b_i(x_i) \propto \psi(x_i) \prod_{j \in N(i)} m_{j \to i}(x_i)
\end{equation}

This equation shows that the belief $b_i(x_i)$ at node $i$ is proportional to the product of the local node potential $\psi(x_i)$ and all incoming messages $m_{j \to i}(x_i)$ from neighboring nodes $j \in N(i)$. The beliefs provide the final approximation to the marginal probabilities after the message passing algorithm has converged, serving as the output of the loopy belief propagation algorithm.

\subsection{Exact on Trees}

For tree-structured graphs, belief propagation converges to exact marginals in a finite number of iterations. The key insight is that trees have no cycles, so each message is computed exactly once during the forward and backward passes of the algorithm. The mathematical foundation relies on the fact that the joint distribution factors according to the tree structure, and the message passing equations correspond exactly to the marginalization operations needed to compute the true marginals. This makes belief propagation on trees both exact and efficient, with computational complexity linear in the number of nodes.

\subsection{Loopy Graphs}

When applied to graphs with cycles, loopy belief propagation faces several challenges that distinguish it from the exact algorithm on trees. The presence of cycles means that messages can circulate indefinitely, potentially preventing convergence to a fixed point. Despite this theoretical limitation, the algorithm often provides surprisingly good approximations in practice, making it a valuable tool for approximate inference in complex graphical models.

The success of loopy belief propagation in applications like error-correcting codes and computer vision demonstrates its practical utility, even when theoretical convergence guarantees are absent. The algorithm's ability to capture local dependencies and propagate information through the graph structure makes it particularly effective for problems where the true posterior has complex dependencies that would be difficult to capture with simpler approximation methods. This practical success has made loopy belief propagation one of the most widely used approximate inference methods in machine learning and computer vision applications.



% \subsection{Visual aids}
% \addcontentsline{toc}{subsubsection}{Visual aids (loopy BP)}

% \begin{figure}[h]
%   \centering
%   \begin{tikzpicture}
%     \begin{axis}[
%       width=0.48\textwidth,height=0.36\textwidth,
%       xlabel={Iteration}, ylabel={Residual}, ymode=log, grid=both]
%       \addplot[bookpurple,very thick] coordinates{(1,1.0) (2,0.5) (3,0.3) (4,0.2) (5,0.18)};
%       \addplot[bookred,very thick,dashed] coordinates{(1,1.0) (2,1.2) (3,1.4) (4,1.7) (5,2.0)};
%     \end{axis}
%   \end{tikzpicture}
%   \caption{Loopy BP: a convergent case (solid) vs. a divergent case (dashed), illustrative.}
%   \label{fig:lbp-conv}
% \end{figure}

% \subsection{Notes and references}

% For background and practical considerations, see \textcite{Bishop2006,GoodfellowEtAl2016,Prince2023}.


% Chapter 19, Section 4

\section{Expectation Propagation \difficultyInline{advanced}}
\label{sec:ep}

Approximates each factor with simpler distribution:
\begin{equation}
p(\vect{x}) = \frac{1}{Z} \prod_i f_i(\vect{x}) \approx \frac{1}{Z} \prod_i \tilde{f}_i(\vect{x})
\end{equation}

Iteratively refine approximations to match moments.

Better than mean field for multi-modal posteriors.


% \subsection{Visual aids}
% \addcontentsline{toc}{subsubsection}{Visual aids (EP)}

% \begin{figure}[h]
%   \centering
%   \begin{tikzpicture}
%     \begin{axis}[
%       width=0.48\textwidth,height=0.36\textwidth,
%       xlabel={$x$}, ylabel={Density}, grid=both]
%       \addplot[bookpurple,very thick,domain=-3:3,samples=100]{0.5*exp(-0.5*((x-1)^2)) + 0.5*exp(-0.5*((x+1)^2))};
%       \addplot[bookred,very thick,dashed,domain=-3:3,samples=100]{exp(-0.5*((x)^2))};
%     \end{axis}
%   \end{tikzpicture}
%   \caption{EP (solid) can better match multi-modal targets than simple mean-field Gaussian (dashed), illustrative.}
%   \label{fig:ep-modes}
% \end{figure}

% \subsection{Notes and references}

% See \textcite{Bishop2006,GoodfellowEtAl2016,Prince2023} for EP algorithms and comparisons to mean field.


% Chapter 19: Real World Applications

\section{Real World Applications}
\label{sec:approx-inference-real-world}


Approximate inference makes complex probabilistic reasoning practical. When exact inference is intractable, approximate methods enable deploying sophisticated probabilistic models in real-world systems requiring fast, scalable inference.

\subsection{Autonomous Systems}

Real-time decision making under uncertainty:

\begin{itemize}
    \item \textbf{Robot navigation in uncertain environments:} Robots operating in homes or warehouses face sensor noise and unpredictable obstacles. Approximate inference (particle filters, variational methods) enables real-time localization and mapping despite uncertainties. The robot continuously updates beliefs about its position and surroundings, making navigation decisions based on approximate posterior distributions computed in milliseconds.
    
    \item \textbf{Drone flight control:} Autonomous drones must track their position, velocity, and orientation while compensating for wind and sensor errors. Extended Kalman filters—a form of approximate inference—provide real-time state estimation enabling stable flight. This makes applications from package delivery to aerial photography practical.
    
    \item \textbf{Agricultural robots:} Farm robots use approximate inference to model crop health, soil conditions, and pest distributions from noisy sensor data. Variational inference enables processing data from multiple robots, building probabilistic maps guiding precision agriculture interventions like targeted watering or pesticide application.
\end{itemize}

\subsection{Personalized Medicine}

Tailoring treatment to individual patients:

\begin{itemize}
    \item \textbf{Genomic data analysis:} Understanding disease risk from genetic variants requires integrating evidence across thousands of genes. Approximate inference in Bayesian models combines genetic data with clinical information, computing posterior probabilities for disease risk and treatment response. This enables precision medicine decisions about preventive care and drug selection.
    
    \item \textbf{Real-time patient monitoring:} ICU monitoring systems track dozens of vital signs, detecting deterioration early. Approximate inference in hierarchical models captures normal variation versus concerning trends, triggering alerts while avoiding false alarms that cause alarm fatigue among medical staff.
    
    \item \textbf{Cancer treatment optimization:} Tumor evolution models use approximate inference to predict how cancers respond to treatments and develop resistance. These predictions help oncologists select treatment sequences maximizing long-term outcomes rather than just immediate tumor reduction.
\end{itemize}

\subsection{Content Recommendation}

Personalization at massive scale:

\begin{itemize}
    \item \textbf{Real-time feed ranking:} Social media platforms rank posts for billions of users continuously. Approximate inference in probabilistic models estimates user preferences from sparse interactions, computing rankings in milliseconds. Variational methods enable scaling to massive user bases while capturing uncertainty in preference estimates.
    
    \item \textbf{Explore-exploit tradeoffs:} Recommendation systems balance showing proven content (exploit) versus trying new items (explore). Approximate Bayesian inference maintains uncertainty estimates about item quality, implementing principled exploration strategies like Thompson sampling. This prevents recommendation systems from getting stuck showing only popular content.
    
    \item \textbf{Cold start recommendations:} New users have minimal history. Approximate inference in hierarchical models shares information across users, providing reasonable recommendations immediately. As users interact, the system refines individual preference estimates through ongoing approximate inference.
\end{itemize}

\subsection{Natural Language Systems}

Understanding language at scale:

\begin{itemize}
    \item \textbf{Document understanding:} Extracting structured information from documents (contracts, medical records, scientific papers) involves uncertain entity recognition and relation extraction. Approximate inference in structured models provides confidence estimates, flagging uncertain extractions for human verification while automating clear cases.
    
    \item \textbf{Conversational AI:} Chatbots maintain beliefs about conversation state and user intent through approximate inference. This handles ambiguity gracefully—when uncertain about user goals, systems ask clarifying questions rather than guessing wrongly.
    
    \item \textbf{Machine translation:} Modern translation uses approximate inference to explore possible translations efficiently. Beam search—a form of approximate inference—enables finding high-quality translations without exhaustively evaluating all possibilities.
\end{itemize}

\subsection{Why Approximation Is Essential}

Practical benefits of approximate inference:
\begin{itemize}
    \item \textbf{Scalability:} Handle complex models on real-world data sizes
    \item \textbf{Speed:} Provide results fast enough for interactive applications
    \item \textbf{Flexibility:} Enable sophisticated models despite computational constraints
    \item \textbf{Uncertainty:} Maintain probabilistic reasoning benefits with practical efficiency
\end{itemize}

These applications demonstrate that approximate inference is not a compromise—it's what makes probabilistic modeling practical at scale.

% Index entries
\index{applications!autonomous systems}
\index{applications!personalized medicine}
\index{applications!recommendation systems}
\index{approximate inference!applications}


% Chapter summary and problems
% Key Takeaways for Chapter 19

\section*{Key Takeaways}
\addcontentsline{toc}{section}{Key Takeaways}

\begin{keytakeaways}
\begin{itemize}[leftmargin=2em]
    \item \textbf{VI vs. MCMC}: bias-variance trade-offs define suitability.
    \item \textbf{ELBO optimisation} turns inference into tractable learning.
    \item \textbf{Amortisation} speeds inference but can underfit the posterior.
\end{itemize}
\end{keytakeaways}



% Exercises (Exercises) for Chapter 19

\section*{Exercises}
\addcontentsline{toc}{section}{Exercises}

\subsection*{Easy}

\begin{problem}[Why Approximate?]
Explain why exact inference is intractable in many models.

\textbf{Hint:} Partition function; high-dimensional integration.
\end{problem}

\begin{problem}[ELBO Connection]
Relate ELBO to KL divergence between $q$ and $p$.

\textbf{Hint:} $\log p(x) = \text{ELBO} + D_{KL}(q||p)$.
\end{problem}

\begin{problem}[Mean-Field Assumption]
State the mean-field independence assumption.

\textbf{Hint:} Factored variational distribution.
\end{problem}

\begin{problem}[MCMC vs. VI]
Compare MCMC and variational inference trade-offs.

\textbf{Hint:} Asymptotic exactness vs. computational speed.
\end{problem}

\subsection*{Medium}

\begin{problem}[Coordinate Ascent VI]
Derive the coordinate ascent update for a simple model.

\textbf{Hint:} Fix all but one factor; optimise w.r.t. remaining factor.
\end{problem}

\begin{problem}[Importance Sampling]
Explain how importance sampling estimates expectations.

\textbf{Hint:} Reweight samples from proposal distribution.
\end{problem}

\subsection*{Hard}

\begin{problem}[Amortised Inference]
Analyse the trade-offs of amortised inference in VAEs.

\textbf{Hint:} Amortisation gap; scalability.
\end{problem}

\begin{problem}[Reparameterisation Gradients]
Derive the reparameterisation gradient for a Gaussian variational distribution.

\textbf{Hint:} $z = \mu + \sigma \epsilon$ where $\epsilon \sim \mathcal{N}(0,1)$.
\end{problem}


\begin{problem}[Advanced Topic 1]
Explain a key concept from this chapter and its practical applications.

\textbf{Hint:} Consider the theoretical foundations and real-world implications.
\end{problem}

\begin{problem}[Advanced Topic 2]
Analyse the relationship between different techniques covered in this chapter.

\textbf{Hint:} Look for connections and trade-offs between methods.
\end{problem}

\begin{problem}[Advanced Topic 3]
Design an experiment to test a hypothesis related to this chapter's content.

\textbf{Hint:} Consider experimental design, metrics, and potential confounding factors.
\end{problem}

\begin{problem}[Advanced Topic 4]
Compare different approaches to solving a problem from this chapter.

\textbf{Hint:} Consider computational complexity, accuracy, and practical considerations.
\end{problem}

\begin{problem}[Advanced Topic 5]
Derive a mathematical relationship or prove a theorem from this chapter.

\textbf{Hint:} Start with the definitions and work through the logical steps.
\end{problem}

\begin{problem}[Advanced Topic 6]
Implement a practical solution to a problem discussed in this chapter.

\textbf{Hint:} Consider the implementation details and potential challenges.
\end{problem}

\begin{problem}[Advanced Topic 7]
Evaluate the limitations and potential improvements of techniques from this chapter.

\textbf{Hint:} Consider both theoretical limitations and practical constraints.
\end{problem}

