% Exercises (Hands-On Exercises) for Chapter 9: Convolutional Networks

\section*{Exercises}
\addcontentsline{toc}{section}{Exercises}

\subsection*{Easy}

\begin{exercisebox}[easy]
\begin{problem}[Receptive Field Calculation]
A CNN has two convolutional layers with 3×3 kernels (no padding, stride 1). Calculate the receptive field of a neuron in the second layer.
\end{problem}
\begin{hintbox}
Each layer expands the receptive field. For the second layer, consider how many input pixels affect it.
\end{hintbox}
\end{exercisebox}


\begin{exercisebox}[easy]
\begin{problem}[Parameter Counting]
Calculate the number of parameters in a convolutional layer with 64 input channels, 128 output channels, and 3×3 kernels (including bias).
\end{problem}
\begin{hintbox}
Each output channel has a 3×3 kernel for each input channel, plus one bias term.
\end{hintbox}
\end{exercisebox}


\begin{exercisebox}[easy]
\begin{problem}[Pooling Operations]
Explain the difference between max pooling and average pooling. When would you prefer one over the other?
\end{problem}
\begin{hintbox}
Consider feature prominence, spatial information retention, and gradient flow.
\end{hintbox}
\end{exercisebox}


\begin{exercisebox}[easy]
\begin{problem}[Translation Equivariance]
Explain what translation equivariance means in the context of CNNs and why it is a desirable property for image processing.
\end{problem}
\begin{hintbox}
If the input is shifted, how does the output change? Consider the relationship $f(T(x)) = T(f(x))$.
\end{hintbox}
\end{exercisebox}


\subsection*{Medium}

\begin{exercisebox}[medium]
\begin{problem}[Output Shape Calculation]
Given an input image of size 224×224×3, apply the following operations and calculate the output shape at each step:
\begin{enumerate}
    \item Conv2D: 64 filters, 7×7 kernel, stride 2, padding 3
    \item MaxPool2D: 3×3, stride 2
    \item Conv2D: 128 filters, 3×3 kernel, stride 1, padding 1
\end{enumerate}
\end{problem}
\begin{hintbox}
Use the formula: $\text{output\_size} = \lfloor \frac{\text{input\_size} + 2 \times \text{padding} - \text{kernel\_size}}{\text{stride}} \rfloor + 1$.
\end{hintbox}
\end{exercisebox}


\begin{exercisebox}[medium]
\begin{problem}[ResNet Skip Connections]
Explain why residual connections (skip connections) help train very deep networks. Discuss the gradient flow through skip connections.
\end{problem}
\begin{hintbox}
Consider the identity mapping $\vect{y} = \vect{x} + F(\vect{x})$ and compute $\frac{\partial \vect{y}}{\partial \vect{x}}$.
\end{hintbox}
\end{exercisebox}


\subsection*{Hard}

\begin{exercisebox}[hard]
\begin{problem}[Dilated Convolutions]
Derive the receptive field for a stack of dilated convolutions with dilation rates [1, 2, 4, 8]. Compare computational cost with standard convolutions achieving the same receptive field.
\end{problem}
\begin{hintbox}
Dilated convolution with rate $r$ introduces $(r-1)$ gaps between kernel elements. Track receptive field growth layer by layer.
\end{hintbox}
\end{exercisebox}


\begin{exercisebox}[hard]
\begin{problem}[Depthwise Separable Convolutions]
Analyse the computational savings of depthwise separable convolutions (as used in MobileNets) compared to standard convolutions. Derive the reduction factor for a layer with $C_{in}$ input channels, $C_{out}$ output channels, and $K×K$ kernel size.
\end{problem}
\begin{hintbox}
Depthwise separable splits into depthwise ($C_{in}$ groups) and pointwise (1×1) convolutions. Compare FLOPs.
\end{hintbox}
\end{exercisebox}


\begin{exercisebox}[hard]
\begin{problem}[Convolutional Layer Design]
Design a convolutional layer architecture for a specific computer vision task. Justify your choices of kernel size, stride, and padding.
\end{problem}
\begin{hintbox}
Consider the trade-off between computational efficiency and feature extraction capability.
\end{hintbox}
\end{exercisebox}


\begin{exercisebox}[hard]
\begin{problem}[Pooling Operations]
Compare different pooling operations (max, average, L2) and their effects on feature maps.
\end{problem}
\begin{hintbox}
Consider the impact on spatial information, gradient flow, and computational efficiency.
\end{hintbox}
\end{exercisebox}


\begin{exercisebox}[hard]
\begin{problem}[Feature Map Visualization]
Explain how to visualize and interpret feature maps in different layers of a CNN.
\end{problem}
\begin{hintbox}
Consider activation maximization, gradient-based methods, and occlusion analysis.
\end{hintbox}
\end{exercisebox}


\begin{exercisebox}[hard]
\begin{problem}[CNN Architecture Search]
Design an automated method for finding optimal CNN architectures for a given dataset.
\end{problem}
\begin{hintbox}
Consider neural architecture search (NAS), evolutionary algorithms, and reinforcement learning approaches.
\end{hintbox}
\end{exercisebox}


\begin{exercisebox}[hard]
\begin{problem}[Transfer Learning Strategies]
Compare different transfer learning approaches for CNNs and when to use each.
\end{problem}
\begin{hintbox}
Consider feature extraction, fine-tuning, and progressive unfreezing strategies.
\end{hintbox}
\end{exercisebox}


\begin{exercisebox}[hard]
\begin{problem}[CNN Interpretability]
Explain methods for understanding and interpreting CNN decisions, including attention mechanisms.
\end{problem}
\begin{hintbox}
Consider gradient-based attribution methods, attention maps, and adversarial analysis.
\end{hintbox}
\end{exercisebox}


