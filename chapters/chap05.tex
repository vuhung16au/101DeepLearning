% Chapter 5: Classical Machine Learning Algorithms

\chapter{Classical Machine Learning Algorithms}
\label{chap:classical-ml}

This chapter reviews traditional machine learning methods that provide context and motivation for deep learning approaches. Understanding these classical algorithms helps appreciate the advantages and innovations of deep learning.

\section*{Learning Objectives}
\addcontentsline{toc}{section}{Learning Objectives}

After studying this chapter, you will be able to:

\begin{enumerate}
    \item \textbf{Understand the mathematical foundations} of classical machine learning algorithms including linear regression, logistic regression, and support vector machines
    \item \textbf{Compare and contrast} different approaches to classification and regression problems
    \item \textbf{Implement and optimize} classical algorithms using both closed-form solutions and iterative methods
    \item \textbf{Apply ensemble methods} like random forests and gradient boosting to improve model performance
    \item \textbf{Evaluate the trade-offs} between classical methods and deep learning approaches
    \item \textbf{Choose appropriate algorithms} based on dataset characteristics, computational constraints, and interpretability requirements
    \item \textbf{Understand the limitations} of classical methods that motivated the development of deep learning
    \item \textbf{Apply regularization techniques} to prevent overfitting in classical machine learning models
\end{enumerate}

 This chapter assumes familiarity with linear algebra, probability theory, and basic optimization concepts from previous chapters.

% Chapter 5, Section 01

\section{Linear Regression \difficultyInline{intermediate}}
\label{sec:linear-regression}

\textbf{Linear regression} is one of the most fundamental and widely-used machine learning algorithms. It models the relationship between input features and a continuous output by finding the best linear function that minimizes prediction errors.

\subsection{Intuition and Motivation}

Imagine you're trying to predict house prices based on features like size, number of bedrooms, and location. Linear regression assumes that the price can be expressed as a weighted sum of these features plus a base price (bias). The algorithm learns the optimal weights that best explain the relationship between features and prices in your training data.

The key insight is that linear relationships are often sufficient for many real-world problems, and they have several advantages:
\begin{itemize}
    \item \textbf{Interpretability:} Each weight tells us how much the output changes when a feature increases by one unit
    \item \textbf{Computational efficiency:} Fast training and prediction
    \item \textbf{Statistical properties:} Well-understood theoretical guarantees
\end{itemize}

\begin{figure}[htbp]
\centering
\begin{tikzpicture}[scale=0.8]
\begin{axis}[
    xlabel={House Size (sq ft)},
    ylabel={Price (\$1000s)},
    title={Linear Regression Example: House Price Prediction},
    grid=major,
    width=10cm,
    height=6cm
]
% Generate some sample data points
\addplot[only marks, mark=*, mark size=2pt, color=bookpurple] coordinates {
    (1000, 200) (1200, 250) (1400, 300) (1600, 350) (1800, 400)
    (2000, 450) (2200, 500) (2400, 550) (2600, 600) (2800, 650)
};
% Add regression line
\addplot[thick, color=bookred, domain=800:3000] {0.2*x + 50};
\node at (axis cs:2500,400) [anchor=west] {$\hat{y} = 0.2x + 50$};
\end{axis}
\end{tikzpicture}
\caption{Linear regression finds the best line that fits the data points, minimizing the sum of squared errors.}
\label{fig:linear-regression-example}
\end{figure}

\subsection{Model Formulation}

For input $\vect{x} \in \mathbb{R}^d$ and output $y \in \mathbb{R}$, linear regression models the relationship as:

\begin{equation}
\hat{y} = \vect{w}^\top \vect{x} + b
\end{equation}

where:
\begin{itemize}
    \item $\vect{w} \in \mathbb{R}^d$ are the \textbf{weights} (regression coefficients)
    \item $b \in \mathbb{R}$ is the \textbf{bias} (intercept term)
    \item $\hat{y}$ is the predicted output
\end{itemize}

\subsection{Ordinary Least Squares}

The goal is to find parameters that minimize the prediction error. We use the \textbf{mean squared error} (MSE) as our loss function:

\begin{equation}
L(\vect{w}, b) = \frac{1}{n} \sum_{i=1}^{n} (y^{(i)} - \hat{y}^{(i)})^2 = \frac{1}{n} \sum_{i=1}^{n} (y^{(i)} - \vect{w}^\top \vect{x}^{(i)} - b)^2
\end{equation}

\subsubsection{Matrix Formulation}

For computational efficiency, we can absorb the bias into the weight vector by adding a constant feature of 1 to each input. Let $\mat{X} \in \mathbb{R}^{n \times (d+1)}$ be the design matrix with an additional column of ones, and $\vect{w} \in \mathbb{R}^{d+1}$ include the bias term.

The closed-form solution (normal equation) is:

\begin{equation}
\vect{w}^* = (\mat{X}^\top \mat{X})^{-1} \mat{X}^\top \vect{y}
\end{equation}

\begin{remark}
The normal equation requires $\mat{X}^\top \mat{X}$ to be invertible. This condition is satisfied when the features are linearly independent and we have at least as many training examples as features.
\end{remark}

\subsection{Regularized Regression}

When we have many features or when features are correlated, the normal equation can become unstable. Regularization helps by adding a penalty term to prevent overfitting.

\subsubsection{Ridge Regression (L2 Regularization)}

\textbf{Ridge regression} adds an L2 penalty to the loss function:

\begin{equation}
L(\vect{w}) = \|\mat{X}\vect{w} - \vect{y}\|^2 + \lambda \|\vect{w}\|^2
\end{equation}

The solution becomes:
\begin{equation}
\vect{w}^* = (\mat{X}^\top \mat{X} + \lambda \mat{I})^{-1} \mat{X}^\top \vect{y}
\end{equation}

where $\lambda > 0$ is the regularization strength.

\begin{example}
For a simple 2D case with features $x_1$ and $x_2$, ridge regression finds:
$$\hat{y} = w_1 x_1 + w_2 x_2 + b$$
The L2 penalty $\lambda(w_1^2 + w_2^2)$ encourages smaller weights, leading to a smoother, more stable solution.
\end{example}

\subsubsection{Lasso Regression (L1 Regularization)}

\textbf{Lasso regression} uses L1 regularization, which promotes sparsity:

\begin{equation}
L(\vect{w}) = \|\mat{X}\vect{w} - \vect{y}\|^2 + \lambda \|\vect{w}\|_1
\end{equation}

Unlike ridge regression, lasso can drive some weights to exactly zero, effectively performing automatic feature selection.

\begin{figure}[htbp]
\centering
\begin{tikzpicture}[scale=0.7]
\begin{axis}[
    xlabel={Regularization Strength ($\lambda$)},
    ylabel={Weight Magnitude},
    title={Regularization Effect on Weights},
    grid=major,
    width=10cm,
    height=6cm,
    legend pos=north east
]
% Ridge regression path
\addplot[thick, color=bookpurple, domain=0:2] {1/(1+x)};
\addplot[thick, color=bookred, domain=0:2] {max(0, 1-x)};
\legend{Ridge (L2), Lasso (L1)}
\end{axis}
\end{tikzpicture}
\caption{Comparison of L1 and L2 regularization effects. L1 can drive weights to zero, while L2 shrinks them smoothly.}
\label{fig:regularization-comparison}
\end{figure}

\subsection{Gradient Descent Solution}

For large datasets, computing the inverse of $\mat{X}^\top \mat{X}$ can be computationally expensive. Gradient descent provides an iterative alternative:

\begin{equation}
\vect{w}_{t+1} = \vect{w}_t - \alpha \nabla_{\vect{w}} L(\vect{w}_t)
\end{equation}

where the gradient is:
\begin{equation}
\nabla_{\vect{w}} L(\vect{w}) = \frac{2}{n} \mat{X}^\top (\mat{X}\vect{w} - \vect{y})
\end{equation}

\subsubsection{Stochastic Gradient Descent}

For very large datasets, we can use stochastic gradient descent (SGD), which updates weights using only a subset of the data at each iteration:

\begin{equation}
\vect{w}_{t+1} = \vect{w}_t - \alpha \nabla_{\vect{w}} L_i(\vect{w}_t)
\end{equation}

where $L_i$ is the loss for a single training example or a small batch.

\subsection{Geometric Interpretation}

Linear regression can be understood geometrically as finding the projection of the target vector $\vect{y}$ onto the column space of the design matrix $\mat{X}$. The residual vector $\vect{y} - \mat{X}\vect{w}^*$ is orthogonal to the column space of $\mat{X}$.

\begin{theorem}[Orthogonality Principle]
The optimal solution $\vect{w}^*$ satisfies:
$$\mat{X}^\top(\vect{y} - \mat{X}\vect{w}^*) = \vect{0}$$
This means the residual vector is orthogonal to all feature vectors.
\end{theorem}


% Chapter 5, Section 02

\section{Logistic Regression \difficultyInline{intermediate}}
\label{sec:logistic-regression}

\textbf{Logistic regression} is a fundamental classification algorithm that models the probability of class membership using a logistic (sigmoid) function. Despite its name, it's actually a classification method, not a regression method.

\subsection{Intuition and Motivation}

Logistic regression extends linear regression to handle classification problems. Instead of predicting continuous values, it predicts probabilities that an input belongs to a particular class. The key insight is to use a sigmoid function to map linear combinations of features to probabilities between 0 and 1.

Think of logistic regression as answering: "Given these features, what's the probability that this example belongs to the positive class?" For example, given a patient's symptoms, what's the probability they have a particular disease?

\begin{figure}[htbp]
\centering
\begin{tikzpicture}[scale=0.8]
\begin{axis}[
    xlabel={Linear Score ($z = \vect{w}^\top \vect{x} + b$)},
    ylabel={Probability $P(y=1|\vect{x})$},
    title={Sigmoid Function: Mapping Linear Scores to Probabilities},
    grid=major,
    width=10cm,
    height=6cm,
    xmin=-6, xmax=6,
    ymin=0, ymax=1
]
\addplot[thick, color=bookpurple, domain=-6:6] {1/(1+exp(-x))};
\node at (axis cs:2,0.8) [anchor=west] {$\sigma(z) = \frac{1}{1 + e^{-z}}$};
\addplot[dashed, color=bookred] coordinates {(-6,0.5) (6,0.5)};
\node at (axis cs:0,0.6) [anchor=east] {Decision boundary at 0.5};
\end{axis}
\end{tikzpicture}
\caption{The sigmoid function smoothly maps any real number to a probability between 0 and 1. The decision boundary is typically at 0.5.}
\label{fig:sigmoid-function}
\end{figure}

\subsection{Binary Classification}

For binary classification with classes $\{0, 1\}$, logistic regression models the probability of the positive class using the sigmoid function $\sigma(z) = \frac{1}{1 + e^{-z}}$, which smoothly maps any real number to a probability between 0 and 1. The prediction probability is $P(y=1|\vect{x}) = \sigma(\vect{w}^\top \vect{x} + b) = \frac{1}{1 + e^{-(\vect{w}^\top \vect{x} + b)}}$, where the linear combination of features and weights determines the input to the sigmoid function, and the output represents the probability that the input belongs to the positive class.

\begin{remark}
For categorical classification with more than 2 classes, we extend logistic regression to softmax regression, which uses the softmax function to compute probabilities for each class that sum to 1. This allows us to handle multi-class problems like image classification with 10 digit classes or natural language processing with hundreds of word categories.
\end{remark}

\subsubsection{Properties of the Sigmoid Function}

The sigmoid function has several important properties:
\begin{itemize}
    \item \textbf{Range:} $\sigma(z) \in (0, 1)$ for all $z \in \mathbb{R}$
    \item \textbf{Monotonic:} $\sigma'(z) = \sigma(z)(1-\sigma(z)) > 0$ for all $z$
    \item \textbf{Symmetric:} $\sigma(-z) = 1 - \sigma(z)$
    \item \textbf{Asymptotic:} $\lim_{z \to \infty} \sigma(z) = 1$ and $\lim_{z \to -\infty} \sigma(z) = 0$
\end{itemize}

\subsection{Cross-Entropy Loss}

Unlike linear regression, we can't use mean squared error for classification because the sigmoid function is non-linear. Instead, we use the \textbf{cross-entropy loss} (negative log-likelihood):

\begin{equation}
L(\vect{w}, b) = -\frac{1}{n} \sum_{i=1}^{n} \left[y^{(i)} \log \hat{y}^{(i)} + (1-y^{(i)}) \log(1-\hat{y}^{(i)})\right]
\end{equation}

where $\hat{y}^{(i)} = P(y=1|\vect{x}^{(i)})$.

\begin{remark}
Cross-entropy loss is the standard loss function for classification tasks in both classical machine learning and deep learning, providing better gradient properties than mean squared error for probability outputs. In deep neural networks, it's commonly used in the final layer for binary and multi-class classification, enabling efficient backpropagation and stable training across various architectures from simple logistic regression to complex convolutional and transformer networks.
\end{remark}

\subsubsection{Derivation of Cross-Entropy Loss}

The cross-entropy loss comes from maximum likelihood estimation. For a single example, the likelihood is:
$$L_i = P(y^{(i)}|\vect{x}^{(i)}) = (\hat{y}^{(i)})^{y^{(i)}} (1-\hat{y}^{(i)})^{1-y^{(i)}}$$

Taking the negative log-likelihood:
$$-\log L_i = -y^{(i)} \log \hat{y}^{(i)} - (1-y^{(i)}) \log(1-\hat{y}^{(i)})$$

\subsection{Gradient Descent for Logistic Regression}

The gradient of the cross-entropy loss with respect to the weights is:

\begin{equation}
\nabla_{\vect{w}} L = \frac{1}{n} \sum_{i=1}^{n} (\hat{y}^{(i)} - y^{(i)}) \vect{x}^{(i)}
\end{equation}

The weight update rule becomes:
\begin{equation}
\vect{w}_{t+1} = \vect{w}_t - \alpha \frac{1}{n} \sum_{i=1}^{n} (\hat{y}^{(i)} - y^{(i)}) \vect{x}^{(i)}
\end{equation}

\begin{remark}
The gradient has a simple form: it's the average of the prediction errors multiplied by the input features. This makes logistic regression particularly efficient to train.
\end{remark}

\subsection{Multiclass Classification}

For $K$ classes, we extend logistic regression to softmax regression (also called multinomial logistic regression), where instead of a single sigmoid function, we use the softmax function to compute the probability of each class:

\begin{equation}
P(y=k|\vect{x}) = \frac{\exp(\vect{w}_k^\top \vect{x} + b_k)}{\sum_{j=1}^{K} \exp(\vect{w}_j^\top \vect{x} + b_j)}
\end{equation}

The softmax function normalizes the exponential scores so that they sum to 1, creating a valid probability distribution over all possible classes.

\subsubsection{Properties of Softmax}

The softmax function has several key properties:
\begin{itemize}
    \item \textbf{Probability distribution:} $\sum_{k=1}^{K} P(y=k|\vect{x}) = 1$
    \item \textbf{Non-negative:} $P(y=k|\vect{x}) \geq 0$ for all $k$
    \item \textbf{Monotonic:} Higher scores lead to higher probabilities
    \item \textbf{Scale invariant:} Adding a constant to all scores doesn't change probabilities
\end{itemize}

\subsection{Categorical Cross-Entropy Loss}

For multiclass classification, we use the categorical cross-entropy loss:

\begin{equation}
L = -\frac{1}{n} \sum_{i=1}^{n} \sum_{k=1}^{K} y_k^{(i)} \log \hat{y}_k^{(i)}
\end{equation}

where $y_k^{(i)}$ is 1 if example $i$ belongs to class $k$, and 0 otherwise (one-hot encoding).

\subsection{Decision Boundaries}

Logistic regression creates linear decision boundaries. For binary classification, the decision boundary is the hyperplane where $P(y=1|\vect{x}) = 0.5$, which occurs when $\vect{w}^\top \vect{x} + b = 0$.

\begin{figure}[htbp]
\centering
\begin{tikzpicture}[scale=0.8]
\begin{axis}[
    xlabel={Feature 1},
    ylabel={Feature 2},
    title={Logistic Regression Decision Boundary},
    grid=major,
    width=10cm,
    height=8cm
]
% Generate some sample data
\addplot[only marks, mark=*, mark size=3pt, color=bookpurple] coordinates {
    (1,1) (1.5,1.5) (2,1) (2.5,2) (3,1.5)
};
\addplot[only marks, mark=square*, mark size=3pt, color=bookred] coordinates {
    (2,3) (2.5,3.5) (3,3) (3.5,3.5) (4,3)
};
% Add decision boundary
\addplot[thick, color=bookblack, domain=0:5] {3-x};
\node at (axis cs:2.5,2.5) [anchor=west] {Decision boundary: $x_1 + x_2 = 3$};
\legend{Class 0, Class 1, Decision boundary}
\end{axis}
\end{tikzpicture}
\caption{Logistic regression finds a linear decision boundary that separates the two classes. Points on one side are classified as class 0, points on the other side as class 1.}
\label{fig:logistic-decision-boundary}
\end{figure}

\subsection{Regularization in Logistic Regression}

Just like linear regression, logistic regression can benefit from regularization to prevent overfitting:

\subsubsection{L2 Regularization (Ridge)}

\begin{equation}
L(\vect{w}) = -\frac{1}{n} \sum_{i=1}^{n} \left[y^{(i)} \log \hat{y}^{(i)} + (1-y^{(i)}) \log(1-\hat{y}^{(i)})\right] + \lambda \|\vect{w}\|^2
\end{equation}

\subsubsection{L1 Regularization (Lasso)}

\begin{equation}
L(\vect{w}) = -\frac{1}{n} \sum_{i=1}^{n} \left[y^{(i)} \log \hat{y}^{(i)} + (1-y^{(i)}) \log(1-\hat{y}^{(i)})\right] + \lambda \|\vect{w}\|_1
\end{equation}

\begin{remark}
The key difference between L1 and L2 regularization lies in their penalty shapes: L2 uses the sum of squared weights $\|\vect{w}\|^2$ which creates smooth, continuous shrinkage of all weights, while L1 uses the sum of absolute weights $\|\vect{w}\|_1$ which can drive some weights to exactly zero, performing automatic feature selection. L2 is better for preventing overfitting when all features are relevant, while L1 is preferred when you want to identify and remove irrelevant features.
\end{remark}

\subsection{Advantages and Limitations}

Logistic regression has several advantages including being simple and interpretable with clear mathematical foundations, fast training and prediction due to efficient gradient computation, providing probability estimates that are useful for decision-making, working well with small datasets where complex models might overfit, and making no assumptions about feature distributions. However, it also has limitations including assuming a linear relationship between features and log-odds which may not hold for complex data, being sensitive to outliers that can significantly affect the decision boundary, potentially not working well with highly correlated features that can cause numerical instability, and being limited to linear decision boundaries which may not be sufficient for non-linearly separable data.


% Chapter 5, Section 03

\section{Support Vector Machines \difficultyInline{intermediate}}
\label{sec:svm}

\textbf{Support Vector Machines} (SVMs) are powerful classification algorithms that find the optimal hyperplane that maximally separates different classes. The key insight is to maximize the margin between classes, leading to better generalization performance.

\subsection{Intuition and Motivation}

Imagine you have two groups of points on a plane that you want to separate with a line. There are many possible lines that could separate them, but SVM finds the line that maximizes the distance to the nearest points from each class. This "maximum margin" approach leads to better generalization because the decision boundary is as far as possible from both classes.

The key concepts are:
\begin{itemize}
    \item \textbf{Support vectors:} The training examples closest to the decision boundary
    \item \textbf{Margin:} The distance between the decision boundary and the nearest support vectors
    \item \textbf{Maximum margin principle:} Choose the hyperplane that maximizes this margin
\end{itemize}

\begin{figure}[htbp]
\centering
\begin{tikzpicture}[scale=0.8]
\begin{axis}[
    xlabel={Feature 1},
    ylabel={Feature 2},
    title={SVM: Maximum Margin Classification},
    grid=major,
    width=10cm,
    height=8cm
]
% Generate some sample data
\addplot[only marks, mark=*, mark size=3pt, color=bookpurple] coordinates {
    (1,1) (1.5,1.5) (2,1) (2.5,2) (3,1.5)
};
\addplot[only marks, mark=square*, mark size=3pt, color=bookred] coordinates {
    (2,3) (2.5,3.5) (3,3) (3.5,3.5) (4,3)
};
% Add decision boundary and margins
\addplot[thick, color=bookblack, domain=0:5] {3-x};
\addplot[dashed, color=bookpurple, domain=0:5] {3-x+1};
\addplot[dashed, color=bookred, domain=0:5] {3-x-1};
\node at (axis cs:2.5,2.5) [anchor=west] {Decision boundary};
\node at (axis cs:2.5,3.5) [anchor=west] {Margin};
\node at (axis cs:2.5,1.5) [anchor=west] {Margin};
\legend{Class -1, Class +1, Decision boundary, Margins}
\end{axis}
\end{tikzpicture}
\caption{SVM finds the hyperplane (line in 2D) that maximizes the margin between classes. The support vectors are the points closest to the decision boundary.}
\label{fig:svm-margin}
\end{figure}

\subsection{Linear SVM}

For binary classification with labels $y \in \{-1, +1\}$, the decision boundary is:

\begin{equation}
\vect{w}^\top \vect{x} + b = 0
\end{equation}

The \textbf{margin} is the distance between the decision boundary and the nearest support vectors. For a point $\vect{x}^{(i)}$, the distance to the hyperplane is:

\begin{equation}
\text{distance} = \frac{|y^{(i)}(\vect{w}^\top \vect{x}^{(i)} + b)|}{\|\vect{w}\|}
\end{equation}

Since we want to maximize the margin, we can set the margin to be $\frac{2}{\|\vect{w}\|}$ by requiring:

\begin{equation}
y^{(i)}(\vect{w}^\top \vect{x}^{(i)} + b) \geq 1 \quad \forall i
\end{equation}

\subsubsection{Optimization Problem}

Maximizing the margin is equivalent to minimizing $\|\vect{w}\|^2$ subject to the constraints:

\begin{equation}
\min_{\vect{w}, b} \frac{1}{2}\|\vect{w}\|^2
\end{equation}

subject to:
\begin{equation}
y^{(i)}(\vect{w}^\top \vect{x}^{(i)} + b) \geq 1 \quad \forall i
\end{equation}

This is a quadratic programming problem that can be solved using Lagrange multipliers.

\subsection{Soft Margin SVM}

In practice, data is rarely linearly separable. The \textbf{soft margin SVM} allows some training examples to be misclassified by introducing slack variables $\xi_i$:

\begin{equation}
\min_{\vect{w}, b, \boldsymbol{\xi}} \frac{1}{2}\|\vect{w}\|^2 + C \sum_{i=1}^{n} \xi_i
\end{equation}

subject to:
\begin{equation}
y^{(i)}(\vect{w}^\top \vect{x}^{(i)} + b) \geq 1 - \xi_i, \quad \xi_i \geq 0
\end{equation}

The parameter $C$ controls the trade-off between:
\begin{itemize}
    \item \textbf{Large margin:} Small $C$ allows more slack, larger margin
    \item \textbf{Training accuracy:} Large $C$ penalizes misclassifications more heavily
\end{itemize}

\subsection{Dual Formulation}

The SVM optimization problem can be reformulated in its dual form, which reveals the support vectors and enables the kernel trick:

\begin{equation}
\max_{\boldsymbol{\alpha}} \sum_{i=1}^{n} \alpha_i - \frac{1}{2} \sum_{i=1}^{n} \sum_{j=1}^{n} \alpha_i \alpha_j y^{(i)} y^{(j)} \vect{x}^{(i)} \cdot \vect{x}^{(j)}
\end{equation}

subject to:
\begin{equation}
\sum_{i=1}^{n} \alpha_i y^{(i)} = 0, \quad 0 \leq \alpha_i \leq C
\end{equation}

The decision function becomes:
\begin{equation}
f(\vect{x}) = \sum_{i=1}^{n} \alpha_i y^{(i)} \vect{x}^{(i)} \cdot \vect{x} + b
\end{equation}

Only examples with $\alpha_i > 0$ are support vectors.

\subsection{Kernel Trick}

For non-linear decision boundaries, we can map inputs to a higher-dimensional space using a \textbf{kernel function} $k(\vect{x}, \vect{x}')$:

\begin{equation}
f(\vect{x}) = \sum_{i=1}^{n} \alpha_i y^{(i)} k(\vect{x}^{(i)}, \vect{x}) + b
\end{equation}

\subsubsection{Common Kernels}

\textbf{Linear kernel:}
\begin{equation}
k(\vect{x}, \vect{x}') = \vect{x}^\top \vect{x}'
\end{equation}

\textbf{Polynomial kernel:}
\begin{equation}
k(\vect{x}, \vect{x}') = (\vect{x}^\top \vect{x}' + c)^d
\end{equation}

\textbf{RBF (Gaussian) kernel:}
\begin{equation}
k(\vect{x}, \vect{x}') = \exp(-\gamma \|\vect{x} - \vect{x}'\|^2)
\end{equation}

\begin{figure}[htbp]
\centering
\begin{tikzpicture}[scale=0.8]
\begin{axis}[
    xlabel={Feature 1},
    ylabel={Feature 2},
    title={SVM with RBF Kernel: Non-linear Decision Boundary},
    grid=major,
    width=10cm,
    height=8cm
]
% Generate some sample data in a circular pattern
\addplot[only marks, mark=*, mark size=3pt, color=bookpurple] coordinates {
    (1,1) (1.5,1.5) (2,1) (2.5,2) (3,1.5) (1,2) (1.5,2.5) (2,2) (2.5,3) (3,2.5)
};
\addplot[only marks, mark=square*, mark size=3pt, color=bookred] coordinates {
    (2,3) (2.5,3.5) (3,3) (3.5,3.5) (4,3) (2,4) (2.5,4.5) (3,4) (3.5,4.5) (4,4)
};
% Add a non-linear decision boundary (circle)
\addplot[thick, color=bookblack, domain=0:5] {sqrt(6.25-(x-2.5)^2)+2.5};
\addplot[thick, color=bookblack, domain=0:5] {-sqrt(6.25-(x-2.5)^2)+2.5};
\node at (axis cs:3.5,3.5) [anchor=west] {Non-linear boundary};
\legend{Class -1, Class +1, Decision boundary}
\end{axis}
\end{tikzpicture}
\caption{SVM with RBF kernel can learn non-linear decision boundaries. The decision boundary here is approximately circular.}
\label{fig:svm-rbf-kernel}
\end{figure}

\subsection{Kernel Properties}

A function $k(\vect{x}, \vect{x}')$ is a valid kernel if and only if it is:
\begin{itemize}
    \item \textbf{Symmetric:} $k(\vect{x}, \vect{x}') = k(\vect{x}', \vect{x})$
    \item \textbf{Positive semi-definite:} For any set of points $\{\vect{x}^{(1)}, \ldots, \vect{x}^{(n)}\}$, the kernel matrix $K_{ij} = k(\vect{x}^{(i)}, \vect{x}^{(j)})$ is positive semi-definite
\end{itemize}

\subsection{Advantages and Limitations}

\textbf{Advantages:}
\begin{itemize}
    \item Effective in high-dimensional spaces
    \item Memory efficient (only stores support vectors)
    \item Versatile (different kernel functions)
    \item Strong theoretical foundation
    \item Works well with small to medium datasets
\end{itemize}

\textbf{Limitations:}
\begin{itemize}
    \item Poor performance on large datasets
    \item Sensitive to feature scaling
    \item No probabilistic output
    \item Kernel selection can be tricky
    \item Computationally expensive for very large datasets
\end{itemize}

\subsection{SVM for Regression}

SVM can also be extended to regression problems (Support Vector Regression, SVR). Instead of finding a hyperplane that separates classes, SVR finds a hyperplane that fits the data within an $\epsilon$-tube:

\begin{equation}
\min_{\vect{w}, b, \boldsymbol{\xi}, \boldsymbol{\xi}^*} \frac{1}{2}\|\vect{w}\|^2 + C \sum_{i=1}^{n} (\xi_i + \xi_i^*)
\end{equation}

subject to:
\begin{align}
y^{(i)} - \vect{w}^\top \vect{x}^{(i)} - b &\leq \epsilon + \xi_i \\
\vect{w}^\top \vect{x}^{(i)} + b - y^{(i)} &\leq \epsilon + \xi_i^* \\
\xi_i, \xi_i^* &\geq 0
\end{align}


% Chapter 5, Section 04

\section{Decision Trees and Ensemble Methods \difficultyInline{intermediate}}
\label{sec:decision-trees}

\textbf{Decision trees} are intuitive, interpretable models that make predictions by asking a series of yes/no questions about the input features. While individual trees can be prone to overfitting, combining multiple trees through ensemble methods often leads to much better performance.

\subsection{Intuition and Motivation}

Think of a decision tree as a flowchart for making decisions, where to classify whether someone will buy a product, you might ask "Is their income > \$50k?" and if yes, ask "Are they under 30?" or if no, ask "Do they have children?" Each question splits the data into smaller, more homogeneous groups that become easier to classify. The key advantages of decision trees include interpretability where they are easy to understand and explain, no feature scaling requirements since they work with mixed data types, handling missing values by being able to deal with incomplete data, and being non-parametric with no assumptions about data distribution, making them flexible and robust across different problem domains.

\begin{figure}[htbp]
\centering
\begin{tikzpicture}[scale=0.7]
\node[draw, rectangle, fill=bookpurple!20] (root) at (0,0) {Income > \$50k?};
\node[draw, rectangle, fill=bookpurple!20] (left) at (-3,-2) {Age < 30?};
\node[draw, rectangle, fill=bookpurple!20] (right) at (3,-2) {Has children?};
\node[draw, rectangle, fill=bookred!20] (ll) at (-4.5,-4) {Buy};
\node[draw, rectangle, fill=bookred!20] (lr) at (-1.5,-4) {Don't buy};
\node[draw, rectangle, fill=bookred!20] (rl) at (1.5,-4) {Buy};
\node[draw, rectangle, fill=bookred!20] (rr) at (4.5,-4) {Don't buy};

\draw[->] (root) -- (left) node[midway, left] {Yes};
\draw[->] (root) -- (right) node[midway, right] {No};
\draw[->] (left) -- (ll) node[midway, left] {Yes};
\draw[->] (left) -- (lr) node[midway, right] {No};
\draw[->] (right) -- (rl) node[midway, left] {Yes};
\draw[->] (right) -- (rr) node[midway, right] {No};
\end{tikzpicture}
\caption{A simple decision tree for predicting product purchases. Each internal node represents a decision, and leaf nodes represent the final prediction.}
\label{fig:decision-tree-example}
\end{figure}

\subsection{Decision Trees}

A decision tree recursively partitions the input space based on feature values, where the algorithm works by starting with all training examples at the root, finding the best feature and threshold to split on that maximizes information gain, creating child nodes for each split that represent the different outcomes, repeating recursively until a stopping criterion is met such as maximum depth or minimum samples per leaf, and assigning a prediction to each leaf node based on the majority class or mean value of the examples that reach that node.

\subsubsection{Splitting Criteria}

The key question is: "Which feature and threshold should we use to split the data?" We want splits that create the most homogeneous child nodes.

\textbf{For classification:}

\textbf{Gini impurity:}
\begin{equation}
\text{Gini} = 1 - \sum_{k=1}^{K} p_k^2
\end{equation}

\textbf{Entropy:}
\begin{equation}
\text{Entropy} = -\sum_{k=1}^{K} p_k \log p_k
\end{equation}

\textbf{For regression:}
\begin{equation}
\text{MSE} = \frac{1}{n} \sum_{i=1}^{n} (y^{(i)} - \bar{y})^2
\end{equation}

where $p_k$ is the proportion of class $k$ examples in a node, and $\bar{y}$ is the mean target value.

\subsubsection{Information Gain}

The \textbf{information gain} measures how much a split reduces impurity:

\begin{equation}
\text{IG} = \text{Impurity(parent)} - \sum_{j} \frac{n_j}{n} \text{Impurity(child}_j\text{)}
\end{equation}

We choose the split that maximizes information gain.

\subsection{Random Forests}

Random forests address the overfitting problem of individual trees by combining multiple trees trained on different subsets of the data, where each tree is trained on a bootstrap sample of the original data and considers only a random subset of features at each split, creating diversity among the trees that reduces overfitting and improves generalization performance.

\subsubsection{Bootstrap Aggregating (Bagging)}

Random forests inject diversity through two complementary mechanisms. First, each tree is trained on a \emph{bootstrap sample}—a dataset created by sampling with replacement from the original training set—so trees see different subsets of examples and learn different decision boundaries. Second, at each split within a tree, the algorithm considers only a \emph{random subset of features} rather than all features, forcing trees to rely on different signals and preventing a few strong predictors from dominating every split. At inference time, the forest aggregates individual tree predictions—by averaging for regression or majority voting for classification—so the ensemble reduces variance relative to any single overfit tree while keeping low bias.

\begin{equation}
\hat{y} = \frac{1}{B} \sum_{b=1}^{B} f_b(\vect{x})
\end{equation}

where $B$ is the number of trees.

\begin{figure}[htbp]
\centering
\begin{tikzpicture}[scale=0.8]
% Draw multiple trees
\foreach \x in {0,2,4,6} {
    \node[draw, rectangle, fill=bookpurple!20] (root\x) at (\x,0) {Tree \pgfmathparse{int(\x/2+1)}\pgfmathresult};
    \node[draw, rectangle, fill=bookred!20] (leaf1\x) at (\x-0.5,-1.5) {Leaf 1};
    \node[draw, rectangle, fill=bookred!20] (leaf2\x) at (\x+0.5,-1.5) {Leaf 2};
    \draw[->] (root\x) -- (leaf1\x);
    \draw[->] (root\x) -- (leaf2\x);
}

% Add ensemble prediction
\node[draw, rectangle, fill=bookred!40] (ensemble) at (3,-3) {Final Prediction};
\draw[->] (leaf10) -- (ensemble);
\draw[->] (leaf12) -- (ensemble);
\draw[->] (leaf14) -- (ensemble);
\draw[->] (leaf16) -- (ensemble);

\node at (3,-4) {Average/Vote};
\end{tikzpicture}
\caption{Random forests combine multiple decision trees. Each tree is trained on a different bootstrap sample and makes its own prediction. The final prediction is the average (regression) or majority vote (classification) of all trees.}
\label{fig:random-forest}
\end{figure}

\subsubsection{Advantages of Random Forests}

Random forests substantially reduce overfitting by averaging many de-correlated trees, which lowers variance without greatly increasing bias. They provide built-in \emph{feature importance} estimates that highlight which inputs most influence predictions, helping interpretation and feature selection. Because each tree is trained on a bootstrap sample, the ensemble is naturally robust to outliers and noisy examples that might mislead a single tree. Training is trivially parallelizable since trees are independent, and, like decision trees, random forests require no feature scaling and handle mixed data types gracefully.

\subsection{Gradient Boosting}

Gradient boosting builds an ensemble sequentially, where each new tree corrects the errors of the previous ensemble by learning to predict the residuals of the current model, unlike random forests where trees are trained independently. This sequential approach allows the ensemble to focus on the most difficult examples and gradually improve its performance through iterative refinement.

\subsubsection{Algorithm}

\begin{algorithm}[htbp]
\caption{Gradient Boosting Algorithm}
\label{alg:gradient-boosting}
\begin{algorithmic}[1]
\State Initialize $\hat{y}^{(0)} = \frac{1}{n} \sum_{i=1}^{n} y^{(i)}$ \Comment{Start with mean for regression}
\For{$m = 1$ to $M$}
    \State Compute residuals: $r_i^{(m)} = y^{(i)} - \hat{y}^{(m-1)}(\vect{x}^{(i)})$ for $i = 1, \ldots, n$
    \State Fit tree $f_m$ to residuals $\{(\vect{x}^{(i)}, r_i^{(m)})\}_{i=1}^{n}$
    \State Update: $\hat{y}^{(m)} = \hat{y}^{(m-1)} + \nu f_m(\vect{x}^{(i)})$ for all $i$
\EndFor
\State \textbf{Return} final ensemble $\hat{y}^{(M)}$
\end{algorithmic}
\end{algorithm}

where $\nu$ is the learning rate (shrinkage parameter) and $M$ is the number of boosting iterations.

\subsubsection{Intuition}

Gradient boosting works by starting with a simple model like the mean for regression, calculating the errors (residuals) of the current model to identify where it fails, training a new model to predict these errors and correct the mistakes, adding the new model to the ensemble with a small weight to avoid overfitting, and repeating this process until convergence where the ensemble can no longer be improved significantly.

\begin{example}
For regression with target values $[10, 20, 30, 40]$:

\begin{algorithmic}[1]
\State Initial prediction: $\hat{y}^{(0)} = 25$ \Comment{mean of targets}
\State Residuals: $[-15, -5, 5, 15]$ \Comment{$r_i = y^{(i)} - \hat{y}^{(0)}$}
\State Fit a small tree to residuals $\{(\vect{x}^{(i)}, r_i)\}$
\State Update ensemble: $\hat{y}^{(1)}(\vect{x}) = 25 + \nu\, f_1(\vect{x})$
\State Recompute residuals and repeat for $m=2,\dots,M$
\end{algorithmic}
\end{example}

\subsection{Advanced Ensemble Methods}

Advanced ensemble methods extend the basic boosting and bagging approaches with sophisticated techniques that improve performance and efficiency. AdaBoost (Adaptive Boosting) is an early boosting algorithm that assigns higher weights to misclassified examples, combines weak learners with weights based on their performance, and focuses on the hardest examples to improve the overall ensemble. Modern gradient boosting implementations like XGBoost and LightGBM add regularization with L1 and L2 penalties to prevent overfitting, pruning to remove splits that don't improve performance, feature subsampling with random feature selection at each split, and efficient implementation optimized for speed and memory usage.

\subsection{Comparison of Ensemble Methods}

\begin{table}[htbp]
\centering
\begin{tabular}{lccc}
\toprule
Method & Bias & Variance & Interpretability \\
\midrule
Single Tree & Low & High & High \\
Random Forest & Low & Medium & Medium \\
Gradient Boosting & Low & Low & Low \\
\bottomrule
\end{tabular}
\caption{Comparison of tree-based methods. Random forests reduce variance through averaging, while gradient boosting reduces both bias and variance through sequential learning.}
\label{tab:ensemble-comparison}
\end{table}

\subsection{Advantages and Limitations}

Decision trees and ensemble methods have several advantages including being interpretable and easy to understand and explain, being flexible by handling mixed data types and missing values, making no assumptions about data distribution, providing feature importance measures that can identify important features, and being robust with less sensitivity to outliers than linear methods. However, they also have limitations including overfitting where individual trees can overfit easily, instability where small changes in data can lead to very different trees, computational cost where ensemble methods can be slow to train, memory usage where storing many trees requires significant memory, and being less effective with high-dimensional data where performance can degrade with many features.


% Chapter 5, Section 05

\section{k-Nearest Neighbors \difficultyInline{intermediate}}
\label{sec:knn}

\textbf{k-Nearest Neighbors} (k-NN) is a simple yet powerful non-parametric algorithm that makes predictions based on the similarity to training examples. It's called "lazy learning" because it doesn't build a model during training—all computation happens at prediction time.

\subsection{Intuition and Motivation}

The k-NN algorithm is based on a simple principle: "similar things are close to each other." If you want to predict whether someone will like a movie, look at what movies similar people (with similar tastes) liked. If you want to predict house prices, look at prices of similar houses in the neighborhood.

The key insight is that we can make good predictions by finding the most similar examples in our training data and using their outcomes as a guide.

\begin{figure}[htbp]
\centering
\begin{tikzpicture}[scale=0.8]
\begin{axis}[
    xlabel={Feature 1},
    ylabel={Feature 2},
    title={k-NN Classification Example (k=3)},
    grid=major,
    width=10cm,
    height=8cm
]
% Generate some sample data
\addplot[only marks, mark=*, mark size=3pt, color=bookpurple] coordinates {
    (1,1) (1.5,1.5) (2,1) (2.5,2) (3,1.5)
};
\addplot[only marks, mark=square*, mark size=3pt, color=bookred] coordinates {
    (2,3) (2.5,3.5) (3,3) (3.5,3.5) (4,3)
};
% Add query point
\addplot[only marks, mark=triangle*, mark size=4pt, color=bookblack] coordinates {
    (2.5,2.5)
};
% Add circles around query point
\draw[thick, color=bookblack, dashed] (2.5,2.5) circle (1.2);
\node at (axis cs:2.5,2.5) [anchor=south] {Query point};
\node at (axis cs:2.5,1.3) [anchor=west] {k=3 neighbors};
\legend{Class 0, Class 1, Query point, Neighbors}
\end{axis}
\end{tikzpicture}
\caption{k-NN finds k nearest neighbors. With k=3, query classified as class 0 (2/3 neighbors).}
\label{fig:knn-example}
\end{figure}

\subsection{Algorithm}

For a query point $\vect{x}$, the k-NN algorithm works by finding the $k$ closest training examples based on a distance metric, then for classification returning the majority class among the $k$ neighbors, and for regression returning the average of the target values of the $k$ neighbors. The key ideas of the algorithm include using distance-based similarity to find relevant examples, leveraging the assumption that similar examples have similar outcomes, and making predictions based on local patterns in the data rather than global models.

\subsubsection{Mathematical Formulation}

For classification, the predicted class is:
\begin{equation}
\hat{y} = \arg\max_{c} \sum_{i=1}^{k} \mathbb{I}(y^{(i)} = c)
\end{equation}

For regression, the predicted value is:
\begin{equation}
\hat{y} = \frac{1}{k} \sum_{i=1}^{k} y^{(i)}
\end{equation}

where $y^{(i)}$ are the target values of the $k$ nearest neighbors.

\subsection{Distance Metrics}

The choice of distance metric significantly affects k-NN performance, where different metrics are used in various machine learning algorithms to measure similarity between data points. Euclidean distance is commonly used in k-NN and clustering algorithms, Manhattan distance is useful in recommendation systems and text analysis, and Minkowski distance provides a flexible framework for different distance measures in various machine learning applications.

\subsubsection{Euclidean Distance}

\begin{definition}[Euclidean Distance]
\begin{equation}
d(\vect{x}, \vect{x}') = \sqrt{\sum_{i=1}^{d} (x_i - x_i')^2}
\end{equation}
This is the most common choice, measuring the straight-line distance between points.
\end{definition}

\subsubsection{Manhattan Distance}

\begin{definition}[Manhattan Distance]
\begin{equation}
d(\vect{x}, \vect{x}') = \sum_{i=1}^{d} |x_i - x_i'|
\end{equation}
Also known as L1 distance, it measures the sum of absolute differences. Useful when features have different scales.
\end{definition}

\subsubsection{Minkowski Distance}

\begin{definition}[Minkowski Distance]
\begin{equation}
d(\vect{x}, \vect{x}') = \left(\sum_{i=1}^{d} |x_i - x_i'|^p\right)^{1/p}
\end{equation}
This is a generalization where:
\begin{itemize}
    \item $p = 1$: Manhattan distance
    \item $p = 2$: Euclidean distance
    \item $p \to \infty$: Chebyshev distance (maximum coordinate difference)
\end{itemize}
\end{definition}

\subsection{Choosing k}

The choice of $k$ is crucial and involves a bias-variance trade-off, where small $k$ values like k=1 result in low bias and high variance with flexible decision boundaries that are sensitive to noise and outliers and may overfit to training data, while large $k$ values like k=n result in high bias and low variance with smooth decision boundaries that may miss local patterns and underfit the data. The optimal $k$ is typically found through cross-validation, balancing the need for local pattern recognition with robustness to noise.

\begin{figure}[htbp]
\centering
\begin{tikzpicture}[scale=0.8]
\begin{axis}[
    xlabel={k value},
    ylabel={Error Rate},
    title={Effect of k on k-NN Performance},
    grid=major,
    width=10cm,
    height=6cm,
    xmin=1, xmax=20
]
% Generate a U-shaped curve
\addplot[thick, color=bookpurple, domain=1:20] {0.1 + 0.05*(x-5)^2/25};
\node at (axis cs:5,0.1) [anchor=south] {Optimal k};
\node at (axis cs:1,0.15) [anchor=east] {High variance};
\node at (axis cs:20,0.15) [anchor=west] {High bias};
\end{axis}
\end{tikzpicture}
\caption{Effect of k on k-NN: small k→high variance, large k→high bias. Optimal k via CV.}
\label{fig:knn-k-selection}
\end{figure}

\subsection{Weighted k-NN}

Instead of giving equal weight to all $k$ neighbors, we can weight them by their distance:

\begin{equation}
\hat{y} = \frac{\sum_{i=1}^{k} w_i y^{(i)}}{\sum_{i=1}^{k} w_i}
\end{equation}

where $w_i = \frac{1}{d(\vect{x}, \vect{x}^{(i)})}$ is the weight based on distance.

\subsection{Computational Considerations}

k-NN has several computational characteristics including no training time since it's a lazy learner that defers all computation to prediction time, memory usage that requires storing all training examples, and scalability issues where performance degrades with large datasets. The naive approach has $O(n \cdot d)$ complexity for each prediction, but optimized approaches using data structures like KD-trees, ball trees, and locality sensitive hashing can significantly reduce this complexity for large-scale applications.

\subsubsection{Speedup Techniques}

Several speedup techniques can be employed to improve the computational efficiency of k-NN algorithms. KD-Trees partition the space into regions using hyperplanes, reducing search time to $O(\log n)$ in low dimensions, though they become less effective in high dimensions where the curse of dimensionality affects their performance. Ball Trees use hyperspheres instead of hyperplanes to partition the space, making them better suited for high-dimensional data, though they are more complex to implement than KD-trees. Locality Sensitive Hashing (LSH) provides approximate nearest neighbor search capabilities, offering significant speedup for very large datasets, though it may sacrifice some accuracy in exchange for computational efficiency, making it suitable for applications where approximate results are acceptable.

\subsection{Advantages and Limitations}

k-NN has several advantages including being simple and easy to understand and implement, making no assumptions about data distribution and working with any data type, being non-parametric with no model to fit, being able to capture complex decision boundaries through local patterns, and being incremental by making it easy to add new training examples. However, it also has limitations including computational cost where it's slow for large datasets, memory usage where it must store all training data, sensitivity to irrelevant features where all features are treated equally, the curse of dimensionality where performance degrades in high dimensions, and lack of interpretability where it's hard to understand why a prediction was made.

\subsection{Curse of Dimensionality}

In high-dimensional spaces, k-NN faces the curse of dimensionality where all points become roughly equidistant due to distance concentration, most of the space is empty making it difficult to find meaningful neighbors, and performance degrades with irrelevant features that can dominate the distance calculations, making the algorithm less effective in high-dimensional spaces.

\begin{example}
In a 1000-dimensional space, even if only 10 features are relevant, the 990 irrelevant features can dominate the distance calculations, making k-NN ineffective.
\end{example}

\subsection{Feature Selection and Scaling}

To improve k-NN performance, several techniques can be employed including feature selection to remove irrelevant features that can hurt performance, feature scaling to normalize features to the same scale so that all features contribute equally to distance calculations, dimensionality reduction using PCA or other techniques to reduce the curse of dimensionality, and distance weighting to weight features by importance so that more relevant features have greater influence on the similarity calculations.


% Chapter 5, Section 06

\section{Comparison with Deep Learning \difficultyInline{intermediate}}
\label{sec:comparison}

Understanding the relationship between classical machine learning methods and deep learning is crucial for choosing the right approach for your problem. While deep learning has achieved remarkable success in many domains, classical methods still have important advantages in certain scenarios.

\subsection{When Classical Methods Excel}

Classical machine learning methods have several advantages that make them preferable in certain situations, including interpretability and debugging where linear models have coefficients that directly show feature importance, decision trees have rules that are human-readable, SVMs have support vectors that provide insight into decision boundaries, and easier debugging where predictions can be traced step by step. For small to medium datasets, classical methods are less prone to overfitting with fewer parameters, have faster training requiring less computational resources, often perform better with limited data through better generalization, and work well with original data without needing data augmentation. In terms of computational efficiency, classical methods have lower memory requirements without needing to store large networks, faster inference through simple mathematical operations, no GPU requirements allowing them to run on standard hardware, and suitability for real-time applications and embedded systems.

\subsection{When Deep Learning Excels}

Deep learning addresses several fundamental limitations of classical methods through automatic feature learning where networks learn relevant features automatically without manual feature engineering, hierarchical representations where lower layers learn simple features and higher layers learn complex combinations, end-to-end learning where a single model handles feature extraction and classification, and adaptive features that adapt to the specific problem. In terms of scalability with data and model size, deep learning has data scalability where performance typically improves with more data, model capacity to handle very large models with millions of parameters, distributed training capabilities that can leverage multiple GPUs/TPUs, and transfer learning where pre-trained models can be fine-tuned for new tasks. For handling complex data types, deep learning excels with images through convolutional networks that excel at computer vision, text through recurrent and transformer networks that handle natural language, audio by processing raw audio signals, and multimodal applications that can combine different data types.

\subsection{Performance Comparison}

\begin{table}[htbp]
\centering
\begin{tabular}{lccc}
\toprule
Aspect & Classical ML & Deep Learning & Best Use Case \\
\midrule
Interpretability & High & Low & Medical diagnosis, finance \\
Training Speed & Fast & Slow & Prototyping, small datasets \\
Inference Speed & Fast & Medium & Real-time applications \\
Data Requirements & Low & High & Small companies, research \\
Computational Cost & Low & High & Resource-constrained environments \\
Feature Engineering & Manual & Automatic & Complex domains \\
\bottomrule
\end{tabular}
\caption{Classical ML vs deep learning comparison.}
\label{tab:ml-vs-dl-comparison}
\end{table}

\subsection{Hybrid Approaches}

In practice, the best solutions often combine classical and deep learning methods through feature engineering with deep learning where deep networks are used to extract features and classical methods like SVM and random forest are applied on the extracted features, combining the interpretability of classical methods with the representation power of deep learning. Ensemble methods combine predictions from classical and deep learning models, using classical methods for interpretable components and deep learning for complex pattern recognition. Two-stage approaches use classical methods for initial screening and apply deep learning for final classification, balancing efficiency and accuracy.

\subsection{Choosing the Right Approach}

The choice between classical and deep learning methods depends on several factors including data characteristics where small datasets (< 10k examples) often favor classical methods, medium datasets (10k-100k examples) make both approaches viable, large datasets (> 100k examples) typically favor deep learning, high-dimensional data excels with deep learning, and structured data often works well with classical methods. Problem requirements include interpretability needs where classical methods are preferred, real-time inference where classical methods are often faster, complex patterns where deep learning is better, and unstructured data where deep learning is necessary. Resource constraints include limited computational resources favoring classical methods, limited labeled data favoring classical methods or transfer learning, need for quick prototyping favoring classical methods, and production deployment requiring consideration of inference costs.

\subsection{Future Directions}

The field continues to evolve with several promising directions including automated machine learning (AutoML) with neural architecture search that automatically designs network architectures, hyperparameter optimization that automatically tunes classical methods, and model selection that automatically chooses between classical and deep learning. Explainable AI includes SHAP values that explain predictions from any model, LIME for local interpretable model-agnostic explanations, and attention mechanisms that help understand what deep networks focus on. Efficient deep learning involves model compression to reduce model size while maintaining performance, quantization to use lower precision arithmetic, and knowledge distillation to transfer knowledge from large to small models.

\subsection{Conclusion}

Classical machine learning methods and deep learning are not competing approaches but complementary tools in the machine learning toolkit, where the key is to understand the strengths and limitations of each approach and choose the right tool for your specific problem. The approach should start simple by beginning with classical methods for baseline performance, consider complexity by only using deep learning if classical methods are insufficient, think about requirements by considering interpretability, speed, and resource constraints, combine approaches by using hybrid methods when appropriate, and stay updated as the field continues to evolve rapidly. The goal is not to use the most complex method, but to use the most appropriate method for your specific problem and constraints.

\begin{remark}[Transformer Architecture Excellence]
Transformer architecture excels compared to traditional deep learning by using self-attention mechanisms that can process entire sequences in parallel, enabling much faster training than sequential RNNs, and by capturing long-range dependencies more effectively through direct attention connections rather than relying on sequential processing. This parallel processing capability and superior sequence modeling make transformers the foundation for modern large language models and state-of-the-art performance in natural language processing tasks.
\end{remark}

\begin{table}[htbp]
\centering
\begin{tabular}{lcc}
\toprule
Aspect & Classical ML & Deep Learning \\
\midrule
Interpretability & High & Low \\
Training Speed & Fast & Slow \\
Inference Speed & Fast & Medium \\
Data Requirements & Low & High \\
Computational Cost & Low & High \\
Feature Engineering & Manual & Automatic \\
Best for Small Data & Yes & No \\
Best for Large Data & No & Yes \\
\bottomrule
\end{tabular}
\caption{Pros and cons comparison: classical ML vs deep learning.}
\label{tab:ml-dl-pros-cons}
\end{table}


% Chapter summary and problems
% Key Takeaways for Chapter 5

\section*{Key Takeaways}
\addcontentsline{toc}{section}{Key Takeaways}

\begin{keytakeaways}
\begin{itemize}[leftmargin=2em]
    \item \textbf{Classical algorithms} like linear regression, logistic regression, and SVMs provide interpretable baselines for machine learning tasks.
    \item \textbf{Trade-offs exist} between model complexity, interpretability, and performance; classical methods excel in low-data regimes.
    \item \textbf{Regularisation} (L1/L2) prevents overfitting and enables feature selection in high-dimensional settings.
    \item \textbf{Ensemble methods} combine weak learners to improve accuracy and robustness through variance reduction.
    \item \textbf{Understanding limitations} of classical methods motivates deep learning's hierarchical representation learning.
\end{itemize}
\end{keytakeaways}




\section*{Problems}
\addcontentsline{toc}{section}{Problems}

\subsection*{Easy Problems}

\begin{problem}[Linear Regression Basics]
\label{prob:linear-regression-basics}
Given the dataset $\{(1, 2), (2, 4), (3, 6), (4, 8)\}$, find the linear regression model $\hat{y} = wx + b$ that minimises the mean squared error.

\textbf{Hint:} Use the normal equation $\vect{w}^* = (\mat{X}^\top \mat{X})^{-1} \mat{X}^\top \vect{y}$ where $\mat{X}$ includes a column of ones for the bias term.
\end{problem}

\begin{problem}[Logistic Regression Decision Boundary]
\label{prob:logistic-decision-boundary}
For a logistic regression model with weights $\vect{w} = [2, -1]$ and bias $b = 0$, find the decision boundary equation and classify the point $(1, 1)$.

\textbf{Hint:} The decision boundary occurs where $P(y=1|\vect{x}) = 0.5$, which means $\vect{w}^\top \vect{x} + b = 0$.
\end{problem}

\begin{problem}[Decision Tree Splitting]
\label{prob:decision-tree-splitting}
Given a node with 10 examples: 6 belong to class A and 4 to class B, calculate the Gini impurity and entropy.

\textbf{Hint:} Gini impurity = $1 - \sum_{k} p_k^2$ and entropy = $-\sum_{k} p_k \log p_k$ where $p_k$ is the proportion of class $k$.
\end{problem}

\begin{problem}[Regularisation Effect]
\label{prob:regularization-effect}
Explain why L1 regularisation (Lasso) can drive some weights to exactly zero, while L2 regularisation (Ridge) cannot.

\textbf{Hint:} Consider the shape of the L1 and L2 penalty functions and their derivatives at zero.
\end{problem}

\subsection*{Medium Problems}

\begin{problem}[Ridge Regression Derivation]
\label{prob:ridge-derivation}
Derive the closed-form solution for ridge regression: $\vect{w}^* = (\mat{X}^\top \mat{X} + \lambda \mat{I})^{-1} \mat{X}^\top \vect{y}$.

\textbf{Hint:} Start with the ridge regression objective function $L(\vect{w}) = \|\mat{X}\vect{w} - \vect{y}\|^2 + \lambda \|\vect{w}\|^2$ and take the gradient with respect to $\vect{w}$.
\end{problem}

\begin{problem}[Random Forest Bias-Variance]
\label{prob:random-forest-bias-variance}
Explain how random forests reduce variance compared to a single decision tree. What happens to bias?

\textbf{Hint:} Consider how averaging multiple models affects the bias and variance of the ensemble.
\end{problem}

\subsection*{Hard Problems}

\begin{problem}[SVM Kernel Trick]
\label{prob:svm-kernel-trick}
Prove that the polynomial kernel $k(\vect{x}, \vect{x}') = (\vect{x}^\top \vect{x}' + c)^d$ is a valid kernel function by showing it corresponds to an inner product in some feature space.

\textbf{Hint:} Expand the polynomial and show it can be written as $\phi(\vect{x})^\top \phi(\vect{x}')$ for some feature map $\phi$.
\end{problem}

\begin{problem}[Ensemble Methods Theory]
\label{prob:ensemble-theory}
Prove that for an ensemble of $B$ independent models with error rate $p < 0.5$, the ensemble error rate approaches 0 as $B \to \infty$.

\textbf{Hint:} Use the binomial distribution and the fact that the ensemble makes an error only when more than half of the models are wrong.
\end{problem}
