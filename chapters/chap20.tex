% Chapter 20: Deep Generative Models

\chapter{Deep Generative Models}
\label{chap:deep-generative-models}

This chapter examines modern approaches to generating new data samples using deep learning.


\begin{learningobjectives}
\objective{VAEs, GANs, and flow/diffusion models conceptually and practically}
\objective{Training objectives and sampling procedures for each model class}
\objective{Generative model evaluation with proper metrics and qualitative checks}
\objective{Trade-offs in likelihood, sample quality, and mode coverage}
\end{learningobjectives}



\section*{Intuition}
\addcontentsline{toc}{section}{Intuition}

Generative models are machine learning systems that learn to create new data samples that resemble a given dataset. For example, after training on thousands of cat photos, a generative model can produce entirely new, realistic cat images that never existed before.

Think of generative models as digital artists who study masterpieces to understand artistic style, then create original works in that same style. Just as a painter learns brushstrokes and color palettes from studying great works, these models learn the underlying patterns and structures in data to generate novel samples.

Generative models learn data distributions in complementary ways: explicit likelihoods, implicit adversarial training, or score-based diffusion. Each chooses a tractable learning signal that captures structure while managing complexity.


% Chapter 20, Section 1

\section{Variational Autoencoders (VAEs) \difficultyInline{advanced}}
\label{sec:vaes}

(See also Chapter 14 for detailed VAE coverage.)

\subsection{Recap}

VAE learns latent representation $\vect{z}$ and decoder $p_{\theta}(\vect{x}|\vect{z})$:
\begin{equation}
\max_{\theta, \phi} \mathbb{E}_{q_{\phi}(\vect{z}|\vect{x})}[\log p_{\theta}(\vect{x}|\vect{z})] - D_{KL}(q_{\phi}(\vect{z}|\vect{x}) \| p(\vect{z}))
\end{equation}

The VAE objective consists of two terms: the reconstruction loss $\mathbb{E}_{q_{\phi}(\vect{z}|\vect{x})}[\log p_{\theta}(\vect{x}|\vect{z})]$ encourages the decoder to accurately reconstruct input data from latent codes, while the KL divergence term $D_{KL}(q_{\phi}(\vect{z}|\vect{x}) \| p(\vect{z}))$ regularizes the encoder to produce latent distributions close to the prior $p(\vect{z})$. This balance ensures both faithful reconstruction and meaningful latent representations.

\subsection{Conditional VAEs}

Generate conditioned on class or attributes:
\begin{equation}
\max \mathbb{E}_{q(\vect{z}|\vect{x}, y)}[\log p(\vect{x}|\vect{z}, y)] - D_{KL}(q(\vect{z}|\vect{x}, y) \| p(\vect{z}))
\end{equation}

Conditional VAEs extend the standard VAE framework by incorporating conditioning information $y$ (such as class labels or attributes) into both the encoder and decoder. The encoder $q(\vect{z}|\vect{x}, y)$ learns to map input data and conditioning information to latent codes, while the decoder $p(\vect{x}|\vect{z}, y)$ generates samples conditioned on both the latent code and the conditioning variable. This enables controlled generation of specific types of content.

\subsection{Disentangled Representations}

\textbf{$\beta$-VAE:} Increase KL weight for disentanglement
\begin{equation}
\mathcal{L} = \mathbb{E}_{q}[\log p(\vect{x}|\vect{z})] - \beta D_{KL}(q(\vect{z}|\vect{x}) \| p(\vect{z}))
\end{equation}

The $\beta$-VAE introduces a hyperparameter $\beta$ that controls the strength of the KL regularization term. When $\beta > 1$, the model is encouraged to learn more independent latent factors, where each dimension of $\vect{z}$ corresponds to a distinct, interpretable attribute of the data. This disentanglement is achieved by penalizing the mutual information between latent dimensions, forcing the model to encode different aspects of variation in separate latent variables.

% \subsection{Visual aids}
% \addcontentsline{toc}{subsubsection}{Visual aids (VAE recap)}

% \begin{figure}[h]
%   \centering
%   \begin{tikzpicture}
%     \begin{axis}[
%       width=0.48\textwidth,height=0.36\textwidth,
%       xlabel={$z_1$}, ylabel={$z_2$}, grid=both]
%       \addplot+[only marks,mark=*,mark size=0.9pt,bookpurple!70] coordinates{(-2,-2) (-1,0) (0,0) (1,1) (2,2)};
%     \end{axis}
%   \end{tikzpicture}
%   \caption{Latent samples from a learned VAE prior/posterior (illustrative).}
%   \label{fig:vae20-latent}
% \end{figure}

% \subsection{Notes and references}

% See \textcite{Kingma2013,GoodfellowEtAl2016,Prince2023} for VAE objectives, conditional VAEs, and disentanglement.


% Chapter 20, Section 2

\section{Generative Adversarial Networks (GANs) \difficultyInline{advanced}}
\label{sec:gans}

\subsection{Core Idea}

Two networks compete:
\begin{itemize}
    \item \textbf{Generator} $G$: Creates fake samples from noise
    \item \textbf{Discriminator} $D$: Distinguishes real from fake
\end{itemize}

\subsection{Objective}

Minimax game:
\begin{equation}
\min_G \max_D \mathbb{E}_{\vect{x} \sim p_{\text{data}}}[\log D(\vect{x})] + \mathbb{E}_{\vect{z} \sim p(\vect{z})}[\log(1 - D(G(\vect{z})))]
\end{equation}

\subsection{Training Procedure}

Alternate updates:
\begin{enumerate}
    \item \textbf{Update D:} Maximize discrimination
    \begin{equation}
    \max_D \mathbb{E}_{\vect{x}}[\log D(\vect{x})] + \mathbb{E}_{\vect{z}}[\log(1 - D(G(\vect{z})))]
    \end{equation}
    
    \item \textbf{Update G:} Minimize discrimination (or maximize $\log D(G(\vect{z}))$)
    \begin{equation}
    \min_G \mathbb{E}_{\vect{z}}[\log(1 - D(G(\vect{z})))]
    \end{equation}
\end{enumerate}

\subsection{Training Challenges}

\textbf{Mode collapse:} Generator produces limited variety

\textbf{Training instability:} Oscillations, non-convergence

\textbf{Vanishing gradients:} When discriminator too strong

\subsection{GAN Variants}

\textbf{DCGAN:} Deep Convolutional GAN with architectural guidelines

\textbf{WGAN:} Wasserstein GAN with improved training stability

\textbf{StyleGAN:} High-quality image generation with style control

\textbf{Conditional GAN:} Generate from class labels

\textbf{CycleGAN:} Unpaired image-to-image translation

% \subsection{Visual aids}
% \addcontentsline{toc}{subsubsection}{Visual aids (GANs)}

% \begin{figure}[h]
%   \centering
%   \begin{tikzpicture}[>=stealth]
%     \tikzstyle{b}=[draw,rounded corners,align=center,minimum width=2.4cm,minimum height=1.0cm]
%     \node[b,fill=bookpurple!10] at (0,0) (z) {Noise $\vect{z}$};
%     \node[b,fill=bookpurple!15] at (3.2,0) (g) {Generator $G$};
%     \node[b,fill=bookpurple!10] at (6.4,0.9) (x) {Real $\vect{x}$};
%     \node[b,fill=bookpurple!10] at (6.4,-0.9) (gx) {$G(\vect{z})$};
%     \node[b,fill=bookpurple!20] at (9.6,0) (d) {Discriminator $D$};
%     \draw[->] (z) -- (g);
%     \draw[->] (g) -- (gx);
%     \draw[->] (x) -- (d);
%     \draw[->] (gx) -- (d);
%   \end{tikzpicture}
%   \caption{GAN training: generator produces samples to fool discriminator.}
%   \label{fig:gan-diagram}
% \end{figure}

% \subsection{Notes and references}

% See \textcite{Goodfellow2014,GoodfellowEtAl2016,Prince2023} for GAN fundamentals and variants.


% Chapter 20, Section 3

\section{Normalizing Flows \difficultyInline{advanced}}
\label{sec:normalizing-flows}

Normalizing flows transform simple probability distributions through a series of invertible mappings to model complex data distributions while maintaining exact likelihood computation.

\subsection{Key Idea}

Transform simple distribution (e.g., Gaussian) through invertible mappings:
\begin{equation}
\vect{x} = f_{\theta}(\vect{z}), \quad \vect{z} \sim p_z(\vect{z})
\end{equation}

The fundamental insight of normalizing flows is that complex probability distributions can be constructed by applying a series of invertible transformations to a simple base distribution. Starting from a tractable distribution like a standard Gaussian $p_z(\vect{z})$, the flow model learns a bijective function $f_{\theta}$ that maps samples from this simple distribution to the complex data distribution. The invertibility requirement ensures that we can compute exact likelihoods and perform exact sampling, making flows particularly attractive for applications requiring precise probability estimates.

\subsection{Change of Variables}

Density transforms as:
\begin{equation}
p_x(\vect{x}) = p_z(f^{-1}(\vect{x})) \left|\det \frac{\partial f^{-1}}{\partial \vect{x}}\right|
\end{equation}

or equivalently:
\begin{equation}
\log p_x(\vect{x}) = \log p_z(\vect{z}) - \log\left|\det \frac{\partial f}{\partial \vect{z}}\right|
\end{equation}

The change of variables formula is the mathematical foundation that enables normalizing flows to compute exact likelihoods. When transforming a random variable through an invertible function, the probability density changes according to the Jacobian determinant of the transformation. The first equation shows how the density at a point $\vect{x}$ in the data space relates to the density at the corresponding point $f^{-1}(\vect{x})$ in the latent space, scaled by the absolute value of the Jacobian determinant. The log-space formulation is computationally more stable and directly relates the log-likelihood of the data to the log-likelihood of the latent variable minus the log-determinant of the transformation.

\subsection{Requirements}

The success of normalizing flows depends critically on the design of the transformation function $f$, which must satisfy two fundamental requirements. First, the function must be invertible, meaning that for every output $\vect{x}$, there exists a unique input $\vect{z}$ such that $\vect{x} = f(\vect{z})$. This bijectivity is essential for both likelihood computation and sampling, as it ensures a one-to-one correspondence between the latent and data spaces. Second, the function must have a tractable Jacobian determinant, as computing $\log|\det \frac{\partial f}{\partial \vect{z}}|$ is required for likelihood evaluation. These constraints significantly limit the class of functions that can be used in practice, leading to the development of specialized architectures that satisfy both requirements while maintaining sufficient expressiveness to model complex distributions.

\subsection{Flow Architectures}

The design of invertible transformations has led to several innovative architectural paradigms that balance expressiveness with computational tractability. Coupling layers represent a particularly elegant solution by partitioning the input dimensions into two halves, where one half is transformed based on the other half, ensuring invertibility while allowing complex dependencies. This approach enables the modeling of intricate conditional relationships while maintaining the ability to compute exact likelihoods through the structured Jacobian.

Autoregressive flows extend this idea by modeling each dimension as a function of all previous dimensions, creating a natural ordering that facilitates both sampling and likelihood computation. This autoregressive structure allows for highly expressive transformations while maintaining computational efficiency through the triangular nature of the resulting Jacobian matrix. Continuous normalizing flows represent a more recent innovation that leverages neural ordinary differential equations to model continuous-time transformations, providing a more flexible framework for learning complex flow dynamics while maintaining the theoretical guarantees of normalizing flows.

\subsection{Advantages}

Normalizing flows offer several compelling advantages that distinguish them from other generative modeling approaches. The ability to compute exact likelihoods provides a principled way to evaluate model performance and compare different architectures, addressing a fundamental limitation of GANs where likelihood estimation is intractable. This exact likelihood computation also enables applications requiring precise probability estimates, such as anomaly detection and uncertainty quantification.

The exact sampling capability ensures that generated samples are drawn from the true learned distribution, eliminating the approximation errors inherent in other methods. Unlike GANs, which require careful balance between generator and discriminator, normalizing flows provide stable training dynamics through standard maximum likelihood optimization. This stability makes flows particularly attractive for applications where reliable convergence is crucial, while their invertibility enables bidirectional mapping between latent and data spaces, opening up possibilities for data compression and representation learning.

% \subsection{Visual aids}
% \addcontentsline{toc}{subsubsection}{Visual aids (flows)}

% \begin{figure}[h]
%   \centering
%   \begin{tikzpicture}
%     \begin{axis}[
%       width=0.48\textwidth,height=0.36\textwidth,
%       xlabel={$z_1$}, ylabel={$z_2$}, grid=both]
%       \addplot+[only marks,mark=*,mark size=0.8pt,bookpurple!60] coordinates{(-1,-1) (-1,1) (1,-1) (1,1)};
%     \end{axis}
%   \end{tikzpicture}
%   \caption{Toy latent samples before/after flow transformation (illustrative).}
%   \label{fig:flow-toy}
% \end{figure}

% \subsection{Notes and references}

% See \textcite{GoodfellowEtAl2016,Prince2023} for flow-based modeling and practical design choices.


% Chapter 20, Section 4

\section{Diffusion Models \difficultyInline{advanced}}
\label{sec:diffusion-models}

Diffusion models learn to reverse a gradual noise corruption process, starting from pure noise and iteratively denoising to generate high-quality samples through a learned denoising network.

\subsection{Forward Process}

Gradually add noise over $T$ steps:
\begin{equation}
q(\vect{x}_t|\vect{x}_{t-1}) = \mathcal{N}(\vect{x}_t; \sqrt{1-\beta_t} \vect{x}_{t-1}, \beta_t \mat{I})
\end{equation}

Eventually $\vect{x}_T \approx \mathcal{N}(\boldsymbol{0}, \mat{I})$.

The forward process systematically corrupts clean data by adding Gaussian noise at each timestep, following a predefined noise schedule $\{\beta_t\}_{t=1}^T$. The parameterization $\sqrt{1-\beta_t} \vect{x}_{t-1}$ ensures that the signal-to-noise ratio decreases gradually, while $\beta_t \mat{I}$ controls the amount of noise added at each step. This process is designed to be analytically tractable, allowing efficient sampling and training. The key insight is that after sufficient timesteps, the data becomes approximately Gaussian noise, providing a well-defined starting point for the reverse process.

\subsection{Reverse Process}

Learn to denoise (reverse diffusion):
\begin{equation}
p_{\theta}(\vect{x}_{t-1}|\vect{x}_t) = \mathcal{N}(\vect{x}_{t-1}; \boldsymbol{\mu}_{\theta}(\vect{x}_t, t), \boldsymbol{\Sigma}_{\theta}(\vect{x}_t, t))
\end{equation}

The reverse process represents the core learning challenge in diffusion models, where a neural network must learn to reverse the forward corruption process. The model $p_{\theta}(\vect{x}_{t-1}|\vect{x}_t)$ predicts the parameters of a Gaussian distribution that should generate the previous timestep $\vect{x}_{t-1}$ given the current noisy state $\vect{x}_t$ and timestep $t$. The mean $\boldsymbol{\mu}_{\theta}(\vect{x}_t, t)$ and covariance $\boldsymbol{\Sigma}_{\theta}(\vect{x}_t, t)$ are learned functions that capture the complex dependencies needed to reverse the noise corruption, effectively learning to "denoise" the data step by step.

\subsection{Training}

Predict noise $\boldsymbol{\epsilon}_{\theta}(\vect{x}_t, t)$ at each step:
\begin{equation}
\mathcal{L} = \mathbb{E}_{t, \vect{x}_0, \boldsymbol{\epsilon}}\left[\|\boldsymbol{\epsilon} - \boldsymbol{\epsilon}_{\theta}(\vect{x}_t, t)\|^2\right]
\end{equation}

The training objective simplifies the complex reverse process by focusing on noise prediction rather than directly modeling the reverse distribution. The loss function $\mathcal{L}$ trains the network $\boldsymbol{\epsilon}_{\theta}(\vect{x}_t, t)$ to predict the noise $\boldsymbol{\epsilon}$ that was added to the clean data $\vect{x}_0$ to produce the noisy observation $\vect{x}_t$. This approach leverages the fact that the forward process is analytically tractable, allowing efficient computation of the training targets. The expectation is taken over random timesteps $t$, clean data samples $\vect{x}_0$, and noise realizations $\boldsymbol{\epsilon}$, ensuring the model learns to denoise across all noise levels and data types.

\subsection{Sampling}

Start from noise and iteratively denoise:
\begin{equation}
\vect{x}_{t-1} = \frac{1}{\sqrt{\alpha_t}}\left(\vect{x}_t - \frac{1-\alpha_t}{\sqrt{1-\bar{\alpha}_t}}\boldsymbol{\epsilon}_{\theta}(\vect{x}_t, t)\right) + \sigma_t \vect{z}
\end{equation}

The sampling process begins with pure Gaussian noise $\vect{x}_T \sim \mathcal{N}(\boldsymbol{0}, \mat{I})$ and iteratively applies the learned denoising function to generate clean samples. The sampling equation combines the predicted noise $\boldsymbol{\epsilon}_{\theta}(\vect{x}_t, t)$ with the current noisy state $\vect{x}_t$ to estimate the previous timestep $\vect{x}_{t-1}$. The coefficients $\alpha_t$ and $\bar{\alpha}_t$ are derived from the noise schedule and ensure consistency with the forward process, while $\sigma_t \vect{z}$ adds stochasticity to prevent the sampling from becoming deterministic. This iterative denoising process gradually transforms noise into realistic data samples.

\subsection{The Algorithm}

The diffusion model sampling algorithm can be expressed as follows:

\begin{algorithm}[h]
\caption{Diffusion Model Sampling Algorithm}
\begin{algorithmic}[1]
\State \textbf{Input:} Trained denoising network $\boldsymbol{\epsilon}_{\theta}(\vect{x}_t, t)$, noise schedule $\{\alpha_t, \bar{\alpha}_t\}_{t=1}^T$
\State \textbf{Output:} Generated sample $\vect{x}_0$
\State
\State Sample initial noise: $\vect{x}_T \sim \mathcal{N}(\boldsymbol{0}, \mat{I})$
\For{$t = T, T-1, \ldots, 1$}
    \State Predict noise: $\hat{\boldsymbol{\epsilon}} = \boldsymbol{\epsilon}_{\theta}(\vect{x}_t, t)$
    \State Compute denoised estimate: $\vect{x}_{t-1} = \frac{1}{\sqrt{\alpha_t}}\left(\vect{x}_t - \frac{1-\alpha_t}{\sqrt{1-\bar{\alpha}_t}}\hat{\boldsymbol{\epsilon}}\right)$
    \If{$t > 1$}
        \State Add stochasticity: $\vect{x}_{t-1} = \vect{x}_{t-1} + \sigma_t \vect{z}$ where $\vect{z} \sim \mathcal{N}(\boldsymbol{0}, \mat{I})$
    \EndIf
\EndFor
\State \textbf{return} $\vect{x}_0$
\end{algorithmic}
\end{algorithm}

\subsection{Advantages}

Diffusion models have emerged as the state-of-the-art approach for high-quality generative modeling, powering breakthrough applications like DALL-E 2, Stable Diffusion, and Midjourney. Their success stems from several key advantages that address fundamental limitations of previous approaches. Unlike GANs, diffusion models provide stable training dynamics without the delicate balance required between generator and discriminator, making them more reliable for large-scale applications.

The strong theoretical foundations of diffusion models provide principled guarantees about convergence and sample quality, addressing the theoretical gaps that plagued earlier generative approaches. The iterative denoising process naturally supports conditioning on various modalities, enabling text-to-image generation, image editing, and other controlled generation tasks. This flexibility, combined with their proven ability to generate photorealistic images and coherent text, has established diffusion models as the dominant paradigm for modern generative AI systems.

% \subsection{Visual aids}
% \addcontentsline{toc}{subsubsection}{Visual aids (diffusion)}

% \begin{figure}[h]
%   \centering
%   \begin{tikzpicture}
%     \begin{axis}[
%       width=0.48\textwidth,height=0.36\textwidth,
%       xlabel={Step $t$}, ylabel={Noise level}, grid=both]
%       \addplot[bookpurple,very thick] coordinates{(0,0.0) (10,0.2) (20,0.4) (30,0.6) (40,0.8) (50,1.0)};
%       \addplot[bookred,very thick,dashed] coordinates{(50,1.0) (40,0.8) (30,0.6) (20,0.4) (10,0.2) (0,0.0)};
%     \end{axis}
%   \end{tikzpicture}
%   \caption{Forward (solid) and reverse (dashed) diffusion noise schedules (illustrative).}
%   \label{fig:diffusion-schedule}
% \end{figure}

% \subsection{Notes and references}

% See \textcite{Ho2020,Prince2023} for DDPM training and sampling.


% Chapter 20, Section 5

\section{Applications and Future Directions \difficultyInline{advanced}}
\label{sec:generative-applications}

Generative models are transforming creative industries, scientific discovery, and everyday applications by enabling the creation of realistic synthetic content across multiple modalities.

\subsection{Current Applications}

The current landscape of generative model applications spans across multiple domains, each demonstrating the transformative potential of these technologies. Image generation has reached unprecedented levels of quality and controllability, with systems like DALL-E and Stable Diffusion enabling users to create photorealistic images from text descriptions. These models have revolutionized creative workflows, allowing artists and designers to rapidly prototype concepts, generate variations, and explore creative possibilities that would be impossible through traditional means.

Text generation has been equally transformative, with large language models like the GPT family demonstrating remarkable capabilities in natural language understanding and generation. These models have found applications in code generation, creative writing, and conversational AI, fundamentally changing how we interact with computers and access information. The ability to generate coherent, contextually appropriate text has opened new possibilities for content creation, education, and human-computer interaction.

Audio and speech generation have advanced significantly, with text-to-speech systems achieving near-human quality and music generation models creating original compositions in various styles. Voice conversion technologies enable personalized speech synthesis, while music generation models assist composers and content creators in exploring new musical ideas. Video generation represents the next frontier, with models capable of predicting future frames, synthesizing new video content, and creating animations that were previously impossible to generate automatically.

Scientific applications of generative models are particularly promising, with models being used to design novel molecules for drug discovery, predict protein structures, and discover new materials with desired properties. These applications demonstrate how generative models can accelerate scientific discovery by exploring vast spaces of possibilities that would be impractical to investigate through traditional experimental approaches.

\subsection{Future Directions}

The future of generative models promises even more sophisticated capabilities and broader applications. Controllability represents a key frontier, where researchers are developing methods for fine-grained control over generation processes, enabling users to specify not just what to generate, but how to generate it with precise control over style, composition, and semantic attributes. This enhanced controllability will make generative models more useful for professional applications where specific requirements must be met.

Efficiency improvements are crucial for making generative models more accessible and practical. Current models often require significant computational resources and time for generation, limiting their widespread adoption. Future developments will focus on faster sampling algorithms, more efficient architectures, and smaller models that can run on consumer hardware while maintaining high quality. Multi-modal integration represents another exciting direction, where unified models will seamlessly work across text, images, audio, and video, enabling more natural and intuitive human-computer interaction.

The incorporation of logical reasoning capabilities will enable generative models to produce more coherent and contextually appropriate outputs, while improved safety mechanisms will prevent the generation of harmful or inappropriate content. Better evaluation metrics will provide more reliable ways to assess generation quality, enabling more systematic progress in the field and better comparison between different approaches.

\subsection{Societal Impact}

The widespread adoption of generative models brings both tremendous opportunities and significant challenges that society must carefully navigate. Copyright and intellectual property issues have become increasingly complex as models are trained on vast datasets containing copyrighted material, raising questions about ownership, attribution, and fair use. The ability to generate highly realistic content has also created new challenges around misinformation and deepfakes, requiring the development of robust detection methods and media literacy education.

The potential for job displacement in creative fields is a legitimate concern, as generative models become increasingly capable of producing professional-quality content. However, these tools also democratize access to creative capabilities, enabling individuals without traditional artistic training to express their ideas visually and musically. The environmental cost of training large generative models is substantial, with some models requiring enormous computational resources that contribute to carbon emissions, highlighting the need for more efficient training methods and renewable energy sources.

Ensuring equitable access to these powerful technologies is crucial for preventing the exacerbation of existing inequalities. The benefits of generative models should be available to all members of society, not just those with access to expensive hardware or specialized knowledge. Responsible development of these technologies requires careful consideration of these societal implications, balancing the tremendous potential for positive impact with the need to address legitimate concerns about misuse and unintended consequences.


% \subsection{Visual aids}
% \addcontentsline{toc}{subsubsection}{Visual aids (applications)}

% \begin{figure}[h]
%   \centering
%   \begin{tikzpicture}
%     \begin{axis}[
%       width=0.48\textwidth,height=0.36\textwidth,
%       ybar, bar width=10pt, grid=both,
%       xlabel={Domain}, ylabel={Adoption (example)}, xtick=data,
%       xticklabels={Image,Text,Audio,Video,Science}]
%       \addplot[bookpurple,fill=bookpurple!40] coordinates{(1,9) (2,8) (3,6) (4,7) (5,5)};
%     \end{axis}
%   \end{tikzpicture}
%   \caption{Illustrative adoption levels across domains.}
%   \label{fig:gen-apps}
% \end{figure}

% \subsection{References}

% See \textcite{GoodfellowEtAl2016,Prince2023,Ho2020} for surveys of applications and frontiers in generative modeling.


% Chapter 20: Real World Applications

\section{Real World Applications}
\label{sec:generative-real-world}


Deep generative models create new data samples, enabling applications from content creation to scientific discovery. Recent advances in GANs, VAEs, and diffusion models have made generation remarkably realistic and controllable.

\subsection{Creative Content Generation}

The realm of creative content generation has been fundamentally transformed by generative models, ushering in a new era of AI-powered creativity and design. AI art and design tools like Midjourney, DALL-E, and Stable Diffusion have democratized visual content creation, enabling anyone to produce professional-quality images from simple text descriptions. Designers leverage these tools for rapid prototyping, generating dozens of concept variations in minutes rather than hours of manual work, while artists use them as creative partners, combining AI-generated elements with traditional techniques to explore new artistic possibilities.

Music composition has been revolutionized by generative models that create original compositions in various styles, from background scores for videos to experimental pieces. These services generate royalty-free music customized to specific moods, tempos, and instrumentation, providing musicians with inspiration and content creators with affordable custom soundtracks. The ability to generate music on demand has opened new possibilities for personalized audio experiences and creative exploration.

Architectural and product design have also benefited from generative models that can explore vast design spaces, proposing innovative variations on building layouts and product designs. Architects use these tools to generate floor plan alternatives that consider multiple constraints like lighting and space efficiency, while product designers can rapidly iterate through form variations, dramatically accelerating the creative process from initial concept to final prototype.

\subsection{Scientific Discovery}

The application of generative models to scientific discovery represents one of the most promising frontiers in AI research, with the potential to accelerate breakthroughs across multiple disciplines. Drug molecule design has been revolutionized by generative models that can propose novel drug candidates with specific desired properties, including optimal binding to target proteins, favorable safety profiles, and ease of synthesis. This approach explores chemical space far more efficiently than traditional trial-and-error synthesis methods, potentially accelerating drug discovery timelines and reducing costs for pharmaceutical companies developing treatments for cancer, infectious diseases, and other conditions.

Materials science has benefited tremendously from generative models that can design new materials with precisely specified properties, from stronger alloys for aerospace applications to more efficient batteries and solar cells for clean energy technologies. These models learn complex relationships between molecular structure and material properties, enabling researchers to propose novel materials for experimental validation that would be impossible to discover through traditional methods. This capability could dramatically accelerate the development of technologies needed for clean energy and sustainability.

Protein structure prediction and design represent another area where generative models are making transformative contributions. These models help predict how proteins fold into their functional three-dimensional structures and design proteins with novel functions for industrial processes, vaccine development, and therapeutic applications. The success of AlphaFold in protein structure prediction demonstrates how generative models can advance our understanding of biological systems and accelerate the development of new treatments and technologies.

\subsection{Data Augmentation and Synthesis}

The ability of generative models to create synthetic training data has become a crucial tool for addressing data scarcity and privacy concerns across multiple domains. Synthetic medical images represent a particularly important application, where generative models create realistic training data that doesn't correspond to real patients, enabling the development of better diagnostic models while protecting patient privacy. This approach is especially valuable for rare diseases where real data is limited, helping to address data imbalances that can bias machine learning models and improving their performance on underrepresented conditions.

Simulation for autonomous vehicles has emerged as another critical application, where generative models create realistic synthetic driving scenarios including rare but dangerous events like pedestrians jaywalking or vehicles running red lights. Self-driving cars can train on these synthetic scenarios to prepare for dangerous situations without risking real-world testing, addressing the "long tail" of rare but critical edge cases that are essential for safe autonomous operation. This approach enables more comprehensive training than would be possible with real-world data alone.

Video game content generation has been transformed by generative models that can create textures, terrain, character models, and even entire game levels. This capability reduces development costs and time while dramatically increasing content variety, enabling procedural generation that creates unique experiences for each player rather than requiring manual crafting of every asset. The result is more dynamic and engaging gaming experiences that can adapt to player preferences and behaviors.

\subsection{Personalization and Adaptation}

The personalization capabilities of generative models are creating new possibilities for customized content that adapts to individual preferences and needs. Avatar creation has become increasingly sophisticated, with apps generating personalized avatars from photos that maintain recognizable features while providing stylized or cartoon representations. These avatars appear in messaging apps, games, and virtual meetings, offering fun and privacy-conscious ways for users to represent themselves in digital spaces.

Text-to-speech personalization represents a particularly meaningful application, where generative models can create natural-sounding speech in a user's own voice from text input. This technology is invaluable for people who have lost their voice due to illness, allowing them to preserve their vocal identity and communicate naturally. It also enables personalized audiobook narration and accessible content delivery in preferred voices, making information more accessible to diverse audiences.

Style transfer and image editing applications have made sophisticated image manipulation accessible to everyone, from professional photographers to casual social media users. These tools can apply artistic styles to photos, change seasons in landscape photography, or realistically age or de-age faces, enabling creative expression and visual storytelling that was previously limited to those with advanced technical skills.

\subsection{Transformative Impact}

The transformative impact of generative models extends far beyond impressive technical demonstrations, fundamentally changing how we approach creative work, scientific discovery, and everyday applications. The democratization of creative tools represents perhaps the most significant shift, making sophisticated artistic and design capabilities accessible to everyone, not just those with years of training and expertise. This accessibility is breaking down barriers to creative expression and enabling new forms of artistic collaboration between humans and AI.

The acceleration of creative and scientific processes through rapid iteration and exploration of vast design spaces is another key transformation. Researchers and creators can now explore possibilities that would be impossible to investigate through traditional methods, dramatically reducing the time from concept to realization. This acceleration is particularly valuable in scientific discovery, where generative models can explore complex domains like chemistry and biology to find solutions that might take years to discover through conventional approaches.

The synthesis capabilities of generative models, particularly their ability to create training data and simulations that would otherwise be unavailable, are opening new possibilities for machine learning and AI development. These capabilities are essential for addressing data scarcity, privacy concerns, and the need for diverse training examples across multiple domains. The practical impact of these applications demonstrates that generative models are not just impressive demonstrations but essential tools that are already transforming how we work, create, and discover.

% Index entries
\index{applications!content generation}
\index{applications!scientific discovery}
\index{applications!data augmentation}
\index{generative models!applications}


% Chapter summary and problems
% Key Takeaways for Chapter 20

\section*{Key Takeaways}
\addcontentsline{toc}{section}{Key Takeaways}

\begin{keytakeaways}
\begin{itemize}[leftmargin=2em]
    \item \textbf{Model families} differ in training signal and guarantees: VAEs use likelihood-based training with explicit density modeling, GANs employ adversarial training for implicit modeling, flows provide exact likelihoods through invertible transformations, and diffusion models learn to reverse noise corruption processes.
    \item \textbf{Evaluation} must consider likelihood, fidelity, diversity, and downstream utility: Different metrics capture different aspects of generation quality, and the choice of evaluation method should align with the intended application and user requirements.
    \item \textbf{Trade-offs} are inevitable: choose for the target application: Each approach offers distinct advantages—VAEs provide stable training and exact likelihoods, GANs excel at sample quality, flows enable exact sampling, and diffusion models achieve state-of-the-art results with strong theoretical foundations.
    \item \textbf{Training dynamics} vary significantly across approaches: VAEs and flows use standard maximum likelihood optimization, GANs require careful balance between generator and discriminator, while diffusion models benefit from stable training with principled noise schedules.
    \item \textbf{Applications} drive architectural choices: The specific requirements of creative content generation, scientific discovery, data augmentation, and personalization should guide the selection of appropriate generative modeling approaches and evaluation metrics.
\end{itemize}
\end{keytakeaways}



% Exercises (Exercises) for Chapter 20

\section*{Exercises}
\addcontentsline{toc}{section}{Exercises}

\subsection*{Easy}

\begin{problem}[VAE vs. GAN]
Compare the training objectives of VAEs and GANs.

\textbf{Hint:} Likelihood-based vs. adversarial.
\end{problem}

\begin{problem}[Normalising Flow Invertibility]
Why must normalising flows be invertible?

\textbf{Hint:} Exact likelihood computation via change of variables.
\end{problem}

\begin{problem}[Diffusion Process]
Describe the forward diffusion process in diffusion models.

\textbf{Hint:} Gradual addition of Gaussian noise.
\end{problem}

\begin{problem}[Sampling Speed]
Compare sampling speed across VAEs, GANs, and diffusion models.

\textbf{Hint:} Single pass vs. iterative refinement.
\end{problem}

\subsection*{Medium}

\begin{problem}[Flow Jacobian]
Derive the change-of-variables formula for a normalising flow.

\textbf{Hint:} $\log p(x) = \log p(z) + \log|\det \frac{\partial f}{\partial z}|$.
\end{problem}

\begin{problem}[Denoising Score Matching]
Explain how diffusion models learn the score function.

\textbf{Hint:} Predict noise; connection to $\nabla_x \log p(x)$.
\end{problem}

\subsection*{Hard}

\begin{problem}[Coupling Layer Design]
Design an invertible coupling layer and prove its properties.

\textbf{Hint:} Affine transformations; partition dimensions.
\end{problem}

\begin{problem}[Guidance Trade-off]
Analyse the trade-off between sample quality and diversity in classifier-free guidance.

\textbf{Hint:} Guidance scale; conditional vs. unconditional scores.
\end{problem}


\begin{problem}[Advanced Topic 1]
Explain a key concept from this chapter and its practical applications.

\textbf{Hint:} Consider the theoretical foundations and real-world implications.
\end{problem}

\begin{problem}[Advanced Topic 2]
Analyse the relationship between different techniques covered in this chapter.

\textbf{Hint:} Look for connections and trade-offs between methods.
\end{problem}

\begin{problem}[Advanced Topic 3]
Design an experiment to test a hypothesis related to this chapter's content.

\textbf{Hint:} Consider experimental design, metrics, and potential confounding factors.
\end{problem}

\begin{problem}[Advanced Topic 4]
Compare different approaches to solving a problem from this chapter.

\textbf{Hint:} Consider computational complexity, accuracy, and practical considerations.
\end{problem}

\begin{problem}[Advanced Topic 5]
Derive a mathematical relationship or prove a theorem from this chapter.

\textbf{Hint:} Start with the definitions and work through the logical steps.
\end{problem}

\begin{problem}[Advanced Topic 6]
Implement a practical solution to a problem discussed in this chapter.

\textbf{Hint:} Consider the implementation details and potential challenges.
\end{problem}

\begin{problem}[Advanced Topic 7]
Evaluate the limitations and potential improvements of techniques from this chapter.

\textbf{Hint:} Consider both theoretical limitations and practical constraints.
\end{problem}

