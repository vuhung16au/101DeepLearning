% Chapter 5, Section 05

\section{k-Nearest Neighbors \difficultyInline{intermediate}}
\label{sec:knn}

\textbf{k-Nearest Neighbors} (k-NN) is a simple yet powerful non-parametric algorithm that makes predictions based on the similarity to training examples. It's called "lazy learning" because it doesn't build a model during training—all computation happens at prediction time.

\subsection{Intuition and Motivation}

The k-NN algorithm is based on a simple principle: "similar things are close to each other." If you want to predict whether someone will like a movie, look at what movies similar people (with similar tastes) liked. If you want to predict house prices, look at prices of similar houses in the neighborhood.

The key insight is that we can make good predictions by finding the most similar examples in our training data and using their outcomes as a guide.

\begin{figure}[htbp]
\centering
\begin{tikzpicture}[scale=0.8]
\begin{axis}[
    xlabel={Feature 1},
    ylabel={Feature 2},
    title={k-NN Classification Example (k=3)},
    grid=major,
    width=10cm,
    height=8cm
]
% Generate some sample data
\addplot[only marks, mark=*, mark size=3pt, color=bookpurple] coordinates {
    (1,1) (1.5,1.5) (2,1) (2.5,2) (3,1.5)
};
\addplot[only marks, mark=square*, mark size=3pt, color=bookred] coordinates {
    (2,3) (2.5,3.5) (3,3) (3.5,3.5) (4,3)
};
% Add query point
\addplot[only marks, mark=triangle*, mark size=4pt, color=bookblack] coordinates {
    (2.5,2.5)
};
% Add circles around query point
\draw[thick, color=bookblack, dashed] (2.5,2.5) circle (1.2);
\node at (axis cs:2.5,2.5) [anchor=south] {Query point};
\node at (axis cs:2.5,1.3) [anchor=west] {k=3 neighbors};
\legend{Class 0, Class 1, Query point, Neighbors}
\end{axis}
\end{tikzpicture}
\caption{k-NN finds the k nearest neighbors to a query point and makes predictions based on their labels. Here, with k=3, the query point would be classified as class 0 (2 out of 3 neighbors are class 0).}
\label{fig:knn-example}
\end{figure}

\subsection{Algorithm}

For a query point $\vect{x}$:

\begin{enumerate}
    \item Find the $k$ closest training examples based on a distance metric
    \item For classification: return the majority class among the $k$ neighbors
    \item For regression: return the average of the target values of the $k$ neighbors
\end{enumerate}

\subsubsection{Mathematical Formulation}

For classification, the predicted class is:
\begin{equation}
\hat{y} = \arg\max_{c} \sum_{i=1}^{k} \mathbb{I}(y^{(i)} = c)
\end{equation}

For regression, the predicted value is:
\begin{equation}
\hat{y} = \frac{1}{k} \sum_{i=1}^{k} y^{(i)}
\end{equation}

where $y^{(i)}$ are the target values of the $k$ nearest neighbors.

\subsection{Distance Metrics}

The choice of distance metric significantly affects k-NN performance. Here are the most common ones:

\subsubsection{Euclidean Distance}

\begin{equation}
d(\vect{x}, \vect{x}') = \sqrt{\sum_{i=1}^{d} (x_i - x_i')^2}
\end{equation}

This is the most common choice, measuring the straight-line distance between points.

\subsubsection{Manhattan Distance}

\begin{equation}
d(\vect{x}, \vect{x}') = \sum_{i=1}^{d} |x_i - x_i'|
\end{equation}

Also known as L1 distance, it measures the sum of absolute differences. Useful when features have different scales.

\subsubsection{Minkowski Distance}

\begin{equation}
d(\vect{x}, \vect{x}') = \left(\sum_{i=1}^{d} |x_i - x_i'|^p\right)^{1/p}
\end{equation}

This is a generalization where:
\begin{itemize}
    \item $p = 1$: Manhattan distance
    \item $p = 2$: Euclidean distance
    \item $p \to \infty$: Chebyshev distance (maximum coordinate difference)
\end{itemize}

\subsection{Choosing k}

The choice of $k$ is crucial and involves a bias-variance trade-off:

\begin{itemize}
    \item \textbf{Small $k$ (e.g., k=1):}
    \begin{itemize}
        \item Low bias, high variance
        \item Flexible decision boundary
        \item Sensitive to noise and outliers
        \item May overfit to training data
    \end{itemize}
    \item \textbf{Large $k$ (e.g., k=n):}
    \begin{itemize}
        \item High bias, low variance
        \item Smooth decision boundary
        \item May miss local patterns
        \item Underfits the data
    \end{itemize}
\end{itemize}

\begin{figure}[htbp]
\centering
\begin{tikzpicture}[scale=0.8]
\begin{axis}[
    xlabel={k value},
    ylabel={Error Rate},
    title={Effect of k on k-NN Performance},
    grid=major,
    width=10cm,
    height=6cm,
    xmin=1, xmax=20
]
% Generate a U-shaped curve
\addplot[thick, color=bookpurple, domain=1:20] {0.1 + 0.05*(x-5)^2/25};
\node at (axis cs:5,0.1) [anchor=south] {Optimal k};
\node at (axis cs:1,0.15) [anchor=east] {High variance};
\node at (axis cs:20,0.15) [anchor=west] {High bias};
\end{axis}
\end{tikzpicture}
\caption{The effect of k on k-NN performance. Small k leads to high variance (overfitting), while large k leads to high bias (underfitting). The optimal k is typically found through cross-validation.}
\label{fig:knn-k-selection}
\end{figure}

\subsection{Weighted k-NN}

Instead of giving equal weight to all $k$ neighbors, we can weight them by their distance:

\begin{equation}
\hat{y} = \frac{\sum_{i=1}^{k} w_i y^{(i)}}{\sum_{i=1}^{k} w_i}
\end{equation}

where $w_i = \frac{1}{d(\vect{x}, \vect{x}^{(i)})}$ is the weight based on distance.

\subsection{Computational Considerations}

k-NN has several computational characteristics:

\subsubsection{Training Time}
\begin{itemize}
    \item \textbf{No training time:} k-NN is a lazy learner
    \item \textbf{Memory usage:} Must store all training examples
    \item \textbf{Scalability:} Performance degrades with large datasets
\end{itemize}

\subsubsection{Prediction Time}
\begin{itemize}
    \item \textbf{Naive approach:} $O(n \cdot d)$ for each prediction
    \item \textbf{Optimized approaches:} Can be reduced using data structures
\end{itemize}

\subsubsection{Speedup Techniques}

\textbf{KD-Trees:}
\begin{itemize}
    \item Partition space into regions
    \item Reduce search time to $O(\log n)$ in low dimensions
    \item Less effective in high dimensions
\end{itemize}

\textbf{Ball Trees:}
\begin{itemize}
    \item Use hyperspheres instead of hyperplanes
    \item Better for high-dimensional data
    \item More complex to implement
\end{itemize}

\textbf{Locality Sensitive Hashing (LSH):}
\begin{itemize}
    \item Approximate nearest neighbor search
    \item Significant speedup for very large datasets
    \item May sacrifice some accuracy
\end{itemize}

\subsection{Advantages and Limitations}

\textbf{Advantages:}
\begin{itemize}
    \item \textbf{Simple:} Easy to understand and implement
    \item \textbf{No assumptions:} Works with any data type
    \item \textbf{Non-parametric:} No model to fit
    \item \textbf{Local patterns:} Can capture complex decision boundaries
    \item \textbf{Incremental:} Easy to add new training examples
\end{itemize}

\textbf{Limitations:}
\begin{itemize}
    \item \textbf{Computational cost:} Slow for large datasets
    \item \textbf{Memory usage:} Must store all training data
    \item \textbf{Sensitive to irrelevant features:} All features are treated equally
    \item \textbf{Curse of dimensionality:} Performance degrades in high dimensions
    \item \textbf{No interpretability:} Hard to understand why a prediction was made
\end{itemize}

\subsection{Curse of Dimensionality}

In high-dimensional spaces, k-NN faces the \textbf{curse of dimensionality}:

\begin{itemize}
    \item \textbf{Distance concentration:} All points become roughly equidistant
    \item \textbf{Empty space:} Most of the space is empty
    \item \textbf{Irrelevant features:} Performance degrades with irrelevant features
\end{itemize}

\begin{example}
In a 1000-dimensional space, even if only 10 features are relevant, the 990 irrelevant features can dominate the distance calculations, making k-NN ineffective.
\end{example}

\subsection{Feature Selection and Scaling}

To improve k-NN performance:

\begin{itemize}
    \item \textbf{Feature selection:} Remove irrelevant features
    \item \textbf{Feature scaling:} Normalize features to the same scale
    \item \textbf{Dimensionality reduction:} Use PCA or other techniques
    \item \textbf{Distance weighting:} Weight features by importance
\end{itemize}

