% Chapter 18, Section 3

\section{Noise-Contrastive Estimation \\difficultyInline{advanced}}
\label{sec:nce}


\subsection{Key Idea}

Turn density estimation into binary classification:
\begin{itemize}
    \item Distinguish data samples from noise samples
    \item Avoids computing partition function
\end{itemize}

\subsection{NCE Objective}

\begin{equation}
\mathcal{L} = \mathbb{E}_{p_{\text{data}}}[\log h(\vect{x})] + k \cdot \mathbb{E}_{p_{\text{noise}}}[\log(1-h(\vect{x}))]
\end{equation}

where:
\begin{equation}
h(\vect{x}) = \frac{p_{\text{model}}(\vect{x})}{p_{\text{model}}(\vect{x}) + k \cdot p_{\text{noise}}(\vect{x})}
\end{equation}

\subsection{Applications}

\begin{itemize}
    \item Word embeddings (word2vec)
    \item Language models
    \item Energy-based models
\end{itemize}

% \subsection{Visual aids}
% \addcontentsline{toc}{subsubsection}{Visual aids (NCE)}

% \begin{figure}[h]
%   \centering
%   \begin{tikzpicture}
%     \begin{axis}[
%       width=0.48\textwidth,height=0.36\textwidth,
%       xlabel={Score $h(x)$}, ylabel={Density}, grid=both]
%       \addplot[bookpurple,very thick,domain=0:1,samples=100]{4*x*(1-x)};
%       \addplot[bookred,very thick,dashed,domain=0:1,samples=100]{2*(1-x)};
%     \end{axis}
%   \end{tikzpicture}
%   \caption{Schematic of classifier scores for data (solid) vs. noise (dashed).}
%   \label{fig:nce-scores}
% \end{figure}

\subsection{Notes and references}

NCE as a technique to bypass partition functions is discussed in \textcite{GoodfellowEtAl2016,Prince2023}.

