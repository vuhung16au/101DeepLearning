% Chapter 15, Section 1

\section{What Makes a Good Representation? \difficultyInline{intermediate}}
\label{sec:good-representations}

\subsection{Desirable Properties}

\textbf{Disentanglement:} Different factors of variation are separated
\begin{itemize}
    \item Changes in one dimension affect one factor
    \item Easier interpretation and manipulation
\end{itemize}

\textbf{Invariance:} Representation unchanged under irrelevant transformations
\begin{itemize}
    \item Translation, rotation invariance for objects
    \item Speaker invariance for speech content
\end{itemize}

\textbf{Smoothness:} Similar inputs have similar representations
\begin{itemize}
    \item Enables generalization
    \item Supports interpolation
\end{itemize}

\textbf{Sparsity:} Few features active for each input
\begin{itemize}
    \item Computational efficiency
    \item Interpretability
\end{itemize}

\subsection{Manifold Hypothesis}

Natural data lies on low-dimensional manifolds embedded in high-dimensional space.

Deep learning learns to:
\begin{itemize}
    \item Discover the manifold structure
    \item Map data to meaningful coordinates on manifold
\end{itemize}

% \subsection{Visual aids}
% \addcontentsline{toc}{subsubsection}{Visual aids (representations)}

% \begin{figure}[h]
%   \centering
%   \begin{tikzpicture}
%     \begin{axis}[
%       width=0.48\textwidth,height=0.36\textwidth,
%       xlabel={$x_1$}, ylabel={$x_2$}, grid=both]
%       \addplot[bookpurple,very thick,domain=-2:2] {sin(deg(x))/1.5};
%       \addplot[bookred,very thick,domain=-2:2] {0.0};
%     \end{axis}
%   \end{tikzpicture}
%   \caption{Complex data manifold (purple) and a learned linearized coordinate (red) locally.}
%   \label{fig:manifold}
% \end{figure}

\subsection{Notes and references}

Desirable properties are discussed in modern DL texts; disentanglement and invariance connect to inductive biases and data augmentation \textcite{GoodfellowEtAl2016,Prince2023}.
