% Chapter 8, Section 1

\section{Gradient Descent Variants \difficultyInline{intermediate}}
\label{sec:gd-variants}

\subsection{Intuition: Following the Steepest Downhill Direction}

Imagine standing on a foggy hillside trying to reach the valley. You feel the slope under your feet and take a step downhill. \textbf{Batch gradient descent} measures the average slope using the whole landscape (dataset) before each step: accurate but slow to gauge. \textbf{Stochastic gradient descent (SGD)} feels the slope at a single point: fast, but noisy. \textbf{Mini-batch} is a compromise: feel the slope at a handful of nearby points to get a reliable yet efficient direction. This trade-off underlies most practical training regimes.\index{optimization!gradient descent}\index{mini-batch}\gls{mini-batch}

Historical note: Early neural network training widely used batch gradient descent, but the rise of large datasets and GPUs made mini-batch SGD the de facto standard \cite{GoodfellowEtAl2016,Prince2023}.

\subsection{Batch Gradient Descent}

Computes gradient using entire training set:
\begin{equation}
\vect{\theta}_{t+1} = \vect{\theta}_t - \alpha \nabla_{\vect{\theta}} \frac{1}{n} \sum_{i=1}^{n} L(\vect{\theta}, \vect{x}^{(i)}, y^{(i)})
\end{equation}

Characteristics:\index{gradient descent!batch}
\begin{itemize}
    \item Deterministic and low-variance updates; well-suited for convex exercises.
    \item Each step uses all \(n\) examples, incurring high computation and memory costs for large datasets.
    \item In deep, non-convex landscapes, stable but often too slow to respond to curvature changes.
\end{itemize}

Example (convex quadratic): Let \(L(\theta)=\tfrac{1}{2}a\theta^2\) with gradient \(a\theta\). Batch GD with step size \(\alpha\) yields \(\theta_{t+1}=(1-\alpha a)\theta_t\). Convergence occurs if \(0<\alpha<\tfrac{2}{a}\), illustrating the learning-rate–curvature interaction.\index{learning rate}\glsadd{gradient-descent}

When to use:
\begin{itemize}
    \item Small datasets that fit comfortably in memory.
    \item Convex or nearly convex objectives (e.g., linear/logistic regression) when precise convergence is desired.
\end{itemize}

Historical context: Full-batch methods trace back to classical numerical optimization. For massive datasets, stochastic approximations dating to \cite{RobbinsMonro1951} became essential; modern deep learning typically favors mini-batches \cite{GoodfellowEtAl2016,WebOptimizationDLBook,D2LChapterOptimization}.

\subsection{Stochastic Gradient Descent (SGD)}\index{SGD@stochastic gradient descent}

Uses a single random example per update:
\begin{equation}
\vect{\theta}_{t+1} = \vect{\theta}_t - \alpha \nabla_{\vect{\theta}} L(\vect{\theta}, \vect{x}^{(i)}, y^{(i)})
\end{equation}

Characteristics:\glsadd{stochastic-gradient-descent}
\begin{itemize}
    \item Very fast, streaming-friendly updates; one pass can begin learning immediately.
    \item Noisy gradients add exploration, helping traverse saddle points and plateaus.\index{saddle point}\index{plateau}
    \item High variance can hamper precise convergence; diminishing \(\alpha_t\) helps stabilize \cite{RobbinsMonro1951}.
\end{itemize}

Practical tips:
\begin{itemize}
    \item Use a decaying schedule (e.g., \(\alpha_t = \alpha_0/(1+\lambda t)\)) or switch to momentum/Adam later for fine-tuning \cite{WebOptimizationDLBook,D2LChapterOptimization}.
    \item Shuffle examples every epoch to avoid periodic bias.
\end{itemize}

Applications: Early CNN training and many online/streaming scenarios employ SGD due to its simplicity and ability to handle large-scale data. In large vision models, SGD with momentum remains competitive \cite{He2016}.

\subsection{Mini-Batch Gradient Descent}\index{gradient descent!mini-batch}

Balances batch and stochastic approaches:
\begin{equation}
\vect{\theta}_{t+1} = \vect{\theta}_t - \alpha \nabla_{\vect{\theta}} \frac{1}{|\mathcal{B}|} \sum_{i \in \mathcal{B}} L(\vect{\theta}, \vect{x}^{(i)}, y^{(i)})
\end{equation}

where $\mathcal{B}$ is a mini-batch (typically 32–1024 examples depending on model and hardware).\glsadd{mini-batch}

Benefits:
\begin{itemize}
    \item Reduces gradient variance
    \item Efficient GPU utilization
    \item Good balance of speed and stability
\end{itemize}

Further guidance:
\begin{itemize}
    \item Larger batches yield smoother estimates but may require proportionally larger learning rates; the ``linear scaling rule'' is a useful heuristic in some regimes.\index{mini-batch}
    \item Very large batches can hurt generalization unless paired with warmup and appropriate regularization. See \cite{WebOptimizationDLBook,D2LChapterOptimization}.
\end{itemize}

Illustrative example (variance vs. batch size): For a fixed \(\alpha\), increasing \(|\mathcal{B}|\) reduces the variance of the stochastic gradient approximately as \(\mathcal{O}(1/|\mathcal{B}|)\), improving stability but diminishing returns once hardware is saturated.

