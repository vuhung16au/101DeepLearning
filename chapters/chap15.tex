% Chapter 15: Representation Learning

\chapter{Representation Learning}
\label{chap:representation-learning}

This chapter discusses the central challenge of deep learning: learning meaningful representations from data.


\section*{Learning Objectives}
\addcontentsline{toc}{section}{Learning Objectives}

After studying this chapter, you will be able to:

\begin{enumerate}
    \item Define representations and desirable properties (invariance, disentanglement, sparsity).
    \item Compare supervised, self-supervised, and contrastive objectives.
    \item Evaluate representations via linear probes and transfer learning.
    \item Relate information-theoretic perspectives to empirical practice.
\end{enumerate}



\section*{Intuition}
\addcontentsline{toc}{section}{Intuition}

Good representations separate task-relevant factors from nuisance variability, where learning signals that compare positive and negative pairs, or predict masked content, shape geometry in embedding spaces to reflect semantics. For example, in facial recognition systems, a good representation would capture identity-related features like facial structure and bone geometry while being invariant to lighting conditions, pose variations, and facial expressions. A good representation is like a highly efficient, specialized library catalogue designed for a specific purpose, where it organizes and indexes information in a way that makes the most relevant details immediately accessible while filtering out irrelevant noise and redundancy.

\section*{Representation Function}
\addcontentsline{toc}{section}{Representation Function}

Representation learning can be mathematically defined as learning a function that maps raw input data from a high-dimensional space to a lower-dimensional, more informative space. Mathematically, representation learning aims to find a mapping function $\mathbf{f}$ that transforms the input data $\mathbf{x} \in \mathbb{R}^{D}$ into a representation $\mathbf{z} \in \mathbb{R}^{d}$, where typically the input dimension $D$ is much greater than the representation dimension $d$ ($D \gg d$). The key is that $\mathbf{z}$ is designed to capture the salient semantic factors of $\mathbf{x}$ while discarding noise and redundancy, where the function $\mathbf{f}$ is learned by defining an objective function that encourages $\mathbf{z}$ to possess desirable properties like invariance, disentanglement, and sparsity.


% Chapter 15, Section 1

\section{What Makes a Good Representation? \difficultyInline{intermediate}}
\label{sec:good-representations}

In the context of deep learning, a representation is a learned transformation that maps high-dimensional raw data into a lower-dimensional space that captures the essential semantic information needed for downstream tasks, where the goal is to extract meaningful features that are useful for classification, generation, or other machine learning objectives.

\subsection{Desirable Properties}

Disentanglement refers to the property where different factors of variation are separated in the representation space, where changes in one dimension affect only one factor, making the representation easier to interpret and manipulate by allowing independent control over different aspects of the data. Invariance ensures that the representation remains unchanged under irrelevant transformations, such as translation and rotation invariance for objects or speaker invariance for speech content, where the model learns to focus on task-relevant features while ignoring nuisance variability. Smoothness means that similar inputs have similar representations, enabling generalization to unseen data and supporting interpolation between known examples, where the representation space maintains a coherent structure that reflects the underlying data manifold. Sparsity refers to the property where few features are active for each input, providing computational efficiency by reducing the number of computations needed and improving interpretability by focusing on the most important features for each specific input.

\subsection{Manifold Hypothesis}

The manifold hypothesis states that natural data lies on low-dimensional manifolds embedded in high-dimensional space, where despite the high dimensionality of raw data, the underlying structure can be captured by a much smaller number of parameters. Deep learning learns to discover the manifold structure by identifying the low-dimensional subspace that contains the essential information, where the model maps data to meaningful coordinates on the manifold that correspond to the underlying factors of variation in the data, enabling efficient representation and manipulation of complex high-dimensional data.

% \subsection{Visual aids}
% \addcontentsline{toc}{subsubsection}{Visual aids (representations)}

% \begin{figure}[h]
%   \centering
%   \begin{tikzpicture}
%     \begin{axis}[
%       width=0.48\textwidth,height=0.36\textwidth,
%       xlabel={$x_1$}, ylabel={$x_2$}, grid=both]
%       \addplot[bookpurple,very thick,domain=-2:2] {sin(deg(x))/1.5};
%       \addplot[bookred,very thick,domain=-2:2] {0.0};
%     \end{axis}
%   \end{tikzpicture}
%   \caption{Complex data manifold (purple) and a learned linearized coordinate (red) locally.}
%   \label{fig:manifold}
% \end{figure}

\subsection{Notes and references}

Desirable properties are discussed in modern deep learning texts, where disentanglement and invariance connect to inductive biases and data augmentation, representing key milestones in understanding how to design effective representation learning systems. The work by Goodfellow and colleagues has been particularly influential in establishing the theoretical foundations for representation learning, while Prince's contributions have advanced our understanding of how these properties emerge in practice. These models have achieved remarkable success in applications ranging from computer vision and natural language processing to scientific discovery and creative applications, demonstrating their versatility and practical impact in modern machine learning systems.

% Chapter 15, Section 2

\section{Transfer Learning and Domain Adaptation \difficultyInline{intermediate}}
\label{sec:transfer-learning}

Transfer learning and domain adaptation enable the reuse of learned representations across different tasks and domains, where knowledge from a source task is leveraged to improve performance on a target task, even when the target task has limited labeled data or operates in a different domain.

\subsection{Transfer Learning}

Transfer learning leverages knowledge from a source task to improve performance on a target task, where the learned representations from the source task are reused to accelerate learning on the target task. Feature extraction involves pre-training on a large dataset like ImageNet, freezing the convolutional layers to preserve the learned features, and training only the final classification layers on the target task, where this approach is particularly effective when the target task has limited labeled data. Fine-tuning starts with a pre-trained model and continues training on the target task with a lower learning rate to avoid destroying the learned features, where optionally freezing early layers can help preserve the low-level features while allowing the higher-level features to adapt to the target task.

\subsection{Domain Adaptation}

Domain adaptation addresses the challenge when training and test distributions differ, where the model needs to adapt to the target domain while maintaining performance on the source domain. Domain-adversarial training learns domain-invariant features by using an adversarial training procedure that encourages the feature extractor to learn representations that are indistinguishable between source and target domains, where this approach helps the model generalize across different data distributions. Self-training uses confident predictions on the target domain to generate pseudo-labels for unlabeled target data, where the model is iteratively trained on these pseudo-labels to improve its performance on the target domain. Multi-task learning involves joint training on both domains, where the model learns to perform well on both the source and target tasks simultaneously, enabling better transfer of knowledge between domains.

\subsection{Few-Shot Learning}

Few-shot learning addresses the challenge of learning from few examples per class, where the model needs to quickly adapt to new tasks with limited labeled data. Meta-learning approaches like MAML (Model-Agnostic Meta-Learning) learn to learn quickly by training the model to adapt to new tasks with just a few gradient steps, where the meta-learner learns initialization parameters that enable fast adaptation to new tasks. Prototypical networks learn a metric space where examples from the same class are close together and examples from different classes are far apart, where classification is performed by computing distances to class prototypes in this learned metric space. Matching networks use attention-based comparison to find the most similar examples in the support set for each query example, where the model learns to attend to relevant examples and make predictions based on these similarities.

% \subsection{Visual aids}
% \addcontentsline{toc}{subsubsection}{Visual aids (transfer)}

% \begin{figure}[h]
%   \centering
%   \begin{tikzpicture}[>=stealth]
%     \tikzstyle{b}=[draw,rounded corners,align=center,minimum width=2.2cm,minimum height=0.9cm]
%     \node[b,fill=bookpurple!10] at (0,0) (src) {Source model};
%     \node[b,fill=bookpurple!15] at (3.2,0) (freeze) {Freeze early layers};
%     \node[b,fill=bookpurple!20] at (6.4,0) (ft) {Fine-tune head};
%     \draw[->] (src) -- (freeze);
%     \draw[->] (freeze) -- (ft);
%   \end{tikzpicture}
%   \caption{Common transfer pipeline: initialize, freeze, fine-tune.}
%   \label{fig:transfer}
% \end{figure}

% \subsection{References}

% See \textcite{Prince2023,GoodfellowEtAl2016} for practical recipes and pitfalls in transfer and domain adaptation.


% Chapter 15, Section 3

\section{Self-Supervised Learning \difficultyInline{intermediate}}
\label{sec:self-supervised}

Learn representations without manual labels by solving pretext tasks.

\subsection{Pretext Tasks}

\textbf{For images:}
\begin{itemize}
    \item \textbf{Rotation prediction:} Predict rotation angle
    \item \textbf{Jigsaw puzzle:} Arrange shuffled patches
    \item \textbf{Colorization:} Predict colors from grayscale
    \item \textbf{Inpainting:} Fill masked regions
\end{itemize}

\textbf{For text:}
\begin{itemize}
    \item \textbf{Masked language modeling:} Predict masked words (BERT)
    \item \textbf{Next sentence prediction:} Predict if sentences are consecutive
    \item \textbf{Autoregressive generation:} Predict next token (GPT)
\end{itemize}

\subsection{Benefits}

\begin{itemize}
    \item Leverage unlabeled data
    \item Learn general-purpose representations
    \item Often outperforms supervised pre-training
\end{itemize}

% \subsection{Visual aids}
% \addcontentsline{toc}{subsubsection}{Visual aids (self-supervised)}

% \begin{figure}[h]
%   \centering
%   \begin{tikzpicture}
%     \begin{axis}[
%       width=0.48\textwidth,height=0.36\textwidth,
%       xlabel={Pretraining epochs}, ylabel={Linear probe accuracy}, grid=both]
%       \addplot[bookpurple,very thick] coordinates{(10,0.55) (20,0.62) (40,0.68) (80,0.72) (160,0.74)};
%     \end{axis}
%   \end{tikzpicture}
%   \caption{Self-supervised pretraining improves downstream linear probe performance (illustrative).}
%   \label{fig:ssl-probe}
% \end{figure}

% \subsection{References}

% For contrastive and masked pretraining approaches, see \textcite{Chen2020,Devlin2018,Prince2023}.


% Chapter 15, Section 4

\section{Contrastive Learning \difficultyInline{intermediate}}
\label{sec:contrastive-learning}

Contrastive learning learns representations by contrasting positive and negative pairs, where the model learns to pull similar examples together and push dissimilar examples apart in the representation space, enabling the discovery of meaningful semantic structure without explicit supervision.

\subsection{Core Idea}

The core idea of contrastive learning is to maximize agreement between different views of the same data (positive pairs) while minimizing agreement with other data (negative pairs), where this approach forces the model to learn representations that are invariant to irrelevant transformations while being sensitive to semantic differences. This is achieved by training the model to distinguish between positive pairs that should have similar representations and negative pairs that should have different representations, where the contrastive loss encourages the model to learn a representation space where similar examples are close together and dissimilar examples are far apart. The key insight is that by learning to distinguish between positive and negative pairs, the model naturally learns to capture the underlying semantic structure of the data without requiring explicit labels.

\subsection{SimCLR Framework}

The SimCLR framework applies two random augmentations to each image, encodes both views using the same encoder network $\vect{z}_i = f(\vect{x}_i)$ and $\vect{z}_j = f(\vect{x}_j)$, and minimizes the contrastive loss (NT-Xent) to learn representations that are invariant to the applied augmentations. The contrastive loss $\ell_{i,j} = -\log \frac{\exp(\text{sim}(\vect{z}_i, \vect{z}_j)/\tau)}{\sum_{k=1}^{2N} \mathbb{I}_{[k \neq i]} \exp(\text{sim}(\vect{z}_i, \vect{z}_k)/\tau)}$ encourages the model to learn representations where positive pairs have high similarity and negative pairs have low similarity, where $\text{sim}(\cdot, \cdot)$ is cosine similarity and $\tau$ is a temperature parameter that controls the sharpness of the similarity distribution.

\subsection{MoCo (Momentum Contrast)}

MoCo (Momentum Contrast) uses a momentum encoder and a queue of negative samples for efficiency, where the momentum encoder is updated using exponential moving average of the main encoder parameters to maintain consistency in the representation space. The queue stores negative samples from previous batches to provide a large and diverse set of negative examples without requiring large batch sizes, where this approach enables efficient contrastive learning with limited computational resources while maintaining high-quality representations.

\subsection{BYOL (Bootstrap Your Own Latent)}

BYOL (Bootstrap Your Own Latent) surprisingly works without negative samples by using an online network that is updated by gradients, a target network that is updated using momentum, and a prediction head on the online network that learns to predict the target network's output. This approach avoids the need for negative samples by learning to predict the target representation from the online representation, where the target network provides a moving target that encourages the online network to learn consistent representations across different augmentations.

\subsection{Applications}

Contrastive learning has achieved state-of-the-art results in image classification where the learned representations transfer well to downstream classification tasks, object detection where the learned features enable accurate localization and classification of objects, and segmentation where the learned representations capture fine-grained spatial information needed for pixel-level predictions. The approach is particularly valuable for medical imaging with limited labels, where the learned representations can be fine-tuned with minimal labeled data while maintaining high performance on diagnostic tasks.


% \subsection{Visual aids}
% \addcontentsline{toc}{subsubsection}{Visual aids (contrastive)}

% \begin{figure}[h]
%   \centering
%   \begin{tikzpicture}
%     \begin{axis}[
%       width=0.48\textwidth,height=0.36\textwidth,
%       xlabel={Epoch}, ylabel={NT-Xent loss}, grid=both]
%       \addplot[bookpurple,very thick] coordinates{(1,6.0) (10,3.2) (20,2.1) (40,1.4) (80,1.0)};
%     \end{axis}
%   \end{tikzpicture}
%   \caption{Contrastive loss decreasing during training (illustrative).}
%   \label{fig:ntxent}
% \end{figure}

% \subsection{References}

% Contrastive learning frameworks (SimCLR, MoCo, BYOL) underpin modern representation learning \textcite{Chen2020,Prince2023}.


% Chapter 15: Real World Applications

\section{Real World Applications}
\label{sec:representation-real-world}


Representation learning—automatically discovering useful features from raw data—is fundamental to modern deep learning success. Good representations make downstream tasks easier and enable transfer learning across domains.

\subsection{Transfer Learning in Computer Vision}

Transfer learning in computer vision reuses learned representations to save time and data, where medical imaging with limited data benefits from starting with representations learned from millions of general images like ImageNet and then fine-tuning on medical data, enabling networks that learned to recognize textures and shapes in everyday photos to adapt to recognize pathologies in X-rays with just a small medical dataset. Custom object detection for businesses allows retailers to detect their specific products on shelves and manufacturers to identify particular defects by using pre-trained vision models and fine-tuning with just hundreds of examples, where the learned representations of edges, textures, and objects transfer effectively, making custom vision systems practical for small businesses. Wildlife monitoring applications use camera traps to monitor endangered species, generating millions of images where transfer learning enables creating species classifiers with limited labeled examples, accelerating research without requiring biologists to manually label vast datasets.

\subsection{Natural Language Processing}

Learned language representations revolutionize text applications through multilingual models that learn representations capturing meaning across languages, where a model trained on English, Spanish, and Chinese text learns that "cat," "gato," and "mao" represent similar concepts, enabling zero-shot translation and allowing improvements in high-resource languages to benefit low-resource languages automatically. Domain adaptation applications use customer service chatbots that employ language models pre-trained on general text and then fine-tuned on company-specific conversations, where the general language understanding including grammar, reasoning, and world knowledge transfers while fine-tuning adds domain expertise, making sophisticated chatbots feasible without training from scratch. Sentiment analysis for brands allows companies to monitor social media sentiment about their products by using general text representations learned from billions of documents and then adapting to specific brand vocabulary, providing accurate sentiment analysis even for newly launched products.

\subsection{Cross-Modal Representations}

Cross-modal representations learn representations spanning multiple modalities, where image-text search systems like Google Images allow searching photos using text descriptions by requiring representations where images and text descriptions of the same concept are similar, where models learn joint representations by training on millions of image-caption pairs, enabling finding relevant images even for queries with no exact text matches. Video understanding applications use YouTube's recommendation and search systems that learn representations combining visual content, audio, speech transcripts, and metadata, where these multi-modal representations understand videos better than any single modality alone, improving search relevance and recommendations. Accessibility tools for visually impaired users generate descriptions of images on web pages using cross-modal representations trained on image-caption pairs, enabling generating relevant, helpful descriptions automatically and making the web more accessible.

\subsection{Impact of Good Representations}

Good representations matter because they provide data efficiency by enabling the solution of new tasks with less labeled data, where the learned representations capture the essential structure of the data and can be reused across different tasks. They enable generalization by providing better performance on diverse, real-world examples, where the learned representations are robust to variations in the data and can handle unseen examples effectively. Knowledge transfer allows expertise learned on one task to help others, where the learned representations can be fine-tuned for new tasks without starting from scratch. Semantic understanding captures meaningful structure in data, where the learned representations reflect the underlying semantic relationships in the data, enabling better understanding and manipulation of the data.

These applications demonstrate that representation learning is not just a theoretical concept—it's the foundation enabling practical deep learning with limited data and computational resources.

% Index entries
\index{applications!transfer learning}
\index{applications!natural language processing}
\index{applications!cross-modal learning}
\index{representation learning!applications}


% Chapter summary and problems
% Key Takeaways for Chapter 15

\section*{Key Takeaways}
\addcontentsline{toc}{section}{Key Takeaways}

\begin{keytakeaways}
\begin{itemize}[leftmargin=2em]
    \item \textbf{Representation quality} is judged by transfer and linear separability.
    \item \textbf{Self-supervision} shapes embeddings via predictive or contrastive signals.
    \item \textbf{Evaluation matters}: consistent probes and protocols enable fair comparison.
\end{itemize}
\end{keytakeaways}



% Exercises (Exercises) for Chapter 15

\section*{Exercises}
\addcontentsline{toc}{section}{Exercises}

\subsection*{Easy}

\begin{exercisebox}[easy]
\begin{problem}[Encoder-Decoder Symmetry]
Why share architecture between $q(z|x)$ and $p(x|z)$?
\end{problem}
\begin{hintbox}
Computation; conceptual symmetry.
\end{hintbox}
\end{exercisebox}


\begin{exercisebox}[easy]
\begin{problem}[KL in ELBO]
State the role of $D_{KL}(q||p)$ in VAE training.
\end{problem}
\begin{hintbox}
Regularisation; posterior matching.
\end{hintbox}
\end{exercisebox}


\begin{exercisebox}[easy]
\begin{problem}[Reparameterisation Trick]
Explain why reparameterisation enables gradient flow.
\end{problem}
\begin{hintbox}
Sampling vs. deterministic path.
\end{hintbox}
\end{exercisebox}


\begin{exercisebox}[easy]
\begin{problem}[Prior Choice]
Justify Gaussian prior for VAE latents.
\end{problem}
\begin{hintbox}
Tractability; simplicity; universality.
\end{hintbox}
\end{exercisebox}


\subsection*{Medium}

\begin{exercisebox}[medium]
\begin{problem}[ELBO Derivation]
Derive the ELBO from Jensen's inequality.
\end{problem}
\begin{hintbox}
$\log \mathbb{E}[X] \geq \mathbb{E}[\log X]$.
\end{hintbox}
\end{exercisebox}


\begin{exercisebox}[medium]
\begin{problem}[Beta-VAE]
Explain how $\beta$-VAE encourages disentanglement.
\end{problem}
\begin{hintbox}
Weighted KL penalty; independence.
\end{hintbox}
\end{exercisebox}


\subsection*{Hard}

\begin{exercisebox}[hard]
\begin{problem}[Posterior Collapse]
Analyse conditions causing posterior collapse and propose mitigation.
\end{problem}
\begin{hintbox}
Strong decoder; KL annealing; free bits.
\end{hintbox}
\end{exercisebox}


\begin{exercisebox}[hard]
\begin{problem}[Importance-Weighted ELBO]
Derive the importance-weighted ELBO and show it tightens the bound.
\end{problem}
\begin{hintbox}
Multiple samples; log-mean-exp.
\end{hintbox}
\end{exercisebox}



\begin{exercisebox}[hard]
\begin{problem}[Advanced Topic 1]
Explain a key concept from this chapter and its practical applications.
\end{problem}
\begin{hintbox}
Consider the theoretical foundations and real-world implications.
\end{hintbox}
\end{exercisebox}


\begin{exercisebox}[hard]
\begin{problem}[Advanced Topic 2]
Analyse the relationship between different techniques covered in this chapter.
\end{problem}
\begin{hintbox}
Look for connections and trade-offs between methods.
\end{hintbox}
\end{exercisebox}


\begin{exercisebox}[hard]
\begin{problem}[Advanced Topic 3]
Design an experiment to test a hypothesis related to this chapter's content.
\end{problem}
\begin{hintbox}
Consider experimental design, metrics, and potential confounding factors.
\end{hintbox}
\end{exercisebox}


\begin{exercisebox}[hard]
\begin{problem}[Advanced Topic 4]
Compare different approaches to solving a problem from this chapter.
\end{problem}
\begin{hintbox}
Consider computational complexity, accuracy, and practical considerations.
\end{hintbox}
\end{exercisebox}


\begin{exercisebox}[hard]
\begin{problem}[Advanced Topic 5]
Derive a mathematical relationship or prove a theorem from this chapter.
\end{problem}
\begin{hintbox}
Start with the definitions and work through the logical steps.
\end{hintbox}
\end{exercisebox}


\begin{exercisebox}[hard]
\begin{problem}[Advanced Topic 6]
Implement a practical solution to a problem discussed in this chapter.
\end{problem}
\begin{hintbox}
Consider the implementation details and potential challenges.
\end{hintbox}
\end{exercisebox}


\begin{exercisebox}[hard]
\begin{problem}[Advanced Topic 7]
Evaluate the limitations and potential improvements of techniques from this chapter.
\end{problem}
\begin{hintbox}
Consider both theoretical limitations and practical constraints.
\end{hintbox}
\end{exercisebox}


