% Chapter 15: Representation Learning

\chapter{Representation Learning}
\label{chap:representation-learning}

This chapter discusses the central challenge of deep learning: learning meaningful representations from data.


\begin{learningobjectives}
\objective{Representations and desirable properties (invariance, disentanglement, sparsity)}
\objective{Supervised, self-supervised, and contrastive learning objectives}
\objective{Representation evaluation via linear probes and transfer learning}
\objective{Information-theoretic perspectives in empirical practice}
\end{learningobjectives}



\section*{Intuition}
\addcontentsline{toc}{section}{Intuition}

Good representations separate task-relevant factors from nuisance variability, where learning signals that compare positive and negative pairs, or predict masked content, shape geometry in embedding spaces to reflect semantics. For example, in facial recognition systems, a good representation would capture identity-related features like facial structure and bone geometry while being invariant to lighting conditions, pose variations, and facial expressions. A good representation is like a highly efficient, specialized library catalogue designed for a specific purpose, where it organizes and indexes information in a way that makes the most relevant details immediately accessible while filtering out irrelevant noise and redundancy.

\section*{Representation Function}
\addcontentsline{toc}{section}{Representation Function}

Representation learning can be mathematically defined as learning a function that maps raw input data from a high-dimensional space to a lower-dimensional, more informative space. Mathematically, representation learning aims to find a mapping function $\mathbf{f}$ that transforms the input data $\mathbf{x} \in \mathbb{R}^{D}$ into a representation $\mathbf{z} \in \mathbb{R}^{d}$, where typically the input dimension $D$ is much greater than the representation dimension $d$ ($D \gg d$). The key is that $\mathbf{z}$ is designed to capture the salient semantic factors of $\mathbf{x}$ while discarding noise and redundancy, where the function $\mathbf{f}$ is learned by defining an objective function that encourages $\mathbf{z}$ to possess desirable properties like invariance, disentanglement, and sparsity.


% Chapter 15, Section 1

\section{What Makes a Good Representation? \difficultyInline{intermediate}}
\label{sec:good-representations}

\subsection{Desirable Properties}

\textbf{Disentanglement:} Different factors of variation are separated
\begin{itemize}
    \item Changes in one dimension affect one factor
    \item Easier interpretation and manipulation
\end{itemize}

\textbf{Invariance:} Representation unchanged under irrelevant transformations
\begin{itemize}
    \item Translation, rotation invariance for objects
    \item Speaker invariance for speech content
\end{itemize}

\textbf{Smoothness:} Similar inputs have similar representations
\begin{itemize}
    \item Enables generalization
    \item Supports interpolation
\end{itemize}

\textbf{Sparsity:} Few features active for each input
\begin{itemize}
    \item Computational efficiency
    \item Interpretability
\end{itemize}

\subsection{Manifold Hypothesis}

Natural data lies on low-dimensional manifolds embedded in high-dimensional space.

Deep learning learns to:
\begin{itemize}
    \item Discover the manifold structure
    \item Map data to meaningful coordinates on manifold
\end{itemize}

% \subsection{Visual aids}
% \addcontentsline{toc}{subsubsection}{Visual aids (representations)}

% \begin{figure}[h]
%   \centering
%   \begin{tikzpicture}
%     \begin{axis}[
%       width=0.48\textwidth,height=0.36\textwidth,
%       xlabel={$x_1$}, ylabel={$x_2$}, grid=both]
%       \addplot[bookpurple,very thick,domain=-2:2] {sin(deg(x))/1.5};
%       \addplot[bookred,very thick,domain=-2:2] {0.0};
%     \end{axis}
%   \end{tikzpicture}
%   \caption{Complex data manifold (purple) and a learned linearized coordinate (red) locally.}
%   \label{fig:manifold}
% \end{figure}

\subsection{Notes and references}

Desirable properties are discussed in modern DL texts; disentanglement and invariance connect to inductive biases and data augmentation \textcite{GoodfellowEtAl2016,Prince2023}.

% Chapter 15, Section 2

\section{Transfer Learning and Domain Adaptation \difficultyInline{intermediate}}
\label{sec:transfer-learning}

\subsection{Transfer Learning}

Leverage knowledge from source task to improve target task:

\textbf{Feature extraction:}
\begin{enumerate}
    \item Pre-train on large dataset (e.g., ImageNet)
    \item Freeze convolutional layers
    \item Train only final classification layers on target task
\end{enumerate}

\textbf{Fine-tuning:}
\begin{enumerate}
    \item Start with pre-trained model
    \item Continue training on target task with lower learning rate
    \item Optionally freeze early layers
\end{enumerate}

\subsection{Domain Adaptation}

Adapt model when training (source) and test (target) distributions differ.

\textbf{Approaches:}
\begin{itemize}
    \item \textbf{Domain-adversarial training:} Learn domain-invariant features
    \item \textbf{Self-training:} Use confident predictions on target domain
    \item \textbf{Multi-task learning:} Joint training on both domains
\end{itemize}

\subsection{Few-Shot Learning}

Learn from few examples per class:
\begin{itemize}
    \item \textbf{Meta-learning:} Learn to learn quickly (MAML)
    \item \textbf{Prototypical networks:} Learn metric space
    \item \textbf{Matching networks:} Attention-based comparison
\end{itemize}

% \subsection{Visual aids}
% \addcontentsline{toc}{subsubsection}{Visual aids (transfer)}

% \begin{figure}[h]
%   \centering
%   \begin{tikzpicture}[>=stealth]
%     \tikzstyle{b}=[draw,rounded corners,align=center,minimum width=2.2cm,minimum height=0.9cm]
%     \node[b,fill=bookpurple!10] at (0,0) (src) {Source model};
%     \node[b,fill=bookpurple!15] at (3.2,0) (freeze) {Freeze early layers};
%     \node[b,fill=bookpurple!20] at (6.4,0) (ft) {Fine-tune head};
%     \draw[->] (src) -- (freeze);
%     \draw[->] (freeze) -- (ft);
%   \end{tikzpicture}
%   \caption{Common transfer pipeline: initialize, freeze, fine-tune.}
%   \label{fig:transfer}
% \end{figure}

% \subsection{References}

% See \textcite{Prince2023,GoodfellowEtAl2016} for practical recipes and pitfalls in transfer and domain adaptation.


% Chapter 15, Section 3

\section{Self-Supervised Learning \difficultyInline{intermediate}}
\label{sec:self-supervised}

Self-supervised learning learns representations without manual labels by solving pretext tasks that can be automatically generated from the data itself, where the model learns useful representations by predicting parts of the input from other parts or by solving auxiliary tasks that don't require human annotation.

\subsection{Pretext Tasks}

Pretext tasks for images include rotation prediction where the model learns to predict the rotation angle of an image, jigsaw puzzle where the model arranges shuffled patches to learn spatial relationships, colorization where the model predicts colors from grayscale images to learn color semantics, and inpainting where the model fills masked regions to learn object structure and context. For text, pretext tasks include masked language modeling where the model predicts masked words as in BERT, next sentence prediction where the model determines if sentences are consecutive to learn discourse relationships, and autoregressive generation where the model predicts the next token as in GPT to learn language modeling capabilities.

\subsection{Benefits}

Self-supervised learning provides several key benefits including the ability to leverage unlabeled data that is often abundant and cheap to obtain, where the model can learn from vast amounts of data without requiring expensive human annotation. The learned representations are often general-purpose and transfer well to downstream tasks, where the model learns to capture the underlying structure of the data rather than task-specific patterns. Self-supervised learning often outperforms supervised pre-training on downstream tasks, where the learned representations are more robust and generalizable because they are not biased toward specific labeled examples.

% \subsection{Visual aids}
% \addcontentsline{toc}{subsubsection}{Visual aids (self-supervised)}

% \begin{figure}[h]
%   \centering
%   \begin{tikzpicture}
%     \begin{axis}[
%       width=0.48\textwidth,height=0.36\textwidth,
%       xlabel={Pretraining epochs}, ylabel={Linear probe accuracy}, grid=both]
%       \addplot[bookpurple,very thick] coordinates{(10,0.55) (20,0.62) (40,0.68) (80,0.72) (160,0.74)};
%     \end{axis}
%   \end{tikzpicture}
%   \caption{Self-supervised pretraining improves downstream linear probe performance (illustrative).}
%   \label{fig:ssl-probe}
% \end{figure}

% \subsection{References}

% For contrastive and masked pretraining approaches, see \textcite{Chen2020,Devlin2018,Prince2023}.


% Chapter 15, Section 4

\section{Contrastive Learning \difficultyInline{intermediate}}
\label{sec:contrastive-learning}

Learn representations by contrasting positive and negative pairs.

\subsection{Core Idea}

Maximize agreement between different views of same data (positive pairs), minimize agreement with other data (negative pairs).

\subsection{SimCLR Framework}

\begin{enumerate}
    \item Apply two random augmentations to each image
    \item Encode both views: $\vect{z}_i = f(\vect{x}_i)$, $\vect{z}_j = f(\vect{x}_j)$
    \item Minimize contrastive loss (NT-Xent):
\end{enumerate}

\begin{equation}
\ell_{i,j} = -\log \frac{\exp(\text{sim}(\vect{z}_i, \vect{z}_j)/\tau)}{\sum_{k=1}^{2N} \mathbb{I}_{[k \neq i]} \exp(\text{sim}(\vect{z}_i, \vect{z}_k)/\tau)}
\end{equation}

where $\text{sim}(\cdot, \cdot)$ is cosine similarity and $\tau$ is temperature.

\subsection{MoCo (Momentum Contrast)}

Uses momentum encoder and queue of negative samples for efficiency.

\subsection{BYOL (Bootstrap Your Own Latent)}

Surprisingly, can work without negative samples using:
\begin{itemize}
    \item Online network (updated by gradients)
    \item Target network (momentum update)
    \item Prediction head on online network
\end{itemize}

\subsection{Applications}

State-of-the-art results in:
\begin{itemize}
    \item Image classification
    \item Object detection
    \item Segmentation
    \item Medical imaging with limited labels
\end{itemize}


% \subsection{Visual aids}
% \addcontentsline{toc}{subsubsection}{Visual aids (contrastive)}

% \begin{figure}[h]
%   \centering
%   \begin{tikzpicture}
%     \begin{axis}[
%       width=0.48\textwidth,height=0.36\textwidth,
%       xlabel={Epoch}, ylabel={NT-Xent loss}, grid=both]
%       \addplot[bookpurple,very thick] coordinates{(1,6.0) (10,3.2) (20,2.1) (40,1.4) (80,1.0)};
%     \end{axis}
%   \end{tikzpicture}
%   \caption{Contrastive loss decreasing during training (illustrative).}
%   \label{fig:ntxent}
% \end{figure}

% \subsection{References}

% Contrastive learning frameworks (SimCLR, MoCo, BYOL) underpin modern representation learning \textcite{Chen2020,Prince2023}.


% Chapter 15: Real World Applications

\section{Real World Applications}
\label{sec:representation-real-world}


Representation learning—automatically discovering useful features from raw data—is fundamental to modern deep learning success. Good representations make downstream tasks easier and enable transfer learning across domains.

\subsection{Transfer Learning in Computer Vision}

Transfer learning in computer vision reuses learned representations to save time and data, where medical imaging with limited data benefits from starting with representations learned from millions of general images like ImageNet and then fine-tuning on medical data, enabling networks that learned to recognize textures and shapes in everyday photos to adapt to recognize pathologies in X-rays with just a small medical dataset. Custom object detection for businesses allows retailers to detect their specific products on shelves and manufacturers to identify particular defects by using pre-trained vision models and fine-tuning with just hundreds of examples, where the learned representations of edges, textures, and objects transfer effectively, making custom vision systems practical for small businesses. Wildlife monitoring applications use camera traps to monitor endangered species, generating millions of images where transfer learning enables creating species classifiers with limited labeled examples, accelerating research without requiring biologists to manually label vast datasets.

\subsection{Natural Language Processing}

Learned language representations revolutionize text applications through multilingual models that learn representations capturing meaning across languages, where a model trained on English, Spanish, and Chinese text learns that "cat," "gato," and "mao" represent similar concepts, enabling zero-shot translation and allowing improvements in high-resource languages to benefit low-resource languages automatically. Domain adaptation applications use customer service chatbots that employ language models pre-trained on general text and then fine-tuned on company-specific conversations, where the general language understanding including grammar, reasoning, and world knowledge transfers while fine-tuning adds domain expertise, making sophisticated chatbots feasible without training from scratch. Sentiment analysis for brands allows companies to monitor social media sentiment about their products by using general text representations learned from billions of documents and then adapting to specific brand vocabulary, providing accurate sentiment analysis even for newly launched products.

\subsection{Cross-Modal Representations}

Cross-modal representations learn representations spanning multiple modalities, where image-text search systems like Google Images allow searching photos using text descriptions by requiring representations where images and text descriptions of the same concept are similar, where models learn joint representations by training on millions of image-caption pairs, enabling finding relevant images even for queries with no exact text matches. Video understanding applications use YouTube's recommendation and search systems that learn representations combining visual content, audio, speech transcripts, and metadata, where these multi-modal representations understand videos better than any single modality alone, improving search relevance and recommendations. Accessibility tools for visually impaired users generate descriptions of images on web pages using cross-modal representations trained on image-caption pairs, enabling generating relevant, helpful descriptions automatically and making the web more accessible.

\subsection{Impact of Good Representations}

Good representations matter because they provide data efficiency by enabling the solution of new tasks with less labeled data, where the learned representations capture the essential structure of the data and can be reused across different tasks. They enable generalization by providing better performance on diverse, real-world examples, where the learned representations are robust to variations in the data and can handle unseen examples effectively. Knowledge transfer allows expertise learned on one task to help others, where the learned representations can be fine-tuned for new tasks without starting from scratch. Semantic understanding captures meaningful structure in data, where the learned representations reflect the underlying semantic relationships in the data, enabling better understanding and manipulation of the data.

These applications demonstrate that representation learning is not just a theoretical concept—it's the foundation enabling practical deep learning with limited data and computational resources.

% Index entries
\index{applications!transfer learning}
\index{applications!natural language processing}
\index{applications!cross-modal learning}
\index{representation learning!applications}


% Chapter summary and problems
% Key Takeaways for Chapter 15

\section*{Key Takeaways}
\addcontentsline{toc}{section}{Key Takeaways}

\begin{keytakeaways}
\begin{itemize}[leftmargin=2em]
    \item \textbf{Representation quality} is judged by transfer and linear separability.
    \item \textbf{Self-supervision} shapes embeddings via predictive or contrastive signals.
    \item \textbf{Evaluation matters}: consistent probes and protocols enable fair comparison.
\end{itemize}
\end{keytakeaways}



% Exercises (Exercises) for Chapter 15

\section*{Exercises}
\addcontentsline{toc}{section}{Exercises}

\subsection*{Easy}

\begin{problem}[Encoder-Decoder Symmetry]
Why share architecture between $q(z|x)$ and $p(x|z)$?

\textbf{Hint:} Computation; conceptual symmetry.
\end{problem}

\begin{problem}[KL in ELBO]
State the role of $D_{KL}(q||p)$ in VAE training.

\textbf{Hint:} Regularisation; posterior matching.
\end{problem}

\begin{problem}[Reparameterisation Trick]
Explain why reparameterisation enables gradient flow.

\textbf{Hint:} Sampling vs. deterministic path.
\end{problem}

\begin{problem}[Prior Choice]
Justify Gaussian prior for VAE latents.

\textbf{Hint:} Tractability; simplicity; universality.
\end{problem}

\subsection*{Medium}

\begin{problem}[ELBO Derivation]
Derive the ELBO from Jensen's inequality.

\textbf{Hint:} $\log \mathbb{E}[X] \geq \mathbb{E}[\log X]$.
\end{problem}

\begin{problem}[Beta-VAE]
Explain how $\beta$-VAE encourages disentanglement.

\textbf{Hint:} Weighted KL penalty; independence.
\end{problem}

\subsection*{Hard}

\begin{problem}[Posterior Collapse]
Analyse conditions causing posterior collapse and propose mitigation.

\textbf{Hint:} Strong decoder; KL annealing; free bits.
\end{problem}

\begin{problem}[Importance-Weighted ELBO]
Derive the importance-weighted ELBO and show it tightens the bound.

\textbf{Hint:} Multiple samples; log-mean-exp.
\end{problem}


\begin{problem}[Advanced Topic 1]
Explain a key concept from this chapter and its practical applications.

\textbf{Hint:} Consider the theoretical foundations and real-world implications.
\end{problem}

\begin{problem}[Advanced Topic 2]
Analyse the relationship between different techniques covered in this chapter.

\textbf{Hint:} Look for connections and trade-offs between methods.
\end{problem}

\begin{problem}[Advanced Topic 3]
Design an experiment to test a hypothesis related to this chapter's content.

\textbf{Hint:} Consider experimental design, metrics, and potential confounding factors.
\end{problem}

\begin{problem}[Advanced Topic 4]
Compare different approaches to solving a problem from this chapter.

\textbf{Hint:} Consider computational complexity, accuracy, and practical considerations.
\end{problem}

\begin{problem}[Advanced Topic 5]
Derive a mathematical relationship or prove a theorem from this chapter.

\textbf{Hint:} Start with the definitions and work through the logical steps.
\end{problem}

\begin{problem}[Advanced Topic 6]
Implement a practical solution to a problem discussed in this chapter.

\textbf{Hint:} Consider the implementation details and potential challenges.
\end{problem}

\begin{problem}[Advanced Topic 7]
Evaluate the limitations and potential improvements of techniques from this chapter.

\textbf{Hint:} Consider both theoretical limitations and practical constraints.
\end{problem}

