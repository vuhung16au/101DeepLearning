% Problems (Exercises) for Chapter 16

\section*{Problems}
\addcontentsline{toc}{section}{Problems}

\subsection*{Easy}

\begin{problem}[Generator Role]
Describe the generator's objective in GANs.

\textbf{Hint:} Fool the discriminator.
\end{problem}

\begin{problem}[Discriminator Role]
Describe the discriminator's objective in GANs.

\textbf{Hint:} Distinguish real from fake.
\end{problem}

\begin{problem}[Mode Collapse]
Define mode collapse and its symptoms.

\textbf{Hint:} Generator produces limited variety.
\end{problem}

\begin{problem}[Training Instability]
Name two causes of GAN training instability.

\textbf{Hint:} Oscillation; gradient vanishing.
\end{problem}

\subsection*{Medium}

\begin{problem}[Nash Equilibrium]
Explain why GAN training seeks a Nash equilibrium.

\textbf{Hint:} Minimax game; no unilateral improvement.
\end{problem}

\begin{problem}[Wasserstein Distance]
State the advantage of Wasserstein distance over JS divergence.

\textbf{Hint:} Gradient behaviour with non-overlapping distributions.
\end{problem}

\subsection*{Hard}

\begin{problem}[Optimal Discriminator]
Derive the optimal discriminator for fixed generator.

\textbf{Hint:} Maximise expected log-likelihood.
\end{problem}

\begin{problem}[Spectral Normalisation]
Analyse how spectral normalisation stabilises GAN training.

\textbf{Hint:} Lipschitz constraint; gradient norms.
\end{problem}

