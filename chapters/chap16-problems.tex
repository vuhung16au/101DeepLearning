% Exercises (Exercises) for Chapter 16

\section*{Exercises}
\addcontentsline{toc}{section}{Exercises}

\subsection*{Easy}

\begin{problem}[Generator Role]
Describe the generator's objective in GANs.

\textbf{Hint:} Fool the discriminator.
\end{problem}

\begin{problem}[Discriminator Role]
Describe the discriminator's objective in GANs.

\textbf{Hint:} Distinguish real from fake.
\end{problem}

\begin{problem}[Mode Collapse]
Define mode collapse and its symptoms.

\textbf{Hint:} Generator produces limited variety.
\end{problem}

\begin{problem}[Training Instability]
Name two causes of GAN training instability.

\textbf{Hint:} Oscillation; gradient vanishing.
\end{problem}

\subsection*{Medium}

\begin{problem}[Nash Equilibrium]
Explain why GAN training seeks a Nash equilibrium.

\textbf{Hint:} Minimax game; no unilateral improvement.
\end{problem}

\begin{problem}[Wasserstein Distance]
State the advantage of Wasserstein distance over JS divergence.

\textbf{Hint:} Gradient behaviour with non-overlapping distributions.
\end{problem}

\subsection*{Hard}

\begin{problem}[Optimal Discriminator]
Derive the optimal discriminator for fixed generator.

\textbf{Hint:} Maximise expected log-likelihood.
\end{problem}

\begin{problem}[Spectral Normalisation]
Analyse how spectral normalisation stabilises GAN training.

\textbf{Hint:} Lipschitz constraint; gradient norms.
\end{problem}


\begin{problem}[Advanced Topic 1]
Explain a key concept from this chapter and its practical applications.

\textbf{Hint:} Consider the theoretical foundations and real-world implications.
\end{problem}

\begin{problem}[Advanced Topic 2]
Analyse the relationship between different techniques covered in this chapter.

\textbf{Hint:} Look for connections and trade-offs between methods.
\end{problem}

\begin{problem}[Advanced Topic 3]
Design an experiment to test a hypothesis related to this chapter's content.

\textbf{Hint:} Consider experimental design, metrics, and potential confounding factors.
\end{problem}

\begin{problem}[Advanced Topic 4]
Compare different approaches to solving a problem from this chapter.

\textbf{Hint:} Consider computational complexity, accuracy, and practical considerations.
\end{problem}

\begin{problem}[Advanced Topic 5]
Derive a mathematical relationship or prove a theorem from this chapter.

\textbf{Hint:} Start with the definitions and work through the logical steps.
\end{problem}

\begin{problem}[Advanced Topic 6]
Implement a practical solution to a problem discussed in this chapter.

\textbf{Hint:} Consider the implementation details and potential challenges.
\end{problem}

\begin{problem}[Advanced Topic 7]
Evaluate the limitations and potential improvements of techniques from this chapter.

\textbf{Hint:} Consider both theoretical limitations and practical constraints.
\end{problem}
