% Key Takeaways for Chapter 20

\section*{Key Takeaways}
\addcontentsline{toc}{section}{Key Takeaways}

\begin{keytakeaways}
\begin{itemize}[leftmargin=2em]
    \item \textbf{Model families} differ in training signal and guarantees: VAEs use likelihood-based training with explicit density modeling, GANs employ adversarial training for implicit modeling, flows provide exact likelihoods through invertible transformations, and diffusion models learn to reverse noise corruption processes.
    \item \textbf{Evaluation} must consider likelihood, fidelity, diversity, and downstream utility: Different metrics capture different aspects of generation quality, and the choice of evaluation method should align with the intended application and user requirements.
    \item \textbf{Trade-offs} are inevitable: choose for the target application: Each approach offers distinct advantages—VAEs provide stable training and exact likelihoods, GANs excel at sample quality, flows enable exact sampling, and diffusion models achieve state-of-the-art results with strong theoretical foundations.
    \item \textbf{Training dynamics} vary significantly across approaches: VAEs and flows use standard maximum likelihood optimization, GANs require careful balance between generator and discriminator, while diffusion models benefit from stable training with principled noise schedules.
    \item \textbf{Applications} drive architectural choices: The specific requirements of creative content generation, scientific discovery, data augmentation, and personalization should guide the selection of appropriate generative modeling approaches and evaluation metrics.
\end{itemize}
\end{keytakeaways}


