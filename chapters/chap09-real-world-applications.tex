% Chapter 9: Real World Applications

\section{Real World Applications}
\label{sec:cnn-real-world}


Convolutional neural networks have revolutionized how computers understand images and videos. Their applications touch nearly every aspect of modern visual technology.

\subsection{Medical Image Analysis}

CNNs help doctors diagnose diseases more accurately and quickly. In cancer detection in radiology, CNNs analyse X-rays, CT scans, and MRIs to detect tumours often invisible to the human eye. For example, mammography systems using CNNs can spot breast cancer earlier than traditional methods, potentially saving thousands of lives annually. The networks learn to recognise subtle patterns that indicate malignancy. For diabetic retinopathy screening, CNNs examine photos of patients' eyes to detect diabetes-related damage before vision loss occurs. This allows automated screening in remote areas without specialist ophthalmologists, making eye care accessible to millions more people worldwide. In skin cancer classification, smartphone apps with CNNs let people photograph suspicious moles for instant preliminary assessment. Whilst not replacing doctors, these tools encourage early medical consultation when something looks concerning.

\subsection{Autonomous Driving}

Self-driving cars rely on CNNs to understand their surroundings. For object detection and tracking, CNNs process camera feeds to identify pedestrians, other vehicles, traffic signs, and lane markings in real-time. The network must work perfectly under varied conditions—rain, snow, nighttime, construction zones—because lives depend on it. In depth estimation, CNNs analyse images to determine how far away objects are, helping vehicles make safe decisions about braking, turning, and merging. This works even with regular cameras, though it's enhanced when combined with other sensors. Through semantic segmentation, CNNs label every pixel in the camera view (road, sidewalk, vehicle, sky, etc.), giving the vehicle complete understanding of its environment. This pixel-level understanding enables precise navigation.

\subsection{Content Moderation and Safety}

Social media platforms use CNNs to keep online spaces safe. For inappropriate content detection, CNNs scan billions of uploaded images and videos daily, automatically flagging harmful content (violence, explicit material, hate symbols) for human review. This happens before most users ever see problematic content. In face blurring for privacy, news organisations and mapping services use CNNs to automatically blur faces and licence plates in photos and street view imagery, protecting people's privacy whilst sharing useful information. For copyright protection, CNNs help platforms identify copyrighted images and videos, preventing unauthorised sharing whilst allowing legitimate uses. This technology processes millions of uploads per hour.

\subsection{Everyday Applications}

CNNs power features you use daily. Photo organisation on your phone automatically groups photos by people, places, and things. Visual search lets you find products by taking photos instead of typing descriptions. Document scanning apps automatically detect document edges and enhance readability. Augmented reality enables filters and effects in camera apps that track faces and scenes.

These applications show how CNNs transform abstract computer vision research into practical tools that improve healthcare, safety, and daily convenience.

% Index entries
\index{applications!medical imaging}
\index{applications!autonomous vehicles}
\index{applications!content moderation}
\index{convolutional networks!applications}
