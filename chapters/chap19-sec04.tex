% Chapter 19, Section 4

\section{Expectation Propagation \difficultyInline{advanced}}
\label{sec:ep}

Expectation propagation approximates complex probability distributions by replacing each factor with a simpler distribution from an exponential family, enabling tractable inference while preserving important statistical properties. The mathematical foundation is:

\begin{equation}
p(\vect{x}) = \frac{1}{Z} \prod_i f_i(\vect{x}) \approx \frac{1}{Z} \prod_i \tilde{f}_i(\vect{x})
\end{equation}

This equation shows that the true distribution $p(\vect{x})$ is approximated by replacing each complex factor $f_i(\vect{x})$ with a simpler factor $\tilde{f}_i(\vect{x})$ from a tractable family. The algorithm iteratively refines these approximations to match the moments of the true distribution, providing a principled way to approximate complex posteriors. Expectation propagation is particularly effective for multi-modal posteriors where mean field approximation would fail, as it can capture multiple modes by using more expressive approximating families.


% \subsection{Visual aids}
% \addcontentsline{toc}{subsubsection}{Visual aids (EP)}

% \begin{figure}[h]
%   \centering
%   \begin{tikzpicture}
%     \begin{axis}[
%       width=0.48\textwidth,height=0.36\textwidth,
%       xlabel={$x$}, ylabel={Density}, grid=both]
%       \addplot[bookpurple,very thick,domain=-3:3,samples=100]{0.5*exp(-0.5*((x-1)^2)) + 0.5*exp(-0.5*((x+1)^2))};
%       \addplot[bookred,very thick,dashed,domain=-3:3,samples=100]{exp(-0.5*((x)^2))};
%     \end{axis}
%   \end{tikzpicture}
%   \caption{EP (solid) can better match multi-modal targets than simple mean-field Gaussian (dashed), illustrative.}
%   \label{fig:ep-modes}
% \end{figure}

% \subsection{Notes and references}

% See \textcite{Bishop2006,GoodfellowEtAl2016,Prince2023} for EP algorithms and comparisons to mean field.
