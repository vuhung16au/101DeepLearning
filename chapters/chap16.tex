% Chapter 16: Structured Probabilistic Models for Deep Learning

\chapter{Structured Probabilistic Models for Deep Learning}
\label{chap:structured-probabilistic-models}

This chapter covers graphical models and their integration with deep learning.


\section*{Learning Objectives}
\addcontentsline{toc}{section}{Learning Objectives}

After studying this chapter, you will be able to:

\begin{enumerate}
    \item Explain directed and undirected graphical models and conditional independencies.
    \item Combine neural networks with graphical structures for hybrid models.
    \item Perform basic inference and understand when approximations are required.
    \item Design learning objectives for structured prediction tasks.
\end{enumerate}



\section*{Intuition}
\addcontentsline{toc}{section}{Intuition}

Graphical structure encodes assumptions about which variables interact, where neural components capture complex local relationships while the graph constrains global behavior, aiding data efficiency and interpretability. For example, in medical diagnosis, a graphical model can encode the causal relationships between symptoms, diseases, and treatments, where neural networks learn complex patterns in individual symptoms while the graph structure ensures that the diagnosis follows logical medical reasoning. A graphical model is like a city's traffic system, where the neural networks are the individual vehicles that can navigate complex local routes, while the graph structure provides the overall traffic rules and road network that ensures efficient and logical flow throughout the entire system.


% Chapter 16, Section 1

\section{Graphical Models \difficultyInline{advanced}}
\label{sec:graphical-models}

Graphical models provide a powerful framework for representing complex probability distributions using graph structures, where nodes represent random variables and edges encode probabilistic dependencies, enabling efficient representation and inference in high-dimensional spaces.

\subsection{Motivation}

Graphical models represent complex probability distributions using graphs, where nodes represent random variables and edges encode probabilistic dependencies, enabling the efficient representation of high-dimensional joint distributions through factorization. This approach is particularly valuable in deep learning because it allows us to encode domain knowledge and structural assumptions directly into the model, where the graph structure provides inductive biases that guide the learning process and improve generalization. The graphical representation makes the model more interpretable by explicitly showing which variables interact and how, where this transparency is crucial for understanding model behavior and debugging complex systems. Furthermore, graphical models enable efficient inference by exploiting the conditional independence structure encoded in the graph, where this computational efficiency is essential for practical applications with large-scale data.

\subsection{Bayesian Networks}

Bayesian networks use directed acyclic graphs (DAGs) to represent conditional dependencies, where the joint probability distribution factorizes as $p(\vect{x}) = \prod_{i=1}^{n} p(x_i | \text{Pa}(x_i))$ with $\text{Pa}(x_i)$ being the parents of $x_i$. In deep learning, Bayesian networks are particularly useful for modeling causal relationships and incorporating domain knowledge. The Naive Bayes classifier exemplifies this approach with
\begin{equation}
p(y, \vect{x}) = p(y) \prod_{i=1}^{d} p(x_i\,|\,y).
\end{equation}
\noindent\textbf{Variables.} $y$: class label; $\vect{x}=(x_1,\dots,x_d)$: features; $p(y)$: class prior; $p(x_i|y)$: class-conditional likelihoods.

\subsection{Markov Random Fields}

Markov Random Fields use undirected graphs with potential functions. The joint distribution factorizes as
\begin{equation}
p(\vect{x}) = \frac{1}{Z} \prod_{c \in \mathcal{C}} \psi_c(\vect{x}_c),
\end{equation}
where $\mathcal{C}$ are cliques, $\psi_c$ are nonnegative potentials over variables in clique $c$, and $Z$ is the partition function ensuring normalization.


% \subsection{Visual aids}
% \addcontentsline{toc}{subsubsection}{Visual aids (graphical models)}

% \begin{figure}[h]
%   \centering
%   \begin{tikzpicture}[>=stealth]
%     % Simple DAG: X1 -> X3 <- X2
%     \node[circle,draw] (x1) at (0,0) {$x_1$};
%     \node[circle,draw] (x2) at (2,0) {$x_2$};
%     \node[circle,draw] (x3) at (1,-1.2) {$x_3$};
%     \draw[->] (x1) -- (x3);
%     \draw[->] (x2) -- (x3);
%   \end{tikzpicture}
%   \caption{A simple Bayesian network (DAG) encoding conditional dependencies.}
%   \label{fig:bn-dag}
% \end{figure}

% \begin{figure}[h]
%   \centering
%   \begin{tikzpicture}
%     % 3x3 MRF grid (undirected)
%     \foreach \i in {0,1,2} {
%       \foreach \j in {0,1,2} {
%         \node[circle,draw,inner sep=1.5pt] (n\i\j) at (\i*0.9,-\j*0.9) {};
%       }
%     }
%     \foreach \i in {0,1,2} {
%       \foreach \j in {0,1} {
%         \draw (n\i\j) -- (n\i\the\numexpr\j+1\relax);
%       }
%     }
%     \foreach \i in {0,1} {
%       \foreach \j in {0,1,2} {
%         \draw (n\i\j) -- (n\the\numexpr\i+1\relax\j);
%       }
%     }
%   \end{tikzpicture}
%   \caption{Undirected MRF grid with pairwise potentials between neighbors.}
%   \label{fig:mrf-grid}
% \end{figure}

% \subsection{Notes and references}

% Foundations and factorization properties are covered extensively in \textcite{Bishop2006,GoodfellowEtAl2016,Prince2023}.

% Chapter 16, Section 2

\section{Inference in Graphical Models \difficultyInline{advanced}}
\label{sec:inference}

\subsection{Exact Inference}

\textbf{Variable elimination:} Marginalize variables sequentially

\textbf{Belief propagation:} Message passing on tree-structured graphs

Complexity exponential in tree-width, often intractable.

\subsection{Approximate Inference}

\textbf{Variational inference:} Optimize tractable approximation (Chapter 19)

\textbf{Sampling methods:} Monte Carlo approaches (Chapter 17)

\textbf{Loopy belief propagation:} Approximate inference on graphs with cycles

% \subsection{Visual aids}
% \addcontentsline{toc}{subsubsection}{Visual aids (inference)}

% \begin{figure}[h]
%   \centering
%   \begin{tikzpicture}
%     \begin{axis}[
%       width=0.48\textwidth,height=0.36\textwidth,
%       xlabel={Tree-width}, ylabel={Complexity (log-scale)}, ymode=log, grid=both]
%       \addplot[bookpurple,very thick] coordinates{(1,1e2) (2,1e3) (3,1e4) (4,1e5)};
%     \end{axis}
%   \end{tikzpicture}
%   \caption{Exact inference complexity grows rapidly with tree-width (illustrative).}
%   \label{fig:treewidth}
% \end{figure}

% Chapter 16, Section 3

\section{Deep Learning and Structured Models \difficultyInline{advanced}}
\label{sec:deep-structured}

Deep learning and structured models combine the representational power of neural networks with the structural constraints of graphical models, where neural networks learn complex feature representations while graphical models enforce structural constraints on the output space, enabling the modeling of complex, structured data with both local and global dependencies.

\subsection{Structured Output Prediction}

Structured output prediction uses graphical models to model output structure, where the goal is to predict complex, structured outputs that satisfy certain constraints and dependencies. Conditional Random Fields (CRFs) provide a powerful framework for structured prediction with $p(\vect{y}|\vect{x}) = \frac{1}{Z(\vect{x})} \exp\left(\sum_c \vect{w}^\top \vect{\phi}_c(\vect{x}, \vect{y}_c)\right)$, where the model learns to assign higher probability to valid output structures while penalizing invalid ones. Applications include sequence labeling tasks like named entity recognition and part-of-speech tagging, where the model must ensure that the predicted labels form coherent sequences, image segmentation where the model must produce spatially coherent segmentations, and parsing where the model must generate syntactically valid parse trees that respect grammatical constraints.

\subsection{Structured Prediction with Neural Networks}

Structured prediction with neural networks combines the representational power of neural networks with the structural constraints of graphical models, where the neural network extracts rich feature representations from the input data while the graphical model enforces structural constraints on the output space. The approach typically involves feature extraction using CNNs or RNNs to learn complex patterns in the input data, followed by structured inference using a CRF layer that ensures the output satisfies the required structural constraints, where the entire system is trained end-to-end using backpropagation to optimize both the feature extraction and structured prediction components simultaneously. An example is CNN-CRF for semantic segmentation, where the CNN learns to extract visual features from images while the CRF layer ensures that the predicted segmentation labels form spatially coherent regions, resulting in more accurate and visually plausible segmentations.

\subsection{Neural Module Networks}

Neural Module Networks compose neural modules based on program structure for visual reasoning, where the model learns to decompose complex visual reasoning tasks into simpler sub-tasks that can be solved by specialized neural modules. This approach enables the model to handle compositional reasoning tasks by learning to combine different modules in a structured way, where each module is responsible for a specific type of reasoning operation such as object detection, spatial relationships, or attribute recognition. The modular design allows the model to generalize to new combinations of reasoning operations and provides interpretability by showing which modules are used for each step of the reasoning process.

\subsection{Graph Neural Networks}

Graph Neural Networks (GNNs) operate on graphs via message passing and permutation-invariant aggregations, where the mathematical definition involves updating node representations through $\mathbf{h}_v^{(l+1)} = \text{UPDATE}^{(l)}(\mathbf{h}_v^{(l)}, \text{AGGREGATE}^{(l)}(\{\mathbf{h}_u^{(l)} : u \in \mathcal{N}(v)\}))$ for each node $v$ and layer $l$. We need GNNs because many real-world problems involve data with inherent graph structure, where traditional neural networks cannot directly handle the irregular and variable-sized nature of graph data. We use GNNs by first representing the data as a graph with nodes and edges, then applying message passing layers that aggregate information from neighboring nodes, where the learned node representations can be used for various downstream tasks like node classification, link prediction, or graph-level prediction.

% \subsection{Visual aids}
% \addcontentsline{toc}{subsubsection}{Visual aids (structured models)}

% \begin{figure}[h]
%   \centering
%   \begin{tikzpicture}[>=stealth]
%     \tikzstyle{b}=[draw,rounded corners,align=center,minimum width=2.2cm,minimum height=0.9cm]
%     \node[b,fill=bookpurple!10] at (0,0) (cnn) {CNN features};
%     \node[b,fill=bookpurple!15] at (3.2,0) (crf) {CRF layer};
%     \node[b,fill=bookpurple!20] at (6.4,0) (mask) {Segmentation mask};
%     \draw[->] (cnn) -- (crf);
%     \draw[->] (crf) -- (mask);
%   \end{tikzpicture}
%   \caption{CNN features feeding a CRF layer for structured output.}
%   \label{fig:cnn-crf}
% \end{figure}


% Chapter 16: Real World Applications

\section{Real World Applications}
\label{sec:structured-prob-real-world}


Structured probabilistic models capture dependencies and uncertainties in complex systems. These models enable reasoning under uncertainty and provide principled frameworks for decision-making in real-world applications.

\subsection{Autonomous Vehicle Decision Making}

Autonomous vehicle decision making requires reasoning about uncertainties in complex environments, where predicting pedestrian behavior involves capturing uncertainty about pedestrian intentions based on their position, posture, and gaze direction, enabling the vehicle to make conservative decisions by slowing down when a pedestrian might cross rather than assuming they won't. Sensor fusion with uncertainty combines cameras, radar, and lidar data using probabilistic graphical models that weigh each sensor according to reliability in current conditions, where cameras work poorly in fog while radar penetrates fog better, providing robust perception despite individual sensor limitations. Planning under uncertainty involves route planning that considers uncertain travel times due to traffic, weather, and road conditions, where structured probabilistic models help vehicles balance expected arrival time with reliability and provide realistic time estimates.

\subsection{Medical Diagnosis and Treatment}

Medical diagnosis and treatment require careful uncertainty quantification, where Bayesian diagnosis systems handle the uncertainty inherent in medical diagnosis by encoding relationships between symptoms, diseases, and test results using structured probabilistic models that compute probability distributions over possible diagnoses, helping doctors order appropriate tests and consider differential diagnoses systematically. Personalized treatment planning uses probabilistic models that integrate genetic markers, tumor characteristics, and treatment histories to estimate probability distributions over treatment outcomes, where doctors use these estimates to discuss risks and benefits with patients, enabling informed shared decisions about treatment options. Drug interaction modeling addresses the risks faced by patients taking multiple medications by using structured models that capture dependencies between drugs while considering individual patient factors like age, kidney function, and genetics to estimate risk probabilities, enabling safer prescribing especially for elderly patients on many medications.

\subsection{Natural Language Understanding}

Natural language understanding involves handling the inherent ambiguity and structure of language, where machine translation quality uses probabilistic models to capture ambiguity in words that have multiple possible translations depending on context, with structured models representing sentence structure helping to select appropriate translations and providing confidence estimates for different interpretations, enabling highlighting of uncertain translations for human review. Information extraction requires understanding relationships between entities to extract structured information about who did what to whom, when, and where, where probabilistic graphical models capture these dependencies and provide uncertainty estimates about extracted information, enabling news aggregators to reconcile potentially conflicting reports from multiple sources. Voice assistant intent recognition handles ambiguous queries like "Book a table for two" that could mean restaurant reservations or furniture arrangements, where structured models use conversation context and user history to estimate intent probabilities, asking clarifying questions when uncertainty is high rather than guessing incorrectly.

\subsection{Social Media Network Analysis}

Social media networks provide rich examples of how deep learning and graph neural networks are applied to understand complex social structures and behaviors. In social media platforms like Facebook, Twitter, and LinkedIn, GNNs are used to model user interactions, content propagation, and community detection, where the graph structure represents users as nodes and relationships like friendships, follows, or interactions as edges. These models can predict user behavior, identify influential users, detect fake news propagation, and recommend relevant content by learning representations that capture both individual user characteristics and their position within the social network structure.

\subsection{Value of Structured Models}

Structured models provide several key advantages including interpretability where the model structure reflects domain knowledge and causal relationships, enabling users to understand how the model makes decisions and what factors influence the predictions. Uncertainty quantification provides principled probability estimates that are crucial for decision-making in high-stakes applications, where knowing not just what the model predicts but how confident it is in that prediction can be the difference between success and failure. Data efficiency is achieved through structure that reduces parameters and sample complexity, where the explicit modeling of dependencies allows the model to learn more effectively from limited data. Reasoning capabilities enable inference, prediction, and decision-making under uncertainty, where the structured approach allows the model to handle complex scenarios that would be difficult for unstructured models.

These applications show how structured probabilistic models provide principled frameworks for dealing with uncertainty in safety-critical and high-stakes applications.

% Index entries
\index{applications!autonomous vehicles}
\index{applications!medical diagnosis}
\index{applications!natural language understanding}
\index{structured probabilistic models!applications}


% Chapter summary and problems
% Key Takeaways for Chapter 16

\section*{Key Takeaways}
\addcontentsline{toc}{section}{Key Takeaways}

\begin{keytakeaways}
\begin{itemize}[leftmargin=2em]
    \item \textbf{Structure} encodes independence, enabling efficient inference.
    \item \textbf{Hybrid models} leverage neural expressivity with graphical constraints.
    \item \textbf{Inference choices} depend on graph type and potential functions.
\end{itemize}
\end{keytakeaways}



% Exercises (Exercises) for Chapter 16

\section*{Exercises}
\addcontentsline{toc}{section}{Exercises}

\subsection*{Easy}

\begin{exercisebox}[easy]
\begin{problem}[Generator Role]
Describe the generator's objective in GANs.
\end{problem}
\begin{hintbox}
Fool the discriminator.
\end{hintbox}
\end{exercisebox}


\begin{exercisebox}[easy]
\begin{problem}[Discriminator Role]
Describe the discriminator's objective in GANs.
\end{problem}
\begin{hintbox}
Distinguish real from fake.
\end{hintbox}
\end{exercisebox}


\begin{exercisebox}[easy]
\begin{problem}[Mode Collapse]
Define mode collapse and its symptoms.
\end{problem}
\begin{hintbox}
Generator produces limited variety.
\end{hintbox}
\end{exercisebox}


\begin{exercisebox}[easy]
\begin{problem}[Training Instability]
Name two causes of GAN training instability.
\end{problem}
\begin{hintbox}
Oscillation; gradient vanishing.
\end{hintbox}
\end{exercisebox}


\subsection*{Medium}

\begin{exercisebox}[medium]
\begin{problem}[Nash Equilibrium]
Explain why GAN training seeks a Nash equilibrium.
\end{problem}
\begin{hintbox}
Minimax game; no unilateral improvement.
\end{hintbox}
\end{exercisebox}


\begin{exercisebox}[medium]
\begin{problem}[Wasserstein Distance]
State the advantage of Wasserstein distance over JS divergence.
\end{problem}
\begin{hintbox}
Gradient behaviour with non-overlapping distributions.
\end{hintbox}
\end{exercisebox}


\subsection*{Hard}

\begin{exercisebox}[hard]
\begin{problem}[Optimal Discriminator]
Derive the optimal discriminator for fixed generator.
\end{problem}
\begin{hintbox}
Maximise expected log-likelihood.
\end{hintbox}
\end{exercisebox}


\begin{exercisebox}[hard]
\begin{problem}[Spectral Normalisation]
Analyse how spectral normalisation stabilises GAN training.
\end{problem}
\begin{hintbox}
Lipschitz constraint; gradient norms.
\end{hintbox}
\end{exercisebox}



\begin{exercisebox}[hard]
\begin{problem}[Advanced Topic 1]
Explain a key concept from this chapter and its practical applications.
\end{problem}
\begin{hintbox}
Consider the theoretical foundations and real-world implications.
\end{hintbox}
\end{exercisebox}


\begin{exercisebox}[hard]
\begin{problem}[Advanced Topic 2]
Analyse the relationship between different techniques covered in this chapter.
\end{problem}
\begin{hintbox}
Look for connections and trade-offs between methods.
\end{hintbox}
\end{exercisebox}


\begin{exercisebox}[hard]
\begin{problem}[Advanced Topic 3]
Design an experiment to test a hypothesis related to this chapter's content.
\end{problem}
\begin{hintbox}
Consider experimental design, metrics, and potential confounding factors.
\end{hintbox}
\end{exercisebox}


\begin{exercisebox}[hard]
\begin{problem}[Advanced Topic 4]
Compare different approaches to solving a problem from this chapter.
\end{problem}
\begin{hintbox}
Consider computational complexity, accuracy, and practical considerations.
\end{hintbox}
\end{exercisebox}


\begin{exercisebox}[hard]
\begin{problem}[Advanced Topic 5]
Derive a mathematical relationship or prove a theorem from this chapter.
\end{problem}
\begin{hintbox}
Start with the definitions and work through the logical steps.
\end{hintbox}
\end{exercisebox}


\begin{exercisebox}[hard]
\begin{problem}[Advanced Topic 6]
Implement a practical solution to a problem discussed in this chapter.
\end{problem}
\begin{hintbox}
Consider the implementation details and potential challenges.
\end{hintbox}
\end{exercisebox}


\begin{exercisebox}[hard]
\begin{problem}[Advanced Topic 7]
Evaluate the limitations and potential improvements of techniques from this chapter.
\end{problem}
\begin{hintbox}
Consider both theoretical limitations and practical constraints.
\end{hintbox}
\end{exercisebox}


