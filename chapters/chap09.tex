% Chapter 9: Convolutional Networks

\chapter{Convolutional Networks}
\label{chap:convolutional-networks}

This chapter introduces convolutional neural networks (CNNs), which are particularly effective for processing grid-structured data like images. We build intuition first, then progressively add mathematical detail and modern algorithms, with historical context and key takeaways throughout \cite{GoodfellowEtAl2016,Prince2023}.

\begin{learningobjectives}
\objective{The intuition of convolution, pooling, and receptive fields and how they induce translation equivariance and parameter sharing}
\objective{Output shapes and parameter counts for common layer configurations (kernel size, stride, padding, channels)}
\objective{Core CNN algorithms: convolution/cross-correlation, pooling, backpropagation through convolution, and residual connections}
\objective{Classic architectures (LeNet, AlexNet, VGG, Inception, ResNet, MobileNet/EfficientNet) and their design trade-offs}
\objective{CNNs to key tasks in vision: classification, detection, and segmentation, and choose appropriate heads and losses}
\objective{Historical milestones that motivated CNN advances and understand why techniques were developed}
\end{learningobjectives}




% Chapter 9, Section 1

\section{The Convolution Operation\index{convolution}\index{cross-correlation}}
\label{sec:convolution}


\subsection*{Intuition}
Convolution extracts local patterns by sliding small filters across the input, producing feature maps that respond strongly where patterns occur. Parameter sharing means the same filter detects the same pattern anywhere, yielding translation equivariance \index{equivariance} and dramatic parameter efficiency \cite{GoodfellowEtAl2016,Prince2023}.

\subsection{Definition}

The \textbf{convolution} operation applies a filter (kernel) across an input:

For discrete 2D convolution:
\begin{equation}
S(i,j) = (I * K)(i,j) = \sum_m \sum_n I(i-m, j-n) K(m, n)
\end{equation}

where $I$ is the input and $K$ is the kernel. In deep learning libraries, the implemented operation is often cross-correlation (no kernel flip).

In practice, we often use \textbf{cross-correlation}:
\begin{equation}
S(i,j) = (I * K)(i,j) = \sum_m \sum_n I(i+m, j+n) K(m, n)
\end{equation}

\subsection{Properties}

\textbf{Parameter sharing:} same kernel applied across spatial locations\index{parameter sharing}. This dramatically reduces parameters compared to fully connected layers acting on flattened images. Parameter sharing yields \emph{translation equivariance}: if the input shifts, the feature map shifts in the same way \cite{GoodfellowEtAl2016,Prince2023}. Formally, letting \(T_\delta\) denote a spatial shift, we have \((T_\delta I) * K = T_\delta (I * K)\) under appropriate boundary conditions.

\textbf{Local connectivity:} each output depends on a local input region (\gls{receptive-field})\index{receptive field}. This local connectivity exploits spatial locality, recognising that nearby pixels are statistically dependent in natural images. Through \emph{compositionality}, stacking layers grows the effective receptive field, enabling detection of increasingly complex patterns—from edges to corners to object parts \cite{GoodfellowEtAl2016}.

\textbf{Linearity and nonlinearity:} a single convolution is linear, so CNNs interleave it with pointwise nonlinearities (e.g., ReLU) to model complex functions. With stride and padding fixed, convolution is a sparse matrix multiplication with a Toeplitz structure \cite{GoodfellowEtAl2016}.

\textbf{Padding and boundary effects}\index{padding}: padding preserves spatial size and mitigates edge shrinkage. Without padding (``valid''), repeated convolutions rapidly reduce feature map size and bias features toward the center.

\textbf{Stride and downsampling}\index{stride}: stride \(s>1\) reduces spatial resolution. While efficient, aggressive striding early can remove fine detail; many designs delay downsampling to later stages (e.g., \textit{conv stem} then stride) \cite{Krizhevsky2012,He2016}.

\textbf{Dilation (atrous convolutions)}\index{dilation}: inserting \emph{holes} between kernel elements increases receptive field without increasing parameters, useful for dense prediction (e.g., segmentation) \cite{GoodfellowEtAl2016}.

\textbf{Invariance via pooling:} convolution is equivariant to translation; combining it with pooling or global aggregation introduces partial \emph{translation invariance} in the representation \cite{GoodfellowEtAl2016}.

\textbf{Multi-channel mixing:} kernels of shape \(k\times k\times C_{\text{in}}\) learn both spatial and cross-channel interactions, while $1\times1$ convolutions mix channels without spatial coupling (used for bottlenecks and dimension reduction in Inception/ResNet).

\subsection{Multi-Channel Convolution}

For input with $C_{\text{in}}$ channels and $C_{\text{out}}$ output channels:
\begin{equation}
S_{c_{\text{out}}}(i,j) = \sum_{c_{\text{in}}=1}^{C_{\text{in}}} (I_{c_{\text{in}}} * K_{c_{\text{out}}, c_{\text{in}}})(i,j) + b_{c_{\text{out}}}
\end{equation}

\paragraph{Understanding Multi-Channel Convolution and S(i,j)}
Multi-channel convolution is the fundamental operation that enables CNNs to process complex inputs like RGB images (3 channels) and produce rich feature representations. The notation $S_{c_{\text{out}}}(i,j)$ represents the output feature map value at spatial position $(i,j)$ for output channel $c_{\text{out}}$. This operation allows the network to learn both spatial patterns (through kernel sliding) and cross-channel interactions (by mixing information from different input channels). Each output channel $S_{c_{\text{out}}}(i,j)$ captures how strongly a particular learned pattern is detected at location $(i,j)$, where higher values indicate stronger pattern matches. This enables the network to detect complex features that span across multiple input channels while maintaining spatial locality and translation equivariance.

\subsection{Hyperparameters}

\textbf{Kernel size:} typical choices are $3 \times 3$ or $5 \times 5$, with stacked $3\times3$ filters often preferred over a single $5\times5$ as demonstrated in VGG \cite{GoodfellowEtAl2016}.

\textbf{Stride:} the step size for sliding the kernel (stride $s$)\index{stride} determines the output size according to:
\begin{equation}
\text{Output size} = \left\lfloor \frac{n - k}{s} \right\rfloor + 1
\end{equation}

\textbf{Padding:} adding zeros around the input\index{padding} offers different strategies. \textbf{Valid} padding means no padding is added. \textbf{Same} padding adds just enough padding to preserve spatial size. \textbf{Full} padding adds maximum padding. For "same" padding with stride 1, the required padding is:
\begin{equation}
p = \left\lfloor \frac{k-1}{2} \right\rfloor
\end{equation}

% \subsection{Visual Aid}
% \begin{figure}[h]
%     \centering
%     % illustrative kernel sliding grid using TikZ
%     \begin{tikzpicture}[scale=0.5]
%         % input grid 5x5
%         \foreach \i in {0,...,5} {\draw (0,\i) -- (5,\i);}
%         \foreach \j in {0,...,5} {\draw (\j,0) -- (\j,5);}
%         % kernel window 3x3 at (1,1)
%         \draw[bookred,very thick] (1,1) rectangle (4,4);
%     \end{tikzpicture}
%     \caption{A $3\times3$ kernel sliding over a $5\times5$ input highlights local pattern extraction.}
%     \label{fig:conv-sliding-window}
% \end{figure}


% Chapter 9, Section 2

\section{Pooling\index{pooling}\index{max pooling}\index{average pooling}}
\label{sec:pooling}


\textbf{Pooling} reduces spatial dimensions and provides translation invariance.

\subsection*{Intuition}
Pooling summarizes nearby activations so that small translations of the input do not significantly change the summary. Max pooling keeps the strongest response, while average pooling smooths responses. It provides a degree of translation \emph{invariance} complementary to convolution's translation \emph{equivariance} \index{invariance}\index{equivariance}. Modern designs sometimes prefer strided convolutions to make downsampling learnable \cite{GoodfellowEtAl2016,He2016}.

\subsection{Max Pooling}

Takes maximum value in each pooling region:
\begin{equation}
\text{MaxPool}(i,j) = \max_{m,n \in \mathcal{R}_{ij}} I(m,n)
\end{equation}

Common: $2 \times 2$ max pooling with stride 2 (halves spatial dimensions).

\paragraph{Understanding Max Pooling}
Max pooling takes the maximum value within each pooling region, preserving only the strongest activation while discarding weaker responses. This operation provides translation invariance by keeping the strongest response in each region, making the output robust to small spatial shifts in the input. Max pooling is particularly effective for detecting the presence of features (like edges or textures) rather than their exact location, making it useful for building hierarchical representations in CNNs. The operation reduces spatial dimensions while maintaining the most important information, helping the network focus on the most salient features.

\begin{examplebox}{Max Pooling Example}
For input of size $H\times W=32\times32$ and a $2\times2$ window with stride $2$, the output is $16\times16$. Channels are pooled independently.
\end{examplebox}

\subsection{Average Pooling}

Computes average:
\begin{equation}
\text{AvgPool}(i,j) = \frac{1}{|\mathcal{R}_{ij}|} \sum_{m,n \in \mathcal{R}_{ij}} I(m,n)
\end{equation}

\paragraph{Understanding Average Pooling}
Average pooling computes the mean value across each pooling region, providing a smooth summary of local activations rather than preserving only the strongest response like max pooling. This operation reduces spatial dimensions while providing translation invariance by averaging nearby activations, making the output less sensitive to small spatial shifts in the input. Unlike max pooling which preserves the strongest features, average pooling creates a smoother representation that can help reduce noise and provide more stable feature maps. It's particularly useful when you want to preserve information about the overall activation level in a region rather than just the peak response.

\subsection{Global Pooling}

Global pooling operations extend the concept of local pooling to cover entire spatial dimensions, providing powerful mechanisms for reducing feature map complexity while preserving essential information. Global Average Pooling (GAP) computes the average value across all spatial locations for each channel, while Global Max Pooling identifies the maximum value across all spatial locations. These operations prove particularly useful for reducing the parameter count before fully connected layers and for connecting convolutional backbones to classification heads, such as using GAP before softmax layers.

The strategic use of global average pooling can replace large fully connected layers by averaging each feature map to a single scalar value, dramatically reducing overfitting and parameter count while maintaining classification performance. This approach has become particularly popular in modern architectures where the final feature maps are globally pooled before classification, eliminating the need for expensive fully connected layers and providing better generalization properties.\cite{GoodfellowEtAl2016}

\subsection{Alternative: Strided Convolutions\index{strided convolution}}
\label{subsec:strided-convs}

Instead of relying on non-learned pooling operators, strided convolutions with stride $s>1$ perform learned downsampling that can adapt to the specific requirements of the task. For kernel size $k$, stride $s$, and padding $p$, the output spatial dimensions are calculated as $H' = \left\lfloor \frac{H - k + 2p}{s} \right\rfloor + 1$ and $W' = \left\lfloor \frac{W - k + 2p}{s} \right\rfloor + 1$, providing precise control over the downsampling process.

The primary advantages of strided convolutions include their learnable nature and ability to combine feature extraction with downsampling in a single operation, making them particularly effective for stage transitions in modern architectures like ResNet. However, these operations may introduce aliasing artifacts if high-frequency content is not properly low-pass filtered before sub-sampling, leading to the development of anti-aliasing variants that apply blurring before strided operations to preserve signal quality.\index{ResNet}\cite{He2016}

\paragraph{Example.} A $3\times3$ convolution with stride $2$ and padding $1$ keeps spatial size roughly halved (e.g., $32\to16$) while learning filters.

\begin{figure}[h]
    \centering
    \begin{tikzpicture}[x=0.25cm,y=0.25cm]
        % input grid
        \foreach \i in {0,...,15} {\draw (0,\i) -- (16,\i);} 
        \foreach \j in {0,...,16} {\draw (\j,0) -- (\j,16);} 
        % stride-2 sampling dots
        \foreach \i in {1,3,5,7,9,11,13,15} {\foreach \j in {1,3,5,7,9,11,13,15} {\fill[bookpurple] (\j,\i) circle (0.12);} }
        % arrow to output
        \draw[->,thick] (8,-1) -- (24,-1) node[midway,below]{stride 2};
        % output grid 8x8
        \begin{scope}[shift={(24,-0)}]
            \foreach \i in {0,...,8} {\draw (0,\i) -- (8,\i);} 
            \foreach \j in {0,...,8} {\draw (\j,0) -- (\j,8);} 
        \end{scope}
    \end{tikzpicture}
    \caption{Downsampling via stride 2: fewer spatial samples after a strided convolution compared to pooling.}
    \label{fig:strided-conv}
\end{figure}


% Chapter 9, Section 3

\section{CNN Architectures \difficultyInline{intermediate}}
\label{sec:cnn-architectures}

\subsection*{Historical Context}
CNNs evolved from early biologically inspired work to practical systems. \textbf{LeNet-5} established the template for digit recognition \cite{LeCun1989}. \textbf{AlexNet} showed large-scale training with ReLU, dropout, and data augmentation could dominate ImageNet \cite{Krizhevsky2012}. \textbf{VGG} emphasized simplicity via small filters, while \textbf{Inception} exploited multi-scale processing with $1\times1$ dimension reduction. \textbf{ResNet} enabled very deep networks via residual connections \cite{He2016}. Efficiency-driven families like \textbf{MobileNet} and \textbf{EfficientNet} target edge devices and compound scaling.
\subsection{LeNet-5 (1998)\index{LeNet-5}}
\label{subsec:lenet}

LeNet-5 demonstrated the viability of CNNs for handwritten digit recognition (MNIST) \cite{LeCun1989}. It combined convolution, subsampling (pooling), and small fully connected layers. The topology consists of \(\text{Conv}(6@5\times5)\to\text{Pool}(2\times2)\to\text{Conv}(16@5\times5)\to\text{Pool}(2\times2)\to\text{FC}(120)\to\text{FC}(84)\to\text{Softmax}(10)\). The network originally used sigmoid or tanh activations, though later works often retrofit ReLU for pedagogy. Its key properties include local receptive fields, parameter sharing, and early evidence of translation invariance via pooling. The impact of LeNet-5 was profound: it established the core conv-pool pattern and end-to-end learning for vision \cite{GoodfellowEtAl2016}.

\begin{figure}[h]
    \centering
    \begin{tikzpicture}[x=0.2cm,y=0.2cm]
        % input
        \draw[fill=bookpurple!10,draw=bookpurple] (0,0) rectangle (12,12);
        \node at (6,13.5) {Input $32\times32$};
        % conv1
        \draw[->,thick] (12,6) -- (16,6);
        \draw[fill=bookpurple!20,draw=bookpurple] (16,1) rectangle (24,11);
        \node[align=center] at (20,12.8) {Conv1\\$6@5\times5$};
        % pool1
        \draw[->,thick] (24,6) -- (28,6);
        \draw[fill=bookpurple!20,draw=bookpurple] (28,3) rectangle (34,9);
        \node[align=center] at (31,10.8) {Pool $2\times2$};
        % conv2
        \draw[->,thick] (34,6) -- (38,6);
        \draw[fill=bookpurple!20,draw=bookpurple] (38,2) rectangle (46,10);
        \node[align=center] at (42,11.8) {Conv2\\$16@5\times5$};
        % pool2
        \draw[->,thick] (46,6) -- (50,6);
        \draw[fill=bookpurple!20,draw=bookpurple] (50,4) rectangle (55,8);
        \node[align=center] at (52.5,9.8) {Pool};
        % fc
        \draw[->,thick] (55,6) -- (59,6);
        \draw[fill=bookpurple!10,draw=bookpurple] (59,4) rectangle (62,8);
        \node at (60.5,9.8) {FC120};
        \draw[->,thick] (62,6) -- (66,6);
        \draw[fill=bookpurple!10,draw=bookpurple] (66,4) rectangle (69,8);
        \node at (67.5,9.8) {FC84};
        \draw[->,thick] (69,6) -- (73,6);
        \draw[fill=bookpurple!10,draw=bookpurple] (73,4) rectangle (76,8);
        \node at (74.5,9.8) {10};
    \end{tikzpicture}
    \caption{LeNet-5 architecture: alternating conv and pooling, followed by small fully connected layers.}
    \label{fig:lenet5}
\end{figure}

\subsection{AlexNet (2012)\index{AlexNet}}

AlexNet won ILSVRC 2012 by a large margin, catalysing deep learning in vision \cite{Krizhevsky2012}. The design features 5 convolutional layers followed by 3 fully connected layers, incorporating local response normalisation (LRN) and overlapping pooling. For optimisation, ReLU activations enabled faster training, heavy data augmentation improved generalisation, and dropout in fully connected layers reduced overfitting. From a systems perspective, AlexNet was trained on 2 GPUs with model parallelism, used large kernels early in the network, and employed stride for rapid downsampling. Its impact was transformative, establishing large-scale supervised pretraining on ImageNet as a standard approach.

\subsection{VGG Networks (2014)\index{VGG}}

VGG emphasised depth with a simple recipe \cite{GoodfellowEtAl2016}: stacks of $3\times3$ convolutions with stride 1 and $2\times2$ max pooling for downsampling. The architecture uses uniform blocks where replacing large kernels by multiple $3\times3$ layers increases nonlinearity and receptive field with fewer parameters. The most common models are VGG-16 and VGG-19, which have very large parameter counts in their dense layers. These networks exhibit trade-offs: they achieve strong accuracy but are memory and computation heavy, leading to their frequent use as feature extractors rather than end-to-end classifiers.

\subsection{ResNet (2015)\index{ResNet}}
\label{subsec:resnet}

ResNet introduced \textbf{identity skip connections} to learn residual functions \cite{He2016}:
\begin{equation}
\vect{y} = \mathcal{F}(\vect{x}, \{\mat{W}_i\}) + \vect{x},
\end{equation}
where $\mathcal{F}$ is typically a small stack of convolutions and normalisation/activation. This architecture enabled unprecedented depth, with 50-, 101-, and 152-layer models achieving stable optimisation. The key to this success lies in gradient flow: the Jacobian includes an identity term, mitigating vanishing gradients. ResNet uses two types of blocks—basic blocks with two $3\times3$ convolutions, and bottleneck blocks with $1\times1$-$3\times3$-$1\times1$ structure—along with projection shortcuts for dimension changes.

\begin{figure}[h]
    \centering
    \begin{tikzpicture}[x=0.25cm,y=0.25cm]
        % input x
        \draw[fill=bookpurple!10,draw=bookpurple] (0,4) rectangle (6,10);
        \node at (3,11.7) {$\vect{x}$};
        % conv block
        \draw[->,thick] (6,7) -- (9,7);
        \draw[fill=bookpurple!20,draw=bookpurple] (9,5.5) rectangle (15,8.5);
        \node at (12,9.8) {$3\times3$};
        \draw[->,thick] (15,7) -- (18,7);
        \draw[fill=bookpurple!20,draw=bookpurple] (18,5.5) rectangle (24,8.5);
        \node at (21,9.8) {$3\times3$};
        % sum
        \draw[->,thick] (24,7) -- (27,7);
        % skip
        \draw[thick] (3,10) to[out=90,in=180] (15,13) to[out=0,in=90] (27,7);
        % output
        \draw[fill=bookpurple!10,draw=bookpurple] (27,4) rectangle (33,10);
        \node at (30,11.7) {$\vect{y}$};
    \end{tikzpicture}
    \caption{ResNet basic residual block with identity skip connection.}
    \label{fig:resnet-block}
\end{figure}

\subsection{Inception/GoogLeNet (2014)\index{Inception}\index{GoogLeNet}}
\label{subsec:inception}

GoogLeNet popularised \textbf{Inception modules} with parallel multi-scale branches and $1\times1$ bottlenecks for efficiency. Each module contains parallel branches with $1\times1$, $3\times3$, and $5\times5$ convolutions, plus a pooled branch, which are then concatenated channel-wise. The efficiency comes from $1\times1$ convolutions that reduce channel dimensions before larger kernels, cutting FLOPs whilst preserving capacity. The impact was significant: Inception achieved competitive accuracy with fewer parameters than VGG, and its design influenced later hybrid networks.

\begin{figure}[h]
    \centering
    \begin{tikzpicture}[x=0.25cm,y=0.25cm]
        % input
        \draw[fill=bookpurple!10,draw=bookpurple] (0,6) rectangle (6,12);
        \node at (3,13.7) {Input};
        % branches
        \draw[->,thick] (6,9) -- (9,9);
        % 1x1
        \draw[fill=bookpurple!20,draw=bookpurple] (9,10.5) rectangle (15,12.5);
        \node at (12,13.7) {$1\times1$};
        % 3x3
        \draw[fill=bookpurple!20,draw=bookpurple] (9,7.5) rectangle (15,9.5);
        \node at (12,10.7) {$3\times3$};
        % 5x5
        \draw[fill=bookpurple!20,draw=bookpurple] (9,4.5) rectangle (15,6.5);
        \node at (12,7.7) {$5\times5$};
        % pool
        \draw[fill=bookpurple!20,draw=bookpurple] (9,1.5) rectangle (15,3.5);
        \node at (12,4.7) {Pool $3\times3$};
        % concat
        \draw[->,thick] (15,11.5) -- (18,11.5);
        \draw[->,thick] (15,8.5) -- (18,8.5);
        \draw[->,thick] (15,5.5) -- (18,5.5);
        \draw[->,thick] (15,2.5) -- (18,2.5);
        \draw[fill=bookpurple!10,draw=bookpurple] (18,1) rectangle (24,13);
        \node at (21,14.7) {Concat};
    \end{tikzpicture}
    \caption{Inception module: parallel multi-scale branches concatenated along channels with $1\times1$ bottlenecks.}
    \label{fig:inception-module}
\end{figure}

\subsection{MobileNet and EfficientNet\index{MobileNet}\index{EfficientNet}}

\textbf{MobileNet.} Prioritizes efficiency for edge devices using depthwise separable convolutions (depthwise $k\times k$ followed by pointwise $1\times1$), drastically reducing FLOPs and parameters while maintaining accuracy.

\textbf{EfficientNet.} Introduces compound scaling to jointly scale depth, width, and resolution with a principled coefficient, yielding strong accuracy/efficiency trade-offs. Variants (B0--B7) demonstrate near-optimal Pareto fronts.

For introductory treatment of these families, see \textit{D2L} \href{https://d2l.ai/chapter_convolutional-modern/index.html}{(modern CNNs)} and \textit{Deep Learning} \href{https://www.deeplearningbook.org/contents/convnets.html}{(convolutional networks)}.


% Chapter 9, Section 4

\section{Applications of CNNs \difficultyInline{intermediate}}
\label{sec:cnn-applications}

\subsection*{Intuition}
Backbones of stacked convolutions extract spatially local features that become increasingly abstract with depth. Task-specific heads (classification, detection, segmentation) transform backbone features into outputs appropriate to the problem \cite{GoodfellowEtAl2016,Prince2023}.
\subsection{Image Classification}

Image classification represents the foundational task of assigning a single label to an entire image, serving as the basis for most computer vision applications. The standard approach combines a convolutional backbone that extracts hierarchical features with a task-specific head that produces the final classification. The backbone processes the input through multiple convolutional layers with downsampling via pooling or strided convolutions, progressively building more abstract and semantically meaningful representations. The head typically uses global average pooling followed by a small fully connected layer or $1\times1$ convolution with softmax activation to produce class probabilities.

This architecture has proven remarkably effective across diverse datasets including CIFAR-10/100 and ImageNet (ILSVRC), establishing the foundation for transfer learning where ImageNet pretraining commonly improves performance on downstream tasks. The hierarchical feature extraction enables the network to learn from low-level edges and textures to high-level object parts and complete objects, making it particularly suitable for natural image classification where spatial structure and local patterns provide strong discriminative signals.

\subsection{Object Detection}

Object detection extends image classification to simultaneously localize and classify multiple objects within a single image, requiring the network to predict bounding boxes and class labels for each detected object. This task presents unique challenges as it combines spatial localization with classification, demanding architectures that can handle variable numbers of objects at different scales and positions.

Region-based approaches, also known as two-stage methods, first generate region proposals and then classify each region. R-CNN pioneered this approach by applying CNN features to region proposals, though it was computationally expensive. Fast R-CNN and Faster R-CNN improved efficiency by integrating feature extraction, with Faster R-CNN learning region proposals through a Region Proposal Network (RPN). Mask R-CNN extended this framework with an additional instance segmentation branch for pixel-level object boundaries.

Single-shot approaches like YOLO and SSD perform detection in a single pass, making dense predictions at multiple scales for real-time performance. YOLO processes the entire image at once, predicting bounding boxes and class probabilities directly, while SSD uses default boxes across different feature map scales for efficient multi-scale detection. These methods employ specialized heads and losses including classification losses (cross-entropy or focal loss), box regression losses (smooth-$\ell_1$ or IoU losses), and non-maximum suppression (NMS) at inference to eliminate duplicate detections.

\subsection{Semantic Segmentation}

Semantic segmentation represents the most fine-grained computer vision task, requiring the assignment of a class label to each individual pixel in the image. This pixel-level classification demands architectures that can maintain spatial resolution while providing rich semantic understanding, making it particularly challenging compared to classification or detection tasks.

Fully Convolutional Networks (FCNs) revolutionized semantic segmentation by replacing dense layers with $1\times1$ convolutions and upsampling through deconvolution to restore input resolution. U-Net introduced the encoder-decoder architecture with skip connections that preserve fine-grained spatial details for precise localization, becoming particularly popular in medical imaging applications where pixel-level accuracy is critical. Atrous or dilated convolutions provide an alternative approach by enlarging the receptive field without losing spatial resolution, enabling the network to capture both local details and global context simultaneously.

The training and evaluation of semantic segmentation models requires specialized losses and metrics that account for the pixel-level nature of the task. Pixel-wise cross-entropy loss provides the foundation for training, while Dice and IoU losses offer better handling of class imbalance. Mean Intersection over Union (mIoU) serves as the standard evaluation metric, measuring the overlap between predicted and ground truth segmentations across all classes to provide a comprehensive assessment of segmentation quality.\cite{Ronneberger2015}

% Chapter 9, Section 5

\section{Core CNN Algorithms \difficultyInline{intermediate}}
\label{sec:cnn-algorithms}

We introduce algorithms progressively, starting from basic cross-correlation to residual learning.

\subsection{Cross-Correlation and Convolution}
Given input $I\in\mathbb{R}^{H\times W\times C_{\text{in}}}$ and kernel $K\in\mathbb{R}^{k\times k\times C_{\text{in}}\times C_{\text{out}}}$, the output feature map $S\in\mathbb{R}^{H'\times W'\times C_{\text{out}}}$ under stride $s$ and padding $p$ is computed by cross-correlation as in \cref{sec:convolution}. Libraries often refer to this as "convolution" \cite{GoodfellowEtAl2016}.

\subsection{Backpropagation Through Convolution}
Backpropagation through convolution follows the chain rule, where gradients flow backward through the cross-correlation operation to update both input and kernel parameters. For loss $\mathcal{L}$ and pre-activation output $S = I * K$, the gradients are computed as $\frac{\partial \mathcal{L}}{\partial K} = I \star \frac{\partial \mathcal{L}}{\partial S}$ and $\frac{\partial \mathcal{L}}{\partial I} = \frac{\partial \mathcal{L}}{\partial S} * K^\text{rot}$, where $\star$ denotes cross-correlation, $*$ denotes convolution, and $K^\text{rot}$ is the kernel rotated by $180^{\circ}$. The mathematical derivation shows that gradient computation mirrors the forward pass but with rotated kernels, enabling efficient parameter updates through cross-correlation operations.\cite{GoodfellowEtAl2016}

\paragraph{Example (shape-aware):} For $I\in\mathbb{R}^{32\times32\times 64}$ and $K\in\mathbb{R}^{3\times3\times64\times128}$ with stride 1 and same padding, $S\in\mathbb{R}^{32\times32\times128}$. The gradient w.r.t. $K$ accumulates over spatial locations and batch.

\subsection{Pooling Backpropagation}
Pooling backpropagation requires careful handling of gradient routing to maintain the mathematical properties of the forward pass. For max pooling, the upstream gradient is routed exclusively to the maximal input in each pooling region, since only the maximum value contributed to the output. For average pooling, the gradient is evenly divided among all elements in each region, reflecting the equal contribution of each input to the average. This gradient routing ensures that the backward pass accurately reflects the forward computation, enabling proper parameter updates during training.

\subsection{Residual Connections}
Residual connections introduce identity skip paths that fundamentally change gradient flow in deep networks. In a residual block with input $\vect{x}$ and residual mapping $\mathcal{F}$, the output becomes $\vect{y}=\mathcal{F}(\vect{x})+\vect{x}$, where the identity connection provides a direct path for information and gradients. The mathematical derivation of backpropagation yields $\frac{\partial \mathcal{L}}{\partial \vect{x}}=\frac{\partial \mathcal{L}}{\partial \vect{y}}\left(\frac{\partial \mathcal{F}}{\partial \vect{x}}+\mat{I}\right)$, where the identity matrix $\mat{I}$ ensures that gradients can flow directly through the skip connection, stabilizing training and enabling very deep networks.\cite{He2016}

\subsection{Progressive Complexity: Depthwise Separable Convolutions}
Depthwise separable convolution factors standard convolution into depthwise (per-channel) and pointwise ($1\times1$) operations, reducing FLOPs and parameters (used by MobileNet). This keeps representational power while improving efficiency.

\paragraph{Parameter comparison.} For $C_{\text{in}}=C_{\text{out}}=c$ and kernel $k\times k$:
\begin{align}
\text{standard} &= k^2 c^2,\\
\text{depthwise separable} &= k^2 c + c^2,\quad \text{saving} \approx 1 - \frac{k^2 c + c^2}{k^2 c^2}.
\end{align}

\subsection{Normalization and Activation}
Batch normalization and ReLU-family activations play crucial roles in modern CNN training by improving optimization dynamics and generalization performance. Batch normalization addresses covariate shift by normalizing activations across the batch dimension, smoothing the loss landscape and enabling faster convergence with higher learning rates. ReLU and its variants provide sparse activations that reduce computational complexity while maintaining gradient flow, with techniques like Leaky ReLU and ELU addressing the "dying ReLU" problem. These components work together to create more stable training dynamics, allowing networks to learn more effectively from complex visual data.\cite{Ioffe2015}

\subsection{Other Useful Variants}
Modern CNN architectures incorporate several additional techniques that address specific challenges in deep learning. Group and Layer Normalization provide alternatives to batch normalization when batch sizes are small or when batch statistics are unreliable, offering different approaches to activation normalization that can improve training stability. Dilated convolutions expand the receptive field without requiring pooling operations, making them particularly effective in segmentation tasks where spatial resolution must be preserved. Anti-aliased downsampling techniques, including blur pooling and low-pass filtering before strided operations, reduce aliasing artifacts in feature maps, leading to more robust representations that better preserve spatial information during downsampling operations.

% \subsection*{Key Takeaways}
% \begin{itemize}
%     \item Cross-correlation is the practical "convolution" in deep learning; parameter sharing and locality are core.
%     \item Residual connections preserve gradient flow, enabling very deep networks.
%     \item Efficient convolutions (depthwise separable) trade compute for accuracy with strong Pareto gains.
%     \item Normalization and careful downsampling are critical for stable optimization and preserving information.
% \end{itemize}



% Chapter 9: Real World Applications

\section{Real World Applications}
\label{sec:cnn-real-world}


Convolutional neural networks have revolutionized how computers understand images and videos. Their applications touch nearly every aspect of modern visual technology.

\subsection{Medical Image Analysis}

CNNs help doctors diagnose diseases more accurately and quickly:

\begin{itemize}
    \item \textbf{Cancer detection in radiology:} CNNs analyze X-rays, CT scans, and MRIs to detect tumors often invisible to the human eye. For example, mammography systems using CNNs can spot breast cancer earlier than traditional methods, potentially saving thousands of lives annually. The networks learn to recognize subtle patterns that indicate malignancy.
    
    \item \textbf{Diabetic retinopathy screening:} CNNs examine photos of patients' eyes to detect diabetes-related damage before vision loss occurs. This allows automated screening in remote areas without specialist ophthalmologists, making eye care accessible to millions more people worldwide.
    
    \item \textbf{Skin cancer classification:} Smartphone apps with CNNs let people photograph suspicious moles for instant preliminary assessment. While not replacing doctors, these tools encourage early medical consultation when something looks concerning.
\end{itemize}

\subsection{Autonomous Driving}

Self-driving cars rely on CNNs to understand their surroundings:

\begin{itemize}
    \item \textbf{Object detection and tracking:} CNNs process camera feeds to identify pedestrians, other vehicles, traffic signs, and lane markings in real-time. The network must work perfectly under varied conditions—rain, snow, nighttime, construction zones—because lives depend on it.
    
    \item \textbf{Depth estimation:} CNNs analyze images to determine how far away objects are, helping vehicles make safe decisions about braking, turning, and merging. This works even with regular cameras, though it's enhanced when combined with other sensors.
    
    \item \textbf{Semantic segmentation:} CNNs label every pixel in the camera view (road, sidewalk, vehicle, sky, etc.), giving the vehicle complete understanding of its environment. This pixel-level understanding enables precise navigation.
\end{itemize}

\subsection{Content Moderation and Safety}

Social media platforms use CNNs to keep online spaces safe:

\begin{itemize}
    \item \textbf{Inappropriate content detection:} CNNs scan billions of uploaded images and videos daily, automatically flagging harmful content (violence, explicit material, hate symbols) for human review. This happens before most users ever see problematic content.
    
    \item \textbf{Face blurring for privacy:} News organizations and mapping services use CNNs to automatically blur faces and license plates in photos and street view imagery, protecting people's privacy while sharing useful information.
    
    \item \textbf{Copyright protection:} CNNs help platforms identify copyrighted images and videos, preventing unauthorized sharing while allowing legitimate uses. This technology processes millions of uploads per hour.
\end{itemize}

\subsection{Everyday Applications}

CNNs power features you use daily:
\begin{itemize}
    \item \textbf{Photo organization:} Your phone groups photos by people, places, and things
    \item \textbf{Visual search:} Find products by taking photos instead of typing descriptions
    \item \textbf{Document scanning:} Apps automatically detect document edges and enhance readability
    \item \textbf{Augmented reality:} Filters and effects in camera apps that track faces and scenes
\end{itemize}

These applications show how CNNs transform abstract computer vision research into practical tools that improve healthcare, safety, and daily convenience.

% Index entries
\index{applications!medical imaging}
\index{applications!autonomous vehicles}
\index{applications!content moderation}
\index{convolutional networks!applications}

% % Chapter 9, Section 6

% \section{Exercises \difficultyInline{intermediate}}
% \label{sec:cnn-exercises}

% \subsection*{Instructions}
% Attempt the following exercises in order. Hints are provided to guide your reasoning.

% \subsection*{Easy}
% \begin{enumerate}[label=E\arabic*.,leftmargin=*]
%     \item Output Size Basics: For input $32\times32$, kernel $5\times5$, stride $1$, padding $2$, what is the output size?\\
%     Hint: Use $\left\lfloor\frac{n-k+2p}{s}\right\rfloor+1$.
%     \item Parameter Count: For a conv layer with $C_{in}=3$, $C_{out}=16$, kernel $3\times3$, how many weights and biases?\\
%     Hint: Weights $=3\cdot3\cdot C_{in}\cdot C_{out}$; biases $=C_{out}$.
%     \item Receptive Field Growth: Two stacked $3\times3$ convs (stride 1, same padding). What is the effective receptive field?\\
%     Hint: Add $k-1$ per layer.
%     \item Pooling Downsampling: How many $2\times2$ max-pooling layers (stride 2) reduce $128\times128$ to $8\times8$?\\
%     Hint: Halve per layer.
%     \item Cross-Correlation vs Convolution: What is the difference?\\
%     Hint: Kernel flip vs no flip.
%     \item GAP Head: Why does global average pooling reduce parameters compared to FC?\\
%     Hint: Eliminates dense connections over spatial maps.
% \end{enumerate}

% \subsection*{Medium}
% \begin{enumerate}[label=M\arabic*.,leftmargin=*]
%     \item Strided Convolution Shape: Input $64\times64\times64$, $3\times3$ conv, $C_{out}=128$, stride 2, padding 1. What is the output shape?\\
%     Hint: Apply formula per spatial dimension.
%     \item FLOPs Estimate: Estimate multiply-accumulate ops for the above layer.\\
%     Hint: $H'W'\cdot k^2\cdot C_{in}\cdot C_{out}$.
%     \item Residual Block Gradient: Explain why adding identity improves gradient flow.\\
%     Hint: Jacobian adds $\mat{I}$.
%     \item Depthwise Separable Savings: Compare parameters of standard vs depthwise+pointwise for $C_{in}=C_{out}=256$, $k=3$.\\
%     Hint: Standard $=k^2 C_{in}C_{out}$; separable $=k^2 C_{in}+C_{in}C_{out}$.
%     \item VGG vs Inception: Contrast design philosophies.\\
%     Hint: Small uniform filters vs multi-branch multi-scale with $1\times1$ bottlenecks.
% \end{enumerate}

% \subsection*{Hard}
% \begin{enumerate}[label=H\arabic*.,leftmargin=*]
%     \item Derive Backprop for Conv: Starting from $S=I\star K$, derive $\partial \mathcal{L}/\partial K$ and $\partial \mathcal{L}/\partial I$.\\
%     Hint: Use index notation and chain rule with shifts.
%     \item Receptive Field in ResNet Stage: For a stage of $N$ residual blocks each with $3\times3$ convs, stride 1, what is the receptive field increase relative to input?\\
%     Hint: Each $3\times3$ adds 2; consider two per block.
%     \item Alias and Stride: Discuss when strided conv without pre-filtering can alias features.\\
%     Hint: Nyquist; low-pass before downsampling.
%     \item Effective Stride and Dilation: Show how dilation increases receptive field without increasing parameters.\\
%     Hint: Holes in kernels; spacing factor.
%     \item Designing a Lightweight Backbone: Propose a backbone achieving $<300$ MFLOPs at $224\times224$ with competitive accuracy.\\
%     Hint: Depthwise separable convs, squeeze-expansion, or compound scaling.
% \end{enumerate}

% \subsection*{Key Takeaways}
% \begin{itemize}
%     \item Start with simple conv-pool stacks; add residuals and normalization as depth grows.
%     \item Prefer small kernels and learn downsampling where appropriate; consider efficiency via separable convs.
%     \item Choose heads tailored to the task (classification/detection/segmentation) and ensure shapes match.
% \end{itemize}




% Chapter summary and problems
% Key Takeaways for Chapter 9

\section*{Key Takeaways}
\addcontentsline{toc}{section}{Key Takeaways}

\begin{keytakeaways}
\begin{itemize}[leftmargin=2em]
    \item \textbf{Convolution and pooling} exploit spatial structure through parameter sharing and local receptive fields, achieving translation equivariance.
    \item \textbf{Classic architectures} progressively deepened networks (LeNet → AlexNet → VGG → ResNet) via innovations like batch normalisation and residual connections.
    \item \textbf{ResNets solve vanishing gradients} by adding skip connections, enabling training of very deep networks.
    \item \textbf{Modern CNNs balance efficiency and accuracy} through depthwise separable convolutions (MobileNet) and compound scaling (EfficientNet).
    \item \textbf{CNNs excel in vision tasks}: classification, object detection, and semantic segmentation, with task-specific heads and losses.
\end{itemize}
\end{keytakeaways}



% Exercises (Hands-On Exercises) for Chapter 9: Convolutional Networks

\section*{Exercises}
\addcontentsline{toc}{section}{Exercises}

\subsection*{Easy}

\begin{exercisebox}[easy]
\begin{problem}[Receptive Field Calculation]
A CNN has two convolutional layers with 3×3 kernels (no padding, stride 1). Calculate the receptive field of a neuron in the second layer.
\end{problem}
\begin{hintbox}
Each layer expands the receptive field. For the second layer, consider how many input pixels affect it.
\end{hintbox}
\end{exercisebox}


\begin{exercisebox}[easy]
\begin{problem}[Parameter Counting]
Calculate the number of parameters in a convolutional layer with 64 input channels, 128 output channels, and 3×3 kernels (including bias).
\end{problem}
\begin{hintbox}
Each output channel has a 3×3 kernel for each input channel, plus one bias term.
\end{hintbox}
\end{exercisebox}


\begin{exercisebox}[easy]
\begin{problem}[Pooling Operations]
Explain the difference between max pooling and average pooling. When would you prefer one over the other?
\end{problem}
\begin{hintbox}
Consider feature prominence, spatial information retention, and gradient flow.
\end{hintbox}
\end{exercisebox}


\begin{exercisebox}[easy]
\begin{problem}[Translation Equivariance]
Explain what translation equivariance means in the context of CNNs and why it is a desirable property for image processing.
\end{problem}
\begin{hintbox}
If the input is shifted, how does the output change? Consider the relationship $f(T(x)) = T(f(x))$.
\end{hintbox}
\end{exercisebox}


\subsection*{Medium}

\begin{exercisebox}[medium]
\begin{problem}[Output Shape Calculation]
Given an input image of size 224×224×3, apply the following operations and calculate the output shape at each step:
\begin{enumerate}
    \item Conv2D: 64 filters, 7×7 kernel, stride 2, padding 3
    \item MaxPool2D: 3×3, stride 2
    \item Conv2D: 128 filters, 3×3 kernel, stride 1, padding 1
\end{enumerate}
\end{problem}
\begin{hintbox}
Use the formula: $\text{output\_size} = \lfloor \frac{\text{input\_size} + 2 \times \text{padding} - \text{kernel\_size}}{\text{stride}} \rfloor + 1$.
\end{hintbox}
\end{exercisebox}


\begin{exercisebox}[medium]
\begin{problem}[ResNet Skip Connections]
Explain why residual connections (skip connections) help train very deep networks. Discuss the gradient flow through skip connections.
\end{problem}
\begin{hintbox}
Consider the identity mapping $\vect{y} = \vect{x} + F(\vect{x})$ and compute $\frac{\partial \vect{y}}{\partial \vect{x}}$.
\end{hintbox}
\end{exercisebox}


\subsection*{Hard}

\begin{exercisebox}[hard]
\begin{problem}[Dilated Convolutions]
Derive the receptive field for a stack of dilated convolutions with dilation rates [1, 2, 4, 8]. Compare computational cost with standard convolutions achieving the same receptive field.
\end{problem}
\begin{hintbox}
Dilated convolution with rate $r$ introduces $(r-1)$ gaps between kernel elements. Track receptive field growth layer by layer.
\end{hintbox}
\end{exercisebox}


\begin{exercisebox}[hard]
\begin{problem}[Depthwise Separable Convolutions]
Analyse the computational savings of depthwise separable convolutions (as used in MobileNets) compared to standard convolutions. Derive the reduction factor for a layer with $C_{in}$ input channels, $C_{out}$ output channels, and $K×K$ kernel size.
\end{problem}
\begin{hintbox}
Depthwise separable splits into depthwise ($C_{in}$ groups) and pointwise (1×1) convolutions. Compare FLOPs.
\end{hintbox}
\end{exercisebox}


\begin{exercisebox}[hard]
\begin{problem}[Convolutional Layer Design]
Design a convolutional layer architecture for a specific computer vision task. Justify your choices of kernel size, stride, and padding.
\end{problem}
\begin{hintbox}
Consider the trade-off between computational efficiency and feature extraction capability.
\end{hintbox}
\end{exercisebox}


\begin{exercisebox}[hard]
\begin{problem}[Pooling Operations]
Compare different pooling operations (max, average, L2) and their effects on feature maps.
\end{problem}
\begin{hintbox}
Consider the impact on spatial information, gradient flow, and computational efficiency.
\end{hintbox}
\end{exercisebox}


\begin{exercisebox}[hard]
\begin{problem}[Feature Map Visualization]
Explain how to visualize and interpret feature maps in different layers of a CNN.
\end{problem}
\begin{hintbox}
Consider activation maximization, gradient-based methods, and occlusion analysis.
\end{hintbox}
\end{exercisebox}


\begin{exercisebox}[hard]
\begin{problem}[CNN Architecture Search]
Design an automated method for finding optimal CNN architectures for a given dataset.
\end{problem}
\begin{hintbox}
Consider neural architecture search (NAS), evolutionary algorithms, and reinforcement learning approaches.
\end{hintbox}
\end{exercisebox}


\begin{exercisebox}[hard]
\begin{problem}[Transfer Learning Strategies]
Compare different transfer learning approaches for CNNs and when to use each.
\end{problem}
\begin{hintbox}
Consider feature extraction, fine-tuning, and progressive unfreezing strategies.
\end{hintbox}
\end{exercisebox}


\begin{exercisebox}[hard]
\begin{problem}[CNN Interpretability]
Explain methods for understanding and interpreting CNN decisions, including attention mechanisms.
\end{problem}
\begin{hintbox}
Consider gradient-based attribution methods, attention maps, and adversarial analysis.
\end{hintbox}
\end{exercisebox}



