% Chapter 17, Section 3

\section{Importance Sampling \difficultyInline{advanced}}
\label{sec:importance-sampling}

Sample from proposal $q(x)$ instead of target $p(x)$:
\begin{equation}
\mathbb{E}_{p}[f(x)] = \mathbb{E}_{q}\left[\frac{p(x)}{q(x)} f(x)\right] \approx \frac{1}{N} \sum_{i=1}^{N} \frac{p(x^{(i)})}{q(x^{(i)})} f(x^{(i)})
\end{equation}

\textbf{Effective when:}
\begin{itemize}
    \item $q$ is easy to sample from
    \item $q$ has heavier tails than $p$
\end{itemize}


% \subsection{Visual aids}
% \addcontentsline{toc}{subsubsection}{Visual aids (importance sampling)}

% \begin{figure}[h]
%   \centering
%   \begin{tikzpicture}
%     \begin{axis}[
%       width=0.48\textwidth,height=0.36\textwidth,
%       xlabel={Sample index}, ylabel={Normalized weight}, grid=both]
%       \addplot[bookpurple,very thick] coordinates{(1,0.05) (2,0.04) (3,0.03) (4,0.02) (5,0.30) (6,0.01) (7,0.01) (8,0.50) (9,0.03) (10,0.01)};
%     \end{axis}
%   \end{tikzpicture}
%   \caption{Weight degeneracy: a few samples dominate (illustrative).}
%   \label{fig:is-weights}
% \end{figure}

% \subsection{Notes and references}

% See \textcite{Bishop2006,GoodfellowEtAl2016,Prince2023} for analyses of variance and effective sample size.

