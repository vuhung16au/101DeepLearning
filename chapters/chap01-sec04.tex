% Chapter 1, Section 4: Structure of This Book

\section{Structure of This Book \difficultyInline{beginner}}
\label{sec:book-structure}

This book is organized into three main parts, each building upon the previous one to provide a comprehensive understanding of deep learning.

\begin{tcolorbox}[
  colback=bookpurple!10,
  colframe=bookpurple,
  colbacktitle=bookpurple!20,
  coltitle=bookpurple,
  boxrule=1.5pt,
  arc=4pt,
  left=15pt,right=15pt,top=15pt,bottom=15pt,
  fonttitle=\bfseries\large,
  title={\textcolor{bookpurple}{Part I: Basic Math and Machine Learning Foundation}}
]
The first part establishes the mathematical and machine learning foundations necessary for understanding deep learning.
\end{tcolorbox}

\begin{description}[leftmargin=0pt,itemsep=8pt,parsep=0pt]
    \item[\textcolor{bookpurple}{\textbf{Chapter 2: Linear Algebra}}] Covers vectors, matrices, and operations essential for understanding neural network computations.
    
    \item[\textcolor{bookpurple}{\textbf{Chapter 3: Probability and Information Theory}}] Introduces probability distributions, expectation, information theory concepts, and their relevance to machine learning.
    
    \item[\textcolor{bookpurple}{\textbf{Chapter 4: Numerical Computation}}] Discusses numerical optimization, gradient-based optimization, and computational considerations.
    
    \item[\textcolor{bookpurple}{\textbf{Chapter 5: Classical Machine Learning Algorithms}}] Reviews traditional machine learning methods that provide context and motivation for deep learning approaches.
\end{description}

\begin{tcolorbox}[
  colback=bookred!10,
  colframe=bookred,
  colbacktitle=bookred!20,
  coltitle=bookred,
  boxrule=1.5pt,
  arc=4pt,
  left=15pt,right=15pt,top=15pt,bottom=15pt,
  fonttitle=\bfseries\large,
  title={\textcolor{bookred}{Part II: Practical Deep Networks}}
]
The second part focuses on practical aspects of designing, training, and deploying deep neural networks.
\end{tcolorbox}

\begin{description}[leftmargin=0pt,itemsep=8pt,parsep=0pt]
    \item[\textcolor{bookred}{\textbf{Chapter 6: Deep Feedforward Networks}}] Introduces the fundamental building blocks of deep learning, including multilayer perceptrons and activation functions.
    
    \item[\textcolor{bookred}{\textbf{Chapter 7: Regularization for Deep Learning}}] Explores techniques to improve generalization and prevent overfitting.
    
    \item[\textcolor{bookred}{\textbf{Chapter 8: Optimization for Training Deep Models}}] Covers modern optimization algorithms and training strategies.
    
    \item[\textcolor{bookred}{\textbf{Chapter 9: Convolutional Networks}}] Details architectures specifically designed for processing grid-structured data like images.
    
    \item[\textcolor{bookred}{\textbf{Chapter 10: Sequence Modeling}}] Examines recurrent and recursive networks for sequential and temporal data.
    
    \item[\textcolor{bookred}{\textbf{Chapter 11: Practical Methodology}}] Provides guidelines for successfully applying deep learning to real-world problems.
    
    \item[\textcolor{bookred}{\textbf{Chapter 12: Applications}}] Showcases deep learning applications across various domains.
\end{description}

\begin{tcolorbox}[
  colback=warmstone!10,
  colframe=warmstone,
  colbacktitle=warmstone!20,
  coltitle=warmstone,
  boxrule=1.5pt,
  arc=4pt,
  left=15pt,right=15pt,top=15pt,bottom=15pt,
  fonttitle=\bfseries\large,
  title={\textcolor{warmstone}{Part III: Deep Learning Research}}
]
The third part delves into advanced topics and current research directions.
\end{tcolorbox}

\begin{description}[leftmargin=0pt,itemsep=8pt,parsep=0pt]
    \item[\textcolor{warmstone}{\textbf{Chapter 13: Linear Factor Models}}] Introduces probabilistic models with linear structure.
    
    \item[\textcolor{warmstone}{\textbf{Chapter 14: Autoencoders}}] Explores unsupervised learning through reconstruction-based models.
    
    \item[\textcolor{warmstone}{\textbf{Chapter 15: Representation Learning}}] Discusses learning meaningful representations from data.
    
    \item[\textcolor{warmstone}{\textbf{Chapter 16: Structured Probabilistic Models}}] Covers graphical models and their integration with deep learning.
    
    \item[\textcolor{warmstone}{\textbf{Chapter 17: Monte Carlo Methods}}] Introduces sampling-based approaches for probabilistic inference.
    
    \item[\textcolor{warmstone}{\textbf{Chapter 18: Confronting the Partition Function}}] Addresses computational challenges in probabilistic models.
    
    \item[\textcolor{warmstone}{\textbf{Chapter 19: Approximate Inference}}] Explores methods for tractable inference in complex models.
    
    \item[\textcolor{warmstone}{\textbf{Chapter 20: Deep Generative Models}}] Examines modern approaches to generating new data samples.
\end{description}

\begin{tcolorbox}[
  colback=softivory,
  colframe=deepcharcoal,
  colbacktitle=deepcharcoal!20,
  coltitle=deepcharcoal,
  boxrule=1.5pt,
  arc=4pt,
  left=15pt,right=15pt,top=15pt,bottom=15pt,
  fonttitle=\bfseries\large,
  title={\textcolor{deepcharcoal}{How to Use This Book}}
]
This book is designed to accommodate different learning paths and experience levels, recognizing that deep learning attracts learners from diverse backgrounds with varying goals and time constraints.

\vspace{8pt}
\textbf{For Beginners:} Start with Part I to build a strong mathematical and conceptual foundation, then proceed sequentially through Part II to develop practical skills and understanding of core deep learning architectures.

\vspace{8pt}
\textbf{For Practitioners:} With a solid mathematical background, you may choose to skip or skim Part I and focus directly on Parts II and III, using the foundational material as a reference when needed.

\vspace{8pt}
\textbf{For Researchers:} Part III provides advanced material relevant to current research directions in deep learning, including cutting-edge topics like generative models and advanced optimization techniques.

\vspace{8pt}
\textbf{Flexible Learning:} Each chapter is relatively self-contained, allowing you to focus on topics most relevant to your interests or professional needs, whether that's computer vision, natural language processing, or theoretical foundations.

\vspace{8pt}
Throughout the book, we balance theoretical rigor with practical insights, providing both mathematical foundations and intuitive explanations that help bridge the gap between abstract concepts and real-world applications. Code examples and exercises (when available) help reinforce concepts and develop practical skills, ensuring that readers can not only understand the theory but also implement and experiment with the ideas presented.
\end{tcolorbox}
