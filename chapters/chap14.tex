% Chapter 14: Autoencoders

\chapter{Autoencoders}
\label{chap:autoencoders}

This chapter explores autoencoders, neural networks designed for unsupervised learning through data reconstruction.


\section*{Learning Objectives}
\addcontentsline{toc}{section}{Learning Objectives}

After studying this chapter, you will be able to:

\begin{enumerate}
    \item Describe the autoencoder framework and common variants (denoising, sparse, contractive).
    \item Explain the role of bottlenecks and regularization in learning useful representations.
    \item Implement training objectives and evaluate reconstruction vs. downstream utility.
    \item Understand the connection between autoencoders and generative models.
\end{enumerate}



\section*{Intuition}
\addcontentsline{toc}{section}{Intuition}

Autoencoders compress input data into a compact code that retains salient information for reconstruction. Constraining capacity (via architecture or penalties) encourages the model to discard noise and redundancies, surfacing structure that transfers to other tasks.


% Chapter 14, Section 1

\section{Undercomplete Autoencoders \difficultyInline{intermediate}}
\label{sec:undercomplete-ae}

\subsection{Architecture}

An autoencoder consists of:
\begin{itemize}
    \item \textbf{Encoder:} $\vect{h} = f(\vect{x})$ maps input to latent representation
    \item \textbf{Decoder:} $\hat{\vect{x}} = g(\vect{h})$ reconstructs from latent code
\end{itemize}

\subsection{Training Objective}

Minimize reconstruction error:
\begin{equation}
L = \|\vect{x} - g(f(\vect{x}))\|^2
\end{equation}

or more generally:
\begin{equation}
L = -\log p(\vect{x} | g(f(\vect{x})))
\end{equation}

\subsection{Undercomplete Constraint}

If $\dim(\vect{h}) < \dim(\vect{x})$, the autoencoder learns compressed representation.

Acts as dimensionality reduction (similar to PCA but non-linear).

% \subsection{Visual aids}
% \addcontentsline{toc}{subsubsection}{Visual aids (autoencoder)}

% \begin{figure}[h]
%   \centering
%   \begin{tikzpicture}[>=stealth]
%     \tikzstyle{b}=[draw,rounded corners,align=center,minimum width=2.0cm,minimum height=0.8cm]
%     \node[b,fill=bookpurple!10] at (0,0) (x) {Input $\vect{x}$};
%     \node[b,fill=bookpurple!15] at (2.8,0) (enc) {Encoder $f$};
%     \node[b,fill=bookpurple!20] at (5.6,0) (h) {Latent $\vect{h}$};
%     \node[b,fill=bookpurple!15] at (8.4,0) (dec) {Decoder $g$};
%     \node[b,fill=bookpurple!10] at (11.2,0) (xh) {Reconstruction $\hat{\vect{x}}$};
%     \draw[->] (x) -- (enc);
%     \draw[->] (enc) -- (h);
%     \draw[->] (h) -- (dec);
%     \draw[->] (dec) -- (xh);
%   \end{tikzpicture}
%   \caption{Undercomplete autoencoder with a low-dimensional bottleneck.}
%   \label{fig:ae-arch}
% \end{figure}

% \subsection{Notes and references}

% Undercomplete autoencoders learn non-linear compressions beyond PCA; see \textcite{GoodfellowEtAl2016,Prince2023} for guidance on architectures and pitfalls (e.g., identity shortcuts).

% Chapter 14, Section 2

\section{Regularized Autoencoders \difficultyInline{intermediate}}
\label{sec:regularized-ae}

Regularized autoencoders extend basic autoencoders by adding various forms of regularization to encourage learning of more useful and robust representations through constraints on the latent space or training process.

\subsection{Sparse Autoencoders}

Sparse autoencoders add a sparsity penalty on hidden activations through the loss function $L = \|\vect{x} - \hat{\vect{x}}\|^2 + \lambda \sum_j |h_j|$, where the L1 penalty encourages learning of sparse, interpretable features by forcing most hidden units to be inactive for any given input, promoting the discovery of meaningful and independent features.

\subsection{Denoising Autoencoders (DAE)}

Denoising autoencoders train to reconstruct clean input from corrupted versions by first corrupting the input $\tilde{\vect{x}} \sim q(\tilde{\vect{x}}|\vect{x})$, then encoding the corrupted input $\vect{h} = f(\tilde{\vect{x}})$, decoding and reconstructing $\hat{\vect{x}} = g(\vect{h})$, and minimizing the loss $L = \|\vect{x} - \hat{\vect{x}}\|^2$. This approach learns robust representations by forcing the model to recover the original signal from noisy inputs, where corruption types include additive Gaussian noise, masking that randomly sets inputs to zero, and salt-and-pepper noise, with the model learning to identify and remove these corruptions while preserving the essential structure of the data.

\subsection{Contractive Autoencoders (CAE)}

Contractive autoencoders add a penalty on the Jacobian of the encoder through the loss function $L = \|\vect{x} - \hat{\vect{x}}\|^2 + \lambda \left\|\frac{\partial f(\vect{x})}{\partial \vect{x}}\right\|_F^2$, where the Frobenius norm penalty encourages locally contractive mappings that are robust to small perturbations by penalizing large gradients in the encoder function.

% \subsection{Visual aids}
% \addcontentsline{toc}{subsubsection}{Visual aids (regularized AEs)}

% \begin{figure}[h]
%   \centering
%   \begin{tikzpicture}
%     \begin{axis}[
%       width=0.48\textwidth,height=0.36\textwidth,
%       xlabel={Noise level}, ylabel={Reconstruction error}, grid=both]
%       \addplot[bookpurple,very thick] coordinates{(0.0,0.05) (0.1,0.06) (0.2,0.08) (0.3,0.12) (0.4,0.20)};
%       \addplot[bookred,very thick,dashed] coordinates{(0.0,0.05) (0.1,0.07) (0.2,0.11) (0.3,0.20) (0.4,0.35)};
%     \end{axis}
%   \end{tikzpicture}
%   \caption{DAE (solid) vs. plain AE (dashed) under increasing input noise (illustrative).}
%   \label{fig:dae-robust}
% \end{figure}

% \begin{figure}[h]
%   \centering
%   \begin{tikzpicture}
%     \begin{axis}[
%       width=0.48\textwidth,height=0.36\textwidth,
%       xlabel={$\Vert \nabla f(x) \Vert_F^2$ penalty $\lambda$}, ylabel={Val. error}, grid=both]
%       \addplot[bookpurple,very thick] coordinates{(0.0,0.15) (0.1,0.12) (0.2,0.11) (0.4,0.12) (0.8,0.16)};
%     \end{axis}
%   \end{tikzpicture}
%   \caption{Contractive penalty tuning vs. validation error (illustrative).}
%   \label{fig:cae-penalty}
% \end{figure}

% Chapter 14, Section 3

\section{Variational Autoencoders \difficultyInline{intermediate}}
\label{sec:vae}

\subsection{Probabilistic Framework}

VAE is a generative model:
\begin{align}
p(\vect{x}) &= \int p(\vect{x}|\vect{z}) p(\vect{z}) d\vect{z} \\
p(\vect{z}) &= \mathcal{N}(\boldsymbol{0}, \mat{I}) \\
p(\vect{x}|\vect{z}) &= \mathcal{N}(\vect{x}; \boldsymbol{\mu}_{\theta}(\vect{z}), \boldsymbol{\sigma}^2_{\theta}(\vect{z})\mat{I})
\end{align}

\subsection{Evidence Lower Bound (ELBO)}

Cannot directly maximize $\log p(\vect{x})$. Instead maximize ELBO:
\begin{equation}
\mathcal{L} = \mathbb{E}_{q(\vect{z}|\vect{x})}[\log p(\vect{x}|\vect{z})] - D_{KL}(q(\vect{z}|\vect{x}) \| p(\vect{z}))
\end{equation}

where $q(\vect{z}|\vect{x}) = \mathcal{N}(\vect{z}; \boldsymbol{\mu}_{\phi}(\vect{x}), \boldsymbol{\sigma}^2_{\phi}(\vect{x})\mat{I})$ is the encoder.

\subsection{Reparameterization Trick}

To backpropagate through sampling:
\begin{equation}
\vect{z} = \boldsymbol{\mu}_{\phi}(\vect{x}) + \boldsymbol{\sigma}_{\phi}(\vect{x}) \odot \boldsymbol{\epsilon}, \quad \boldsymbol{\epsilon} \sim \mathcal{N}(\boldsymbol{0}, \mat{I})
\end{equation}

Enables end-to-end gradient-based training.

\subsection{Generation}

Sample from prior $\vect{z} \sim \mathcal{N}(\boldsymbol{0}, \mat{I})$ and decode to generate new data.

% \subsection{Visual aids}
% \addcontentsline{toc}{subsubsection}{Visual aids (VAE)}

% \begin{figure}[h]
%   \centering
%   \begin{tikzpicture}
%     \begin{axis}[
%       width=0.48\textwidth,height=0.36\textwidth,
%       xlabel={$z_1$}, ylabel={$z_2$}, grid=both]
%       \addplot+[only marks,mark=*,mark size=0.9pt,bookpurple!70] coordinates{(-1,-1) (-1,0) (-1,1) (0,-1) (0,0) (0,1) (1,-1) (1,0) (1,1)};
%     \end{axis}
%   \end{tikzpicture}
%   \caption{Samples from a 2D latent Gaussian prior (illustrative).}
%   \label{fig:vae-latent}
% \end{figure}

\subsection{Notes and references}

VAEs provide a principled probabilistic framework for representation learning and generation \textcite{Kingma2013,GoodfellowEtAl2016,Prince2023}.

% Chapter 14, Section 4

\section{Applications of Autoencoders \difficultyInline{intermediate}}
\label{sec:ae-applications}

Autoencoders find widespread applications in dimensionality reduction, anomaly detection, denoising, and generative modeling, where their ability to learn compressed representations makes them valuable for various machine learning tasks.

\subsection{Dimensionality Reduction}

Autoencoders learn compact representations for visualization tasks similar to t-SNE and UMAP, where they can reduce high-dimensional data to lower dimensions while preserving important structure and relationships. They serve as effective preprocessing tools for downstream tasks by extracting meaningful features that can improve the performance of subsequent machine learning models. The learned representations are particularly valuable for feature extraction in domains where the original data is high-dimensional and contains redundant information, enabling more efficient processing and analysis.

\subsection{Anomaly Detection}

Autoencoders excel at anomaly detection because high reconstruction error indicates anomalies, where the model learns to reconstruct normal patterns well but struggles with unusual or anomalous data. This principle is applied in fraud detection systems that identify suspicious transactions by measuring reconstruction error, in manufacturing quality control where defects produce high reconstruction errors, and in network intrusion detection where unusual network patterns are flagged based on their reconstruction difficulty.

\subsection{Denoising}

Denoising autoencoders (DAEs) remove noise from various types of data by learning to reconstruct clean signals from corrupted inputs, where they are particularly effective for image denoising by learning to identify and remove various types of noise while preserving important image features. They also excel at audio signal processing where they can separate speech from background noise, and in sensor data applications where they can filter out measurement noise while preserving the underlying signal patterns.


% \subsection{Visual aids}
% \addcontentsline{toc}{subsubsection}{Visual aids (AE applications)}

% \begin{figure}[h]
%   \centering
%   \begin{tikzpicture}
%     \begin{axis}[
%       width=0.48\textwidth,height=0.36\textwidth,
%       xlabel={Input sample}, ylabel={Reconstruction error}, grid=both]
%       \addplot[bookpurple,very thick] coordinates{(1,0.02) (2,0.03) (3,0.02) (4,0.45) (5,0.04) (6,0.03)};
%     \end{axis}
%   \end{tikzpicture}
%   \caption{Spike in reconstruction error indicating an anomaly (illustrative).}
%   \label{fig:ae-anomaly}
% \end{figure}

\subsection{References}

Autoencoders have evolved significantly since their introduction, with key milestones including the development of denoising autoencoders that improved robustness, sparse autoencoders that learned interpretable features, and variational autoencoders that enabled generative modeling. The work by Goodfellow and colleagues has been particularly influential in establishing the theoretical foundations and practical applications of autoencoders, while Prince's contributions have advanced our understanding of their connections to other machine learning techniques. These models have achieved remarkable success in applications ranging from image compression and denoising to drug discovery and creative applications, demonstrating their versatility and practical impact in modern machine learning systems.


% Chapter 14: Real World Applications

\section{Real World Applications}
\label{sec:autoencoder-real-world}


Autoencoders learn to compress and reconstruct data, finding compact representations that capture essential information. This capability enables numerous practical applications in compression, denoising, and anomaly detection.

\subsection{Image and Video Compression}

Efficient storage and transmission of visual data:

\begin{itemize}
    \item \textbf{Next-generation image compression:} Traditional formats like JPEG use hand-crafted compression algorithms. Learned autoencoder-based compression achieves better quality at the same file size or smaller files at the same quality. This matters for websites, cloud storage, and mobile apps where bandwidth and storage costs are significant.
    
    \item \textbf{Video streaming optimization:} Netflix and YouTube experiment with autoencoder-based video compression to stream higher quality video at lower bitrates. This reduces buffering, saves bandwidth costs, and enables HD streaming in areas with limited internet connectivity. The autoencoders learn to preserve perceptually important details humans notice while discarding subtle information we don't.
    
    \item \textbf{Satellite imagery compression:} Earth observation satellites generate terabytes of imagery daily. Autoencoder compression reduces transmission bandwidth from space to ground stations, allowing more frequent imagery updates or higher resolution within bandwidth constraints. This improves applications from weather forecasting to agriculture monitoring.
\end{itemize}

\subsection{Denoising and Enhancement}

Improving signal quality in degraded data:

\begin{itemize}
    \item \textbf{Medical image enhancement:} Denoising autoencoders improve quality of MRI and CT scans, reducing radiation exposure needed for diagnostic-quality images or enabling faster scanning. The autoencoder learns the manifold of healthy tissue appearance, removing noise while preserving medically relevant details like tumor boundaries.
    
    \item \textbf{Old photo restoration:} Consumer apps use autoencoders to remove scratches, stains, and aging artifacts from old photographs. The models learn the structure of clean images and infer what damaged regions likely looked like originally. This helps preserve family histories and restore historical photographs.
    
    \item \textbf{Audio enhancement:} Autoencoders clean up audio recordings, removing background noise, hum, or compression artifacts. This improves voice clarity in phone calls, enhances podcast quality, and helps restore old audio recordings. Unlike simple filtering, autoencoders understand speech structure and preserve natural sound.
\end{itemize}

\subsection{Anomaly Detection}

Identifying unusual patterns in complex systems:

\begin{itemize}
    \item \textbf{Credit card fraud detection:} Autoencoders learn to represent normal spending patterns compactly. Fraudulent transactions often don't fit these patterns well, resulting in poor reconstruction. High reconstruction error flags potential fraud for investigation. This catches novel fraud schemes without requiring examples of every possible type of fraud.
    
    \item \textbf{Industrial equipment monitoring:} Manufacturing plants use autoencoders to monitor vibration patterns, temperatures, and other sensor data from machinery. Normal operation reconstructs well; unusual patterns indicating bearing wear, misalignment, or impending failure show high reconstruction error, triggering maintenance before catastrophic breakdowns.
    
    \item \textbf{Cybersecurity threat detection:} Network security systems use autoencoders trained on normal traffic patterns. Malware, intrusions, and data exfiltration create unusual patterns that reconstruct poorly, alerting security teams. This detects zero-day attacks and insider threats that evade signature-based detection.
\end{itemize}

\subsection{Why Autoencoders Excel}

Key advantages in practical applications:
\begin{itemize}
    \item \textbf{Unsupervised learning:} Don't require labeled examples, just normal data
    \item \textbf{Dimensionality reduction:} Capture essential information compactly
    \item \textbf{Noise robustness:} Learn underlying structure despite corrupted inputs
    \item \textbf{Reconstruction ability:} Can generate clean versions of corrupted data
\end{itemize}

These applications show how autoencoders bridge classical compression and modern deep learning, providing practical solutions for data efficiency, quality enhancement, and anomaly detection.

% Index entries
\index{applications!compression}
\index{applications!denoising}
\index{applications!anomaly detection}
\index{autoencoders!applications}


% Chapter summary and problems
% Key Takeaways for Chapter 14

\section*{Key Takeaways}
\addcontentsline{toc}{section}{Key Takeaways}

\begin{keytakeaways}
\begin{itemize}[leftmargin=2em]
    \item \textbf{Bottlenecks and noise} force representations to capture structure, not memorisation.
    \item \textbf{Regularised variants} (denoising, sparse, contractive) improve robustness and usefulness.
    \item \textbf{Utility beyond reconstruction}: learned codes transfer to downstream tasks.
\end{itemize}
\end{keytakeaways}



% Exercises (Exercises) for Chapter 14

\section*{Exercises}
\addcontentsline{toc}{section}{Exercises}

\subsection*{Easy}

\begin{exercisebox}[easy]
\begin{problem}[Undercomplete AE]
Explain why undercomplete AEs can avoid trivial identity mapping.
\end{problem}
\begin{hintbox}
Bottleneck limits capacity.
\end{hintbox}
\end{exercisebox}


\begin{exercisebox}[easy]
\begin{problem}[Denoising Noise]
How does noise type affect learned features?
\end{problem}
\begin{hintbox}
Gaussian vs. masking vs. salt-and-pepper.
\end{hintbox}
\end{exercisebox}


\begin{exercisebox}[easy]
\begin{problem}[Sparse Codes]
List benefits of sparsity in latent codes.
\end{problem}
\begin{hintbox}
Interpretability, robustness, compression.
\end{hintbox}
\end{exercisebox}


\begin{exercisebox}[easy]
\begin{problem}[Contractive Penalty]
What does a Jacobian penalty encourage?
\end{problem}
\begin{hintbox}
Local invariance.
\end{hintbox}
\end{exercisebox}


\subsection*{Medium}

\begin{exercisebox}[medium]
\begin{problem}[Loss Choices]
Compare MSE vs. cross-entropy for images.
\end{problem}
\begin{hintbox}
Data scale/likelihood assumptions.
\end{hintbox}
\end{exercisebox}


\begin{exercisebox}[medium]
\begin{problem}[Regularisation Trade-offs]
Contrast denoising vs. contractive penalties.
\end{problem}
\begin{hintbox}
Noise robustness vs. local smoothing.
\end{hintbox}
\end{exercisebox}


\subsection*{Hard}

\begin{exercisebox}[hard]
\begin{problem}[Jacobian Penalty]
Derive gradient of contractive loss w.r.t. encoder parameters.
\end{problem}
\begin{hintbox}
Chain rule through Jacobian norm.
\end{hintbox}
\end{exercisebox}


\begin{exercisebox}[hard]
\begin{problem}[Generative Link]
Explain links between AEs and VAEs/flows.
\end{problem}
\begin{hintbox}
Likelihood vs. reconstruction objectives.
\end{hintbox}
\end{exercisebox}



\begin{exercisebox}[hard]
\begin{problem}[Advanced Topic 1]
Explain a key concept from this chapter and its practical applications.
\end{problem}
\begin{hintbox}
Consider the theoretical foundations and real-world implications.
\end{hintbox}
\end{exercisebox}


\begin{exercisebox}[hard]
\begin{problem}[Advanced Topic 2]
Analyse the relationship between different techniques covered in this chapter.
\end{problem}
\begin{hintbox}
Look for connections and trade-offs between methods.
\end{hintbox}
\end{exercisebox}


\begin{exercisebox}[hard]
\begin{problem}[Advanced Topic 3]
Design an experiment to test a hypothesis related to this chapter's content.
\end{problem}
\begin{hintbox}
Consider experimental design, metrics, and potential confounding factors.
\end{hintbox}
\end{exercisebox}


\begin{exercisebox}[hard]
\begin{problem}[Advanced Topic 4]
Compare different approaches to solving a problem from this chapter.
\end{problem}
\begin{hintbox}
Consider computational complexity, accuracy, and practical considerations.
\end{hintbox}
\end{exercisebox}


\begin{exercisebox}[hard]
\begin{problem}[Advanced Topic 5]
Derive a mathematical relationship or prove a theorem from this chapter.
\end{problem}
\begin{hintbox}
Start with the definitions and work through the logical steps.
\end{hintbox}
\end{exercisebox}


\begin{exercisebox}[hard]
\begin{problem}[Advanced Topic 6]
Implement a practical solution to a problem discussed in this chapter.
\end{problem}
\begin{hintbox}
Consider the implementation details and potential challenges.
\end{hintbox}
\end{exercisebox}


\begin{exercisebox}[hard]
\begin{problem}[Advanced Topic 7]
Evaluate the limitations and potential improvements of techniques from this chapter.
\end{problem}
\begin{hintbox}
Consider both theoretical limitations and practical constraints.
\end{hintbox}
\end{exercisebox}


