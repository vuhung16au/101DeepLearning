% Chapter 14: Autoencoders

\chapter{Autoencoders}
\label{chap:autoencoders}

This chapter explores autoencoders, neural networks designed for unsupervised learning through data reconstruction.


\begin{learningobjectives}
\objective{The autoencoder framework and common variants (denoising, sparse, contractive)}
\objective{The role of bottlenecks and regularization in learning useful representations}
\objective{Training objectives and evaluate reconstruction vs. downstream utility}
\objective{The connection between autoencoders and generative models}
\end{learningobjectives}



\section*{Intuition}
\addcontentsline{toc}{section}{Intuition}

Autoencoders are like skilled artists who can recreate a masterpiece from just a few key brushstrokes - they learn to compress complex data into essential features while discarding unnecessary details. The compression process involves reducing the dimensionality of input data while preserving the most important information needed for accurate reconstruction, where the goal is to learn a compact representation that captures the underlying structure and patterns in the data. This compression enables the model to identify and retain only the salient features that are necessary for understanding the data, while discarding noise and redundancies that don't contribute to the essential structure, ultimately surfacing meaningful patterns that can transfer to other tasks and applications.


% Chapter 14, Section 1

\section{Undercomplete Autoencoders \difficultyInline{intermediate}}
\label{sec:undercomplete-ae}

Undercomplete autoencoders are neural networks with a bottleneck architecture that forces the model to learn compressed representations by constraining the latent space to have fewer dimensions than the input space.

\subsection{Architecture}

An autoencoder consists of an encoder $\vect{h} = f(\vect{x})$ that maps input to latent representation and a decoder $\hat{\vect{x}} = g(\vect{h})$ that reconstructs from latent code, where the encoder compresses the input data into a lower-dimensional representation and the decoder attempts to reconstruct the original input from this compressed representation. Autoencoders differ from PCA in that they can learn non-linear transformations and capture complex patterns in the data, while PCA is limited to linear transformations, making autoencoders more powerful for handling non-linear data structures. Shannon's Information Theory is relevant to autoencoders because it provides the theoretical foundation for understanding how much information can be compressed without loss, where the bottleneck constraint forces the model to learn the most efficient representation that preserves the essential information needed for reconstruction. The architecture typically consists of a symmetric network where the encoder gradually reduces dimensionality through hidden layers, creating a bottleneck at the latent layer, and the decoder mirrors this structure to reconstruct the original input, with the mathematical formulation being $\vect{h} = f(\vect{x}) = \sigma(\mat{W}_e \vect{x} + \vect{b}_e)$ for the encoder and $\hat{\vect{x}} = g(\vect{h}) = \sigma(\mat{W}_d \vect{h} + \vect{b}_d)$ for the decoder, where $\sigma$ is the activation function and the weights are learned through training.

\begin{figure}[htbp]
\centering
\begin{tikzpicture}[scale=0.8]
% Input layer
\foreach \i in {1,2,3,4,5,6,7,8}
    \node[circle, draw, fill=orange!70, minimum size=0.6cm] (x\i) at (0,-\i*0.8) {$x_\i$};

% Hidden layer 1
\foreach \i in {1,2,3,4,5}
    \node[circle, draw, fill=bookpurple!30, minimum size=0.6cm] (h1\i) at (3,-\i*1.2) {$h_1^{(\i)}$};

% Latent representation
\foreach \i in {1,2,3}
    \node[circle, draw, fill=bookpurple!50, minimum size=0.6cm] (z\i) at (6,-\i*1.5) {$z_\i$};

% Hidden layer 2
\foreach \i in {1,2,3,4,5}
    \node[circle, draw, fill=bookpurple!30, minimum size=0.6cm] (h2\i) at (9,-\i*1.2) {$h_2^{(\i)}$};

% Output layer
\foreach \i in {1,2,3,4,5,6,7,8}
    \node[circle, draw, fill=blue!60!black, text=white, minimum size=0.6cm] (xh\i) at (12,-\i*0.8) {$\hat{x}_\i$};

% Connections from input to hidden1
\foreach \i in {1,2,3,4,5,6,7,8}
    \foreach \j in {1,2,3,4,5}
        \draw[->] (x\i) -- (h1\j);

% Connections from hidden1 to latent
\foreach \i in {1,2,3,4,5}
    \foreach \j in {1,2,3}
        \draw[->] (h1\i) -- (z\j);

% Connections from latent to hidden2
\foreach \i in {1,2,3}
    \foreach \j in {1,2,3,4,5}
        \draw[->] (z\i) -- (h2\j);

% Connections from hidden2 to output
\foreach \i in {1,2,3,4,5}
    \foreach \j in {1,2,3,4,5,6,7,8}
        \draw[->] (h2\i) -- (xh\j);

% Labels
\node at (0, 0.5) {\textbf{Input Layer}};
\node at (3, 0.5) {\textbf{Encoder}};
\node at (6, 0.5) {\textbf{Latent Representation}};
\node at (9, 0.5) {\textbf{Decoder}};
\node at (12, 0.5) {\textbf{Output Layer}};
\end{tikzpicture}
\caption{Undercomplete autoencoder architecture with a bottleneck at the latent layer.}
\label{fig:autoencoder-architecture}
\end{figure}

\subsection{Training Objective}

The training objective is to minimize reconstruction error, where the loss function $L = \|\vect{x} - g(f(\vect{x}))\|^2$ measures the difference between the original input and its reconstruction, or more generally $L = -\log p(\vect{x} | g(f(\vect{x})))$ for probabilistic reconstruction, where the model learns to encode and decode data by minimizing the reconstruction error between input and output.

\subsection{Undercomplete Constraint}

The undercomplete constraint requires that $\dim(\vect{h}) < \dim(\vect{x})$, forcing the autoencoder to learn compressed representations by constraining the latent space to have fewer dimensions than the input space, where this bottleneck forces the model to learn the most important features and discard redundant information, acting as dimensionality reduction similar to PCA but with non-linear transformations.

% \subsection{Visual aids}
% \addcontentsline{toc}{subsubsection}{Visual aids (autoencoder)}

% \begin{figure}[h]
%   \centering
%   \begin{tikzpicture}[>=stealth]
%     \tikzstyle{b}=[draw,rounded corners,align=center,minimum width=2.0cm,minimum height=0.8cm]
%     \node[b,fill=bookpurple!10] at (0,0) (x) {Input $\vect{x}$};
%     \node[b,fill=bookpurple!15] at (2.8,0) (enc) {Encoder $f$};
%     \node[b,fill=bookpurple!20] at (5.6,0) (h) {Latent $\vect{h}$};
%     \node[b,fill=bookpurple!15] at (8.4,0) (dec) {Decoder $g$};
%     \node[b,fill=bookpurple!10] at (11.2,0) (xh) {Reconstruction $\hat{\vect{x}}$};
%     \draw[->] (x) -- (enc);
%     \draw[->] (enc) -- (h);
%     \draw[->] (h) -- (dec);
%     \draw[->] (dec) -- (xh);
%   \end{tikzpicture}
%   \caption{Undercomplete autoencoder with a low-dimensional bottleneck.}
%   \label{fig:ae-arch}
% \end{figure}

% \subsection{Notes and references}

% Undercomplete autoencoders learn non-linear compressions beyond PCA; see \textcite{GoodfellowEtAl2016,Prince2023} for guidance on architectures and pitfalls (e.g., identity shortcuts).

% Chapter 14, Section 2

\section{Regularized Autoencoders \difficultyInline{intermediate}}
\label{sec:regularized-ae}

Regularized autoencoders extend basic autoencoders by adding various forms of regularization to encourage learning of more useful and robust representations through constraints on the latent space or training process.

\subsection{Sparse Autoencoders}

Sparse autoencoders add a sparsity penalty on hidden activations through the loss function $L = \|\vect{x} - \hat{\vect{x}}\|^2 + \lambda \sum_j |h_j|$, where the L1 penalty encourages learning of sparse, interpretable features by forcing most hidden units to be inactive for any given input, promoting the discovery of meaningful and independent features.

\subsection{Denoising Autoencoders (DAE)}

Denoising autoencoders train to reconstruct clean input from corrupted versions by first corrupting the input $\tilde{\vect{x}} \sim q(\tilde{\vect{x}}|\vect{x})$, then encoding the corrupted input $\vect{h} = f(\tilde{\vect{x}})$, decoding and reconstructing $\hat{\vect{x}} = g(\vect{h})$, and minimizing the loss $L = \|\vect{x} - \hat{\vect{x}}\|^2$. This approach learns robust representations by forcing the model to recover the original signal from noisy inputs, where corruption types include additive Gaussian noise, masking that randomly sets inputs to zero, and salt-and-pepper noise, with the model learning to identify and remove these corruptions while preserving the essential structure of the data.

\subsection{Contractive Autoencoders (CAE)}

Contractive autoencoders add a penalty on the Jacobian of the encoder through the loss function $L = \|\vect{x} - \hat{\vect{x}}\|^2 + \lambda \left\|\frac{\partial f(\vect{x})}{\partial \vect{x}}\right\|_F^2$, where the Frobenius norm penalty encourages locally contractive mappings that are robust to small perturbations by penalizing large gradients in the encoder function.

% \subsection{Visual aids}
% \addcontentsline{toc}{subsubsection}{Visual aids (regularized AEs)}

% \begin{figure}[h]
%   \centering
%   \begin{tikzpicture}
%     \begin{axis}[
%       width=0.48\textwidth,height=0.36\textwidth,
%       xlabel={Noise level}, ylabel={Reconstruction error}, grid=both]
%       \addplot[bookpurple,very thick] coordinates{(0.0,0.05) (0.1,0.06) (0.2,0.08) (0.3,0.12) (0.4,0.20)};
%       \addplot[bookred,very thick,dashed] coordinates{(0.0,0.05) (0.1,0.07) (0.2,0.11) (0.3,0.20) (0.4,0.35)};
%     \end{axis}
%   \end{tikzpicture}
%   \caption{DAE (solid) vs. plain AE (dashed) under increasing input noise (illustrative).}
%   \label{fig:dae-robust}
% \end{figure}

% \begin{figure}[h]
%   \centering
%   \begin{tikzpicture}
%     \begin{axis}[
%       width=0.48\textwidth,height=0.36\textwidth,
%       xlabel={$\Vert \nabla f(x) \Vert_F^2$ penalty $\lambda$}, ylabel={Val. error}, grid=both]
%       \addplot[bookpurple,very thick] coordinates{(0.0,0.15) (0.1,0.12) (0.2,0.11) (0.4,0.12) (0.8,0.16)};
%     \end{axis}
%   \end{tikzpicture}
%   \caption{Contractive penalty tuning vs. validation error (illustrative).}
%   \label{fig:cae-penalty}
% \end{figure}

% Chapter 14, Section 3

\section{Variational Autoencoders \difficultyInline{intermediate}}
\label{sec:vae}

Variational Autoencoders (VAEs) are probabilistic generative models that learn to encode data into a latent space and generate new samples by combining the encoder-decoder architecture with variational inference techniques.

\subsection{Probabilistic Framework}

VAE is a generative model with the marginal likelihood $p(\vect{x}) = \int p(\vect{x}|\vect{z}) p(\vect{z}) d\vect{z}$, where the prior $p(\vect{z}) = \mathcal{N}(\boldsymbol{0}, \mat{I})$ is a standard normal distribution and the likelihood $p(\vect{x}|\vect{z}) = \mathcal{N}(\vect{x}; \boldsymbol{\mu}_{\theta}(\vect{z}), \boldsymbol{\sigma}^2_{\theta}(\vect{z})\mat{I})$ is parameterized by neural networks that output the mean and variance of the reconstruction distribution.

\subsection{Evidence Lower Bound (ELBO)}

Since we cannot directly maximize $\log p(\vect{x})$, we instead maximize the ELBO $\mathcal{L} = \mathbb{E}_{q(\vect{z}|\vect{x})}[\log p(\vect{x}|\vect{z})] - D_{KL}(q(\vect{z}|\vect{x}) \| p(\vect{z}))$, where the first term encourages accurate reconstruction and the second term regularizes the encoder distribution $q(\vect{z}|\vect{x}) = \mathcal{N}(\vect{z}; \boldsymbol{\mu}_{\phi}(\vect{x}), \boldsymbol{\sigma}^2_{\phi}(\vect{x})\mat{I})$ to match the prior.

\subsection{Reparameterization Trick}

The reparameterization trick enables backpropagation through stochastic nodes by writing sampling as a deterministic function of parameters and an auxiliary noise variable. For a Gaussian encoder
\begin{equation}
q_{\phi}(\vect{z}|\vect{x}) = \mathcal{N}\big(\vect{z};\, \boldsymbol{\mu}_{\phi}(\vect{x}),\; \mathrm{diag}(\boldsymbol{\sigma}^2_{\phi}(\vect{x}))\big),
\end{equation}
sample via
\begin{equation}
\vect{z} = \boldsymbol{\mu}_{\phi}(\vect{x}) + \boldsymbol{\sigma}_{\phi}(\vect{x}) \odot \boldsymbol{\epsilon},\qquad \boldsymbol{\epsilon} \sim \mathcal{N}(\boldsymbol{0}, \mat{I}).
\end{equation}
This yields a low-variance gradient estimator of the ELBO because the expectation over $q_{\phi}(\vect{z}|\vect{x})$ becomes an expectation over $\boldsymbol{\epsilon}$ independent of $\phi$:
\begin{align}
\nabla_{\phi} \, \mathbb{E}_{q_{\phi}(\vect{z}|\vect{x})}[\,f(\vect{z})\,] &= \nabla_{\phi} \, \mathbb{E}_{\boldsymbol{\epsilon} \sim \mathcal{N}(0,\mat{I})}\big[ f\big(\boldsymbol{\mu}_{\phi}(\vect{x}) + \boldsymbol{\sigma}_{\phi}(\vect{x}) \odot \boldsymbol{\epsilon}\big) \big] \\
&= \mathbb{E}_{\boldsymbol{\epsilon}}\big[ \nabla_{\phi} f\big(\boldsymbol{\mu}_{\phi}(\vect{x}) + \boldsymbol{\sigma}_{\phi}(\vect{x}) \odot \boldsymbol{\epsilon}\big) \big].
\end{align}
Thus gradients flow through $\boldsymbol{\mu}_{\phi}$ and $\boldsymbol{\sigma}_{\phi}$ via the deterministic mapping while preserving stochasticity through $\boldsymbol{\epsilon}$.

\subsection{Generation}

Generation in VAEs involves sampling from the prior $\vect{z} \sim \mathcal{N}(\boldsymbol{0}, \mat{I})$ and decoding to generate new data, where the decoder network learns to map latent samples to realistic data samples by training on the reconstruction task. This process enables the generation of new samples that follow the learned data distribution, where the quality of generated samples depends on how well the model has learned to capture the underlying data structure and the effectiveness of the latent space representation. The generation process is particularly powerful because it allows for controlled sampling and interpolation in the latent space, enabling the creation of new data points that maintain the statistical properties of the training data while potentially exploring new combinations of learned features.

% \subsection{Visual aids}
% \addcontentsline{toc}{subsubsection}{Visual aids (VAE)}

% \begin{figure}[h]
%   \centering
%   \begin{tikzpicture}
%     \begin{axis}[
%       width=0.48\textwidth,height=0.36\textwidth,
%       xlabel={$z_1$}, ylabel={$z_2$}, grid=both]
%       \addplot+[only marks,mark=*,mark size=0.9pt,bookpurple!70] coordinates{(-1,-1) (-1,0) (-1,1) (0,-1) (0,0) (0,1) (1,-1) (1,0) (1,1)};
%     \end{axis}
%   \end{tikzpicture}
%   \caption{Samples from a 2D latent Gaussian prior (illustrative).}
%   \label{fig:vae-latent}
% \end{figure}

\begin{algorithm}[htbp]
\caption{Variational Autoencoder Training Algorithm}
\label{alg:vae-training}
\begin{algorithmic}[1]
\State \textbf{Input:} Dataset $\mathcal{D} = \{\vect{x}^{(i)}\}_{i=1}^{N}$, learning rate $\alpha$
\State Initialize encoder parameters $\phi$ and decoder parameters $\theta$
\For{epoch = 1 to max\_epochs}
    \For{batch $\mathcal{B} \subset \mathcal{D}$}
        \State Compute encoder outputs: $\boldsymbol{\mu}_{\phi}(\vect{x})$, $\boldsymbol{\sigma}_{\phi}(\vect{x})$
        \State Sample $\boldsymbol{\epsilon} \sim \mathcal{N}(\boldsymbol{0}, \mat{I})$
        \State Reparameterize: $\vect{z} = \boldsymbol{\mu}_{\phi}(\vect{x}) + \boldsymbol{\sigma}_{\phi}(\vect{x}) \odot \boldsymbol{\epsilon}$
        \State Compute decoder outputs: $\boldsymbol{\mu}_{\theta}(\vect{z})$, $\boldsymbol{\sigma}_{\theta}(\vect{z})$
        \State Compute reconstruction loss: $\mathcal{L}_{rec} = \mathbb{E}_{q(\vect{z}|\vect{x})}[\log p(\vect{x}|\vect{z})]$
        \State Compute KL divergence: $\mathcal{L}_{KL} = D_{KL}(q(\vect{z}|\vect{x}) \| p(\vect{z}))$
        \State Total loss: $\mathcal{L} = \mathcal{L}_{rec} - \mathcal{L}_{KL}$
        \State Update parameters: $\phi \leftarrow \phi - \alpha \nabla_{\phi} \mathcal{L}$, $\theta \leftarrow \theta - \alpha \nabla_{\theta} \mathcal{L}$
    \EndFor
\EndFor
\State \textbf{Return} trained encoder $q(\vect{z}|\vect{x})$ and decoder $p(\vect{x}|\vect{z})$
\end{algorithmic}
\end{algorithm}

\subsection{Notes and references}

VAEs provide a principled probabilistic framework for representation learning and generation, representing a significant milestone in generative modeling that combines the power of neural networks with variational inference techniques. The work by Kingma and Welling in 2013 introduced the reparameterization trick that made VAEs trainable, while subsequent research has extended VAEs to various domains including image generation, text modeling, and molecular design. These models have achieved remarkable success in applications ranging from image compression and denoising to drug discovery and creative applications, demonstrating their versatility and practical impact in modern machine learning systems.

% Chapter 14, Section 4

\section{Applications of Autoencoders \difficultyInline{intermediate}}
\label{sec:ae-applications}

Autoencoders find widespread applications in dimensionality reduction, anomaly detection, denoising, and generative modeling, where their ability to learn compressed representations makes them valuable for various machine learning tasks.

\subsection{Dimensionality Reduction}

Autoencoders learn compact representations for visualization tasks similar to t-SNE and UMAP, where they can reduce high-dimensional data to lower dimensions while preserving important structure and relationships. They serve as effective preprocessing tools for downstream tasks by extracting meaningful features that can improve the performance of subsequent machine learning models. The learned representations are particularly valuable for feature extraction in domains where the original data is high-dimensional and contains redundant information, enabling more efficient processing and analysis.

\subsection{Anomaly Detection}

Autoencoders excel at anomaly detection because high reconstruction error indicates anomalies, where the model learns to reconstruct normal patterns well but struggles with unusual or anomalous data. This principle is applied in fraud detection systems that identify suspicious transactions by measuring reconstruction error, in manufacturing quality control where defects produce high reconstruction errors, and in network intrusion detection where unusual network patterns are flagged based on their reconstruction difficulty.

\subsection{Denoising}

Denoising autoencoders (DAEs) remove noise from various types of data by learning to reconstruct clean signals from corrupted inputs, where they are particularly effective for image denoising by learning to identify and remove various types of noise while preserving important image features. They also excel at audio signal processing where they can separate speech from background noise, and in sensor data applications where they can filter out measurement noise while preserving the underlying signal patterns.


% \subsection{Visual aids}
% \addcontentsline{toc}{subsubsection}{Visual aids (AE applications)}

% \begin{figure}[h]
%   \centering
%   \begin{tikzpicture}
%     \begin{axis}[
%       width=0.48\textwidth,height=0.36\textwidth,
%       xlabel={Input sample}, ylabel={Reconstruction error}, grid=both]
%       \addplot[bookpurple,very thick] coordinates{(1,0.02) (2,0.03) (3,0.02) (4,0.45) (5,0.04) (6,0.03)};
%     \end{axis}
%   \end{tikzpicture}
%   \caption{Spike in reconstruction error indicating an anomaly (illustrative).}
%   \label{fig:ae-anomaly}
% \end{figure}

\subsection{References}

Autoencoders have evolved significantly since their introduction, with key milestones including the development of denoising autoencoders that improved robustness, sparse autoencoders that learned interpretable features, and variational autoencoders that enabled generative modeling. The work by Goodfellow and colleagues has been particularly influential in establishing the theoretical foundations and practical applications of autoencoders, while Prince's contributions have advanced our understanding of their connections to other machine learning techniques. These models have achieved remarkable success in applications ranging from image compression and denoising to drug discovery and creative applications, demonstrating their versatility and practical impact in modern machine learning systems.


% Chapter 14: Real World Applications

\section{Real World Applications}
\label{sec:autoencoder-real-world}


Autoencoders learn to compress and reconstruct data, finding compact representations that capture essential information. This capability enables numerous practical applications in compression, denoising, and anomaly detection.

\subsection{Image and Video Compression}

Autoencoders enable efficient storage and transmission of visual data through next-generation image compression that achieves better quality at the same file size or smaller files at the same quality compared to traditional formats like JPEG, where learned compression algorithms matter significantly for websites, cloud storage, and mobile apps where bandwidth and storage costs are substantial. Video streaming optimization applications by Netflix and YouTube experiment with autoencoder-based video compression to stream higher quality video at lower bitrates, reducing buffering, saving bandwidth costs, and enabling HD streaming in areas with limited internet connectivity by learning to preserve perceptually important details humans notice while discarding subtle information we don't. Satellite imagery compression applications handle the terabytes of imagery generated daily by Earth observation satellites, where autoencoder compression reduces transmission bandwidth from space to ground stations, allowing more frequent imagery updates or higher resolution within bandwidth constraints and improving applications from weather forecasting to agriculture monitoring.

\subsection{Denoising and Enhancement}

Autoencoders improve signal quality in degraded data through medical image enhancement where denoising autoencoders improve quality of MRI and CT scans, reducing radiation exposure needed for diagnostic-quality images or enabling faster scanning by learning the manifold of healthy tissue appearance and removing noise while preserving medically relevant details like tumor boundaries. Old photo restoration applications use autoencoders to remove scratches, stains, and aging artifacts from old photographs, where the models learn the structure of clean images and infer what damaged regions likely looked like originally, helping preserve family histories and restore historical photographs. Audio enhancement applications clean up audio recordings by removing background noise, hum, or compression artifacts, improving voice clarity in phone calls, enhancing podcast quality, and helping restore old audio recordings, where unlike simple filtering, autoencoders understand speech structure and preserve natural sound.

\subsection{Anomaly Detection}

Autoencoders identify unusual patterns in complex systems through credit card fraud detection where they learn to represent normal spending patterns compactly, with fraudulent transactions often not fitting these patterns well and resulting in poor reconstruction, where high reconstruction error flags potential fraud for investigation and catches novel fraud schemes without requiring examples of every possible type of fraud. Industrial equipment monitoring applications use autoencoders to monitor vibration patterns, temperatures, and other sensor data from machinery, where normal operation reconstructs well but unusual patterns indicating bearing wear, misalignment, or impending failure show high reconstruction error, triggering maintenance before catastrophic breakdowns. Cybersecurity threat detection systems use autoencoders trained on normal traffic patterns, where malware, intrusions, and data exfiltration create unusual patterns that reconstruct poorly, alerting security teams and detecting zero-day attacks and insider threats that evade signature-based detection.

\subsection{Why Autoencoders Excel}

Autoencoders excel in practical applications because they enable unsupervised learning without requiring labeled examples, just normal data, where they capture essential information compactly through dimensionality reduction and learn underlying structure despite corrupted inputs through noise robustness. Their reconstruction ability allows them to generate clean versions of corrupted data, making them particularly valuable for applications where data quality is important but labels are scarce or expensive to obtain.

\subsection{Autoencoders Compared to other NN Algorithms}

\begin{table}[htbp]
\centering
\begin{tabular}{lccc}
\toprule
Feature & Autoencoder (AE) & Supervised NN (e.g., MLP, CNN) & Generative Adversarial Network (GAN) \\
\midrule
Primary Goal & Data Compression, Feature Learning, Anomaly Detection & Classification or Regression (Mapping input to a specific output label) & Data Generation (Creating new, realistic samples) \\
Learning Type & Unsupervised (Learns from data structure itself) & Supervised (Requires labelled data) & Unsupervised (Learns via a competitive game) \\
Input/Output & Input (x) = Output (x′) (It learns the identity function under constraints) & Input (x) → Output (y) (Target is a label) & Input (Random Noise) → Output (Realistic Data Sample) \\
Loss Function & Reconstruction Error (e.g., Mean Squared Error) & Classification Loss (e.g., Cross-Entropy) or Regression Loss (e.g., MSE) & Adversarial Loss (between Generator and Discriminator) \\
Key Use Case & Dimensionality reduction, Denoising, Anomaly detection (poor reconstruction = anomaly) & Image recognition, Natural Language Processing, Stock prediction & Image synthesis, Deepfakes, Generating realistic data \\
\bottomrule
\end{tabular}
\caption{Comparison of Autoencoders with other neural network algorithms.}
\label{tab:autoencoder-comparison}
\end{table}

These applications show how autoencoders bridge classical compression and modern deep learning, providing practical solutions for data efficiency, quality enhancement, and anomaly detection.

% Index entries
\index{applications!compression}
\index{applications!denoising}
\index{applications!anomaly detection}
\index{autoencoders!applications}


% Chapter summary and problems
% Key Takeaways for Chapter 14

\section*{Key Takeaways}
\addcontentsline{toc}{section}{Key Takeaways}

\begin{keytakeaways}
\begin{itemize}[leftmargin=2em]
    \item \textbf{Bottlenecks and noise} force representations to capture structure, not memorisation.
    \item \textbf{Regularised variants} (denoising, sparse, contractive) improve robustness and usefulness.
    \item \textbf{Utility beyond reconstruction}: learned codes transfer to downstream tasks.
\end{itemize}
\end{keytakeaways}



% Exercises (Exercises) for Chapter 14

\section*{Exercises}
\addcontentsline{toc}{section}{Exercises}

\subsection*{Easy}

\begin{problem}[Undercomplete AE]
Explain why undercomplete AEs can avoid trivial identity mapping.

\textbf{Hint:} Bottleneck limits capacity.
\end{problem}

\begin{problem}[Denoising Noise]
How does noise type affect learned features?

\textbf{Hint:} Gaussian vs. masking vs. salt-and-pepper.
\end{problem}

\begin{problem}[Sparse Codes]
List benefits of sparsity in latent codes.

\textbf{Hint:} Interpretability, robustness, compression.
\end{problem}

\begin{problem}[Contractive Penalty]
What does a Jacobian penalty encourage?

\textbf{Hint:} Local invariance.
\end{problem}

\subsection*{Medium}

\begin{problem}[Loss Choices]
Compare MSE vs. cross-entropy for images.

\textbf{Hint:} Data scale/likelihood assumptions.
\end{problem}

\begin{problem}[Regularisation Trade-offs]
Contrast denoising vs. contractive penalties.

\textbf{Hint:} Noise robustness vs. local smoothing.
\end{problem}

\subsection*{Hard}

\begin{problem}[Jacobian Penalty]
Derive gradient of contractive loss w.r.t. encoder parameters.

\textbf{Hint:} Chain rule through Jacobian norm.
\end{problem}

\begin{problem}[Generative Link]
Explain links between AEs and VAEs/flows.

\textbf{Hint:} Likelihood vs. reconstruction objectives.
\end{problem}


\begin{problem}[Advanced Topic 1]
Explain a key concept from this chapter and its practical applications.

\textbf{Hint:} Consider the theoretical foundations and real-world implications.
\end{problem}

\begin{problem}[Advanced Topic 2]
Analyse the relationship between different techniques covered in this chapter.

\textbf{Hint:} Look for connections and trade-offs between methods.
\end{problem}

\begin{problem}[Advanced Topic 3]
Design an experiment to test a hypothesis related to this chapter's content.

\textbf{Hint:} Consider experimental design, metrics, and potential confounding factors.
\end{problem}

\begin{problem}[Advanced Topic 4]
Compare different approaches to solving a problem from this chapter.

\textbf{Hint:} Consider computational complexity, accuracy, and practical considerations.
\end{problem}

\begin{problem}[Advanced Topic 5]
Derive a mathematical relationship or prove a theorem from this chapter.

\textbf{Hint:} Start with the definitions and work through the logical steps.
\end{problem}

\begin{problem}[Advanced Topic 6]
Implement a practical solution to a problem discussed in this chapter.

\textbf{Hint:} Consider the implementation details and potential challenges.
\end{problem}

\begin{problem}[Advanced Topic 7]
Evaluate the limitations and potential improvements of techniques from this chapter.

\textbf{Hint:} Consider both theoretical limitations and practical constraints.
\end{problem}

