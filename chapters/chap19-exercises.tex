% Exercises (Exercises) for Chapter 19

\section*{Exercises}
\addcontentsline{toc}{section}{Exercises}

\subsection*{Easy}

\begin{exercisebox}[easy]
\begin{problem}[Why Approximate?]
Explain why exact inference is intractable in many models.
\end{problem}
\begin{hintbox}
Partition function; high-dimensional integration.
\end{hintbox}
\end{exercisebox}


\begin{exercisebox}[easy]
\begin{problem}[ELBO Connection]
Relate ELBO to KL divergence between $q$ and $p$.
\end{problem}
\begin{hintbox}
$\log p(x) = \text{ELBO} + D_{KL}(q||p)$.
\end{hintbox}
\end{exercisebox}


\begin{exercisebox}[easy]
\begin{problem}[Mean-Field Assumption]
State the mean-field independence assumption.
\end{problem}
\begin{hintbox}
Factored variational distribution.
\end{hintbox}
\end{exercisebox}


\begin{exercisebox}[easy]
\begin{problem}[MCMC vs. VI]
Compare MCMC and variational inference trade-offs.
\end{problem}
\begin{hintbox}
Asymptotic exactness vs. computational speed.
\end{hintbox}
\end{exercisebox}


\subsection*{Medium}

\begin{exercisebox}[medium]
\begin{problem}[Coordinate Ascent VI]
Derive the coordinate ascent update for a simple model.
\end{problem}
\begin{hintbox}
Fix all but one factor; optimise w.r.t. remaining factor.
\end{hintbox}
\end{exercisebox}


\begin{exercisebox}[medium]
\begin{problem}[Importance Sampling]
Explain how importance sampling estimates expectations.
\end{problem}
\begin{hintbox}
Reweight samples from proposal distribution.
\end{hintbox}
\end{exercisebox}


\subsection*{Hard}

\begin{exercisebox}[hard]
\begin{problem}[Amortised Inference]
Analyse the trade-offs of amortised inference in VAEs.
\end{problem}
\begin{hintbox}
Amortisation gap; scalability.
\end{hintbox}
\end{exercisebox}


\begin{exercisebox}[hard]
\begin{problem}[Reparameterisation Gradients]
Derive the reparameterisation gradient for a Gaussian variational distribution.
\end{problem}
\begin{hintbox}
$z = \mu + \sigma \epsilon$ where $\epsilon \sim \mathcal{N}(0,1)$.
\end{hintbox}
\end{exercisebox}



\begin{exercisebox}[hard]
\begin{problem}[Advanced Topic 1]
Explain a key concept from this chapter and its practical applications.
\end{problem}
\begin{hintbox}
Consider the theoretical foundations and real-world implications.
\end{hintbox}
\end{exercisebox}


\begin{exercisebox}[hard]
\begin{problem}[Advanced Topic 2]
Analyse the relationship between different techniques covered in this chapter.
\end{problem}
\begin{hintbox}
Look for connections and trade-offs between methods.
\end{hintbox}
\end{exercisebox}


\begin{exercisebox}[hard]
\begin{problem}[Advanced Topic 3]
Design an experiment to test a hypothesis related to this chapter's content.
\end{problem}
\begin{hintbox}
Consider experimental design, metrics, and potential confounding factors.
\end{hintbox}
\end{exercisebox}


\begin{exercisebox}[hard]
\begin{problem}[Advanced Topic 4]
Compare different approaches to solving a problem from this chapter.
\end{problem}
\begin{hintbox}
Consider computational complexity, accuracy, and practical considerations.
\end{hintbox}
\end{exercisebox}


\begin{exercisebox}[hard]
\begin{problem}[Advanced Topic 5]
Derive a mathematical relationship or prove a theorem from this chapter.
\end{problem}
\begin{hintbox}
Start with the definitions and work through the logical steps.
\end{hintbox}
\end{exercisebox}


\begin{exercisebox}[hard]
\begin{problem}[Advanced Topic 6]
Implement a practical solution to a problem discussed in this chapter.
\end{problem}
\begin{hintbox}
Consider the implementation details and potential challenges.
\end{hintbox}
\end{exercisebox}


\begin{exercisebox}[hard]
\begin{problem}[Advanced Topic 7]
Evaluate the limitations and potential improvements of techniques from this chapter.
\end{problem}
\begin{hintbox}
Consider both theoretical limitations and practical constraints.
\end{hintbox}
\end{exercisebox}

