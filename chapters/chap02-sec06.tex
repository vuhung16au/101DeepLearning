% Chapter 2, Section 6: Eigendecomposition

\section{Eigendecomposition \difficultyInline{beginner}}
\label{sec:eigendecomposition}

Eigendecomposition is a powerful tool for understanding and analyzing linear transformations, with important applications in deep learning.

\subsection{Eigenvalues and Eigenvectors}

\begin{definition}[Eigenvector and Eigenvalue]
An \emph{eigenvector} of a square matrix $\mat{A} \in \mathbb{R}^{n \times n}$ is a non-zero vector $\vect{v}$ such that:
\begin{equation}
    \mat{A}\vect{v} = \lambda\vect{v}
\end{equation}
where $\lambda \in \mathbb{R}$ (or $\mathbb{C}$) is the corresponding \emph{eigenvalue}.
\end{definition}

The eigenvector's direction is preserved under the transformation $\mat{A}$, with only its magnitude scaled by $\lambda$.

\begin{example}
Consider $\mat{A} = \begin{bmatrix} 3 & 1 \\ 0 & 2 \end{bmatrix}$. We can verify that $\vect{v}_1 = \begin{bmatrix} 1 \\ 0 \end{bmatrix}$ is an eigenvector:
\begin{equation}
    \mat{A}\vect{v}_1 = \begin{bmatrix} 3 & 1 \\ 0 & 2 \end{bmatrix}\begin{bmatrix} 1 \\ 0 \end{bmatrix} = \begin{bmatrix} 3 \\ 0 \end{bmatrix} = 3\vect{v}_1
\end{equation}
So $\lambda_1 = 3$ is an eigenvalue.
\end{example}

\subsection{Finding Eigenvalues}

To find eigenvalues, we solve the \emph{characteristic equation}:
\begin{equation}
    \det(\mat{A} - \lambda\mat{I}) = 0
\end{equation}

This gives a polynomial of degree $n$ called the characteristic polynomial, which has $n$ roots (counting multiplicities) in $\mathbb{C}$.

\begin{example}
For $\mat{A} = \begin{bmatrix} 2 & 1 \\ 1 & 2 \end{bmatrix}$:
\begin{equation}
    \det\begin{bmatrix} 2-\lambda & 1 \\ 1 & 2-\lambda \end{bmatrix} = (2-\lambda)^2 - 1 = \lambda^2 - 4\lambda + 3 = 0
\end{equation}
Solving gives $\lambda_1 = 3$ and $\lambda_2 = 1$.
\end{example}

\subsection{Eigendecomposition}

If a matrix $\mat{A} \in \mathbb{R}^{n \times n}$ has $n$ linearly independent eigenvectors, it can be decomposed as:

\begin{equation}
    \mat{A} = \mat{V}\boldsymbol{\Lambda}\mat{V}^{-1}
\end{equation}

where:
\begin{itemize}
    \item $\mat{V}$ is the matrix whose columns are eigenvectors: $\mat{V} = [\vect{v}_1, \vect{v}_2, \ldots, \vect{v}_n]$
    \item $\boldsymbol{\Lambda}$ is a diagonal matrix of eigenvalues: $\boldsymbol{\Lambda} = \text{diag}(\lambda_1, \lambda_2, \ldots, \lambda_n)$
\end{itemize}

This is called the \emph{eigendecomposition} or \emph{spectral decomposition}.

\subsection{Symmetric Matrices}

Symmetric matrices have particularly nice properties.

\begin{theorem}[Spectral Theorem for Symmetric Matrices]
If $\mat{A}$ is a real symmetric matrix ($\mat{A} = \mat{A}\transpose$), then:
\begin{enumerate}
    \item All eigenvalues are real
    \item Eigenvectors corresponding to different eigenvalues are orthogonal
    \item $\mat{A}$ can be decomposed as:
    \begin{equation}
        \mat{A} = \mat{Q}\boldsymbol{\Lambda}\mat{Q}\transpose
    \end{equation}
    where $\mat{Q}$ is an orthogonal matrix ($\mat{Q}\transpose\mat{Q} = \mat{I}$) of eigenvectors.
\end{enumerate}
\end{theorem}

This decomposition is fundamental in many algorithms, including Principal Component Analysis (PCA).

\subsection{Properties of Eigenvalues}

For a matrix $\mat{A} \in \mathbb{R}^{n \times n}$:

\begin{itemize}
    \item $\text{trace}(\mat{A}) = \sum_{i=1}^{n} A_{ii} = \sum_{i=1}^{n} \lambda_i$
    \item $\det(\mat{A}) = \prod_{i=1}^{n} \lambda_i$
    \item If $\mat{A}$ is invertible, eigenvalues of $\mat{A}^{-1}$ are $1/\lambda_i$
    \item Eigenvalues of $\mat{A}^k$ are $\lambda_i^k$
\end{itemize}

\subsection{Positive Definite Matrices}

\begin{definition}[Positive Definite]
A symmetric matrix $\mat{A}$ is \emph{positive definite} if for all non-zero $\vect{x} \in \mathbb{R}^n$:
\begin{equation}
    \vect{x}\transpose\mat{A}\vect{x} > 0
\end{equation}
Equivalently, all eigenvalues of $\mat{A}$ are positive.
\end{definition}

\begin{definition}[Positive Semi-definite]
$\mat{A}$ is \emph{positive semi-definite} if $\vect{x}\transpose\mat{A}\vect{x} \geq 0$ for all $\vect{x}$, i.e., all eigenvalues are non-negative.
\end{definition}

Positive definite matrices are crucial in optimization, as they ensure that local minima are global minima for quadratic functions.

\subsection{Applications in Deep Learning}

Eigendecomposition has several important applications:

\begin{enumerate}
    \item \textbf{Principal Component Analysis (PCA):} Finds directions of maximum variance by computing eigenvectors of the covariance matrix.
    
    \item \textbf{Optimization:} The Hessian matrix's eigenvalues determine the curvature of the loss surface. Positive definite Hessians indicate convexity.
    
    \item \textbf{Spectral Normalization:} Constrains the largest eigenvalue of weight matrices to stabilize training of GANs.
    
    \item \textbf{Graph Neural Networks:} Graph Laplacian eigendecomposition defines spectral graph convolutions.
    
    \item \textbf{Understanding Dynamics:} Eigenvalues of recurrent weight matrices affect gradient flow and stability.
\end{enumerate}

\subsection{Computational Considerations}

Computing eigendecomposition:
\begin{itemize}
    \item Full eigendecomposition: $O(n^3)$ for dense matrices
    \item Power iteration for dominant eigenvector: $O(kn^2)$ for $k$ iterations
    \item Iterative methods (e.g., Lanczos) for sparse matrices
\end{itemize}

For large-scale deep learning, we often use:
\begin{itemize}
    \item Approximations (e.g., power iteration)
    \item Focus on top-$k$ eigenvalues/eigenvectors
    \item Specialized algorithms for specific structures (e.g., symmetric, sparse)
\end{itemize}

Understanding eigendecomposition provides insight into the geometric properties of linear transformations and is essential for many advanced deep learning techniques.
